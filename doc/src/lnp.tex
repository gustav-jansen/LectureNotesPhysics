\documentclass[oneside,graybox,envcountchap,sectrefs]{svmult}

\usepackage{mathptmx}
\usepackage{helvet}
\usepackage{courier}
\usepackage{braket} % used for Dirac notation
\usepackage{algorithmicx}
\usepackage{algpseudocode} % together for code
\usepackage{amsfonts}
\usepackage{simplewick}
\usepackage{type1cm}         
\usepackage{exercise}
\usepackage{makeidx}         % allows index generation
\usepackage{graphicx}        % standard LaTeX graphics tool
                             % when including figure files
\usepackage{multicol}        % used for the two-column index
\usepackage[bottom]{footmisc}% places footnotes at page bottom
\usepackage[usenames,dvipsnames,x11names]{xcolor}
 \usepackage{listings}
 \usepackage{epic}
 \usepackage{eepic}
 \usepackage{a4wide}
 \usepackage{color}
 \usepackage{amsmath}
 \usepackage{amssymb}
 \usepackage[T1]{fontenc}
 \usepackage{cite} % [2,3,4] --> [2--4]
 \usepackage{shadow}
 \usepackage{hyperref}
 \usepackage{bezier}
 \usepackage{pstricks}
%\usepackage{refcheck}
\setcounter{tocdepth}{2}
\usepackage{textcomp,type1ec,pdfpages}
\usepackage{bera}
%\usepackage{natbib}
\usepackage{chapterbib}
\definecolor{dkgreen}{rgb}{0,0.6,0}
\definecolor{gray}{rgb}{0.5,0.5,0.5}
\definecolor{mauve}{rgb}{0.58,0,0.82}

 \lstset{language=c++}
 \lstset{alsolanguage=[90]Fortran}
 \lstset{alsolanguage=python}
% \lstset{basicstyle=\small}
 \lstset{backgroundcolor=\color{white}}
 \lstset{frame=single}
 \lstset{stringstyle=\ttfamily}
 \lstset{keywordstyle=\color{red}\bfseries}
 \lstset{commentstyle=\itshape\color{blue}}
 \lstset{showspaces=false}
 \lstset{showstringspaces=false}
 \lstset{showtabs=false}
 \lstset{breaklines}
 

% Default settings for code listings
% \lstnewenvironment{Python}[1]{
\lstset{%frame=tb,
  language=c++,
  alsolanguage=python,
  %aboveskip=3mm,
 % belowskip=3mm,
  showstringspaces=false,
  columns=flexible,
  basicstyle={\footnotesize\ttfamily},
  numbers=none,
  numberstyle=\tiny\color{gray},
  commentstyle=\color{dkgreen},
  stringstyle=\color{mauve},
 frame=single,  
  breaklines=true,
  %%%% FOR PYTHON 
  otherkeywords={\ , \}, \{},
  keywordstyle=\color{blue},
  emph={void, ||, &&, break, class,continue, delete, else,
  for, if, include, return,try,while},
  emphstyle=\color{black}\bfseries,
  emph={[2]True, False, None, self},
  emphstyle=[2]\color{dkgreen},
  emphstyle=[2]\color{red},
  emph={[3]from, import, as},
  emphstyle=[3]\color{blue},
  upquote=true,
  morecomment=[s]{"""}{"""},
  commentstyle=\color{green}\slshape, %%% cambie gray por green
  emph={[4]1, 2, 3, 4, 5, 6, 7, 8, 9, 0},
  emphstyle=[4]\color{blue},
  breakatwhitespace=true,
  tabsize=2
}

\renewcommand{\lstlistlistingname}{Code Listings}
\renewcommand{\lstlistingname}{Code Listing}
\definecolor{gray}{gray}{0.5}
\definecolor{green}{rgb}{0,0.5,0}

\lstnewenvironment{Python}[1]{
\lstset{
language=python,
basicstyle=\footnotesize\setstretch{1},
stringstyle=\color{red},
showstringspaces=false,
alsoletter={1234567890},
otherkeywords={\ , \}, \{},
keywordstyle=\color{blue},
emph={access,and,break,class,continue,def,del,elif ,else,%
except,exec,finally,for,from,global,if,import,in,is,%
lambda,not,or,pass,print,raise,return,try,while},
emphstyle=\color{black}\bfseries,
emph={[2]True, False, None, self},
emphstyle=[2]\color{red},
emph={[3]from, import, as},
emphstyle=[3]\color{blue},
upquote=true,
morecomment=[s]{"""}{"""},
commentstyle=\color{dkgreen}\slshape, % el color era gray pero lo cambie a verde
emph={[4]1, 2, 3, 4, 5, 6, 7, 8, 9, 0},
emphstyle=[4]\color{blue},
framexleftmargin=1mm, framextopmargin=1mm, rulesepcolor=\color{blue},
breakatwhitespace=true,
tabsize=2
}}{}


\lstnewenvironment{C++}[1]{
\lstset{
language=c++,
% basicstyle=\ttfamily\small\setstretch{1},
basicstyle=\footnotesize\setstretch{1},
stringstyle=\color{red},
showstringspaces=false,
alsoletter={1234567890},
otherkeywords={\ , \}, \{},
keywordstyle=\color{blue},
emph={access,and,break,class,continue,def,del,elif ,else,%
except,exec,finally,for,from,global,if,import,in,is,%
lambda,not,or,pass,print,raise,return,try,while},
emphstyle=\color{black}\bfseries,
emph={[2]True, False, None, self},
emphstyle=[2]\color{red},
emph={[3]from, import, as},
emphstyle=[3]\color{blue},
upquote=true,
morecomment=[s]{"""}{"""},
commentstyle=\color{dkgreen}\slshape, % el color era gray pero lo cambie a verde
emph={[4]1, 2, 3, 4, 5, 6, 7, 8, 9, 0},
emphstyle=[4]\color{blue},
% literate=*{:}{{\textcolor{blue}:}}{1}%
% {=}{{\textcolor{blue}=}}{1}%
% {-}{{\textcolor{blue}-}}{1}%
% {+}{{\textcolor{blue}+}}{1}%
% {*}{{\textcolor{blue}*}}{1}%
% {!}{{\textcolor{blue}!}}{1}%
% {(}{{\textcolor{blue}(}}{1}%
% {)}{{\textcolor{blue})}}{1}%
% {[}{{\textcolor{blue}[}}{1}%
% {]}{{\textcolor{blue}]}}{1}%
% {<}{{\textcolor{blue}<}}{1}%
% {>}{{\textcolor{blue}>}}{1},%
framexleftmargin=1mm, framextopmargin=1mm, rulesepcolor=\color{blue},
breakatwhitespace=true,
tabsize=2
}}{}

\makeindex             % used for the subject index
                       % please use the style svind.ist with
                       % your makeindex program

%%%%%%%%%%%%%%%%%%%%%%%%%%%%%%%%%%%%%%%%%%%%%%%%%%%%%%%%%%%%%%%%%%%%%

\begin{document}
\frontmatter

\begin{titlepage}
\title{An advanced course in computational nuclear physics}
 \subtitle{Bridging the scales from quarks to neutron stars}
%\date{2016}
\author{Morten Hjorth-Jensen, Maria Paola Lombardo, and Ubirajara van Kolck, Editors}
\end{titlepage}
\maketitle



\preface
This graduate-level text collects and synthesizes ten series of
lectures on the nuclear quantum many-body problem - starting from our
present understanding of the underlying forces with a presentation of
recent advances within the field of lattice quantum chromodynamics,
via effective field theories to central many-body methods like Monte
Carlo methods, coupled cluster theories, the similarity renormalization group approach, Green's function methods 
and large-scale
diagonalization approaches.

In particular algorithmic and computational advances show promise for
breakthroughs in predictive power including proper error estimates, a
better understanding of the underlying effective degrees of freedom
and of the respective forces at play.

Enabled by recent advances in theoretical, experimental and numerical
techniques, the modern and state-of-the art applications considered in
this volume span the entire range from our smallest components, quarks
and gluons as the mediators of the strong force to the computation of
the equation of state for neutron star matter.

 

The present lectures provide a proper exposition of the underlying
theoretical and algorithmic approaches as well as strong ties to the
numerical implementation of the exposed methods. Several of the
lectures provides a proper link to actual numerical software and benchmark calculations. The
codes provided with this text will enable the reader to build upon these and develop her/his
own insights about these methods, as well as using the corresponding
codes for developing own programs for tackling challenging nuclear
many-body problems.



\tableofcontents

\mainmatter
\title{Motivation and overarching aims}
\author{Morten Hjorth-Jensen, Maria Paola Lombardo, and Ubirajara van Kolck}
\institute{Morten Hjorth-Jensen  \at Department of Physics and Astronomy and National Superconducting Cyclotron Laboratory, Michigan State University, East Lansing, Michigan, USA and Department of Physics, University of Oslo, Oslo, Norway, \email{hjensen@msu.edu}, \and Maria Paola Lombardo \at INFN, Laboratori Nazionali di Frascati, Frascati, Italy, \email{mariapaola.lombardo@lnf.infn.it}, \and Ubirajara van Kolck \at Department of Physics, University of Arizona, Tucson, Arizona, USA and Institut de Physique Nucléaire, Orsay, France, \email{vankolck@ipno.in2p3.fr}}
\maketitle





Nuclear physics has recently experienced several discoveries and
technological advances that address the fundamental questions of the
field, in particular how nuclei emerge from the strong dynamics
of quantum chromodynamics (QCD).
Many of these advances have been made possible by significant
investments in frontier research facilities worldwide over the last
two decades. Some of these discoveries are the detection of perhaps
the most exotic state of matter, the quark-gluon plasma, which is
believed to have existed in the very first moments of the Universe.  
Recent experiments have validated the standard solar model
and established that neutrinos have mass. High-precision
measurements of the quark structure of the nucleon are challenging
existing theoretical understanding.  Nuclear physicists have started
to explore a completely unknown landscape of nuclei with extreme
neutron-to-proton ratios using radioactive and short-lived ions,
including rare and very neutron-rich isotopes.  These experiments push
us towards the extremes of nuclear stability.  Moreover, these rare
nuclei lie at the heart of nucleosynthesis processes in the universe
and are therefore an important component in the puzzle of matter
generation in the universe.

A firm experimental and theoretical understanding of nuclear stability
in terms of the basic constituents is a huge intellectual endeavor.
Experiments indicate that developing a comprehensive description of
all nuclei and their reactions requires theoretical and experimental
investigations of rare isotopes with unusual neutron-to-proton ratios
that are very different from their stable counterparts.  These rare
nuclei are difficult to produce and study experimentally since they
can have extremely short lifetimes. Theoretical approaches to these
nuclei involve solving the nuclear many-body problem.

Accompanying these developments, a qualitative change has swept the
nuclear theory landscape thanks to a combination of techniques that are
allowing, for the first time, to construct links between QCD and
the nuclear many-body problem. This transformation has been brought by a dramatic
improvement in the capability of numerical calculations both in QCD,
via lattice simulations, and in the nuclear many-body problem via first principle or {\em ab initio} 
many-body methods that employ non-relativistic
Hamiltonians. Simultaneously, effective field
theories attempt at building  a bridge between the two numerical approaches,
allowing to convert the results of lattice QCD into input Hamiltonians that can be used in {\em ab initio}
methods.

Algorithmic and computational advances hold promise for
breakthroughs in predictive power including proper error estimates,
enhancing the already strong ties between theory and experiment.
These advances include better {\em ab initio} many-body methods as well as a
better understanding of the underlying effective degrees of freedom
and the respective forces at play.  Similarly, we have recently witnessed a significant improvement in numerical
algorithms and high-performance computing.
This provides us with important new insights about the stability
of nuclear matter and allows us to relate these novel understandings to
the underlying laws of motion, the corresponding forces and the
pertinent fundamental building blocks of nuclear matter.

Important issues such as whether we can explain from first-principle
methods the existence of magic numbers and their vanishing as we add
more and more nucleons, how the binding energy of neutron-rich nuclei
behaves, or the radii, neutron skins, and many many other probes that
extract information about many-body correlations as nuclei evolve
towards their limits of stability --- these are all fundamental
questions which, combined with recent experimental and theoretical
advances, will allow us to advance our basic knowledge about the
limits of stability of matter, and, hopefully, help us in gaining a
better understanding of visible matter.

It is within this framework the present set of lectures finds its rationale.
This text collects and synthesizes ten series of lectures on the
nuclear many-body problem, starting from our present understanding of
the underlying forces with a presentation of recent advances within
the field of lattice QCD, via effective field theories to central
many-body methods like various Monte Carlo approaches, coupled-cluster
theory, the similarity renormalization group approach, Green's
function methods and large-scale diagonalization methods.  The
applications span from our smallest components, quarks and gluons as
the mediators of the strong force to the computation of the equation
of state for infinite nuclear matter and neutron star matter.  The
lectures provide a proper exposition of the underlying theoretical and
algorithmic approaches as well as strong ties to the numerical
implementation of the exposed methods.  The various chapters propose
exercises meant to deepen the exposed theory as well as  actual numerical software.
The latter will enable the reader to build upon these and develop
her/his own insights about these methods, as well as using these codes
for developing her/his own programs for tackling complicated many-body
problems.  Proper benchmarks for the various programs are also provided, allowing thereby potential readers and users to check the 
correctness, installation and compilation  of the various programs. All codes are properly linked in the various chapters and available via the github link \url{https://github.com/ManyBodyPhysics/LectureNotesPhysics}. 


\begin{acknowledgement}
The different chapters are based on lectures given at the Doctoral Training program {\em Computational Nuclear Physics - Hadrons, Nuclei and Dense Matter} held at  the European Center for Theoretical Nuclear Physics
and Related Areas (ECT*) in Trento, Italy, from April 13 to May 22, 2015, the Nuclear Talent course {\em Many-body methods for nuclear physics}, held at GANIL, Caen, France, from July 5 to July 25,  2015, and the Nuclear Talent course {\em High-performance computing and computational tools for nuclear physics} held at North Carolina State University from July 11 to July 29, 2016. For more information about the Nuclear Talent courses see \url{http://www.nucleartalent.org}. For the Doctoral Training program of the ECT*, see \url{http://www.ectstar.eu/node/799}.
The support for organizing these series of lectures from the ECT*, GANIL and the University of Basse Normadie at Caen, North Carolina State University, Los Alamos National Laboratory and Michigan State University is greatly acknowledged.   Some of the lectures (chapters 8 and 10) are co-authored by students (Michigan State University) who  attended the abovementioned Nuclear Talent courses.


The work of MHJ is supported by NSF Grant No.~PHY-1404159 (Michigan State University). 
\end{acknowledgement}

%\bibliographystyle{spphys}
%\bibliography{lnplib}

\include{chapter2}\label{chap:chapter2}
\title{Lattice quantum chromodynamics}\label{chap:latticeqcd}
\author{Tetsuo Hatsuda} 
\institute{Tetsuo Hatsuda \at Nishina Center, RIKEN, Saitama 351-0198, Japan,  \email{thatsuda@riken.jp}}
\maketitle
\abstract{
%Each chapter should be preceded by an abstract (10--15 lines long) that summarizes the content. The abstract will appear \textit{online} at \url{www.SpringerLink.com} and be available with unrestricted access. This allows unregistered users to read the abstract as a teaser for the complete chapter. As a general rule the abstracts will not appear in the printed version of your book unless it is the style of your particular book or that of the series to which your book belongs.\newline\indent
%Please use the 'starred' version of the new Springer \texttt{abstract} command for typesetting the text of the online abstracts (cf. source file of this chapter template \texttt{abstract}) and include them with the source files of your manuscript. Use the plain \texttt{abstract} command if the abstract is also to appear in the printed version of the book.
Concepts and applications of lattice quantum chromodynamics (LQCD) are introduced.
After discussing how to define quarks and gluons on the Euclidean hypercubic lattice, 
 the strong coupling expansion  and the weak coupling expansions are reviewed 
 to see the vital role played by the quantum fluctuations in QCD.
 Fundamental techniques for numerical LQCD simulations  such as the Markov Chain Monte Carlo method and the
  Hybrid Monte Carlo method are discussed in some details.  As 
  examples of the high precision LQCD simulations,  numerical results of the heavy quark potential and the hadron masses
   are shown.  Recent LQCD results on the baryon-baryon interactions are briefly discussed.
   }

% macros %%%%%%%%%%%%%%%%%%%%%%%%%%%%%%%%%%%
%\renewcommand{\theequation}{\thesection.\arabic{equation}}
\newcommand{\beq}{\begin{eqnarray}}
\newcommand{\eeq}{\end{eqnarray}}
\newcommand{\la}{\langle}
\newcommand{\ra}{\rangle}
%% vector-notation
\newcommand{\vx}{\vec{x}}
\newcommand{\vy}{\vec{y}}
\newcommand{\vk}{\vec{k}}
\newcommand{\LRDelta}{\overleftrightarrow{\Delta}}
\newcommand{\vgamma}{\vec{\gamma}} 
\newcommand{\vA}{\vec{A}}
\newcommand{\vr}{\vec{r}}
\newcommand{\vn}{\vec{n}}
\newcommand{\vm}{\vec{m}}
\newcommand{\vv}{\vec{v}}
\newcommand{\vp}{\vec{p}}
\newcommand{\vsigma}{\vec{\sigma}}
\newcommand{\vL}{\vec{L}}
\newcommand{\vS}{\vec{S}}
\newcommand{\vT}{\vec{T}}
%% lattice
\newcommand{\Rs}{R_{\rm s}}
\newcommand{\Rt}{R_{\rm t}}
\newcommand{\Ns}{N_{\rm s}}
\newcommand{\Ntau}{N_{\tau}}
\newcommand{\at}{a_{\rm t}}
\newcommand{\as}{a_{\rm s}}
\newcommand{\Lamqcd}{ \Lambda_{_{\rm QCD}} } 
\newcommand{\Lamlat}{ \Lambda_{_{\rm LAT}} } 
\newcommand{\Lammsb}{ \Lambda_{_{\overline{\rm MS} }}  } 
%%%%%%%%%%%%%%%%%%%%%%%%%%%%%%%%%%%%%%%%%%%%%



%%%%%%%%%%%%%
\section{Introduction}
%%%%%%%%%%%%%

In this chapter, we introduce  lattice quantum chromodynamics (LQCD) 
originally proposed by K. Wilson in 1974 \cite{Wilson:1974sk}. What makes the LQCD unique and powerful is that
it can allow first-principle,  gauge invariant and non-perturbative calculations of strongly interacting
quarks and gluons.  After the first numerical attempts by M. Creutz \cite{Creutz:1980zw},
 LQCD simulations have been extensively applied to study heavy quark potentials,
hadron masses,  hadronic matrix elements, QCD phase transition at finite temperature, 
and so on.  In recent years, there are also progresses 
in deriving the  baryon-baryon interactions, which are particularly relevant to 
nuclear physics and astrophysics. Throughout this chapter, we will focus on the theoretical concepts, numerical techniques and some
applications to hadron masses and nuclear forces.  LQCD at finite temperature and/or baryon density will
not be covered.  The interested readers may want to consult the  following review articles for further details, or may even 
want to run the open source codes or to use the open LQCD configurations.

\vspace{0.3cm}

 \noindent
 {\bf Review articles:}   Here is a list a few articles which are useful to learn more about LQCD.  
 \begin{itemize}
\item Comprehensive review on QCD can be seen in \cite{Brambilla:2014jmp}.
\item The origin of LQCD is discussed in \cite{Wilson:2004de}.
\item The basic concepts  of LQCD are summarized in the monographs \cite{Creutz:1984mg,Rothe:1992nt}.
 \item Recent progresses of LQCD can be seen in the reviews \cite{Hoelbling:2014uea,Ukawa:2015eka} and references therein. 
 \end{itemize}

\vspace{0.3cm}

 \noindent
 {\bf Open source codes:} 
 It takes a lot of time and energy to develop the LQCD code from scratch. To lower the bar,
 several source codes for LQCD simulations have been released for public use.
  \begin{itemize}
\item Bridge++: \url{http://bridge.kek.jp/Lattice-code/index_e.html}
\item Lattice ToolKit: \url{http://nio-mon.riise.hiroshima-u.ac.jp/LTK/}
\item OpenQCD: \url{http://luscher.web.cern.ch/luscher/openQCD/index.html}
\item USQCD: \url{http://usqcd-software.github.io/}
\end{itemize}

\vspace{0.3cm}
 \noindent
{\bf LQCD configurations:}  Outputs of large scale LQCD simulations  is a {\it big data}  
called "LQCD configurations".   Physicists can study various aspects of QCD by using these configurations.
 The International Lattice Data Grid (ILDG) is a project to share the configurations around the world.
  \begin{itemize}
 \item ILDG \url{http://plone.jldg.org/wiki/index.php/Main_Page}
\end{itemize}
 
%================================
  \subsection{Euclidean QCD action}
%================================
LQCD is formulated on the Euclidean spacetime lattice.
Observables in the Minkowski spacetime are obtained by the analytic continuation of the 
imaginary-time $\tau$ to the real-time $t$.  In terms of the time evolution operator in quantum mechanics,
it corresponds to the continuation from the imaginary time evolution, $e^{-H \tau} $, to the real time
evolution, $e^{-iHt}$, where $H$ being the Hamiltonian of the system.  The functional integral representation of the 
the Euclidean QCD partition function ${\cal Z}$  on a finite spatial box $L^3$  and the temperature $T$ is given by
\beq
\label{eq:Z-QCD}
{\cal Z}(T,V,J) = \int [dA d\bar{q} dq] e^{- \int_{0}^{1/T}  d\tau \int_{L^3} d^3 x \left( {\cal L}_{\rm QCD}^{\rm E} + J \Xi \right) }
\eeq
where the Euclidean QCD Lagrangian in terms of quarks $q^{\alpha=1,2,3}$ and gluons $A_{\mu=1,2,3,4}^{b=1, \cdots, 8}$ 
is given by
\beq
{\cal L}_{\rm QCD}^{\rm E} = \bar{q}^{\alpha} (\Gamma_{\mu}  D_{\mu}^{\alpha \beta} + m \delta^{\alpha \beta}) q^{\beta} 
+ \frac{1}{4} G_{\mu \nu}^b G_{\mu \nu}^b,
\eeq
Here the Euclidean version of the $\gamma$-matrices, $\Gamma_\mu$, is defined in
 the Appendix [Four vectors and Dirac matrices].
The quark mass matrix in the flavor space ($u, d, s, \cdots$) is denoted by $m$ with the flavor indices suppressed.
 The color covariant derivative is defined by
\beq
D_{\mu}^{\alpha \beta} = \partial_{\mu} \delta^{\alpha \beta} + ig A_{\mu}^{\alpha \beta}, 
\eeq
with  $x_{\mu}= (\tau, \vx)$,  $\partial_{\mu} = (\partial_{\tau}, \nabla )$ and the $3 \times 3$ matrix field, $A_{\mu} = A_{\mu}^b t^b$. 
The explicit form of the color SU(3) generators $t^{a=1, \cdots, 8}$ is given in the Appendix [SU$(N)$ algebra]. 
The field strength tensor is $G_{\mu \nu}=G_{\mu \nu}^b t^b$ with
 $G_{\mu \nu}^b= \partial_{\mu}A_{\nu}^b - \partial_{\nu} A_{\mu}^b -g f_{bcd} A_{\mu}^c A_{\nu}^d$.
 The arbitrary external fields (such as the external source of the quarks and gluons,
 external electroweak fields etc) are denoted by $J$, while the corresponding dynamical operators 
  are denoted by $\Xi (A, \bar{q}, q)$.  The functional integration measure for the 
 c-number gluons and the Grassmann-number quarks is defined by
 \beq
 [dA d\bar{q} dq] \equiv \prod_{x, {\rm color, spin, flavor}}  dA_{\mu}^b(x) d\bar{q}^{\alpha}(x)  dq^{\beta} (x).
\eeq 
See Appendix [Gaussian and Grassmann integrals] for the examples of integration with these measures.
The temporal boundary condition of the gluon (quark) field is periodic (anti-periodic) due to its
c-number (Grassmann-number) character;  $A_{\mu}^b (\tau=0, \vx)=A_{\mu}^b (\tau=1/T, \vx)$,
$\bar{q}^{\alpha}(\tau=0,\vx)=-\bar{q}^{\alpha}(\tau=1/T,\vx)$, and 
$q^{\beta}(\tau=0,\vx)=-q^{\beta}(\tau=1/T,\vx)$.    On the other hand, the spatial boundary conditions are 
not constrained and can be taken to be either periodic or anti-periodic; the difference should disappear in the 
thermodynamic limit, $L \rightarrow \infty$.
 Throughout this chapter, we take $T = 1/L $   to
  study hadrons and hadron-hadron interactions at zero temperature in the thermodynamic limit.

Further details of the functional integral formulation of the general many-body systems of fermions and bosons
can be seen in the textbook \cite{Negele:1988vy}.

%================================
 \subsection{Quantum fluctuations}
%================================
In weak coupling perturbation theory, one assumes that the QCD coupling $g$  is small and expand the partition function
 ${\cal Z}$ in terms of a power series of $g$.  Such a perturbative expansion in QCD is justified, however, only in  limited 
 circumstances  such as at extreme high  temperature/density  or at very short distances. This is because  the 
  renormalized QCD coupling (or often called the running coupling) becomes small only in the processes 
  with the energy scale much above 1 GeV. This is called the asymptotic freedom.
 To calculate low-energy hadron properties below 1 GeV,
  we need to go beyond perturbation theory and to evaluate the functional integral with full quantum fluctuations.
 The lattice QCD provides a way to carry out this task numerically in a gauge invariant manner. 
 

  
%%%%%%%%%%%%%%%%%%%%%%
\section{Lattice QCD: theoretical basis } 
%%%%%%%%%%%%%%%%%%%%%%

%================================
\subsection{Wilson line}
%================================

% FIG %%%%%%%%%%%%%%%%%%%%%%%%%%
\begin{figure}[t]
\begin{center}
%\framebox[74mm]{\rule[-26mm]{0mm}{52mm}}
\includegraphics[scale=0.6]{Chapter3-figures/wilson-line.eps}
 \end{center}
\caption{(a) The Wilson line in the Euclidean spacetime. (b)
 Basic quark bilinears with gauge invariance.}
\label{fig:wilson-line}
\end{figure}
%%%%%%%%%%%%%%%%%%%%%%%%%%%%%


Let us first start with the {\it Wilson line} which is defined on 
  a path $P$ connecting the point $y$  and $x$ in the continuous Euclidean spacetime 
   as shown in Fig.\ref{fig:wilson-line}(a).
By parametrizing the path in terms of a coordinate
   $z(s)$ with $z(s=0)= y$ and $z(s=1)=x$, the Wilson line reads
\beq
\label{eq:5.wilson-line}
U_P(x,y;A) & = & 
{\rm P} \ {\rm exp} \left( ig \int_P dz_{\mu} A_{\mu} \right)
={\rm P} \ {\rm exp} \left( ig \int_0^1 ds\ \lambda_{\mu}
 A_{\mu} \right)
  \nonumber \\
  &  & \! \! \! \! \!  
  \! \! \! \! \! \! \! \! \! \! \! \! \! \! = \sum_{n=0}^{\infty} 
 {(ig)^n \over n!} \int_0^1 ds_1 \int_0^1 ds_2 \cdot \cdot \cdot \int_0^1 ds_n \ 
 {\rm P}[\lambda \cdot A(s_1)\ \cdot \cdot \cdot \
  \lambda \cdot A(s_n)],
 \eeq
 where   
 $\lambda_{\mu} = dz_{\mu}/ds$. The path ordered symbol ${\rm P}$ is 
  necessary because $A_{\mu} = A_{\mu}^a t^a$ is a 
  matrix in the color space.   
 
  The Wilson line has the following  properties
   which can be proved from  the definition of $U_P$ (Exercise \ref{prob:1}) :
\begin{enumerate}
  \item[(i)]  It can be broken into parts at any arbitrary points on the path;
\beq
\label{eq:5.wilson-line-i}
U_P(x,y;A) = U_{P_2}(x,z(s);A) U_{P_1}(z(s),y;A).
\eeq
\item[(ii)] It satisfies a differential equation,
\beq
\label{eq:5.wilson-line-ii}
{d \over ds} U_P(z(s),y;A)= 
 \left[ ig \lambda(s)\cdot  A(z(s)) \right] \ U_P(z(s),y;A).
\eeq
\item[(iii)] Under the local gauge transformation  
$A_{\mu}^V(x) = V(x)[A_{\mu}(x) V^{\dagger}(x) -(i/g) V(x) (\partial_{\mu}  V^{\dagger}(x))$, 
 it transforms covariantly, 
\beq
\label{eq:5.wilson-line-iii}
U_P(x,y;A) \rightarrow U_P(x,y;A^V) = V(x) U_P(x,y;A) V^{\dagger}(y).
\eeq
 \end{enumerate}

The Wilson line is useful to define the non-local and gauge-invariant objects.
In particular, the gauge-invariant quark bilinear
$\bar{q}(x) U_P(x,y;A) q(y)$, and the gauge-invariant {\it Wilson loop}
 ${\rm tr} \ U_P(x,x;A)$ turn out to be important building blocks to define
   the QCD action on the lattice.  Here ``tr" implies the trace 
 over  color indices.
   
%===========================  
\subsection{Lattice gluons}  
%===========================

% FIG %%%%%%%%%%%%%%%%%%%%%%%%%%
\begin{figure}[t]
\begin{center}
%\framebox[74mm]{\rule[-26mm]{0mm}{52mm}}
\includegraphics[scale=0.6]{Chapter3-figures/cube.eps}
 \end{center}
\caption{A hypercubic lattice in Euclidean spacetime with
 a lattice constant $a$ and the lattice size $L$.
 Quarks $q(n)$ (gluons $U_{\mu} (n)$ ) are
 defined on the sites (links).}
\label{fig:cube}
\end{figure}
%%%%%%%%%%%%%%%%%%%%%%%%%%%%%%


Let us consider a four dimensional  hyper-cubic lattice
 with a lattice spacing $a$ and the four dimensional volume $L^4$.
 Each lattice site is specified by  $n_{\mu}$ corresponding to the
 Euclidean coordinates through $x_{\mu} = a n_{\mu}$ (see Fig.\ref{fig:cube}).
   The {\it link variable} (the shortest  Wilson line on the lattice) is an SU(3)
 matrix connecting the neighboring sites $n$ and $n + \hat{\mu}$,
\beq
\label{eq:5.wilson-link}
U_{\mu}(n) = {\rm exp} \left(  ig a A_{\mu} (n) \right) .
\eeq
 Here $\hat{\mu}$ implies a vector  pointing the 
 direction of $\mu$ with a length $a$. 
 Since it is the minimal
     Wilson line, we do not need the path ordering symbol ${\rm P}$.
   Also, any non-minimal Wilson line on the lattice  is represented by a
    product of link-variables.  For later purpose, we introduce the link variable pointing the 
    opposite direction as  $U_{-\mu}(n+\hat{\mu}) = [U_{\mu}(n)]^{\dagger}$.

Let us now  define the smallest closed loop;
\beq\label{eq:5.wilson-plaq}
 U_{\mu \nu}(n) = 
 U_{\nu}^{\dagger}(n) U_{\mu}^{\dagger}(n+ \hat{\nu} ) 
  U_{\nu}(n+ \hat{\mu} )U_{\mu}(n ).
\eeq
Under local gauge
 transformation (rotation under  arbitrary  SU(3) matrix $V(n)$), we have 
\beq
 U_{\mu}(n) \rightarrow  V(n) U_{\mu}(n) V^{\dagger}(n+\hat{\mu}), \ \ 
 U_{\mu \nu}(n) \rightarrow  U_{\mu \nu}^V(n)=V(n) U_{\mu \nu}(n) V^{\dagger}(n),
 \eeq
which are the  direct consequence of Eq.(\ref{eq:5.wilson-line-iii}).

In the naive continuum limit where $a \rightarrow 0$, we have
\beq
 \label{eq:5.wilson-plaq-cont}
 U_{\mu}(n) -1 
   &=& iga A_{\mu}(n)  + O(a^2), \\
 \label{eq:5.wilson-plaq-cont-2}
 U_{\mu \nu}(n) -1 
  &=& \exp \left( iga^2 t^b (G_{\mu \nu}^b(n) + O(a^3) ) \right) -1
  = ig a^2 G_{\mu \nu}(n)  + O(a^4), \\
{\rm tr} \left(    U_{\mu \nu}(n) -1 \right) 
 &=&   {\rm tr} \left[  iga^2 t^b (G_{\mu \nu}^b(n) + O(a^3)) - \frac{1}{2} g^2 a^4 G_{\mu \nu}(n)^2 + O(a^5) \right] \nonumber \\ 
 &=& - \frac{1}{4} g^2 a^4 (G_{\mu \nu}^b(n))^2 + O(a^5) .
\eeq
Here  Eq.(\ref{eq:5.wilson-plaq-cont-2}) is obtained by 
 using the Baker-Campbell-Hausdorff formula, 
  ${\rm exp}X\cdot {\rm exp}Y =
  {\rm exp}(X + Y + [X,Y]/2 + \cdot \cdot \cdot )$.  (Exercise \ref{prob:2}). 


Finally, 
a  gluon action on the lattice, which reduces to the Yang-Mills
 action in the leading order of the  naive continuum limit ($a \rightarrow 0$), reads
\beq
\label{eq:5.wilson-action-cont}
 S_{\rm G}  &=& \beta \sum_{\rm Pl} \left[ 1 - \frac{1}{N_c} {\rm Re} \ {\rm tr} \ U_{\mu \nu}(n) \right] \\
 &=& \beta  a^4 \sum_n \sum_{\mu < \nu} \left[ 1 - \frac{1}{2N_c} {\rm tr}  \left( U_{\mu \nu}(n) +  U_{\mu \nu}^{\dagger}(n) \right) \right] \nonumber \\
 & = & \frac{1}{g^2} \sum_n \sum_{\mu \neq \nu} {\rm tr} \left[ 1 -U_{\mu \nu} (n) \right] 
  \ \ \ \ \xrightarrow{a\rightarrow 0}   \ {1 \over 4} \int d^4x  \ G_{\mu \nu}^b (x)^2 , \nonumber
 \eeq
where $\sum_{\rm Pl} $ is a sum over all non-oriented {\it plaquettes} (minimum square tile on the lattice with the area $a^2$). 
Note that  $\beta \equiv \frac{2N_c}{g^2}$ with $N_c$ being the number of colors ($N_c=3$ for QCD) 
 should not be  confused with the inverse temperature. 
    The lattice gluon action is not unique in the sense
 that one may add arbitrary non-minimal terms which vanish
 in the continuum limit ($a\rightarrow 0$).
  
%============================
\subsection{Lattice fermions}
%============================

There exist three types of gauge invariant objects made of nearest neighbor fermions as shown in Fig.\ref{fig:wilson-line}(b);
\beq
\label{eq:5.nn-link}
\bar{q}(n) q(n), \ \ 
\bar{q}(n+\hat{\mu}) U_{\mu}(n) q(n), \ \
\bar{q}(n-\hat{\mu}) U_{-\mu}(n) q(n).
\eeq
Here one may put any $\gamma$-matrices  between $\bar{q}$ and $q$
 without spoiling the color gauge invariance.
A special combination of the above   terms is called the Wilson's fermion action 
\beq
\label{eq:5.wilson-fermion}
\! \! \! 
S_{\rm F}
 &=& a^4 \sum_n 
  \left[
    m   \bar{q}(n)q(n) - \frac{1}{2a}
    \sum_{\pm \mu} \bar{q}(n+\hat{\mu}) {\Gamma}_{\mu} U_{\mu}(n) q(n)   
    - \frac{r}{2a} \sum_{\pm \mu}   \left( \bar{q}(n+\hat{\mu}) U_{\mu}(n) q(n) - \bar{q}(n)q(n) \right) 
  \right]  \nonumber \\ 
\label{eq:5.wilson-fermion2}
 & \equiv & a^4 \sum_{n',n} \bar{q}(n')
 \left( m \delta_{n',n} + D_{_{\rm W}}(n',n;U) \right) q(n)  \\
 \label{eq:5.wilson-fermion3}
 & & \xrightarrow[a \rightarrow 0] \ \  \int d^4x\  \bar{q} (x)  \left(\Gamma_{\mu} D_{\mu} + m - \frac{ar}{2} D_{\mu}^2 \right)   q(x),
\eeq
where the Wilson's Dirac operator in Eq.(\ref{eq:5.wilson-fermion2})  with the Wilson's parameter $r$ reads
\beq
\label{eq:5.wilson-dirac}
D_{\rm W}(n',n;U) 
  = - \frac{1}{2a} \sum_{\pm \mu}
 \left[
 \delta_{n',n+\hat{\mu}} (r+ {\Gamma}_{\mu} ) U_{\mu}(n)
   - r \delta_{n',n}
 \right]      .
\eeq
 To take the continuum limit in Eq.(\ref{eq:5.wilson-fermion3}), 
we use  the midpoint prescription,  $(f(x+a)-f(x-a))/2a = f'(x) + O(a^2)$
 and $(f(x+a)+f(x-a) -2f(x))/a^2 = f''(x) + O(a^2)$.
 One of the important properties of $ D_{\rm W}(n',n;U) $ is its $\Gamma_5$ Hermiticity (Exercise  \ref{prob:3}),
 \beq
 \label{eq:gamma5-hermite}
 \Gamma_5 D_{\rm W} \Gamma_5 = D_{\rm W}^{\dagger},
 \eeq
where $\Gamma_5$ is given in Appendix [Four vectors and Dirac matrices].  Note that the 
 Hermitian conjugate is taken for color, spin and  spacetime.
 
The dispersion relation (relation between the energy and momentum)
for free fermion can be obtained from Eq.(\ref{eq:5.wilson-fermion}) by taking $U_{\mu}=1$ (or equivalently $g=0$)
and substituting  the Fourier transform,
$q(n) = \int_{-\pi/a}^{\pi/a} \frac{d^4p}{(2\pi)^4} e^{i p_{\mu} n_{\mu}} q(p)$.
 This leads to $S_{\rm F}^{\rm (free)}= \int \frac{d^4p}{(2\pi)^4} \bar{q}(-p) {\cal G}_{\rm F}^{-1} q(p)$ with the
 free fermion propagator (Exercise  \ref{prob:4}),
 \beq
 \label{eq:DF}
 {\cal G}_{\rm F} (p) &=& \frac{-i  \sum_{\mu} \bar{p}_{\mu} \Gamma_{\mu}  + m(p) }{\sum_{\mu} \bar{p}_{\mu}^2 + m^2(p)},  \\
 \label{eq:Mp}
 \ \ \ \ \bar{p}_{\mu} &=& \frac{1}{a} \sin (p_{\mu} a), \ \  m(p)  =  m(0) + \frac{r}{a}  \sum_{\mu}      \left( 1-\cos (p_{\mu} a) \right)  .
 \eeq
Since $\sin (p_{\mu} a)$ becomes zero
 for $p_{\mu}a$=$(0,0,0,0)$, $(\pi, 0,0,0)$,$ \cdot $
  $(0,\pi,\pi,\pi)$, $(\pi,\pi,\pi,\pi)$, there arise
  $2^4 =16$ degenerate fermions  if we take  $r=0$.
 This is called the fermion doubling problem on the 
 lattice. In fact, there is a no-go theorem
 by  Nielsen and Ninomiya: 
 The  fermion doubling always exists, if the free fermion action  on the lattice
 has (i) bilinearity in quark field, 
  (ii) translational invariance,
   (iii) hermiticity (in the Minkowski spacetime),
    (iv) locality in spacetime, and (v) exact chiral symmetry.
Indeed, (i)-(v) are all satisfied for $r=0$.

The doubling in the dispersion relation in  the Minkowski spacetime is easily seen by Wick rotating 
$p_4 \rightarrow i E$ in Eq.(\ref{eq:DF}).  As an illustration, let us take the case with massless fermion ($m(0)=0$)
in (1+1)-dimension.  Then, the zero of the denominator after the Wick rotation for small $a$ gives,
\beq
E^2(p) \simeq \left( \frac{1}{a} \sin (pa) \right)^2 + \left( \frac{r}{a}  (1-\cos(pa)) \right)^2 ,
\eeq
whose positive energy solution is plotted in Fig.\ref{fig:dispersion} for several values of $r$.
One finds that the unphysical massless pole at $pa=\pi$ is lifted up as $r$ increases.

% FIG %%%%%%%%%%%%%%%%%%%%%%%%%%
\begin{figure}[t]
\begin{center}
%\framebox[74mm]{\rule[-26mm]{0mm}{52mm}}
\includegraphics[scale=0.75]{Chapter3-figures/dispersion.eps}
 \end{center}
\caption{The  dispersion relation for massless fermion in (1+1)-dimension on the lattice for different values of 
 the Wilson's parameter $r$.  The linear dispersion in the continuum ($E=p$) is also shown for comparison.}
\label{fig:dispersion}
\end{figure}
%%%%%%%%%%%%%%%%%%%%%%%%%%%%%% 
 
 In general, $r\neq 0$  leads to a mass splitting   of 16 fermions:
$    m(p)  \simeq  m(0)\  (^{\forall }p_{\mu} \rightarrow 0) $ and 
$m(p)= m(0) +  \frac{2r}{a}N_{\pi}  \   (^{\exists }p_{\mu} \rightarrow \pi/a) $,
 where $N_{\pi}(=1,2,3,4)$ being the number of $\pi$'s
 in  $p_{\mu}a$.  This implies that we can select only one light
 fermion by choosing $m(0) \simeq 0$ and all the other
 15 fermions have masses of $O(1/a) $  for positive  $r$.  
 A price to pay  is that the non-vanishing  $r$ breaks chiral symmetry
 explicitly for finite $a$, i.e. $\{ \gamma_5, D_{\rm W} \} \neq 0$ even for $m(0)=0$.   
  Namely, the Nielsen-Ninomiya's no-go theorem is evaded by breaking the condition (v).

  Better way to  evade the no-go theorem 
  is to break the condition (v)  in a way that the definition of
 chiral symmetry is modified.  
 Suppose we consider a modified chiral rotation in the flavor space,
 \beq
 \label{eq:5.mod-chiral}
 q \rightarrow   {\rm e}^{-i \theta_{\rm {_A}} \hat{\Gamma}_5   } q, 
  \ \ \bar{q} \rightarrow \bar{q} {\rm e}^{-i \theta_{\rm {_A}} \Gamma_5 }  \ \ 
   {\rm with} \ \  \hat{\Gamma}_5 =  \Gamma_5 (1-2 a D_{_{\rm GW}})  ,
\eeq
which  reduces to the standard axial rotation for $a\rightarrow 0$.
Here  $D_{\rm GW}$ is a generalized Dirac operator which is 
 constructed so that   
  $\bar{q} D_{\rm GW} q$ is  invariant under
 Eq.(\ref{eq:5.mod-chiral}) even for finite $a$;
\beq
 \label{eq:5.GW-1}
\Gamma_5 D_{\rm GW} + D_{\rm GW} \hat{\Gamma}_5 =0 ,
\eeq
or equivalently  $\{ \Gamma_5, D_{_{\rm GW}} \} 
 = 2a D_{_{\rm GW}} \Gamma_5 D_{_{\rm GW}}$.
 This     is called the Ginsparg-Wilson relation.
  An explicit form of $D_{\rm GW}$ may be constructed as
\beq
 \label{eq:5.overlap-op}
 D_{\rm GW} = 
 \frac{1}{2a} \left( 1+  \frac{X}{\sqrt{X^{\dagger}X}} \right)  \ \  
 {\rm with} \ \ 
 X \equiv   D_{\rm W}^{(r=1)} - m_0  ,
\eeq
where $m_0 a$ being a dimensionless parameter of $O(1)$.
 Unlike the case of $m$ in the Wilson fermion,
 $m_0$  is not directly related to the physical fermion mass.
 Nevertheless,  if we choose the region $0< m_0 a < 2$,  
  there exists
 an  exact massless mode for $N_{\pi}=0$ for finite $a$,
 and  other 15 modes have a large mass 
 $(2/a)(2N_{\pi}-m_0 a)>0$.   (Exercise  \ref{prob:5}).
 
 Going back to the Wilson's fermion action, Eq.(\ref{eq:5.wilson-fermion}).
it  can be  conveniently rewritten  as 
\beq
 \label{eq:5.wilson-fermion-lat}
S_{\rm F}    
    & = & \sum_{n',n} \bar{\psi}(n') F(n',n;U) \psi(n) ,\\
\label{eq:5.wilson-fermion-op}
F(n',n;U) &=& \delta_{n'n} 
- \kappa \sum_{\pm \mu} \delta_{n',n+\hat{\mu}}
 (r + {\Gamma}_{\mu}) U_{\mu}(n),
\eeq
where we have redefined the quark field 
 as $\psi = a^{3/2} q /\sqrt{2\kappa }$ with
 $\kappa = [2(ma + 4r)]^{-1}$
 being the  {\it hopping parameter}. 
  If the quark mass $m$ is large, $\kappa$ is
 small and the ``hopping" to the 
  neighboring lattice site is suppressed.
   
%=================================
\subsection{Partition function on the lattice}
%=================================

The functional integration over quarks and gluons in continuum QCD
in Eq.(\ref{eq:Z-QCD}) is now transformed to the integration over quarks 
  on each site and gluons on each link in lattice QCD. 
 With Eq.(\ref{eq:5.wilson-action-cont}) and
   Eq.(\ref{eq:5.wilson-fermion-lat}), the partition function without the external field ($J=0$) reads
\beq
\label{eq:5.lattice-Z}
{\cal Z}  =  \int[dU d\bar{\psi}d\psi]   {\rm e}^{-  S_{\rm G}(U) - S_{\rm F}(\bar{\psi},\psi,U) }  
    =  \int[dU]\ {\rm Det}\ F(U) \ {\rm e}^{- S_{\rm G}(U) }
    = \int[dU]\  \ {\rm e}^{- S_{\rm eff}(U) }
\eeq
To obtain the second equality, we have explicitly carried out the integration over the
 Grassmann variables, $\bar{\psi}$ and $\psi$, 
 by using the formula in Appendix [Gaussian and Grassmann integrals]. Here ${\rm Det}$ implies the determinant in 
  spacetime, color, flavor and spin degrees of freedom.
In the third equality, the exponent  is defined as 
 \beq
\label{eq:Seff}
 S_{\rm eff}(U) \equiv S_{\rm G}(U) - {\rm ln Det} F(U).
 \eeq
 
 The integration over the group element $[dU] = \prod_{\mu,n} dU_{\mu}(n) $ can be defined through 
  the {\it Haar measure} $dU$ which has the property,
  $ d(V_{\rm L}U V_{\rm R}^{\dagger})=dU$
  with  $V_{\rm L,R}$ being arbitrary group elements. Such a measure is
   unique for compact groups such as SU$(N)$.
 If we parametrize the group element as $U=\exp(i \theta_a t^a)$,  one can define the 
 distance in the group space as $ds^2 = g_{ab}  (\theta) d\theta_a d\theta_b$, where the 
 metric is given by $g_{ab}=  {\rm tr} (L_aL_b) =  {\rm tr} (R_a R_b) $ with 
 \beq
 \label{eq:LR-form}
 L_a= -i U^{-1} ( {\partial} U/{\partial \theta_a}), \ \ \ 
 R_a= -i ({\partial}U/{\partial \theta_a} ) U^{-1}.
 \eeq   
 Then the Haar measure can be   explicitly written as
  \beq
  \label{eq:Haar-measure}
   dU = {\cal N} \sqrt{ \det g} \ \prod_a d\theta_a,
  \eeq
  with an overall normalization factor ${\cal N}$.
  
 The followings are some examples of the  SU($N$)  group integration, which can be proved
 by the invariant property of the Haar measure (except for the first one which determines the 
 normalization of the measure) (Exercise \ref{prob:6}):
 \beq
 \label{eq:group-int-0}
  & &   \int dU \  1  =  1 , \\ 
 \label{eq:group-int-1}
 & &   \int dU \ U_{ij}  =  0, \\
 \label{eq:group-int-2}
 & &    \int  dU \ U_{ij} U_{k\ell}^{\dagger}  =  \frac{1}{N} \delta_{i\ell} \delta_{jk} , \\
 \label{eq:group-int-4}
 & &  \int dU \ U_{ij} U_{k\ell}  U_{i'j'}^{\dagger} U_{k'\ell'}^{\dagger}
 = \frac{ \delta_{ij'} \delta_{ji'} \delta_{k\ell'} \delta_{\ell k'} + {(j' \leftrightarrow \ell',  i' \leftrightarrow k' ) }}{N^2-1}
   -  \frac{\delta_{ij'} \delta_{jk'} \delta_{k\ell'} \delta_{\ell i'} + { (j' \leftrightarrow \ell',  i' \leftrightarrow k' )}  }{N}, \\
 \label{eq:group-int-B}
 & &  \int dU \ U_{i_1 j_1} \cdots U_{i_N j_N} = 
 \frac{1}{N!} \epsilon_{i_1 \cdots i_N}    \epsilon_{j_1 \cdots j_N} .
 \eeq    

   
Similar to the statistical systems such as the Ising model, 
 observables are obtained by averaging over the statistical weight  as
 \beq
 \langle {\cal O} \rangle = \frac{1}{{\cal Z}}  \int[dU]\ {\cal O}(U) \ {\rm e}^{- S_{\rm eff}(U) }.
 \eeq
    Due to the gauge invariance of the Haar  measure,
 gauge non-invariant quantities have vanishing expectation values
  (Elitzer's theorem).  For example, consider the expectation value of the link variable,
 \beq
 \label{eq:elitzer}
 \langle U_{\mu} (n) \rangle 
 = \frac{1}{{\cal Z}}  \int [dU] U_{\mu} (n)  \ {\rm e}^{- S_{\rm eff}(U) }  = V(n)  \langle U_{\mu} (n) \rangle ,
\eeq
where we have made a change of variable, $U_{\mu}(n) \rightarrow V(n) U_{\mu}(n)$ with $V(n)$ 
being the $SU(N)$ matrix, and used the  gauge invariance of the Haar measure as well as ${\rm Det}\ F(U)$ and 
$S_{\rm G}(U)$.  Since Eq.(\ref{eq:elitzer}) must be true for arbitrary $V(n)$, we have $ \langle U_{\mu} (n) \rangle =0$.
 
Some examples of the non-vanishing  observables are shown  in Fig.\ref{fig:correlation}:
 (a) and (b) correspond to the mesic and baryonic correlations, respectively,  while (c) is a correlation related to the 
baryon-baryon interactions.  The filled circles are the spacetime points where the quarks and anti-quarks
are created or absorbed. Each  line with arrow indicates the quark propagator $F^{-1}(n,n';U)$ 
connecting two spacetime points $n$ and $n'$.  Thus, the explicit forms of the
 mesic and baryonic correlations are
 \beq
 \label{eq:correlation-M}
C_{\rm M}(n,n') 
 &=& \frac{1}{{\cal Z}}  \int [dU] \ F_{\alpha \beta}^{-1}(n,n';U) F_{\beta \alpha}^{-1}(n',n;U)  \ {\rm e}^{- S_{\rm eff}(U) } ,\\
\label{eq:correlation-B}
C_{\rm B}(n,n') 
 &=& \frac{1}{{\cal Z}}  \int [dU] \ \epsilon_{\alpha \beta \gamma} \epsilon_{\alpha' \beta' \gamma'} 
 F_{\alpha \alpha'}^{-1}(n,n';U) F_{\beta \beta'}^{-1}(n,n';U) F_{\gamma \gamma'}^{-1}(n,n';U)  \ {\rm e}^{- S_{\rm eff}(U) } ,
  \eeq
where all the color indices are contracted so that $C_{\rm M}$ and $C_{\rm B}$ are gauge invariant. 
 Other quantum numbers such as spin and flavor  associated with $F^{-1}$ are
 not shown explicitly.
  Spacetime, spin and flavor dependences of  $C_{\rm M}(n,n')$ in (a) and $C_{\rm B}(n,n')$ in (b)
  have all the information on the hadronic states in various different channels, while
  $C_{\rm BB}(n,m,n',m')$   in (c) has the information on baryon-baryon interactions.
      
% FIG %%%%%%%%%%%%%%%%%%%%%%%%%%
 \begin{figure}[t]
%\vspace{-1cm}
\begin{center}
%\framebox[74mm]{\rule[-26mm]{0mm}{52mm}}
\includegraphics[scale=0.2]{Chapter3-figures/correlations.eps}
 \end{center}
%\vspace{-1cm}
\caption{
(a) Single meson correlation representing the propagation of a  meson created at point $n'$ and absorbed at point $n$.
(b)  Single baryonic correlation representing the propagation of a  baryon created at point $n'$ and absorbed at point $n$.
(c)  Two baryon correlation which contains the information on baryon-baryon interaction.}
\label{fig:correlation}
\end{figure}
%%%%%%%%%%%%%%%%%%%%%%%%%%%%%%%
 
   
%===========================================
\subsection{Strong coupling expansion and quark confinement}
%===========================================


% FIG %%%%%%%%%%%%%%%%%%%%%%%%%%
\begin{figure}[t]
\begin{center}
%\framebox[74mm]{\rule[-26mm]{0mm}{52mm}}
\includegraphics[scale=0.55]{Chapter3-figures/wilson-L.eps}
 \end{center}
\caption{A rectangular  Wilson loop with the
 temporal (spatial) size ${\cal T}$ ($R$).}
\label{fig:wilson-L}
\end{figure}
%%%%%%%%%%%%%%%%%%%%%%%%%%%%%%

One of the remarkable properties of QCD is the   confinement of quarks inside 
hadrons.  Simplest setup to see this phenomena is to consider
the potential $V(R)$ between  an infinitely heavy quark $Q$ and an anti-quark $\bar{Q}$ with
a fixed separation $R$.
It corresponds to Fig.\ref{fig:wilson-L} and can be written as
\beq
 \label{eq:5.wilson-loop-lattice}
\la W(C) \ra  
 & = & \la {\rm tr}\ \prod_{{\rm link}\in C}  U_{\mu}(n) \ra , \\
 \label{eq:5.wilson-loop-asym}
 & \propto  & 
  {\rm e}^{-V(R) {\cal T} } \simeq 
 \exp \left[ -\left( K R + b + \frac{c}{R} + \cdots \right) {\cal T} \right] ,
\eeq  
where we  have taken a limit     
${\cal T} \gg R \rightarrow \infty$ in   Eq.(\ref{eq:5.wilson-loop-asym}).
 Remembering the fact that the real time $t$ and the imaginary time $\tau$ are
 related as  $\tau=it$, the exponential falloff of  $\la W(C) \ra$ in $\tau$ implies the 
  temporal oscillation in $t$, and its $R$-dependent coefficient is nothing but the 
 interaction  energy between $Q$ and $\bar{Q}$.  
 
In Eq.(\ref{eq:5.wilson-loop-asym}), 
   $K > 0$  implies 
 the existence of a string-like
linear confining potential.
 It also implies 
 the area law  of the Wilson loop 
 $\la W(C) \ra \sim \exp(-K{\cal A})$
 where ${\cal A}= R \times {\cal T} $ is 
  the area inside the path $C$.
    In full QCD where pair creation of light quarks 
 are allowed,  
  the linear rising potential becomes eventually 
    flat at long distances due to the breaking of the string,
  $Q\bar{Q} \rightarrow (Q\bar{q})(q\bar{Q})$.
  
  To make the analysis simple, let us now consider the 
  SU$(N_c)$ Yang-Mills theory
  without light quarks: This is called the quenched 
  approximation and corresponds to take $F(U) =1$.
In this case,  the Wilson loop can be evaluated analytically
 in the strong coupling regime ($g \rightarrow \infty$).
First of all,  $S_{\rm G}$ is proportional to $1/g^2$,
so that  one can make an expansion,  
$\exp (-S_{\rm G}) = 1 - S_{\rm G}+ S_{\rm G}^2 /2 + \cdots $
 and finds (Exercise \ref{prob:7})
\beq
\label{eq:5.wilson-strong}
\la W(C) \ra = {1 \over {\cal Z}}
 \int [dU] \ {\rm tr} \prod_{{\rm link}\in C} U_{\mu}(n) \  
 \sum_{\ell =0}^{\infty} \frac{1}{\ell !} (-S_{\rm G})^{\ell}.
\eeq
 
Only the first three integrals, Eqs.(\ref{eq:group-int-0},\ref{eq:group-int-1},\ref{eq:group-int-2}),
 are necessary to extract the leading contribution to 
 $\la W(C) \ra $  in the strong coupling.
   Key observation is that all the $U$'s 
  from the Wilson loop and $U^{\dagger}$'s 
 from $(-S_{\rm G})^{\ell}$ should be paired in the leading order of $1/g^2$
 in   Eq.(\ref{eq:5.wilson-strong}).
  This means that the area inside the Wilson loop is tiled up 
 with minimum number of plaquettes as shown in
  Fig.\ref{fig:strong-c}. All the structures other than 
  the minimal surface are higher orders in  $1/g^2$.

% FIG %%%%%%%%%%%%%%%%%%%%%%%%%%
 \begin{figure}[t]
\begin{center}
%\framebox[74mm]{\rule[-26mm]{0mm}{52mm}}
\includegraphics[scale=0.65]{Chapter3-figures/strong-c.eps}
 \end{center}
\caption{A minimum surface in which the Wilson loop is tiled 
   up by the fundamental plaquettes in the 
 strong coupling limit.}
\label{fig:strong-c}
\end{figure} 
%%%%%%%%%%%%%%%%%%%%%%%%%%%%%%

 In the evaluation of the numerator of
 Eq.(\ref{eq:5.wilson-strong}), 
 each plaquette has a contribution $1/g^2$. 
 Also each  integration on the link
   gives a factor $1/N_c$
   and the contraction of the color indices gives a factor $N_c$
 on each  site.
  On the other hand,
 ${\cal Z}$  (the denominator of Eq.(\ref{eq:5.wilson-strong})
 is   unity in the leading order. 
    Thus, one arrives at the formula in the lowest order
 of the strong coupling expansion,
\beq
 \label{eq:5.strong-estimate}
\frac{1}{N_c} \la W(C) \ra 
& & \xrightarrow[g^2\rightarrow \infty] 
 \  \frac{1}{N_c} \cdot  
\left(\frac{1}{g^2} \right)^{N_{\rm plaq}} \cdot  
\left(\frac{1}{N_c} \right)^{N_{\rm link}} \cdot N_c^{N_{\rm site}}  , \nonumber \\
& & \ \ \ \ \ \ \ \ \ \ \ =\left(\frac{1}{N_c g^2} \right)^{\frac{R {\cal T}}{a^2}} 
= \exp \left( - \frac{\ln N_c g^2}{a^2} R {\cal T} \right) ,
\eeq
where we have used a relation, $N_{\rm link}-N_{\rm site}+1 
 = N_{\rm plaq}$ and 
$N_{\rm plaq} a^2 = R {\cal T} ={\cal A}$.
 Since it shows the area law, \index{area law}
  the confinement is naturally obtained in the strong coupling
  with the linear rising potential, 
\beq
 \label{eq:5.linear-potential}
V(R) = K R \ \ {\rm with} \ \ K = \frac{1}{a^2} \ln (N_c g^2).
\eeq

 If we consider higher orders of the strong coupling expansion,
   ``rough" surfaces should be taken 
 into account.
 Nevertheless, the confining feature is
 stable against small perturbations in $1/g^2$.
  In fact, there exits a theorem 
  that,  for sufficiently large $g$, the strong coupling expansion 
  converges and shows confinement for all compact gauge groups  
  in all spacetime dimensions.
  

  A question here is that whether the real world 
  corresponds to the strong coupling region discussed above.
  The answer is no;  the real world corresponds to  the weak coupling regime. 
  For compact QED (quantum electrodynamics
  formulated in terms of the  U(1) link variable),
  the confinement  $K>0$ in the strong coupling regime 
   changes to  $K=0$ in the weak coupling regime.
  On the other hand,  in QCD in four spacetime dimensions 
 with $N_c=3$,   the confinement feature is expected to persist even in the weak coupling regime. 
 Indeed, there are  strong evidences for this statement 
  from  LQCD simulations.
  Its analytic proof, however,  is  still missing and is being one of the 
   most challenging problems in mathematical  physics.
   

%===============================================
\subsection{Weak coupling expansion and continuum limit}
%===============================================

  Lattice QCD can be regarded as an effective field theory
   with an ultraviolet (UV)  cutoff in the coordinate space. 
   The gauge coupling  $g$ is then interpreted as
  a bare coupling defined at the scale $a$ where
  quantum fluctuations with the wave length shorter than 
  $a$ are integrated out.  In non-Abelian gauge theories,
   it can be shown that $g(a)$ decreases logarithmically as $a$ decreases
   unless the number of matter fields is not too large.  
    This is called the {\it asymptotic freedom}, and is essential for taking
    the continuum limit ($a \rightarrow 0$) to remove the 
    lattice artifact. 
        
 For simplicity, let us consider the case with massless fermions, where 
 observables
   ${\cal O}$ such as the string tension and the hadron masses  depend only
   on the coupling $g$ and the  regularization scale $a$. 
   Then,  from the dimensional ground, one can  write  
     \beq
\label{eq:5.O-dim}
 {\cal O} (g(a),a) = a^{-d} X(g(a)),
 \eeq
 where $d$ is the mass-dimension of ${\cal O}$ and $X$ is a dimensionless
 function of $g$.  The $a$-independence of the observable implies
 \beq
\label{eq:5.O-RG}
 a{d{\cal O} \over da} 
  =  
  \left( a \frac{\partial}{\partial a} 
 - \beta_{_{\rm LAT}}  \frac{\partial}{\partial g} \right) {\cal O}(g(a), a)  = 0 , 
\ \   \beta_{_{\rm LAT}}(g) =  -a {dg(a) \over da} .
\eeq
By integrating the first equation in Eq.(\ref{eq:5.O-RG}),  we find
\beq
\label{eq:5.lat-F}
 X(g) = 
 \exp \left( -d \int^g {dg' \over \beta_{_{\rm LAT}}(g')} \right) .
 \eeq
Suppose that 
 the beta-function can be expanded in terms of $g$ for small $a$:
$\beta_{_{\rm LAT}}(g)   = 
  - \beta_0 g^3 - \beta_1  g^5 + \cdot \cdot \cdot $. Here
  $\beta_0$ and $\beta_1$ can be shown to be independent of the 
   regularization scheme and are known to be 
\beq
\beta_0= \frac{1}{(4\pi)^2} \left( \frac{11N_c}{3}-\frac{2N_f}{3} \right),
\ \ \  \beta_1= \frac{1}{(4\pi)^4} \left( \frac{34N_c^2}{3}-\frac{38N_f}{3} \right),
\eeq
           for QCD with $N_c$ colors and     $N_f$ fermions.
  
% FIG %%%%%%%%%%%%%%%%%%%%%%%%%%
 \begin{figure}[t]
\begin{center}
%\framebox[74mm]{\rule[-26mm]{0mm}{52mm}}
%\includegraphics[scale=0.6]{as-scale.eps}
\includegraphics[scale=0.30]{Chapter3-figures/crossover.eps} 
 \end{center}
\caption{Crossover behavior of the dimensionless
 string tension $Ka^2$ from the strong coupling
 regime  $\beta=2N_c/g^2 \rightarrow 0$ to the weak coupling
  (asymptotic scaling) regime  $\beta=2N_c/g^2 \rightarrow \infty$
  for SU($N_c=2$) Yang-Mills theory.  The figure is adapted from
  \cite{Creutz:1980zw}.
  }
\label{fig:as-scale}
\end{figure}
%%%%%%%%%%%%%%%%%%%%%%%%%%%%%%%%

        
By integrating  the second equation in Eq.(\ref{eq:5.O-RG}) with the above expansion of
$\beta_{_{\rm LAT}}(g)$, one finds  
 \beq
\label{eq:5.a-vs-L}
 a = \Lambda_{_{\rm LAT}}^{-1} \cdot  
 \exp  \left( -\frac{1}{2\beta_0 g^2}  \right)  \cdot
  (\beta_0 g^2)^{-{\beta_1 \over 2 \beta_0^2}} 
 \cdot (1 + O(g^2)).
 \eeq
Here  $\Lamlat$ is called the scale parameter on the lattice, and  
   $g(a)$ can be expressed in terms of $a$ and $\Lamlat$ ((Exercise \ref{prob:8}),
 \beq
\label{eq:5.lat-running-g}
{1 \over g^2(a)} = \beta_0 \ln \left( 1 \over a^2 \Lamlat^2 \right)
+ {\beta_1 \over \beta_0} \ln \ln \left( 1 \over a^2 \Lamlat^2 \right)
+ \cdot \cdot \cdot .
\eeq
 This is the asymptotic freedom
 in which $g(a)$ decreases as $a$ decreases.  This also justified the 
 assumption that the beta-function can be expanded by $g(a)$ for small $a$.
 
 
Direct way to extract the actual value of $\Lamlat$  is to
 carry out  numerical simulations of  
 a certain physical quantity (such as the string tension)
 and compare the result with the experimental value. 
 For example, the string tension, which has mass-dimension
 two ($d=2$) should  behave as
\beq
\label{eq:5.lat-K}
K a^2 = C_K \exp \left( - {1 \over \beta_0 g^2} \right) 
(\beta_0 g^2)^{-\beta_1 /\beta_0^2} = C_K \Lamlat^2 ,
\eeq
with $C_K$ being a dimensionless numerical constant independent
 of $g$.  As  can be seen from this example, 
 the functional form of the physical quantities 
 for $g\sim 0$  is  severely constrained.  
This is called the {\em asymptotic scaling} which 
 is used to check whether the system is 
  close enough to the continuum limit.
Shown in Fig.\ref{fig:as-scale} is a historic numerical study, which shows 
 a crossover of $Ka^2$  from the strong coupling regime to the weak coupling regime
 in SU(2) Yangs-Mills theory. 
  


%===============================
\subsection{Running coupling}
%===============================

% FIG %%%%%%%%%%%%%%%%%%%%%%%%%%
\begin{figure}[t]
\begin{center}
%\framebox[74mm]{\rule[-26mm]{0mm}{52mm}}
%\includegraphics[scale=0.6]{as-scale.eps}
\includegraphics[scale=0.45]{Chapter3-figures/running-g.eps} 
 \end{center}
\caption{Perturbative running coupling $\frac{1}{4\pi}\bar{g}^2(\lambda)$ as a function of $\lambda^{-1}$.
In the short distance limit ($\lambda =1$), the running coupling coincides with the bare coupling $\bar{g}(\lambda=1)=g(a)$,
while, in the long distance regime ($\lambda \ll 1$), the running coupling grows. }
\label{fig:running-g}
\end{figure}
%%%%%%%%%%%%%%%%%%%%%%%%%%%%%%%


Let us now consider an observable ${\cal O}$ which depends 
not only on $g(a)$ and $a$ but also on some external dimensionful parameter.
For concreteness, we consider the heavy quark potential $V(R; g(a),a)$ in the 
quenched approximation.  Since it has the dimension of energy, one may write
\beq
V(R,g(a),a) = R^{-1} \tilde{V} (R/a, g(a)).
\eeq
Then the cutoff independence of the observable, $a \frac{d}{da}V(R,g(a),a)=0$,  leads to
\beq
\left( \lambda \frac{\partial}{\partial \lambda} + \beta_{_{\rm LAT}} \frac{\partial}{\partial g} \right) \tilde{V} (\lambda,g)=0,
\label{eq:RG-R}
\eeq
where we have introduced 
a dimensionless  scaling parameter $\lambda$ through $R=\lambda a$.

]The solution of the {\it renormalization group equation}, Eq.(\ref{eq:RG-R}), reads
\beq
\label{eq:RG-sol}
 \tilde{V} (\lambda,g(a)) = \tilde{V} (1, \bar{g}(\lambda)).
 \eeq
 Here $\bar{g}(\lambda)$ is called the {\it running coupling} which is  a solution of
\beq
\lambda \frac{d\bar{g}}{d\lambda}  = - \beta_{_{\rm LAT}} (\bar{g}(\lambda)),
\eeq
with the boundary condition, $\bar{g}(\lambda=1)= g(a)$.
One can show that  Eq.(\ref{eq:RG-sol})  satisfies Eq.(\ref{eq:RG-R}) explicitly by applying the partial derivatives 
 or more generally by the method of characteristics in Appendix [Method of characteristics].
Then, we eventually arrive at the formula 
\beq
V(R,g(a), a) = \frac{a}{R} V (a, \bar{g}(R/a), a) .
\eeq
If $R$ is  in the interval,  $a < R \ll \Lamlat^{-1}$, the running coupling $\bar{g}$ is small enough, so that 
one may use the perturbative expansion of $\beta_{_{\rm LAT}}$ to obtain
\beq
\bar{g}^2(R/a) \simeq \frac{g^2(a)}{1-2\beta_0 g^2(a) \ln (R/a)} =  \frac{1}{2\beta_0 \ln (1/(R\Lamlat) ) }. 
\label{eq:running-R}
\eeq 
In Fig.\ref{fig:running-g}, the behavior of $\frac{1}{4\pi}\bar{g}^2(\lambda)$  as a function of $\lambda^{-1}$ is shown.
The bare coupling $\frac{1}{4\pi}g^2(a)$ appears as the boundary condition at shortest distance $\lambda=R/a=1$,
while $\frac{1}{4\pi} \bar{g}^2(\lambda)$ grows  as $\lambda$ increases.
The latter implies that the strong interaction has anti-screening feature.

For $R$ sufficiently close to $a$, one may evaluate the potential by using perturbation theory as
$V (a, \bar{g}(R/a), a)  = - C_{{\rm F}} \frac{\bar{g}(R/a)^2}{4\pi a}$, so that we finally obtain
\beq
\label{eq:RG-Coulomb}
V(R,g(a),a) \simeq - C_{{\rm F}} \frac{\bar{g}^2(R/a)}{4\pi R} \ \ \ \ \ (a < R \ll \Lamlat^{-1}),
\eeq
where $C_{{\rm F}}=4/3$ for $N_c=3$ is given in Appendix [SU($N$) algebra].
Eq.(\ref{eq:RG-Coulomb}) is nothing but the Coulomb potential with the running coupling constant.
Note that the left hand side is $a$-independent, while the right hand side has
logarithmic $a$-dependence through $\bar{g}$.  This is due to the use of perturbation theory;
such a logarithmic $a$-dependence  is cancelled by the next-to-leading order term.
Note also that the similar analysis can be done for any other observables.
The QCD thermal pressure at finite temperature $P(T,g(a),a)$ is a typical example, in which 
the weak coupling analysis (i.e. the description by  quark-gluon plasma picture)
is justified  under the condition  $ \Lamlat \ll T < a^{-1}$. 



%%%%%%%%%%%%%%%%%%%
\section{Lattice QCD:  numerical simulations}
%%%%%%%%%%%%%%%%%%%


Suppose we have a lattice having $\Ns$ ($\Ntau$) number of sites
in each spatial (temporal) direction. Then 
 the total number of links is 
   $  \Ns^3 \times \Ntau  \times 4$. 
 Therefore the total  number of gluon integrations $\int [dU]$
 for a moderate lattice size  $\Ns = \Ntau =32$ reads
\beq
\label{eq:5.lattice-size}
 ( \Ns^3 \times \Ntau \times 4 )_{\rm links} \times 8_{\rm color} 
 \sim 3 \times 10^7 .
 \eeq
 This is 
 hopelessly a large number  
  for standard methods of numerical integration。
   In this case,  the  Monte Carlo (MC) integration,
   which is a
  statistical way to evaluate the integral, plays a 
   powerful role. For rapidly varying integrand,
  the MC integration should be supplemented by the 
 importance sampling 
   to have better accuracy, in which the rapidly varying
   part is sampled more than the slowly varying part.
     
     
 %==========================
\subsection{Importance sampling}
\label{ss:IS}
%==========================
    
     Let us consider the general partition function ${\cal Z}= \int [d\phi] \exp(-S(\phi)) $,
    with some c-number field $\phi$  and try to calculate  an observable ${\cal O} $ by
  \beq
  \langle {\cal O} \rangle = \frac{1}{{\cal Z}} \int [d \phi] \ {\cal O} (\phi) e^{-S(\phi)}.
  \eeq   
  The basic procedure of the MC integration
  with the {\it importance sampling} consists of two steps:
  \begin{enumerate}
  \item[(I)]   Generate a set of field configurations, $ \{  \phi^{(1)}, \phi^{(2)}, \cdots, \phi^{(N)} \}$, 
with $\phi^{(n)}$  being arranged to appear with 
 a probability in ``equilibrium", $W_{\rm eq}[\phi]={\cal Z}^{-1} \exp (-S(\phi))$.
 \item[(II)]  The field configurations thus generated 
 are  used to calculate the expectation value,
 \beq
\label{eq:5.obs-average}
\la {\cal O} \ra = \frac{1}{N}\sum_{n=1}^N {\cal O}^{(n)}
\pm \sqrt{{\sigma^2 \over N}}, \ \ \ \ \ 
\sigma^2 =  \frac{1}{N-1} \sum_{n=1}^N \la {\cal O}^{(n)} - \la {\cal O} \ra \ra^2 ,
\eeq
with ${\cal O}^{(n)}={\cal O}(\phi^{(n)})$.
\end{enumerate}

 %=========================================
\subsection{Markov chain Monte Carlo (MCMC)}
%==========================================
  
 For large  number of integration variables such as Eq.(\ref{eq:5.lattice-size}), 
 it  is essential to develop an appropriate scheme to carry out  Step (I) in Sec. \ref{ss:IS}. 
 The {\it Markov chain Monte Carlo (MCMC)}  method is one of such schemes. 
    
Let us consider a chain of configurations  generated successively starting from an
initial configuration,
\beq
\label{eq:M-chain}
 \phi_{0} \rightarrow  \phi_{1} \rightarrow \phi_{2}
 \rightarrow \cdots \rightarrow \phi_{i} \rightarrow  \phi_{i+1}
\rightarrow \cdots ,  
\eeq
where the ``{\it update}" of the $i$-th configuration ($\phi$)  to
the $(i+1)$-th configulation ($\phi'$)  is governed by the conditional probability
$P$ or equivalently  the transition matrix $\vT$, 
\beq
\label{eq:M-transition}
P(\phi \rightarrow \phi') = (\vT)_{\phi \phi'} 
\eeq
which has the property, $\sum_{\phi'} P(\phi \rightarrow \phi')=1$.
Eq.(\ref{eq:M-chain}) generated by Eq.(\ref{eq:M-transition}) is called the {\it Markov chain} since
it is governed by the Markov process where the conditional probability depends only on the 
neighbouring pair. The probability distribution $W[\phi]$ (with the properties, $W[\phi] \ge 0$ and 
$\sum_{\phi} W[\phi]=1$) is updated successively by Eq.(\ref{eq:M-transition}),
\beq
\label{eq:5.MC-update}
W'[\phi' ] =  \sum_\phi  W[\phi] P(\phi \rightarrow \phi'). 
\eeq

If  the Markov chain is {\it irreducible} (any $\phi$ and $\phi'$ are connected with each other) and  {\it aperiodic}
(absence of $\phi$ which appears periodically) \footnote{Rigorous definitions are as follows. (i) The Markov chain is said to be 
irreducible if one can find a finite positive integer $n (< \infty)$  such that  $(\vT^n)_{\phi \phi'} > 0 $ for all $\phi$ and $\phi'$.
(ii)  The period, $d(\phi)$, is defined by the greatest common divisor  of the set of positive integers $n (\ge 1)$
such that  $(\vT^n)_{\phi \phi} > 0 $ is satisfied. If $d(\phi)=1$ for all $\phi$, the Markov chain is 
said to be aperiodic \cite{Haggstrom:2002}.},  there exists a theorem that the Markov chain
 has a unique equilibrium distribution $W_{\rm eq}$ satisfying 
\beq
\label{eq:transition-W}
W_{\rm eq}[\phi']  =  \sum_\phi W_{\rm eq}[\phi] P (\phi \rightarrow \phi'),
\eeq
and it can be reached by $\vT^{\infty}$ starting from arbitrary initial distribution.
For a heuristic proof of this theorem, see Exercise \ref{prob:9}. For mathematical proof,  see  \cite{Haggstrom:2002}.

As is easily seen, a  sufficient but not necessary condition for $P$ to lead $W_{\rm eq}$  is the {\it detailed balance}:
\beq
\label{eq:5.det-balance}
W_{\rm eq}[\phi] P (\phi \rightarrow \phi' )  =
 W_{\rm eq}[\phi'] P (\phi' \rightarrow \phi ).
\eeq
There also exits specific algorithm without the detailed balance in MCMC  \cite{SuwaTodo:2010}.


The Markov chain Eq.(\ref{eq:M-chain})  takes 
certain {\it thermalization time} to reach equilibration. 
Also, the nearby  field configurations  are strongly correlated  during the {\it autocorrelation time}.
To calculate the actual average in Eq.(\ref{eq:5.obs-average}), we then 
need to discard non-themalized configurations and also 
thin out the configurations to avoid the autocorrelations. 
This is schematically shown for an observable ${\cal O}$ in Fig.\ref{fig:auto}.
The thermalization time can be estimated by monitoring the behavior of ${\cal O}$  under 
successive update, while the autocorrelation time can be estimated by 
calculating the correlations of ${\cal O}$ for different configurations.

 
% FIG %%%%%%%%%%%%%%%%%%%%%%%%%%
\begin{figure}[t]
\begin{center}
%\framebox[74mm]{\rule[-26mm]{0mm}{52mm}}
%\includegraphics[scale=0.6]{as-scale.eps}
\includegraphics[scale=0.37]{Chapter3-figures/auto.eps} 
 \end{center}
\caption{Schematic illustration on the behavior of ${\cal O}$ under successive update starting from
certain initial configuration. Blue crosses correspond to ${\cal O}^{(n)}$ to be used for the actual average in
Eq.(\ref{eq:5.obs-average}).}
\label{fig:auto}
\end{figure}
%%%%%%%%%%%%%%%%%%%%%%%%%%%%%%
    
    
    
%===============================
\subsection{Hybrid Monte Carlo (HMC)}
%===============================

 Most widely used method   for 
  generating configurations in LQCD 
  is  the hybrid Monte Carlo (HMC) method \cite{Duane:1987de} and its variations.
 The basic procedure of the HMC can be summarized as follows:
 First, we rewrite the partition function by introducing a conjugate momentum field $\pi$,
 so that ${\cal Z}$ is transformed to a phase space functional integral,
\beq
{\cal Z}= \int [d\phi]\ e^{-S(\phi)}= \int [d\Phi] \  e^{- H(\Phi)}, \ \  \ \ \ 
H(\Phi) = \frac{1}{2} \pi^2 + S(\phi) ,
\eeq
where $\Phi\equiv(\phi,\pi)$ and $[d\Phi]\equiv [d\phi d\pi]$.
Then we follow the steps below:
 \begin{enumerate}
 \item[1.]   Start with arbitrary chosen initial configuration, $\phi$.
 \item[2.]  Generate $\pi$ with the Gaussian distribution, 
 \beq 
 P_{\rm G}(\pi) \propto \exp(-\pi^2/2).
 \eeq
\item[3.]  Evolve $\Phi $ under transition probability $P_{\rm H}$ with the {\it reversibility condition}, 
\beq
\label{eq:reversible}
P_{\rm H}(\Phi \rightarrow \Phi') = P_{\rm H}(\Phi'_r \rightarrow \Phi_r),   \ \ \  \Phi_r \equiv (\phi,-\pi). 
\eeq
\item[4.] Accept the configuration $\Phi'$ with the probability, 
\beq 
\label{eq:MET-test}
P_{\rm A} (\Phi \rightarrow \Phi')  = {\rm min.} \{ 1, e^{-\Delta H} \} ,
\eeq
where  $\Delta H=H(\Phi')- H(\Phi)$.  This is called the {\it Metropolis test}  \cite{Metropolis_1953}.
\item[5.] If the new configuration $\Phi'$ is accepted, go to Step 2 with $\phi'$.
 Otherwise, keep the original $\phi$ and go to Step 2. 
\end{enumerate}

The above procedure satisfies the detailed balance  Eq.(\ref{eq:5.det-balance}) 
with $W_{\rm eq}[\phi]=\exp(-S(\phi))$. 
In fact, the Step 4  satisfies the detail balance in phase space (Exercise \ref{prob:10}),
\beq
\label{eq:DT-balance}
e^{-H(\Phi)} P_{\rm A} (\Phi\rightarrow \Phi') = e^{-H(\Phi')} P_{\rm A} (\Phi' \rightarrow \Phi) .
\eeq
Then, we have
\beq
e^{-S(\phi)} P (\phi \rightarrow \phi' ) 
&=& e^{-S(\phi)} \int [d\pi d\pi'] P_{\rm G}(\pi)  P_{\rm H}(\Phi \rightarrow \Phi') P_{\rm A} (\Phi \rightarrow \Phi') , \nonumber \\
&=& \int [d\pi d\pi']\ e^{-H(\Phi)}  P_{\rm H}(\Phi \rightarrow \Phi') P_{\rm A} (\Phi \rightarrow \Phi') ,\nonumber \\
&=& \int [d\pi d\pi']\ e^{-H(\Phi')}  P_{\rm H}(\Phi \rightarrow \Phi') P_{\rm A} (\Phi' \rightarrow \Phi) ,\nonumber \\
&=& \int [d\pi d\pi']\ e^{-H(\Phi')}  P_{\rm H}(\Phi'_r \rightarrow \Phi_r)  P_{\rm A} (\Phi' \rightarrow \Phi) ,\nonumber \\
&=& \int [d\pi d\pi']\ e^{-H(\Phi')}  P_{\rm H}(\Phi' \rightarrow \Phi ) P_{\rm A} (\Phi' \rightarrow \Phi) ,\nonumber \\
&=& e^{-S(\phi')} \int [d\pi d\pi'] P_{\rm G}(\pi')  P_{\rm H}(\Phi' \rightarrow \Phi)  P_{\rm A} (\Phi' \rightarrow \Phi) 
= e^{-S(\phi')}  P (\phi' \rightarrow \phi ) ,
\eeq
where we have used Eq.(\ref{eq:reversible}) to obtain the 4th line, and also used $H(\Phi)=H(\Phi_r)$ 
 to obtain the 5th line. 

Note that $P_{\rm H}$ can be chosen to be any transition probability as long as it satisfies
Eq.(\ref{eq:reversible}).  In practice, the deterministic procedure based on the Molecular Dynamics (MD) 
evolution along the  ``computer"  time $s$ is useful:
\beq
\label{eq:MD}
\frac{d}{ds} 
\left(
\begin{array}{cc}
 \phi  \\
  \pi 
\end{array}
\right)
=
\left(
\begin{array}{cc}
 0 &  1   \\
 -1  & 0   
\end{array}
\right)
\left(
\begin{array}{cc}
  {\delta H (\phi,\pi)}/{\delta \phi}   \\
  {\delta H (\phi,\pi)}/{\delta \pi}
\end{array}
\right) =
\left(
\begin{array}{cc}
 \pi  \\
  -  {\delta S (\phi)}/{\delta \phi}
\end{array}
\right) ,
\eeq
which leads to
\beq
P_{\rm H}(\Phi \rightarrow \Phi')  = \delta (\Phi' - \Phi(s)),
\eeq
 on the phase space trajectories, $\Phi = \Phi(0) \rightarrow \Phi(s)$. 
 
If we do not introduce the MD before the Metropolis test $P_{\rm A}$, the procedure is essentially the MCMC
with the Metropolis test.  It becomes, however,
 very slow for non-local action such as Eq.(\ref{eq:Seff})
where $S_{\rm G}(U)$ is local in spacetime while ${\rm Ln Det} F(U)$ is non-local.
The MD is a nice way to evolve the whole variables on the lattice at once.
 The  computer time $s$ needs to be discretized with a step size $\varepsilon$,  which brings 
 inevitable numerical error in MD. However,   the Metropolis test in Step 4 eliminates such error so that
no extrapolation to $\epsilon$ is required in HMC.

 There are numerical algorithms in MD  to satisfy the reversibility and preserve the phase space area 
 exactly for finite $\varepsilon$.   The
  {\it leapfrog integrator} is one of such algorithms widely used in LQCD 
  (see Appendix [Leapfrog integrator in molecular dynamics]).
 Since this  conserves the Hamiltonian with $0(\varepsilon^2$) accuracy, the acceptance rate
 in Step 4 can be kept high.


 In LQCD simulations, we need to treat the unitary matrices $U_{\mu}(n)$ as dynamical variables, i.e.
  the MD should be performed on the SU($N_c$)  group manifold.
 The appropriate choice of the conjugate momentum would be the element of the Lie algebra,
  $ P_l =  R_l^a t^a = -i  (d{U_l}/ds) U_l^{-1} $ (see Eq.(\ref{eq:LR-form})) where
  we have abbreviated the link index $n$ and site index $\mu$  as $l$ for simplicity.
 This  leads to the
  equation of motion for $U_l$,
\beq
\label{eq:EOM-U}
\frac{d U_l}{ds}= i P_l U_l .  
\eeq
The effective Hamiltonian is naturally written as 
\beq
\label{eq:EOM-H}
H= {\rm tr} \sum_{l} P_l^2 + S_{\rm eff}(U) , 
\eeq
 where ${\rm tr}$ is over color indices with the normalization given in Eq.(\ref{eq:tt}).
 Then the  time-parameter independence  $\frac{dH}{ds}=0$ leads to the equation of motion for $P_l$
 (Exercise \ref{prob:11}),
  \beq
 \label{eq:EOM-P} 
 \frac{dP_l}{ds} = - i  \sum_{i,j} t^a \left( t^a U_l \right)_{ij} \frac{\partial S_{\rm eff}(U)}{\partial (U_l)_{ij}}.
\eeq
In the actual simulations, the ${\rm ln Det}F(U) $ part of the effective action is treated by
introducing a set of bosonic variables (pseudofermions) through the identity,
\beq
{\rm Det} F = ({\rm Det} F^{-1})^{-1} = \int [d\chi^* d\chi] \ \exp \left( -\sum_{IJ} \chi^*_{I} F_{IJ}^{-1} \chi_J \right),
\eeq
where $I$ and $J$ stand for all possible internal and spacetime indices carried by $F$.    
For further details of HMC (and its variations) with pseudofermions, 
consult the recent review \cite{Schaefer:2012tq} and references therein.


%===========================
\subsection{Error estimate}
%===========================

There are two kinds of errors  in the data obtained from  LQCD simulations.\\

\noindent {\bf Systematic errors:} \\
 They are related to the lattice spacing $a$, the lattice volume $L^3$,
  and the quark masses ($m$).  During 
  the continuum extrapolation ($a\rightarrow 0$) and the thermodynamic extrapolation ($L \rightarrow \infty$) 
  under the  guidance of  the asymptotic scaling for small $a$ 
 and the finite size scaling for  large $L$, some systematic errors are brought in.
 Also,  one often needs to make extrapolation to the physical quark mass by using 
 lattice data  with heavier  quark masses. This  also brings some
 systematic errors.\\
 
 \noindent
{\bf Statistical error:} \\
It originates from the importance sampling.  A very useful procedure to estimate such error
 commonly used in LQCD is the {\it jackknife resampling method}. (The name
  originates from the ``jackknife"  which is an easy and  portable  tool  for general purposes).
  Let us consider the mean and the unbiased variance of a certain quantity $ {\cal O}$,
  \beq
   \la {\cal O} \ra = \frac{1}{N} \sum_{n=1}^N {\cal O}^{(n)}  \pm \sqrt{\frac{\sigma^2({\cal O} )}{N}}, \ \ \ \
   \sigma^2({\cal O}) = \left( \frac{N}{N-1} \right)  \frac{1}{N} \sum_{n=1}^N ( {\cal O}^{(n)}  - \la {\cal O} \ra )^2,
  \eeq
where the factor $\frac{N}{N-1}$ is called the Bessel's correction.
 The jackknife samples are obtained by
 \beq
  {\cal O}^{(n)}_J= \frac{1}{N-1} \sum_{n' \neq n}  {\cal O}^{(n')}      \ \ \ \  (n=1, \cdots, N).
 \eeq
 If we need to make a quick estimate of the 
  mean and the variance of a function $f({\cal O})$, we have
 \beq
\label{eq:jack}
 \la f({\cal O}_J)\ra = \frac{1}{N} \sum_{n=1}^{N} f( {\cal O}^{(n)}_J) \pm  \sqrt{\frac{\sigma_J^2(f)}{N}}, \ \ \ \ 
  \sigma_J^2(f) = (N-1) \sum_{n=1}^N ( f( {\cal O}^{(n)}_J) -  \la f( {\cal O}_J)\ra)^2. 
\eeq
For $f( {\cal O})={\cal O}$, we recover the original  mean and variance;
 $\la  {\cal O}_J \ra = \la  {\cal O} \ra$ and  $\sigma_J^2({\cal O})=\sigma^2({\cal O})$. (Exercise \ref{prob:12}).
One can generalize this procedure by dividing $N$ into $N_b=N/n_b$  
with the bin-size $n_b$ and create the $N_b$ jackknife samples. 
 Eq.(\ref{eq:jack})  corresponds to the case with $n_b=1$.

 
 
%===========================
\subsection{Heavy quark potential}
%===========================

 As one of the examples of the accurate  inter-quark potential obtained from LQCD simulations,
 we  show in Fig.\ref{fig:QQbar-pot}  the  
 dimensionless ${\rm Q}\bar{\rm Q}$ potential 
$[V(R)-V(R_0)]\times R_0$
 as a function of $R/R_0$ extracted from the calculation of the   
   Wilson loop in the quenched approximation.
 $R$ is a distance between the heavy quarks
  and $R_0$ is called the {\it Sommer scale}   defined by
 $  \left. R^2 \frac{dV(R)}{dR} \right|_{R=R_0}=1.65$.
 
 Simulations with different lattice couplings $6/g^2$
  correspond to different lattice spacings $a$.
 The latter can be fixed, e.g., by   
 taking a phenomenological value
  $R_0\simeq 0.5\ {\rm fm}$. 
    The lattice spacings in the physical uniit in the figure are
   $a=0.094\ {\rm fm}$ (squares: $\beta=6/g^2=6.0$),
    $a= 0.069\ {\rm fm}$ (circles: $\beta=6.2$)
  and  $a=0.051\ {\rm fm}$ (triangles: $\beta=6.4$).
  Since there exists no appreciable $a$ dependence of the potential,
 the system is already close enough to the continuum limit. 
  
 Fig.\ref{fig:QQbar-pot} clearly shows that the 
  heavy quark potential has a linear confining part
   at long distance and an attractive Coulombic part 
    at short distance. The LQCD results agree
     not only qualitatively but also quantitatively  
  with an empirical linear + Coulomb potential
  (the Cornell potential)    shown by the solid line,
   $V(r) = Kr - b/r + {\rm const}$ with $b=0.295$.  
   
% FIG %%%%%%%%%%%%%%%%%%%%%%%%%%   
\begin{figure}[t]
\begin{center}
%\framebox[74mm]{\rule[-26mm]{0mm}{52mm}}
%\includegraphics[scale=0.6]{as-scale.eps}
\includegraphics[scale=0.5]{Chapter3-figures/QQbar-pot.eps} 
 \end{center}
\caption{A dimensionless ${\rm Q}\bar{\rm Q}$ potential as a 
 function of a dimensionless quark$-$anti-quark
 separation $R/R_0$ with $R_0$ being the Sommer scale. 
 Different  symbols correspond to different lattice couplings
  $g(a)$    and hence different lattice spacings. 
  The solid line shows an empirical Cornell potential. 
  The figure is adapted from \cite{Bali:2000gf}.
  }
\label{fig:QQbar-pot}
\end{figure}
%%%%%%%%%%%%%%%%%%%%%%%%%%%%%%


%============================
\subsection{Masses of light hadrons}
%============================

 Meson masses and baryon masses can be 
 calculated with high accuracy by LQCD simulations
  with dynamical quarks.  The starting point is the hadronic 
  correlation functions $C_{\rm H=M,B} (n,n')$ in Eqs.(\ref{eq:correlation-M},\ref{eq:correlation-B}) 
  integrated over the spatial coordinates, $\vn$ and $\vn'$,   
 \beq
 \label{eq:5.D-tau}
 C_{\rm H}(\tau) = \sum_{\vn, \vn'} C_{\rm H} (n,n') 
 \xrightarrow[\tau \rightarrow \infty]
   \ |Z_{\rm H}|^2 {\rm e}^{-M_{\rm H} \tau}   ,
 \eeq   
where $\tau= (n_4-n_4')a$ is the temporal distance between the source at $n'$ and the sink at $n$,
 and  $M_{\rm H}$ ($Z_{\rm H}$) corresponds to the mass (the pole residue) 
 of a lightest bound state in each channel.
 If   the temporal extent of the 
  lattice is infinite, one can extract the 
   hadron mass from the formula,
  $M_{\rm H}= -(1/\tau) \ln C_{\rm H}(\tau)|_{\tau \rightarrow \infty} $.
  In practice, the {\it effective mass}  defined below is more useful,
 \beq
aM_{\rm H}^{\rm eff}(\tau) = \ln \left( \frac{C_{\rm H}(\tau)}{C_{\rm H}(\tau+a) } \right) .
\eeq
The asymptotic plateau of the effective mass at large $\tau$ corresponds to the hadron mass.
     In actual simulations,  the temporal extent is limited ($0 \le \tau/a \le \Ntau$), so that 
    the exponential damping 
    of  Eq.(\ref{eq:5.D-tau}) is replaced by
     $C_{\rm H} \rightarrow \exp[-M_{\rm H} \tau ] \pm  \exp [-M_{\rm H} (\Ntau a - \tau)]$
    where $+ (-)$ for the periodic (anti-periodic) boundary condition.
     
 Shown in the left panel of Fig.\ref{fig:hadron-mass} is the 
 dimensionless effective masses  ($a M_{\rm H}^{\rm eff}$)  against
 $\tau/a=(n_4-n_4')$ \cite{Durr:2008zz}.
  Data points are the effective masses for the pion $(\pi)$, the kaon $(K)$, the nucleon ($N$), the cascade baryon ($\Xi$) and 
  the omega baryon $(\Omega$)  calculated on the lattice with $a \simeq 0.085$ fm and 
  the pion mass $M_{\pi} \simeq 190$ MeV.  Reasonable plateau above $\tau/a > 9$ can
  be seen within the error bars.
 
 Shown in the right panel of Fig.\ref{fig:hadron-mass} is the 
 $M_{\pi}^2$-dependence of the $N$ and $\Omega$ masses for three different values of 
  $a$ \cite{Durr:2008zz}.
 The crosses are the values extrapolated to the continuum limit and to the physical pion mass.
 The $N$ and $\Omega$ masses predicted from LQCD and corresponding experimental numbers
 are 
\beq
\label{eq:mass-LQCD}
M_N^{\rm LQCD} &=& 0.936(25)(22)  \ {\rm GeV}, \ \ \ M_\Omega^{\rm LQCD} =1.676(20)(15)\ {\rm  GeV}, \\
M_N^{\rm exp.} &=& 0.939 \ {\rm  GeV}, \ \ \  \ \ \ \ \ \ \ \ \ \ \ \ \  M_\Omega^{\rm exp.} =1.672 \ {\rm  GeV}.
\eeq
  Note that the numbers in the first (second) parenthesis  in Eq.(\ref{eq:mass-LQCD}) 
 represent the statistical (systematic) errors on the last digits.

% FIG %%%%%%%%%%%%%%%%%%%%%%%%%%   
\begin{figure}[t]
\begin{center}
%\framebox[74mm]{\rule[-26mm]{0mm}{52mm}}
%\includegraphics[scale=0.6]{as-scale.eps}
\includegraphics[scale=0.29]{Chapter3-figures/effective-mass.eps} \ \ 
\includegraphics[scale=0.28]{Chapter3-figures/NO_scaling.eps}
 \end{center}
\caption{(Left) The effective masses of hadrons against the 
 temporal separation between the source and the sink. 
(Right)  Hadron masses under the changes of the (pion mass)$^2$ as well as the lattice spacing $a$.
The figures are taken from \cite{Durr:2008zz}
  }
\label{fig:hadron-mass}
\end{figure}
%%%%%%%%%%%%%%%%%%%%%%%%%%%%%%%

Shown in Fig.\ref{fig:pn-mass} are the high precision numerical results of the 
 hadron mass splittings obtained by the 
 QCD+QED  lattice simulations with dynamical  $u,d,s,c$ quarks \cite{Borsanyi:2014jba}.
 The horizontal lines are the
experimental values and the grey shaded regions represent the experimental errors.
Red dots are the lattice results with their uncertainties denoted by the vertical error bars.
The neutron-proton mass differences from numerical simulations 
and the corresponding experimental numbers are
\beq
(M_n-M_p)^{\rm LQCD+QED} &=&\Delta N = 1.51(16)(23)\   {\rm MeV}, \\
(M_n-M_p)^{\rm exp.} &=& 1.29\   {\rm MeV}. 
\eeq

% FIG %%%%%%%%%%%%%%%%%%%%%%%%%%
\begin{figure}[t]
\begin{center}
%\framebox[74mm]{\rule[-26mm]{0mm}{52mm}}
%\includegraphics[scale=0.6]{as-scale.eps}
\includegraphics[scale=0.2]{Chapter3-figures/pn-mass.eps}
 \end{center}
\caption{Mass splittings in channels that are stable under the strong and electromagnetic interactions.
$\Delta N=M_n-M_p$, $\Delta \Sigma = M_{\Sigma^-}- M_{\Sigma^+}$,
$\Delta \Xi =M_{\Xi^-}- M_{\Xi^+}$,
$\Delta D= M_{D^{\pm}} - M_{D^0}$,
$\Delta \Xi_{cc} = \Delta \Xi_{cc}^{++} - \Delta \Xi_{cc}^{+}$,
$\Delta CG = \Delta N-\Delta \Sigma + \Delta \Xi$.
The figure is taken from
\cite{Borsanyi:2014jba}.
  }
\label{fig:pn-mass}
\end{figure}
%%%%%%%%%%%%%%%%%%%%%%%%%%%

Since all hadrons are composite particles of quarks and gluons,
there are numerous excited states  \cite{RPP}.  To extract the properties of the
excited hadrons from LQCD,  only looking at the asymptotic form as shown in Eq.(\ref{eq:5.D-tau}) is not
 sufficient, and more sophisticated methods such as  
 the maximal entropy method (MEM)  and  the variational method  are required.
 Interested readers should consult the reviews \cite{Asakawa:2000tr,Fodor:2012gf} and references there in.


%%%%%%%%%%%%%%%%%%%%%%%%
\section{Lattice QCD and nuclear force}
%%%%%%%%%%%%%%%%%%%%%%%%

Understanding of  the nuclear force from QCD
 is one of the most challenging problems in nuclear physics.
 Below the pion production threshold,
  the notion of the $NN$ potential (either in the coordinate space or in
  the  momentum space) has been known to be  useful, since it can be 
   used not only to describe the two-body system but also to 
   study  nuclear many-body problems through ab-initio calculations \cite{this_book}.
  Several high precision phenomenological $NN$ forces 
  have been constructed to reproduce
  the   neutron-proton and proton-proton scattering data (about 4500 data points)
   with a $\chi^2/{\rm dof} \sim 1$. However, 
     they have typically 20-40 fitting parameters:
   e.g. the CD Bonn potential, AV18 potential and N$^3$LO chiral effective field 
   theory have 38, 40, and 24 parameters, respectively \cite{Machleidt:2007ms}.
  If one tries to extend these to hyperon-nucleon and hyperon-hyperon interactions,
  the task becomes extremely tough since  the number of parameters 
   increases and the scattering data are scarce.
 
    Under this situation, it is highly desirable to
  study  baryon-baryon interactions from  first principle
  LQCD simulations, where  all the hadronic interactions are
   controlled only by the QCD coupling $g$  and 
    the quark mass $m$  whose values are
    pretty well determined at present by the precision QCD simulations \cite{Aoki:2013ldr}.
 
 The finite volume method (FVM), a
  theoretical framework to study hadron-hadron interactions
  from LQCD, was first  proposed 
  by L\"{u}scher \cite{luescher}: For two hadrons in a finite
  box with a spatial size $L^3$,  
  an exact relation between  the energy spectra in the box
  and the elastic scattering phase shift can be 
   derived.  If the range of the hadronic interaction $R_{\rm QCD}$  is sufficiently
  smaller than the size of the box $R_{\rm QCD}<L/2$, the behavior of the 
  the equal-time Bethe-Salpeter amplitude (or more precisely 
  the Nambu-Bethe-Salpeter (NBS) amplitude)
     $\psi (\vr)$ in the interval $R_{\rm QCD} < | \vr | < L/2 $
    has sufficient information to relate the phase shift and the 
  energy shift $\Delta E =M_{\rm HH}-2M_{\rm H}$.
  
   The HAL QCD method was proposed as another theoretical framework to study the hadron-hadron interactions from LQCD
 by Ishii, Aoki and Hatsuda \cite{Ishii:2006ec}
  and was further developed by  HAL QCD Collaboration \cite{HALQCD:2012aa}.
    The starting point is the same equal-time NBS amplitude  $\psi (\vr)$: 
  Instead of looking at the amplitude 
  outside the range of the interaction,
   the internal region $ |\vr | < R_{\rm QCD}$ is considered and
  an energy-independent  non-local potential $U(\vr, \vr')$  is deduced from  $\psi (\vr)$.
    Since $U(\vr, \vr')$ in QCD
   is spatially  localized due to the confinement
   of quarks and gluons, it is affected  only weakly
   by the finite lattice volume. Physical quantities such 
 as the scattering phase shifts, bound state spectra,  and resonance energies
  can be calculated by solving  the
  integro-differential differential equation satisfied by $\psi (\vr)$ with $U(\vr, \vr')$.
   
  Recently, a detailed comparison between the FVM and the HAL QCD method has been carried out:
  Although they agree with each other quite accurately for non-resonant pion-pion scattering,
   large signal to noise ratio  inherent in the effective mass $\Delta E^{\rm eff}(\tau)$  for 
    baryon-baryon scatterings prevents FVM to extract scattering observables   \cite{Iritani:2015dhu}.
   Therefore, we will focus on  the HAL QCD method below.
  
%=============================================     
\subsection{Master equation for baryon-baryon interaction}
%=============================================  
  
Let us consider  the baryon-baryon correlation in Fig.\ref{fig:correlation}(c) and 
define the equal-time NBS amplitude $\psi_{\ell}(\vr,\tau)$  from its large $\tau$ behavior:
 \beq
 C_{\rm BB}(\vr, \tau) 
 =\sum_{\vn',\vm'}   \left. C_{\rm BB}(n,m,n',m') \right|_{n_4=m_4,n_4'=m_4'} 
 \rightarrow \sum_{\ell} a_{\ell}  \psi_{\ell} (\vr,\tau)  e^{-E_{\ell} \tau}   ,
 \eeq 
 where $\vr=(\vn-\vm) a$, $\tau=(n_4-n_4')$, and $\psi_{\ell} (\vr,\tau)$ being the 
  NBS wave function for $\ell$-th scattering state on the lattice. 
  For large lattice size, $E_{\ell}$ is very dense, so that it is impossible to identify each level.  
 This causes a fatal problem in FVM as mentioned above.  On the other hand,
 if we define  $ C_{\rm BB}(\vr;\tau)= {\cal R}(\vr,\tau) e^{-2M_{\rm B} \tau}$, 
  the following  integro-differential equation can be derived below the inelastic threshold ($\tau > M_{\pi}^{-1}$),
\beq
  \left\{
  \frac1{4M_{\rm B}}\frac{\partial^2}{\partial \tau^2} 
  -\frac{\partial}{\partial \tau}
  - H_0
  \right\}
  {\cal R}(\vr,\tau)
  =
  \int d^3 r'
  U(\vr, \vr')
  {\cal R}(\vr', \tau),
  \label{eq.tdep}
\eeq
 with  $H_0= - \nabla^2/M_{\rm B}$.
 This is the master equation which has the 
correct information of the S-matrix and hence the scattering phase shift for
elastic $BB$ scatterings \cite{HALQCD:2012aa}.
  
  If we further focus on the 
  energies  much below  the inelastic threshold,
   the velocity expansion
   of $U(\vr,\vr')$ in terms of its  non-locality can be adopted.
   In fact, the potential with hermiticity, 
   rotational invariance, parity symmetry,   and time-reversal invariance may be expanded as \cite{okubo}
\begin{eqnarray} 
\label{eq:U-del}
  U(\vr,\vr')    &=& V(\vr, \vv) \delta(\vr-\vr'), \\
 V(\vr, \vv)    & =&
   \underbrace{V_{\rm C}(r) + V_{\rm T}(r) S_{12}}_{\rm LO} 
  + \underbrace{V_{\rm LS}(r) {\vL} \cdot {\vS}}_{\rm NLO}  
  +\underbrace{{O}(\vv^2)}_{{\rm N}^2{\rm LO}}
  + \cdots , 
   \label{eq:V-pot}
\end{eqnarray} 
   where $\vv = \vp/(M_{\rm B}/2) $, $\vL = \vr \times \vp $, 
   $\vp = -i \nabla$ and 
   $S_{12}=3(\vsigma_1 \cdot \vr)(\vsigma_2 \cdot \vr)/r^2 - \vsigma_1 \cdot \vsigma_2$.
     The central potential $V_{\rm C}$ and 
  the tensor potential $V_{\rm T}$ are classified as
  the leading order (LO) potentials since they
  are of $O(\vv^0)$. The next-to-leading (NLO) potential of  
  $O(\vv)$ is the spin-orbit potential  $V_{\rm LS}(r)$. 

%=============================================
\subsection{Baryon-baryon interaction in flavor SU(3) limit}
%=============================================  
  
 % FIG %%%%%%%%%%%%%%%%%%%%%%%%%%
\begin{figure}[t]
\begin{center}
\includegraphics[scale=0.55]{Chapter3-figures/Vc_27_npa2.eps}\ \ \ \ \ \ \ 
\includegraphics[scale=0.55]{Chapter3-figures/Vc_10s_npa2.eps}\\
\includegraphics[scale=0.55]{Chapter3-figures/Vc_1_npa2.eps}\ \ \ \ \ \ \ 
\includegraphics[scale=0.55]{Chapter3-figures/Vt_10s_npa2.eps}
 \end{center}
\caption{The baryon-baryon potentials from LQCD simulations in the flavour
SU(3) limit with several different masses of pseudo-scalar meson. 
The figures are taken from \cite{Inoue:2011ai}.
  }
\label{fig:NN-potential}
\end{figure}
%%%%%%%%%%%%%%%%%%%%%%%%%%%%%
  
 To obtain qualitative understandings of the nuclear force from QCD,
  the $S$-wave interaction between octet baryons  
 in the flavour SU(3) limit  would be a good starting point.
 In this case,  two baryon states with a given angular momentum
are labelled by the irreducible flavour multiplets,
\begin{equation}
 {\bf 8} \otimes {\bf 8} 
 = \underbrace{{\bf 27} \oplus {\bf 8}_s \oplus {\bf 1}}_{\mbox{symmetric}} ~ 
  \oplus \underbrace{{\bf 10}^* \oplus {\bf 10} \oplus {\bf 8}_a}_{\mbox{anti-symmetric}} \ . 
\end{equation}
Here ``symmetric" and ``anti-symmetric" stand for the symmetry under the
flavour exchange of two baryons.
For the system in the orbital S-wave, the Pauli principle between two baryons imposes 
${\bf 27}$, ${\bf 8}_s$ and ${\bf 1}$ to be spin singlet  ($^1S_0$) while 
${\bf 10}^*$, ${\bf 10}$ and ${\bf 8}_a$ to be spin triplet ($^3S_1$). 
Since there are no mixings among different multiplets in the SU(3) limit, 
one may define the corresponding potentials as
 \beq
^1S_0 \ &:& \  V^{({\bf 27})}(r), \ V^{({\bf 8}_s)}(r), \ V^{({\bf 1})}(r), 
\\ 
^3S_1 \ &:& \ V^{({\bf 10}^*)}(r), \ V^{({\bf 10})}(r), \ V^{({\bf 8}_a)}(r) ~.
\eeq
The diagonal potential ($B_1B_2 \rightarrow B_1 B_2)$ and  
 the off-diagonal potentials ($B_1B_2 \rightarrow B_3 B_4$) in the particle basis, 
 are obtained by  suitable combinations
of $V^{(\alpha)}(r)$ with $\alpha={\bf 27},{\bf 8}_s,{\bf 1},{\bf10}^*,{\bf 10},{\bf 8}_a$.

% FIG %%%%%%%%%%%%%%%%%%%%%%%%%%
\begin{figure}[t]
\begin{center}
\includegraphics[scale=0.25]{Chapter3-figures/phase-shift.eps}\ \ \ \ 
\includegraphics[scale=0.27]{Chapter3-figures/neutron-matter.eps}
 \end{center}
\caption{(Left) Phase shifts of the NN scattering as a
function of energy in the laboratory frame, extracted from
LQCD data at the pion mass 469 MeV in the flavor-SU(3) limit.
 The black and gray dashed lines are the results of the partial wave
analysis (PWA) of the experimental data. (Right) Ground state energy per neutron for the
pure neutron matter as a function of the Fermi momentum.
 The APR with black dotted line (black solid line)   corresponds to the empirical equation of state without (with) the
phenomenological three nucleon force \cite{Akmal:1998cf}.
 \cite{Akmal:1998cf}.
The figures are taken from \cite{Inoue:2013nfe}.
  }
\label{fig:NN-phase_shift}
\end{figure}
%%%%%%%%%%%%%%%%%%%%%%%%%%%%%%


Shown in Fig.\ref{fig:NN-potential}  are the results of the exploratory study of the potentials
obtained by LQCD simulations on the lattice  with $a\simeq 0.12$ fm, 
$L\simeq 4$ fm and 3 degenerate flavours  \cite{Inoue:2011ai}.  Corresponding pion mass ranges from
 469 MeV to 1171 MeV. 
\begin{itemize}
\item  The upper left panel is the 
central potential $V^{({\bf 27})}(r)$ to which the  $^1S_0$ nucleon-nucleon potential belongs.
 It  has a repulsive core at short distance and an attractive pocket at intermediate distance.
As the pion mass decreases,  the repulsive core gets stronger and the attractive tail gets
 longer.  
\item As shown in the lower left panel, the structure of the potential is quite different for $V^{({\bf 1})}(r)$ to which the
  flavour singlet $H$ dibaryon (composed of $uuddss$) belongs.   There is no repulsive core and the attraction 
  increases as the pion mass decreases.  Such a feature is consistent with the 
   notion of  the quark Pauli principle previous discussed in phenomenological quark models \cite{Oka-Fujiwara}.
\item The upper and lower right panels of  Fig.\ref{fig:NN-potential}  are the 
central potential and the tensor potential of $V^{({\bf 10}^*)}(r)$, respectively. The
$^3S_1$ nucleon-nucleon potential belongs to this channel. The central part has a similar
structure as the $^1S_0$ channel, while the tensor part has strong attraction and grows rapidly  as the 
 pin mass decreases.  The latter aspect is qualitatively consistent with the one-pion-exchange picture at
 long distances.
 \end{itemize}
   
 Shown in  Fig.\ref{fig:NN-phase_shift} (left) is  the nucleon-nucleon scattering phase shifts
  in the $^1S_0$ channel and $^3S_1$ channel obtained by using the 
   potentials, $V^{({\bf 27})}(r)$ and $V^{({\bf 10}^*)}(r)$,  for the pion mass 469 MeV \cite{Inoue:2013nfe}.
   The qualitative feature of the 
   phase shift in the $^1S_0$ channel is similar to the experimental one denoted by the 
   black solid  line, despite the fact that pion mass in the simulation is  more than 3 times
   heavier than the physical value. In the  $^3S_1$ channel, the deuteron bound state is 
   not formed yet due to heavy pion mass, so that the phase shift starts from 0 at zero energy
    in contrast the the experimental one denoted by the black dashed line.
    There is however a tendency that the attraction in the $^3S_1$ channel is larger than
   the $^1S_0$ channel even for the heavy pion mass.
  Shown in  Fig.\ref{fig:NN-phase_shift} (right) is  the energy per particle $E/A$ as a function of the 
  fermi momentum $k_{\rm F}$ for pure neutron matter calculated by using the Brueckner-Hartree-Fock method
  with the neutron-neutron potential in the $^1S_0$ channel in Fig.\ref{fig:NN-phase_shift} (left). 
 As the pion mass decreases, the equation of state becomes stiffer due to the growth of the repulsive  core. 
 The APR with black dotted line (black solid line)   corresponds to the empirical equation of state without (with) the
phenomenological three nucleon force  \cite{Akmal:1998cf}.

 We note that calculations of the baryon-baryon interactions  with  
  (2+1)-flavour LQCD on a large volume ($L \simeq 8.2 {\rm fm}$, $a\simeq 0.085$ fm) at
   nearly the physical quark mass ($m_{\pi}\simeq146$ MeV, $m_{K}\simeq$ 525 MeV)
   are underway  \cite{Doi:2015oha}.
 
  
%%%%%%%%%%%%%
\section{Exercises}
%%%%%%%%%%%%%

\begin{prob} \label{prob:1}
Prove the properties of the Wilson line, Eqs.(\ref{eq:5.wilson-line-i}), (\ref{eq:5.wilson-line-ii}), and (\ref{eq:5.wilson-line-iii}).
\end{prob}

\begin{prob}\label{prob:2}
Derive the expression on $U_{\mu \nu}(n)-1$ in 
Eq.(\ref{eq:5.wilson-plaq-cont-2}) by using the Baker-Campbell-Hausdorff formula.
\end{prob}

\begin{prob}\label{prob:3}
Show the $\Gamma_5$ Hermiticity of the Dirac operator in Eq.(\ref{eq:gamma5-hermite}).
\end{prob}

\begin{prob}\label{prob:4}
Derive  the free fermion propagator on the lattice in the momentum representation,
Eqs.(\ref{eq:DF}) and (\ref{eq:Mp}). 
\end{prob}

\begin{prob}\label{prob:5}
Analyze the dispersion relation of the free fermion associated with the 
Dirac operator, $D_{\rm GW}$ in Eq.(\ref{eq:5.overlap-op}).
\end{prob}

\begin{prob}\label{prob:6}
Derive the group integration formulas, Eq.(\ref{eq:group-int-1})- Eq.(\ref{eq:group-int-B}), by taking appropriate 
contractions of the color indices.
\end{prob}


\begin{prob}\label{prob:7}
Derive the formula for the Wilson loop in the strong coupling limit Eq.(\ref{eq:5.wilson-strong}) by using the group integration formulas
Eqs.(\ref{eq:group-int-0})-(\ref{eq:group-int-2}). 
\end{prob}

\begin{prob}\label{prob:8}
Derive the lattice coupling $g(a)$ as a function of the lattice spacing $a$, Eq.(\ref{eq:5.lat-running-g}).
\end{prob}

\begin{prob}\label{prob:9}
Show the convergence of $W[\phi]$ to the equilibrium distribution $W_{\rm eq}[\phi]$
under the Markov process by introducing the distance $D= \sum_{\phi} \left| W[\phi] -W_{\rm eq}[\phi] \right| $
 and by studying its behavior under update.
\end{prob}

\begin{prob}\label{prob:10}
Prove that the Metropolis test in Eq.(\ref{eq:MET-test})
satisfies the detailed balance Eq.(\ref{eq:DT-balance}).
\end{prob}

\begin{prob}\label{prob:11}
Derive the equation of motion for $P_l$ in Eq.(\ref{eq:EOM-P})
from Eq.(\ref{eq:EOM-U}) and  Eq.(\ref{eq:EOM-H}).
\end{prob}

\begin{prob}\label{prob:12}
Prove that the jackknife average and variance for $f({\cal O})={\cal O}$ reduce to the 
standard mean and unbiased variance, respectively.
\end{prob}

\begin{prob}\label{prob:13}
Prove that  the leapfrog integrator satisfies the reversibility Eq.(\ref{eq:reversible}) exactly.
Also prove that  the leapfrog integrator preserves the phase space area exactly by
evaluate the Jacobian, $d\phi' d\pi'= J d\phi d\pi $. 
\end{prob}
 

%%%%%%%%%%%%%%%%
\begin{acknowledgement}
%%%%%%%%%%%%%%%%
%If you want to include acknowledgments of assistance and the like at the end of an individual chapter 
%please use the \verb|acknowledgement| environment -- it will automatically render Springer's preferred layout.
The author thanks Takumi Doi and Atsushi Nakamura for useful comments and information on various aspects of
LQCD simulations.
He also thank the members of HAL QCD Collaboration for  fruitful discussions on the hadron-hadron interactions 
on the lattice.
This work was supported in part by MEXT SPIRE and JICFuS and also by RIKEN iTHES Project.
\end{acknowledgement}
%

\section{Appendix}
\addcontentsline{toc}{section}{Appendix}

%%%%%%%%%%%%%%%%%%%%%%%%%%%%
\subsubsection*{\center{Four vectors and Dirac matrices}}
%%%%%%%%%%%%%%%%%%%%%%%%%%%%

In the (3+1)-dimensional  Minkowski spacetime, coordinates, derivatives 
and four vectors with $\mu=0,1,2,3$ are
\beq
x^{\mu} = (t, \vx),
\ \ \partial^{\mu}   =  (\partial_t, -\nabla),
\ \  A^{\mu} =(A^0, \vA).
\label{eq:B.vector-M}
\eeq
In the 4-dimensional Euclidean space, 
 we define the corresponding vectors  for $\mu=4,1,2,3$ as
 \beq
(x_{\mu})^{\rm E} = (\tau=it, \vx),
\ \ (\partial_{\mu})^{\rm E}   =  (\partial_\tau = -i \partial_{t}, \nabla),
\ \  (A_{\mu})^{\rm E}  =(A_4=iA^0,\vA).
\label{eq:B.vector-E}
\eeq


In the (3+1)-dimensional  Minkowski spacetime with the 
 metric $g^{\mu \nu}={\rm diag}(1,-1,-1,-1)$,
  the Dirac matrices satisfy the following relations for $\mu=0,1,2,3$,
\beq
\{ \gamma^{\mu}, \gamma^{\nu} \} = 2 g^{\mu \nu},
\ \ \left( \gamma^{\mu} \right)^{\dagger} = \gamma^0 \gamma^{\mu} \gamma^0,
\ \ \gamma^5 &=& i \gamma^0 \gamma^1 \gamma^2 \gamma^3 
 = \gamma_5 = \left( \gamma_5 \right)^{\dagger}.
\label{eq:B.gamma-min}
\eeq
In the standard Dirac representation, we have
\beq
\gamma^0 = 
\left(  \begin{array}{cc}
        1   & 0   \\
        0   & -1  \\
        \end{array}  \right) , \ \ 
\gamma^j = 
\left(  \begin{array}{cc}
        0           & \sigma_j   \\
        -\sigma_j   & 0          \\
        \end{array}  \right) , \ \ 
\gamma^5 =
\left(  \begin{array}{cc}
        0   & 1   \\
        1   & 0   \\
        \end{array}  \right)  ,
\eeq
where $\sigma_j$ are the Pauli matrices;        
$
%\beq
\sigma_1 = 
\left(  \begin{array}{cc}
        0   & 1   \\
        1   & 0   \\
        \end{array}  \right) , 
\sigma_2 = 
\left(  \begin{array}{cc}
        0           & -i   \\
        i           & 0    \\
        \end{array}  \right) , 
\sigma_3 =
\left(  \begin{array}{cc}
        1   & 0    \\
        0   & -1   \\
        \end{array}  \right)  .
%\label{eq:B.Pauli-matrix}
%\eeq
$

In the 4-dimensional Euclidean space  with the metric
 $\delta_{\mu \nu}={\rm diag}(1,1,1,1)$, we define
  the Euclidean Dirac matrices as
 \beq
  \Gamma_{\mu}  \equiv
 \left(\gamma_4= \gamma^0, -i \vgamma \right),  
 \ \ \Gamma_{-\mu} \equiv  - \Gamma_{\mu}, 
  \ \ {\rm and}   \ \ \Gamma_5 \equiv \gamma^5,
  \eeq
 which satisfy the relations,
 \beq
 \{ \Gamma_{\mu}, \Gamma_{\nu} \}  = 2 \delta_{\mu \nu},
\ \ \Gamma_{\mu}^{\dagger} = \Gamma_{\mu} , 
\ \ \ \ ({\rm for} \ \mu=1,2,3,4,5) 
\label{eq:B.gamma-lat} 
\eeq 

%%%%%%%%%%%%%%%%%%%%%
\subsubsection*{\center{SU$(N)$ algebra}}
%%%%%%%%%%%%%%%%%%%%%

Let ${\cal T}^a$ ($a=1, \cdots , N^2-1$) are the 
 Hermitian generators of the SU$(N)$ group. They satisfy
  the Lie algebra  \index{Lie algebra}
\beq
\left[  {\cal T}^a , {\cal T}^b \right] = i f_{abc} {\cal T}^c ,
\eeq
where $f_{abc}$ is the structure constant \index{structure constants}
 being totally anti-symmetric in its indices.
 $({\cal T}^b)^2$ commutes with every generator ${\cal T}^a$
  and   is called the quadratic Casimir operator. 
  
 For $N=2$, $f_{abc}$ reduces to the  anti-symmetric
 tensor $\epsilon_{ijk}$ with  $\epsilon_{123}=1$.
For $N=3$, the non-vanishing components of $f_{abc}$ read
$f_{123}=1, f_{147}=-f_{156}=f_{246}=f_{257}=f_{345}=-f_{367}=1/2,  
f_{458}=f_{678}= {\sqrt{3}}/{2}$.
   
  In the fundamental representation, 
 ${\cal T}^a$ is written by the $N \times N$ matrices $t^a$ as
\beq
t^a = \frac{1}{2} \lambda_a,
\eeq
 where $\lambda_a$ for $N=2$ reduce to  the Pauli matrices 
 $\sigma_i$, while  those for $N=3$ reduce to the Gell-Mann matrices.
 
 Some useful relations of $t^a$ for general $N$ are
\beq
\label{eq:tt}
{\rm tr} (t^a t^b) = \frac{1}{2} \delta_{ab}, 
\ \ \ t_{ij}^a t_{kl}^b  =  \frac{1}{2} (\delta_{il}\delta_{jk} -\frac{1}{N} \delta_{ij}\delta_{kl} ),
\ \ \ (t^a t^a)_{ij} = C_{{\rm F}} \delta_{ij},
\ \ \ {\rm with} \ C_{{\rm F}}=\frac{N^2-1}{2N}.
\eeq
   
In the adjoint representation, \index{adjoint representation}
 ${\cal T}^a$ is written by  $(N^2-1) \times (N^2-1)$ matrices $T^a$ as
\beq
(T^a)_{bc} = -i f_{abc} ,
\eeq
which satisfy the relations
\beq
  {\rm tr} (T^a T^b) &=& N \delta_{ab}, 
  \ \ \  (T^a T^a)_{bc} = C_{{\rm A}} \delta_{bc},
  \ \ \  {\rm with} \ C_{{\rm A}}= N.
\eeq



%%%%%%%%%%%%%%%%%%%%%%%%%%%%%%
\subsubsection*{\center{Gaussian and Grassmann integrals}}
%%%%%%%%%%%%%%%%%%%%%%%%%%%%%%
 
Basic Gaussian and Grassmann integrals are 
\beq
 \label{eq:C.gauss-x}
& &\int_{-\infty}^{+\infty} \frac{dx}{\sqrt{2\pi}}\ {\rm e}^{-ax^2/2} 
    =\frac{1}{\sqrt{a}}, \\
 \label{eq:C.gauss-z}
& & \int \frac{dz^* dz}{2 \pi i}\ {\rm e}^{-b |z|^2} = \frac{1}{b}, \\
 \label{eq:C.gauss-xi}
& & \int d\bar{\xi} d\xi\  {\rm e}^{- c \bar{\xi} \xi} = c .
\eeq     
Here $x$ ($z$) is a real (complex) number, while $\bar{\xi}$ and
 $\xi$ are anti-commuting Grassmann numbers ($\{ \xi, \bar{\xi} \} =0$,
 and  $\xi^2 = \bar{\xi}^2=0$).
 $a$ and $b$ are assumed to be real and positive numbers, while $c$ is an
  arbitrary complex number.
 Eq.(\ref{eq:C.gauss-z}) can be shown by rewriting the integral 
 in terms of the real and imaginary parts of $z$ or in terms of the 
 polar coordinates of $z$.   Eq.(\ref{eq:C.gauss-xi})
  can be shown by noting that 
   ${\rm e}^{-c\bar{\xi}\xi} = 1- c \bar{\xi}{\xi}$
   and $ \int d\xi = \partial/\partial \xi $
   (integral = derivative) for Grassmann variables.
  
Generalization of the above results to the case of multiple variables
 is straightforward.  For $x =(x_1, \cdots, x_n)$, 
  $z =(z_1, \cdots, z_n)$, $\xi =(\xi_1, \cdots, \xi_n)$,
 and $\bar{\xi} =(\bar{\xi}_1, \cdots, \bar{\xi}_n)$ with
  $\{ \xi_k, \xi_l \} = \{ \bar{\xi}_k, \bar{\xi}_l \} 
   = \{ \xi_k, \bar{\xi}_l \} =0$,  we have
 \beq
 \label{eq:C.gauss-xn}
& &\int \prod_{l=1}^n \frac{dx_l}{ \sqrt{2\pi} }\ 
 {\rm e}^{-\frac{1}{2} x A x} = \frac{1}{\sqrt{{\rm Det}\ A}}, \\
 \label{eq:C.gauss-zn}
& & \int \prod_{l=1}^n \frac{dz_l^* dz_l}{2 \pi i}\ 
{\rm e}^{- z^* B z} = \frac{1}{{\rm Det}\ B}, \\
 \label{eq:C.Z-grass-xin}
& & \int \prod_{l=1}^n d\bar{\xi}_l d\xi_l \
  {\rm e}^{- \bar{\xi}C \xi} = {\rm Det}\ C .
\eeq  
Here $A$ is a non-singular and real-symmetric matrix whose
 eigenvalues $a_l$ satisfy $a_l > 0$ for all $l$.
 $B$ is a non-singular complex matrix whose complex eigenvalues $b_l$
  obtained by the 
  biunitary transformation ($U B V^{\dagger}$)  
  satisfy ${\rm Re}\ b_l > 0$ for all $l$.
 $C$ is an arbitrary complex matrix with no conditions. 
 Note that $B$ and $C$ do not have to be Hermitian matrices.
 In field theories, the label ``$l$" summarizes all possible indices including
  spin, flavor, color, spacetime points etc. and 
  ``${\rm Det}$" denotes the determinant for all these indices. 
 


%%%%%%%%%%%%%%%%%%%%%%%%%%
\subsubsection*{\center{Method of characteristics}}
%%%%%%%%%%%%%%%%%%%%%%%%%%

We need to construct  a general  solution of the 
following partial differential equation,
\beq
 \left( \lambda \frac{\partial}{\partial \lambda} + \beta(g) \frac{\partial}{\partial g} \right) f(\lambda,g)=0.
 \label{eq:MCA-1}
 \eeq
 For this purpose, we introduce the running coupling $\bar{g}(\lambda)$ through 
 $\lambda d\bar{g}/d\lambda = -\beta(\bar{g})$ whose formal solution reads
 \beq
 \lambda = \exp \left( - \int_g^{\bar{g}(\lambda) }  \frac{dg'}{\beta(g')}  \right) .
 \label{eq:MCA-2}
  \eeq
 Then the solution of Eq.(\ref{eq:MCA-1}) can be written as 
 \beq
 f(\lambda, g) = f (1, \bar{g} (\lambda)).
 \eeq
 This can be explicitly checked  by applying the partial derivative on both sides,
 \beq
  \lambda \partial_{\lambda}  f(\lambda,g)&=&
  \lambda (\partial_{\lambda} \bar{g})   (\partial_{\bar{g}} f) = - \beta (\bar{g})  (\partial_{\bar{g}} f) , \nonumber \\
  \beta \partial_g f(\lambda,g) &=&
  \beta(g) \left. (\partial \bar{g}/\partial g) \right|_{\lambda}  (\partial_{\bar{g}} f) = \beta (\bar{g})  (\partial_{\bar{g}} f) .
 \eeq 
 where we have used the  relation $\partial \bar{g}/\partial g = \beta(\bar{g})/\beta(g)$ 
  obtained from   Eq.(\ref{eq:MCA-2}).
 
 In general, the first-order partial differential equation (PDE)  can be transformed to 
 a set of ordinary   differential equations (ODEs) and can  be solved by  {\it the method of characteristics}. 
 As an illustration,  let us consider the following PDE, 
 \beq
  a(t,x) \partial_t u(t,x) + b(t,x) \partial_x f(t,x) = c(t,x).
\label{eq:MCA-3}
 \eeq
This is equivalent to the coupled ODEs,
\beq
\frac{d \bar{t}}{ds}  = a (\bar{t}, \bar{x}) , \ \ 
\frac{d \bar{x}}{ds}  = b (\bar{t}, \bar{x}) , \ \ 
\frac{d f(\bar{t},\bar{x} ) } {ds}  = a (\bar{t}, \bar{x}) ,
\label{eq:MCA-4}
\eeq
where $s$ parametrizes the  "flow" of the coordinates. $(\bar{t}(s), \bar{x}(s))$.
This is called the {\it characteristic curve} as shown in Fig.\ref{fig:char-c}.

\begin{figure}[t]
\begin{center}
%\framebox[74mm]{\rule[-26mm]{0mm}{52mm}}
%\includegraphics[scale=0.6]{as-scale.eps}
\includegraphics[scale=0.35]{Chapter3-figures/char-c.eps} 
 \end{center}
\caption{Schematic illustration of the characteristic curve.}
\label{fig:char-c}
\end{figure}

The function $f$ can be obtained by integrating the last equation of Eq.(\ref{eq:MCA-4}) 
on the  characteristic curve from the initial point $(t_{\rm i},x_{\rm i})$ to the final point $(t_{\rm f},x_{\rm f})\equiv (t,x)$,
\beq
f(t,x)= f (t_{\rm i}, x_{\rm i}) + h (t, x, t_{\rm i}, x_{\rm i})
\label{eq:MCA-6}
\eeq
where  $h$ stands for an integration of the known function $c(\bar{t},\bar{x})$ on the
characteristic curve.  Eq.(\ref{eq:MCA-6}) implies  that the desired function at $(t,x)$ is obtained 
essential by a "pullback" of the point to $(t_{\rm i},x_{\rm i})$ along the characteristic curve. 
It is a straightforward exercise  to generalize the above derivation to the system with more coordinates,
$(t, \vx)$.  

%%%%%%%%%%%%%%%%%%%%%%%%%%%%%%%%%%
 \subsubsection*{\center{Leapfrog integrator in molecular dynamics}}
%%%%%%%%%%%%%%%%%%%%%%%%%%%%%%%%%%

Let us start with a Tayler expansion of the field $\phi$:
\beq
\label{eq:LF-phi}
\phi(s+\varepsilon) &=& \phi(s) + \varepsilon \dot{\phi}(s) + \frac{\varepsilon^2}{2} \ddot{\phi}(s) + O(\varepsilon^3), \nonumber \\
&=&  \phi(s) + \varepsilon {\pi}(s) + \frac{\varepsilon^2}{2} \dot{\pi}(s) + O(\varepsilon^3), \nonumber  \\
&=& \phi(s) + \varepsilon {\pi}(s+\varepsilon/2) + O(\varepsilon^3), 
\eeq
where we have used the equation of motion, $\dot{\phi}(s) \equiv d\phi(s)/ds = \pi(s)$.
To evaluate ${\pi}(s+\varepsilon/2)$, we take the midpoint prescription which does not have $O(\varepsilon^2)$ error,
 \beq
 \label{eq:LF-pi}
{\pi}(s+\varepsilon/2)&=&{\pi}(s-\varepsilon/2) + \varepsilon \dot{\pi}(s) + O(\varepsilon^3) \nonumber \\
&=&  {\pi}(s-\varepsilon/2)   - \varepsilon \frac{\delta S(\phi)}{\delta \phi(s)} + O(\varepsilon^3) .
\eeq
Eqs.(\ref{eq:LF-phi}) and (\ref{eq:LF-pi})  give a procedure to move the molecular dynamics one-step forward,
$(\phi(s), \pi(s-\varepsilon/2))  \rightarrow (\phi(s+\varepsilon), \pi(s+\varepsilon/2))$. 
The initial and final steps need to receive special care,
\beq
\pi(\varepsilon/2) = \pi(0) - \frac{1}{2} \varepsilon \frac{\delta S(\phi)}{\delta \phi(s)} + O(\varepsilon^2), \ \ \ 
\pi(s_{\rm f}) =\pi(s_{\rm f}-\varepsilon/2) -  \frac{1}{2} \varepsilon \frac{\delta S(\phi)}{\delta \phi(s_{\rm f})} + 
O(\varepsilon^2),
 \eeq
 which have only $O(\epsilon^2)$ accuracy.  An illustration of this leapfrog integrator is shown in Fig.(\ref{fig:LF}).
 Since the initial and final steps introduce  $O(\varepsilon^2)$ error irrespective of the 
length of the MD trajectory, and  the intermediate steps introduce 
$O(\varepsilon^3) \times \varepsilon^{-1} =O(\varepsilon^2)$ error as a whole,  one finds
 $\Delta H = O(\varepsilon^2)$ after one MD trajectory before the Metropolis test.

The leapfrog integrator satisfies the reversibility and symplectic property, which can be checked explicitly
by using the above definitions (Exercise \ref{prob:13}).
 
 \begin{figure}[t]
\begin{center}
%\framebox[74mm]{\rule[-26mm]{0mm}{52mm}}
%\includegraphics[scale=0.6]{as-scale.eps}
\includegraphics[scale=0.3]{Chapter3-figures/leapfrog.eps} 
 \end{center}
\caption{The leapfrog integrator.}
\label{fig:LF}
\end{figure}


%%%% references %%%%%%%%%
\begin{thebibliography}{99.}%

\bibitem{Wilson:1974sk}
  K.~G.~Wilson,
 ``Confinement of Quarks,''
  Phys.\ Rev.\ D {\bf 10} (1974) 2445.
  
 \bibitem{Creutz:1980zw}
  M.~Creutz,
  ``Monte Carlo Study of Quantized SU(2) Gauge Theory,''
  Phys.\ Rev.\ D {\bf 21} (1980) 2308. 
  
  \bibitem{Brambilla:2014jmp}
  N.~Brambilla {\it et al.},
  ``QCD and Strongly Coupled Gauge Theories: Challenges and Perspectives,''
  Eur.\ Phys.\ J.\ C {\bf 74} (2014) 2981
 %doi:10.1140/epjc/s10052-014-2981-5
  [arXiv:1404.3723 [hep-ph]].
   
  \bibitem{Wilson:2004de}
  K.~G.~Wilson,
  ``The Origins of lattice gauge theory,''
  Nucl.\ Phys.\ Proc.\ Suppl.\  {\bf 140} (2005) 3
  %doi:10.1016/j.nuclphysbps.2004.11.271
  [hep-lat/0412043].
  
\bibitem{Creutz:1984mg}
  M.~Creutz,
  ``Quarks, gluons and lattices,''
  Cambridge Monographs on Mathematical Physics (Cambridge Univ. Press, UK, 1985) pp. 1-169.
    
\bibitem{Rothe:1992nt}
  H.~J.~Rothe,
  ``Lattice gauge theories: An Introduction,''
 World Sci.\ Lect.\ Notes Phys. vol.82 (2012) pp. 1-606.
  %%CITATION = 00327,43,1;%%  

 \bibitem{Hoelbling:2014uea}
  C.~Hoelbling,
  ``Lattice QCD: concepts, techniques and some results,''
  Acta Phys.\ Polon.\ B {\bf 45} (2014)  2143
 % doi:10.5506/APhysPolB.45.2143
  [arXiv:1410.3403 [hep-lat]]. 

 \bibitem{Ukawa:2015eka}
  A.~Ukawa,
  ``Kenneth Wilson and lattice QCD,''
  J.\ Statist.\ Phys.\  {\bf 160} (2015) 1081
 % doi:10.1007/s10955-015-1197-x
  [arXiv:1501.04215 [hep-lat]]. 
 
\bibitem{Negele:1988vy}
  J.~W.~Negele and H.~Orland,
  ``Quantum Many Particle Systems,''
  FRONTIERS IN PHYSICS, vol.68 (Addison-Wesley, USA, 1988) pp. 1-459.    

\bibitem{Haggstrom:2002}
  O. H\"{a}ggstr\"{o}m,
  ``Finite Markov Chains and Algorithmic Applications,''
 London Mathematical Society, Student Texts, vol. 52   (Cambridge Univ. Press, UK, 2002) pp. 1-114.    

\bibitem{SuwaTodo:2010}
 H. Suwa and S.Todo,
 ``Markov Chain Monte Carlo Method without Detailed Balance,"
 Phys.\ Rev.\ Lett.\ {\bf 105} (2010) 120603
 [ArXiv:1007.2262 [cond-mat]].

 \bibitem{Duane:1987de}
  S.~Duane, A.~D.~Kennedy, B.~J.~Pendleton and D.~Roweth,
  ``Hybrid Monte Carlo,''
  Phys.\ Lett.\ B {\bf 195} (1987) 216.
%  doi:10.1016/0370-2693(87)91197-X
  
\bibitem{Metropolis_1953}
 N. Metropolis, A. W. Rosenbluth, M. N. Rosenbluth, A. H. Teller and E. Teller,
 ``Equation of State Calculations by Fast Computing Machines,''
 J. Chem. Phys. {\bf 21} (1953) 1087.
 %http://dx.doi.org/10.1063/1.1699114 
  
\bibitem{Schaefer:2012tq}
  S.~Schaefer,
 ``Status and challenges of simulations with dynamical fermions,''
  PoS LATTICE {\bf 2012} (2012) 001
  [arXiv:1211.5069 [hep-lat]].  
  
 \bibitem{Bali:2000gf}
  G.~S.~Bali,
  ``QCD forces and heavy quark bound states,''
  Phys.\ Rept.\  {\bf 343} (2001) 1
%  doi:10.1016/S0370-1573(00)00079-X
  [hep-ph/0001312]. 

\bibitem{Durr:2008zz}
  S.~Durr {\it et al.},
  ``Ab-Initio Determination of Light Hadron Masses,''
  Science {\bf 322} (2008) 1224
 % doi:10.1126/science.1163233
  [arXiv:0906.3599 [hep-lat]].
  
\bibitem{Borsanyi:2014jba}
  S.~Borsanyi {\it et al.},
 ``Ab initio calculation of the neutron-proton mass difference,''
  Science {\bf 347} (2015) 1452
%  doi:10.1126/science.1257050
  [arXiv:1406.4088 [hep-lat]].  
  
\bibitem{RPP}   
The Review of Particle Physics (2015),
\url{http://pdg.lbl.gov/ } 
  
\bibitem{Asakawa:2000tr}
  M.~Asakawa, T.~Hatsuda and Y.~Nakahara,
 ``Maximum entropy analysis of the spectral functions in lattice QCD,''
  Prog.\ Part.\ Nucl.\ Phys.\  {\bf 46} (2001) 459
 % doi:10.1016/S0146-6410(01)00150-8
  [hep-lat/0011040].
  
\bibitem{Fodor:2012gf}
  Z.~Fodor and C.~Hoelbling,
 ``Light Hadron Masses from Lattice QCD,''
  Rev.\ Mod.\ Phys.\  {\bf 84} (2012) 449
%  doi:10.1103/RevModPhys.84.449
  [arXiv:1203.4789 [hep-lat]].
   
\bibitem{this_book}
Consult other chapters of this volume.
 
 \bibitem{Machleidt:2007ms}
  R.~Machleidt,
  ``Nuclear forces from chiral effective field theory,''
  arXiv:0704.0807 [nucl-th].

\bibitem{Aoki:2013ldr}
  S.~Aoki {\it et al.},
  ``Review of lattice results concerning low-energy particle physics,''
  Eur.\ Phys.\ J.\ C {\bf 74} (2014) 2890
%  doi:10.1140/epjc/s10052-014-2890-7
  [arXiv:1310.8555 [hep-lat]].

\bibitem{luescher}
 M.~L\"{u}scher,
``Two particle states on a torus and their relation to the scattering matrix,''
Nucl. \ Phys.\ B {\bf 354} (1991) 531

\bibitem{Ishii:2006ec}
N.~Ishii, S.~Aoki and T.~Hatsuda,
``The Nuclear Force from Lattice QCD,''
Phys.\ Rev.\ Lett.\  {\bf 99} (2007) 022001

\bibitem{HALQCD:2012aa}
  N.~Ishii {\it et al.} [HAL QCD Collaboration],
"Hadron-hadron interactions from imaginary-time Nambu-Bethe-Salpeter wave function on the lattice,"
 Phys.\ Lett.\ B {\bf 712} (2012) 437
%\UTF{00A0}\UTF{00A0}doi:10.1016/j.physletb.2012.04.076
[arXiv:1203.3642 [hep-lat]].

\bibitem{Iritani:2015dhu}
T.~Iritani [HALQCD Collaboration],
``Lattice QCD studies on baryon interactions from L\"uscher's finite volume method and HAL QCD method,''
arXiv:1511.05246 [hep-lat].

\bibitem{okubo}
S. Okubo S, R.E. Marshak,
``Velocity dependence of the two-nucleon interaction,"
Ann. of Phys. {\bf 4} (1958) 166.

\bibitem{Oka-Fujiwara}
M. Oka, K. Shimizu, K. Yazaki, 
 ``Quark cluster model of baryon baryon interaction,''
 {\it Prog.\ Theor.\ Phys.\ Suppl.}\  {\bf 137}  (2000) 1.  
% Y. Fujiwara, Y. Suzuki and C. Nakamoto,
%  ``Baryon baryon interactions in the SU(6) quark model and their applications
%  to light nuclear systems,''
%  {\it Prog.\ Part.\ Nucl.\ Phys.}\  {\bf 58} (2007) 439.

\bibitem{Inoue:2011ai}
  T.~Inoue {\it et al.} [HAL QCD Collaboration],
  ``Two-Baryon Potentials and H-Dibaryon from 3-flavor Lattice QCD Simulations,''
  Nucl.\ Phys.\ A {\bf 881} (2012) 28
%  doi:10.1016/j.nuclphysa.2012.02.008
  [arXiv:1112.5926 [hep-lat]].
  
  \bibitem{Inoue:2013nfe}
  T.~Inoue {\it et al.} [HAL QCD Collaboration],
  ``Equation of State for Nucleonic Matter and its Quark Mass Dependence from the Nuclear Force in Lattice QCD,''
  Phys.\ Rev.\ Lett.\  {\bf 111} (2013)  112503
%  doi:10.1103/PhysRevLett.111.112503
  [arXiv:1307.0299 [hep-lat]].
  
 \bibitem{Akmal:1998cf}
  A.~Akmal, V.~R.~Pandharipande and D.~G.~Ravenhall,
  ``The Equation of state of nucleon matter and neutron star structure,''
  Phys.\ Rev.\ C {\bf 58} (1998) 1804
%  doi:10.1103/PhysRevC.58.1804
  [nucl-th/9804027]. 
  
  \bibitem{Doi:2015oha}
  T.~Doi {\it et al.} [HAL QCD Collaboration],
  ``First results of baryon interactions from lattice QCD with physical masses (1) -- General overview and two-nucleon forces --,''
  arXiv:1512.01610 [hep-lat].
 


  
\end{thebibliography}
\label{chap:chapter3}
% common phrases
\newcommand{\ie}{\textit{i.e.}}
\newcommand{\eg}{\textit{e.g.}}

\newcommand{\cf}{\textit{cf.}\xspace}
\newcommand{\adhoc}{\textit{ad hoc}\xspace}
\newcommand{\etal}{\textit{et al.}\xspace}
\newcommand{\perse}{\textit{per se}\xspace}
\newcommand{\etc}{\textit{etc.}\xspace}

\newcommand{\apriori}{\textit{a priori}\xspace}
\newcommand{\aposteriori}{\textit{a posteriori}\xspace}


% basic math
\newcommand{\ii}{\mathrm{i}}
\newcommand{\eex}{\mathrm{e}}

\newcommand{\hc}{\mathrm{h.c.}}

\newcommand{\abs}[1]{\left|#1\right|}
\newcommand{\conj}[1]{\overline{#1}}


% derivatives
\newcommand{\dd}{\mathrm{d}}

\newcommand{\vNabla}{\boldsymbol{\nabla}}
\newcommand{\Laplace}{\vNabla^2}
\newcommand{\dAlem}{\Box}

% three-vectors
%\newcommand{\vx}{\mathbf{x}}
%\newcommand{\vy}{\mathbf{y}}

%\newcommand{\vk}{\mathbf{k}}
%\newcommand{\vq}{\mathbf{q}}

%\newcommand{\vA}{\mathbf{A}}

\newcommand{\vZero}{\mathbf{0}}

\newcommand*\rvec[1]{%
\ensuremath{\overset{\smash{\raisebox{-1.5pt}{\tiny$\rightarrow$}}}{#1}}%
}
\newcommand*\lvec[1]{%
\ensuremath{\overset{\smash{\raisebox{-1.5pt}{\tiny$\leftarrow$}}}{#1}}%
}
\newcommand*\lrvec[1]{%
\ensuremath{\overset{\smash{\raisebox{-1.5pt}{\tiny$\leftrightarrow$}}}{#1}}%
}

\newcommand{\vNablaLR}{\lrvec{\vNabla}}

\newcommand{\skvec}[1]{\vv{#1}}

\newcommand{\vD}{\boldsymbol{D}}

% EFT
\newcommand{\LL}{\mathcal{L}}

\newcommand{\eps}{\varepsilon}

\newcommand{\Mhi}{\ensuremath{M_{\rm hi}}}
\newcommand{\Mlo}{\ensuremath{M_{\rm lo}}}

% units
\newcommand{\GeV}{\ensuremath{\mathrm{GeV}}}
\newcommand{\MeV}{\ensuremath{\mathrm{MeV}}}
\newcommand{\fm}{\ensuremath{\mathrm{fm}}}

% masses etc.
\newcommand{\mpi}{m_\pi}
\newcommand{\msigma}{m_\sigma}
\newcommand{\mrho}{m_\rho}

\newcommand{\MN}{M_N}

\newcommand{\Md}{M_d}
\newcommand{\Bd}{B_d}

\newcommand{\fpi}{f_\pi}

\newcommand{\gamt}{\gamma_t}
%\newcommand{\at}{a_t}
\newcommand{\rt}{r_t}
\newcommand{\rhot}{\rho_t}

\newcommand{\gamd}{\gamt}

% matrices
%\NewEnviron{skmattwod}{%
%\left(\begin{array}{cc}
%  \BODY
%\end{array}\right)
%}

%\NewEnviron{skvectwod}{%
%\left(\begin{array}{c}
%  \BODY
%\end{array}\right)
%}

\newcommand{\twodmat}[1]{\begin{skmattwod}#1\end{skmattwod}}
\newcommand{\twodvec}[1]{\begin{skvectwod}#1\end{skvectwod}}

\newcommand{\diag}{\mathrm{diag}}

\newcommand{\one}{\mathbf{1}}
\newcommand{\leviciv}{\epsilon}

\newcommand{\tr}{\mathrm{tr}}
\newcommand{\Tr}{\mathrm{Tr}}

\newcommand{\idx}[2]{{}^{#1}{}_{#2}}
\newcommand{\idxx}[3]{\big(#1\big)\idx{#2}{#3}}
\newcommand{\idxxx}[3]{\left(#1\right)^{\!\!#2}_{\,#3}}

% misc.
\newcommand{\mathspace}{\ \ }
\newcommand{\mathtext}[1]{\mathspace\text{#1}\mathspace}

\newcommand{\OO}{\mathcal{O}}

% QM
%\newcommand{\bra}[1]{\langle #1|}
\newcommand{\braauto}[1]{\left\langle #1\right|}
\newcommand{\brabig}[1]{\big\langle #1\big|}

%\newcommand{\ket}[1]{|#1\rangle}
\newcommand{\ketauto}[1]{\left|#1\right\rangle}
\newcommand{\ketbig}[1]{\big|#1\big\rangle}

%\newcommand{\braket}[2]{\langle #1|#2\rangle}
\newcommand{\braketauto}[2]{\left\langle #1\middle|#2\right\rangle}
\newcommand{\braketbig}[2]{\big\langle #1\big|#2\big\rangle}

\newcommand{\mbraket}[3]{\langle #1|#2|#3\rangle}
\newcommand{\mbraketauto}[3]{\left\langle #1\middle|#2\middle|#3\right\rangle}
\newcommand{\mbraketbig}[3]{\big\langle #1\big|#2\big|#3\big\rangle}

\newcommand{\comm}[2]{[#1,#2]}
\newcommand{\commauto}[2]{\left[#1,#2\right]}
\newcommand{\commbig}[2]{\big[#1,#2\big]}

\newcommand{\acomm}[2]{\{#1,#2\}}
\newcommand{\acommauto}[2]{\left\{#1,#2\right\}}
\newcommand{\acommbig}[2]{\big\{#1,#2\big\}}

% custom environments and boxes
%\NewEnviron{subalign}[1][]{%
%\begin{subequations}\begin{align}
%  \BODY
%\end{align}\label{#1}\end{subequations}
%}

%\NewEnviron{spliteq}{%
%\begin{equation}\begin{split}
%  \BODY
%\end{split}\end{equation}
%}

\newcommand*{\vcenteredhbox}[1]
{\begingroup\setbox0=\hbox{#1}\parbox{\wd0}{\box0}\endgroup}

% nuclear states and channels

\newcommand{\ThreeSOne}{\ensuremath{{}^3S_1}\xspace}
\newcommand{\OneSNot}{\ensuremath{{}^1S_0}\xspace}

\newcommand{\Triton}{\ensuremath{{}^3\mathrm{H}}\xspace}
\newcommand{\ThreeH}{\Triton}
\newcommand{\ThreeHe}{\ensuremath{{}^3\mathrm{He}}\xspace}
\newcommand{\FourHe}{\ensuremath{{}^3\mathrm{He}}\xspace}
\newcommand{\SixHe}{\ensuremath{{}^6\mathrm{He}}\xspace}
\newcommand{\SixLi}{\ensuremath{{}^6\mathrm{Li}}\xspace}

\newcommand{\epspol}{\vec{\epsilon}_\gamma}

\newcommand{\epsgamma}{\vec{\epsilon}_{s_\gamma}}
\newcommand{\epsgammaout}{\vec{\epsilon}^{\;*}_{s_\gamma}}

\newcommand{\epsd}{\vec{\epsilon}_{s_d}}
\newcommand{\epsdout}{\vec{\epsilon}^{\;*}_{s_d}}

% coupling constants

\newcommand{\y}{y}
\newcommand{\yt}{\ensuremath{\y_t}}
\newcommand{\ys}{\ensuremath{\y_s}}

\newcommand{\sigt}{\ensuremath{g_t}}
\newcommand{\sigs}{\ensuremath{g_s}}

\newcommand{\st}{\ensuremath{{s,t}}}
\newcommand{\yst}{\ensuremath{\y_\st}}
\newcommand{\sigst}{\ensuremath{g_\st}}

\title{General aspects of effective field theories and few-body applications}

\author{Hans-Werner Hammer and Sebastian K\"onig}
\institute{Hans-Werner Hammer
\at Institut f\"ur Kernphysik,
Technische Universit\"at Darmstadt,
64289 Darmstadt, 
Germany,
\email{Hans-Werner.Hammer@physik.tu-darmstadt.de},
\and Sebastian K\"onig
\at Department of Physics,
The Ohio State University,
Columbus, Ohio 43210,
USA
\email{koenig.389@osu.edu}
}

\graphicspath{{Chapter4-figures/}}

\maketitle
\abstract{Effective field theory provides a powerful framework to 
exploit a separation of scales in physical systems. In these lectures,
we discuss some general aspects of effective field theories and their 
application to few-body physics. 
In particular, we consider an effective field theory for 
non-relativistic particles with resonant short-range interactions
where certain parts of the interaction need to be treated nonperturbatively.
As an application, we discuss the so-called \emph{pionless effective field
theory} for low-energy nuclear physics. The extension to include long-range 
interactions mediated by photon and pion-exchange is also addressed.}



%\input{chap4_intro}
\section{Introduction: dimensional analysis and the separation of scales}
\label{sec:EFT-Intro}

Effective field theory (EFT) provides a general approach to calculate low-energy 
observables by exploiting scale separation.  The origin of the EFT approach can 
be traced to the development of the renormalization group~\cite{Wilson-83} and 
the intuitive understanding of ultraviolet divergences in quantum field 
theory~\cite{Lepage-89}.  A concise formulation of the underlying principle was 
given by Weinberg~\cite{Weinberg:1978kz}: If one 
starts from the most general Lagrangian consistent with the symmetries of the 
underlying theory, one will get the most general S-matrix consistent with these 
symmetries.  As a rule, such a most general Lagrangian will contain infinitely 
many terms.  Only together with a power counting scheme that orders these terms 
according to their importance at low energies one obtains a predictive paradigm 
for a low-energy theory.

The Lagrangian and physical observables are typically 
expanded in powers of a low-momentum scale $\Mlo$,
which can be a typical external momentum or an internal
infrared scale, over a high-momentum scale 
$\Mhi\gg\Mlo$.\footnote{Note there are often 
more than two scales, which complicates the power counting. Here we focus on
the simplest case to introduce the general principle.}
This expansion provides the basis for the power counting scheme.
It depends on the system to which physical scales 
$\Mhi$ and $\Mlo$ correspond to.  

As an example, we take a theory that is made of two
particle species, two light bosons with mass $\Mlo$ and heavy
bosons with mass $\Mhi \gg \Mlo$.\footnote{For further examples,
see the lectures by Kaplan \cite{Kaplan:1995uv,Kaplan:2005es}.}
We consider now soft processes
in which the energies and momenta are of the order of the 
light particle mass (the so-called soft scale). 
Under such conditions, the short-distance
physics related to the heavy particles can never be resolved explicitly.
However, it can be represented by light-particle contact interactions 
with increasing dimension (number of derivatives). To illustrate this,
we consider the scattering of the light particles mediated by
heavy-particle exchange, with
$g$ the heavy-light coupling constant. The corresponding interaction
Lagrangian is given by
%----------------------
\begin{equation}
 {\mathcal L}_{\rm int}=g\left(\chi^\dagger \phi \phi +
 \phi^\dagger \phi^\dagger \chi\right)\,,
\end{equation}
%----------------------
where $\phi$ denotes the light boson field and
$\chi$ is the heavy boson field. As depicted in
Fig.~\ref{fig:resosat}, one can represent such exchange diagrams by a 
sum of local operators of the light fields with increasing
number of derivatives. In a symbolic notation, the leading order 
scattering amplitude can be written as
%----------------------
\begin{equation}
 T\sim \frac{g^2}{\Mhi^2 - q^2} = \frac{g^2}{\Mhi^2} + 
 \frac{g^2 \, q^2}{\Mhi^4} + \cdots \,,
\end{equation}
%----------------------
with $q^2$ the squared 4-momentum transfer.  We will come back
to this example in more detail in section~\ref{sec:EFT-Basics}.
%%%%%%%%%%%%%%%%%%%%%%%%%%%%%%%%%%%%%%%%%%%%%%%%%%%%%%%%%%%%%%%%
\begin{figure}[tb]
\centerline{\includegraphics*[width=8cm]{BosonContact-2body-simple}}
\caption{Expansion of heavy-particle exchange
between light particles in terms of contact interactions between
light particles. The solid and dashed lines denote light and heavy
particles, respectively. The circle and square
denote contact interactions with zero and two derivatives, in order.
\label{fig:resosat}}
\end{figure}
%%%%%%%%%%%%%%%%%%%%%%%%%%%%%%%%%%%%%%%%%%%%%%%%%%%%%%%%%%%%%%%%%

In many cases, the corresponding high-energy theory
is either not known or can not easily be solved. 
Still, EFT offers a predictive
and systematic framework for performing calculations in the
light-particle sector. We denote by $Q$ a typical energy
or momentum of the order of $\Mlo$ and by $\Mhi$ the
hard scale where the EFT will break down. In many cases, this
scale is set by the masses of the heavy particles not considered
explicitly and thus replaced by contact interactions as in the example above.  
In such a setting, any matrix element or Green's
function admits an expansion in the small parameter 
$Q/\Mhi$~\cite{Weinberg:1978kz}
%----------------------
\begin{equation}
 {\mathcal M} = \sum\limits_{\nu} \left(\frac{Q}{\Mhi}\right)^{\nu}  
 \,{\mathcal F} \left(\frac{Q}{\Lambda}, g_i\right)
\end{equation}
%----------------------
where ${\mathcal F}$ is a function of order one (this is the naturalness
assumption), $\Lambda$
a regularization scale (related to the UV divergences appearing
in the loop graphs) and the $g_i$ denotes a collection of
coupling constants, often called low-energy constants (LECs).
These parameterize (encode) the unknown high-energy (short-distance)
physics and must be determined by a fit to data (or
can be directly calculated if the corresponding high-energy theory
is known/can be solved). The counting index $\nu$ in general depends 
on the fields in the effective theory, the number of derivatives and
the number of loops. This defines the so-called power counting 
which allows to categorize all contributions to any matrix element
at a given order. It is important to stress that $\nu$ must be
bounded from below to define a sensible EFT. In QCD, \eg, this is 
a consequence of the spontaneous breaking of chiral symmetry. 

The contributions with the lowest possible value of $\nu$ 
define the so-called leading order (LO) contribution, the first
corrections with the second smallest allowed value of $\nu$ the
next-to-leading order (NLO) terms and so on. In contrast to more
conventional perturbation theory, the small parameter is not a 
dimensionless coupling constant (like, \eg, in Quantum Electrodynamics) 
but rather a ratio of two scales. Typically, one
expands in the ratio of a small energy or momentum and 
the hard scale $\Mhi$. A prototype of such
a perturbative EFT is chiral perturbation theory that exploits the
strictures of the spontaneous and explicit chiral symmetry breaking
in QCD~\cite{Gasser:1983yg,Gasser:1984gg}.  Here, the light degrees
of freedom are the pions, that are generated through the symmetry violation.
Heavier particles like \eg vector mesons only appear indirectly
as they generate local four-pion interactions with four, six, etc
derivatives.

In these lectures, we also consider EFTs with bound states, where 
certain contributions need to resummed nonperturbatively. In
section~\ref{sec:EFT-Basics}, we start with some general considerations.
This is followed by the explicit discussion of an EFT for 
non-relativistic bosons with short-range interactions 
and large scattering length in section~\ref{sec:EFT-Bosons}.  The extension of 
this framework 
to low-energy nucleons is presented in section~\ref{sec:EFT-Nucleons}
Finally, we will discuss the inclusion of long-range interactions
mediated by photon and pion exchange in~\ref{sec:EFT-Beyond}.

%\input{chap4_basics}
\section{Theoretical foundations of effective field theory}
\label{sec:EFT-Basics}

As mentioned in the introduction, effective field theories are described by 
writing down Lagrangians with an infinite number of terms, restricted only by 
symmetry considerations, and ordered by a scheme referred to as ``power 
counting.''  In this section, we discuss the meaning and importance of all 
these ingredients.

\subsection{Top-down vs.\ bottom-up approaches}

Generally, there are two different motivations for working with an EFT.  Given a 
known quantum field theory, which can be solved to compute a given quantity of 
interest, it can be beneficial to switch to an effective description valid only 
in a limited energy regime simply because carrying out the calculation is more 
efficient with the effective theory.  With such a solvable underlying theory, 
the parameters (``low-energy constants'') of the effective theory can be 
computed directly by considering some number of (simple) processes, \ie, one 
does not need experimental input beyond what was needed to fix the parameters 
of the underlying theory.  This approach, based on a reduction of expressions 
from the underlying to the effective picture is called a ``top-down'' approach.

An alternative procedure, somewhat closer to what we described at the outset, 
is to start ``bottom up,'' \ie, by simply writing down the effective 
Lagrangian directly---or more precisely only those terms of the infinitely 
many which are needed to achieve a given desired accuracy.  Being able to that 
of course requires that as a first step one has already figures out which terms 
are allowed and how they should be ordered.

Our approach here is to work top down in the pedagogical sense, \ie, postpone 
the discussion of the bottom-up approach and its ingredients until later in 
this section, and instead dive into the matter starting with examples that show 
how effective low-energy theories can arise from more fundamental ones.  We 
assume that the reader is familiar the material from a standard 
(relativistic) quantum field theory course.

\subsubsection{Integrating out exchange particles: part I}
\label{sec:EFT-IntOut-1}

As was also mentioned in the introduction, the very first step in 
the construction of an EFT is to identify the relevant degrees of freedom to 
work with, as well as those which are irrelevant and thus do not need to 
be kept explicitly (with emphasis on the last word, because 
\emph{implicitly} the physics of left-out degrees of freedom should and does 
enter in the effective description).

Let us illustrate this by showing how integrating out a ``heavy'' 
particle gives rise to contact interactions between the remaining degrees 
of freedom (see the example in section \ref{sec:EFT-Intro}).  We 
stress that the particles which are integrated out can in fact be lighter than 
what is left (like it is the case in pionless EFT)---what really matters for 
the procedure is which particles are assumed to appear in \emph{asymptotic} 
states, and what is the typical energy/momentum scale between those.  In that 
spirit, we are not making explicit assumptions about the mass hierarchy of the 
particles in the following.  For the illustration here, we consider two scalar 
fields (complex and relativistic) with Yukawa interactions and start with a 
Lagrangian for two species:
%
\begin{equation}
 \LL = {-}\phi^\dagger\left(\dAlem + m_\phi^2\right)\phi
 - \chi^\dagger\left(\dAlem + m_\chi^2\right)\chi
 + g\left(\phi^\dagger\phi^\dagger\chi + \hc\right) \,.
\label{eq:L-phi-chi-rel}
\end{equation}
%
Suppose now we are only interested in interactions between $\phi$ particles 
at energy scales much smaller than $m_\chi$, so that the explicit 
$\chi$ exchange generated by the interaction term in 
Eq.~\eqref{eq:L-phi-chi-rel} cannot be resolved.  In that case, we can derive a 
new effective Lagrangian that only contains $\phi$ degrees of freedom, a 
process referred to as ``integrating out'' the field $\chi$ stemming from its 
implementation in the path-integral formalism.  In effect, that amounts to 
using the equations of motion, which we do here.  From the Euler-Lagrange 
equation for $\chi^\dagger$, we directly get
%
\begin{equation}
 \chi = \left(\dAlem + m_\chi^2\right)^{-1} g\phi\phi \,.
\label{eq:chi-g-phiphi}
\end{equation}
%
Defining the Klein-Gordon propagator
%
\begin{equation}
 D_\chi(x-y)
 = \int\frac{\dd^4p}{(2\pi)^4} \eex^{{-}\ii p(x-y)}
 \frac{\ii}{p^2-m_\chi^2 + \ii\eps} \,,
\label{eq:D-chi-KG}
\end{equation}
%
satisfying
%
\begin{equation}
 \left(\dAlem + m_\chi^2\right)D_\chi(x-y) = {-}\ii \delta^{(4)}(x-y) \,,
\label{eq:chi-GF}
\end{equation}
%
we can write out Eq.~\eqref{eq:chi-g-phiphi} in configuration space as
%
\begin{equation}
 \chi(x) = \ii g \int\dd^4y \, D_\chi(x-y) \phi(y) \phi(y) \,.
\end{equation}
%
Inserting this back into the Lagrangian~\eqref{eq:L-phi-chi-rel}, we obtain 
%
\begin{equation}
 \LL(x) = {-}\phi^\dagger(x)\left(\dAlem + m_\phi^2\right)\phi(x)
 - \ii g^2 \phi^\dagger(x) \phi^\dagger(x)
 \int\dd^4y \, D_\chi(x-y) \phi(y) \phi(y) \,,
\label{eq:L-phi-chi-rel-nl}
\end{equation}
%
where we have written out the spacetime dependence of all fields and used 
Eq.~\eqref{eq:chi-GF} to cancel the terms involving $\chi^\dagger(x)$.  So far, 
we have made only exact manipulations, but the resulting 
Lagrangian~\eqref{eq:L-phi-chi-rel-nl} is \emph{non-local}, \ie, it depends on 
fields evaluated at different spacetime points.  To simplify it further, we 
want to exploit the fact that $\chi$ is considered ``heavy'' compared to the 
scales we want to describe.  Mathematically, this means that $D_\chi(x-y)$ is 
peaked at distances that are small compared to $1/m_\chi^2$.  There are several 
ways to implement this knowledge.  A particularly intuitive version is to expand 
the propagator~\eqref{eq:D-chi-KG} in momentum space,
%
\begin{equation}
 \frac{\ii}{p^2-m_\chi^2 + \ii\eps}
 = \frac{{-}\ii}{m_\chi^2}\left(1 + \frac{p^2}{m_\chi^2} + \cdots\right) \,,
\label{eq:D-chi-expansion}
\end{equation}
%
and then Fourier-transform back to configuration space.  The first term gives a 
simple delta function, and terms with powers of $p^2$ induce operators with 
derivatives acting on $\delta(x-y)$.  Inserting the leading term into 
Eq.~\eqref{eq:L-phi-chi-rel-nl}, we arrive at the effective \emph{local} 
Lagrangian
%
\begin{equation}
 \LL_{\text{eff}}(x) = {-}\phi^\dagger(x)\left(\dAlem + m_\phi^2\right)\phi(x)
 - \frac{g^2}{m_\chi^2} \phi^\dagger(x) \phi^\dagger(x) \, \phi(x) \, \phi(x)
 + \cdots \,.
\label{eq:L-phi-chi-rel-eff}
\end{equation}
%
The ellipses contain operators with derivatives acting on $\phi(x)$, obtained 
from those acting on the delta functions from the propagator after integrating 
by parts.  A diagrammatic illustration of the procedure is shown in
Fig.~\ref{fig:BosonContact-2body}.

%%%%%%%%%%%%%%%%%%%%%%%%%%%%%%%%%%%%%%%%%%%%%%%%%%%%%%%%%%%%%%%%%%%%%%%%%%%%%%
\begin{figure}[htbp]
\centering
\includegraphics[clip,width=0.75\textwidth]{BosonContact-2body}
\caption{Chain of contact interactions obtained by integrating out an exchange 
particle.}
\label{fig:BosonContact-2body}
\end{figure}
%%%%%%%%%%%%%%%%%%%%%%%%%%%%%%%%%%%%%%%%%%%%%%%%%%%%%%%%%%%%%%%%%%%%%%%%%%%%%%

We note that an alternative derivation of the above 
result, discussed for example in Ref.~\cite{Donoghue:1992}, is given by 
Taylor-expanding the field product $\phi(y) \phi(y)$ about $y=x$ under the 
integral and then using the properties of the propagator.  This directly gives 
terms with an increasing number of derivatives acting on $\phi^\dagger(x) 
\phi^\dagger(x)$, and those with an odd number of derivatives are found to 
vanish, in agreement with Eq.~\eqref{eq:D-chi-expansion} featuring only even 
powers of $p^2$.

\subsubsection{Emergence of many-body forces}

We now add a third field $\Phi$ to the Lagrangian:
%
\begin{multline}
 \LL = {-}\phi^\dagger\left(\dAlem + m_\phi^2\right)\phi
 - \chi^\dagger\left(\dAlem + m_\chi^2\right)\chi
 - \Phi^\dagger\left(\dAlem + m_\Phi^2\right)\Phi \\
 \null + g\left(\phi^\dagger\phi^\dagger\chi + \hc\right)
 + g'\left(\Phi^\dagger\phi\chi + \hc\right) \,.
\label{eq:L-phi-chi-Phi-rel}
\end{multline}
%
The new interaction term is chosen such that $\Phi$ can ``decay'' into a $\phi$ 
and a $\chi$, thus acting like a heavier version of the $\phi$.  In spite of 
the simplicity of this bosonic toy model, it is useful to think about $\phi$ 
and $\Phi$ as the nucleon and its $\Delta$ excitation, respectively, and about 
$\chi$ as a pion field.
%
If we first integrate out the $\Phi$ field following the procedure described in 
the previous section, we find
%
\begin{equation}
 \Phi(x) = \ii g' \int\dd^4y \, D_\Phi(x-y) \phi(y) \chi(y) \,,
\end{equation}
%
and thus
%
\begin{equation}
 \LL_{\text{eff}} = {-}\phi^\dagger\left(\dAlem + m_\phi^2\right)\phi
 - \chi^\dagger\left(\dAlem + m_\chi^2\right)\chi
 + g\left(\phi^\dagger\phi^\dagger\chi + \hc\right)
 + \frac{g'^2}{m_\Phi^2} \phi^\dagger\chi^\dagger\phi\chi
 + \cdots \,,
\label{eq:L-Phi-phi-chi-rel-eff-1}
\end{equation}
%
where we have only kept the leading (no derivatives) induced contact 
interaction.  Proceeding as before for the $\chi$ field, we now get
%
\begin{equation}
 \left(\dAlem + m_\chi^2\right)\chi = g\phi\phi
 + \frac{g'^2}{m_\Phi^2} \phi^\dagger \phi \, \chi 
 + \cdots \,.
\label{eq:chi-g-phiphi-etc-1}
\end{equation}
%
This can no longer be solved exactly because we now have a $\chi$ on the 
right-hand side.  However, using the general operator identify
%
\begin{equation}
 \left(\hat{A}-\hat{B}\right)^{-1}
 = \hat{A}^{-1} + \hat{A}^{-1}\hat{B}\,\hat{A}^{-1} + \cdots \,,
\label{eq:AB-inv}
\end{equation}
%
we can write down a formal iterative solution:
%
\begin{equation}
 \chi = \left(\dAlem + m_\chi^2\right)^{-1} g\phi\phi
 + \left(\dAlem + m_\chi^2\right)^{-1}\frac{g'^2}{m_\Phi^2}\phi^\dagger\phi
 \left(\dAlem + m_\chi^2\right)^{-1} g\phi\phi + \cdots \,,
\label{eq:chi-g-phiphi-etc-2}
\end{equation}
%
with each of the inverse differential operators giving a propagator when 
written out.  Those, in turn, each give factors of ${-}\ii/m_\Phi^2$ times a 
delta function, plus additional terms with derivatives.

\begin{prob}
{\emph Exercise:} Derive Eq.~\eqref{eq:AB-inv}.
\end{prob}

Inserting the above result back into the Eq.~\eqref{eq:L-Phi-phi-chi-rel-eff-1}, 
we see that in addition to the two-body contact operator $(\phi^\dagger\phi)^2$ 
obtained previously, we now also get all kinds of higher-body interactions.  For 
example, we get a three-body force through
%
\begin{equation}
 \frac{g'^2}{m_\Phi^2} \phi^\dagger\chi^\dagger\phi\chi
 \rightarrow \frac{g'^2 g^2}{m_\Phi^2 m_\chi^4} (\phi^\dagger\phi)^3 \,.
\end{equation}
%
In Fig.~\ref{fig:BosonContact-3body} it is illustrated diagrammatically how 
such a term arises subsequently, starting from a diagram derived from the 
original Lagrangian~\eqref{eq:L-phi-chi-Phi-rel} with three fields.


%%%%%%%%%%%%%%%%%%%%%%%%%%%%%%%%%%%%%%%%%%%%%%%%%%%%%%%%%%%%%%%%%%%%%%%%%%%%%%
\begin{figure}[htbp]
\centering
\includegraphics[clip,width=0.85\textwidth]{BosonContact-3body}
\caption{Emergence of a three-body contact interaction.}
\label{fig:BosonContact-3body}
\end{figure}
%%%%%%%%%%%%%%%%%%%%%%%%%%%%%%%%%%%%%%%%%%%%%%%%%%%%%%%%%%%%%%%%%%%%%%%%%%%%%%

\subsection{Nonrelativistic field theory}

Relativistic effects and exact Lorentz invariance are not crucial to describe 
systems at low energies, where ``low'' means ``much smaller than the particles' 
rest mass.''  Based on that, one typically starts with a nonrelativistic 
framework and writes down effective Lagrangians of so-called Schr\"odinger 
fields, \eg,
%
\begin{equation}
 \LL_{\phi,\text{free}}
 = \phi^\dagger \left(\ii\partial_t + \frac{\Laplace}{2m}\right)\phi
\label{eq:L-phi-free-nonrel}
\end{equation}
%
for a free scalar particle, where $\phi^\dagger(t,\vx)$ is the field operator 
that creates a particle at time $t$ and position $\vx$, and $\phi(t,\vx)$)
correspondingly destroys it.  Written in terms of momentum-space ladder 
operators $\hat{a}_{\mathbf{p}}$,$\hat{a}^\dagger_{\mathbf{p}}$ (as they appear in standard 
many-body quantum mechanics), we have
%
\begin{equation}
 \phi(t,\vx)
 = \int\!\frac{\dd^3p}{(2\pi)^3} \hat{a}_{\mathbf{p}} \,\eex^{-\ii E_{\mathbf{p}} t} 
 \eex^{\ii{\mathbf{p}}\cdot\vx} \,,
\end{equation}
%
and analogously for $\phi^\dagger(t,\vx)$.  Note that here $E_{\mathbf{p}} = 
{\mathbf{p}}^2/(2m)$ is the kinetic energy alone, and that creation and destruction 
operators are completely separated.  Intuitively, this makes perfect sense: 
At low energies, virtual particle-antiparticle pairs would be highly 
off-shell, thus giving rise to very short-range effects that we can 
simply describe as contact interactions.  Other effects, such as self-energy 
corrections to the particle mass, are automatically accounted for by using the 
physical value for $m$ in Eq.~\eqref{eq:L-phi-free-nonrel}.  With this in mind, 
one can proceed in the bottom-up approach and construct an interacting theory 
by supplementing the free Lagrangian with all allowed contact operators.  In 
particular, like Eq.~\eqref{eq:L-phi-free-nonrel} they should all be invariant 
under Galilei transformations, the low-energy remnant of the Poincar{\'e} 
group.  Before we come back to this, however, we find it instructive to 
explicitly consider the low-energy limit of a relativistic theory.

\subsubsection{Nonrelativistic limit of a bosonic field}
\label{sec:EFT-NonRelBos}

Let us make the connection of Eq.~\eqref{eq:L-phi-free-nonrel} to a
relativistic complex Klein--Gordon field $\Phi$, the Lagrangian for which can be 
written as
%
\begin{equation}
 \LL_{\varphi,\text{free}}
 = {-}\varphi^\dagger\left(\partial_t^2 - \Laplace + m^2\right)\varphi 
\,.
\end{equation}
%
Using integration by parts, this can be shown to be equivalent to the more 
common form written with $(\partial_\mu\varphi^\dagger)(\partial^\mu\varphi)$.  
This implies the Klein--Gordon equation for the field operator,
%
\begin{equation}
 \left(\partial_t^2 - \Laplace + m^2\right)\varphi = 0 \,,
\end{equation}
%
the most general solution of which is typically written as (with a 
four-vectors $x=(t,\vx)$, $p=(p_0,{\mathbf{p}})$, and a Lorentz-invariant integration 
measure)
%
\begin{equation}
 \varphi(x)
 = \int\!\frac{\dd^3p}{(2\pi)^3} \frac{1}{\sqrt{2\omega_{\mathbf{p}}}}\left(
 \hat{a}_{\mathbf{p}}\,\eex^{-\ii p\cdot x} + \hat{b}^\dagger_{\mathbf{p}}\,\eex^{\ii p\cdot x}
 \right)\Bigg|_{p_0=\omega_{\mathbf{p}}} \,,
\label{eq:Phi}
\end{equation}
%
where $\omega_{\mathbf{p}} = \sqrt{{\mathbf{p}}^2+m^2}$.  With this convention where $p_0$ is 
chosen positive, modes created by $\hat{a}^\dagger_{\mathbf{p}}$ correspond to particles 
(propagating forward in time), whereas $\hat{b}^\dagger_{\mathbf{p}}$ creates an 
antiparticle (positive-energy state propagating backwards in time).  That we 
have both stems from the fact that the complex scalar field corresponds to two 
real ones (completely decoupled in the absence of interactions), each of which 
comes with its own pair of creation and annihilation operators.  To take 
the nonrelativistic limit, we have to consider the particle and antiparticles 
separately.  Defining
%
\begin{equation}
 \varphi_a(x) = \int\!\frac{\dd^3p}{(2\pi)^3} \frac{1}{\sqrt{2\omega_{\mathbf{p}}}}
 \hat{a}_{\mathbf{p}}\,\eex^{-\ii p\cdot x}\Bigg|_{p_0=\omega_{\mathbf{p}}}
 \equiv \eex^{-\ii m t} \phi_a(x) \,,
\label{eq:Phi-phi-a}
\end{equation}
%
and plugging this into the Klein--Gordon equation, we get
%
\begin{equation}
 \eex^{-\ii m t} \left[\partial_t^2 - 2\ii m\,\partial_t
 - \vNabla^2\right] \phi_a(x) = 0 \,,
\label{eq:KG-phi-a}
\end{equation}
%
where the quadratic mass term has canceled.  Since $\phi_a(x) = \eex^{\ii m t} 
\varphi_a(x)$, we see from Eq.~\eqref{eq:Phi-phi-a} that in the Fourier 
transform each time derivative acting on $\phi_a(x)$ brings down a factor
%
\begin{equation}
 \omega_p - m = \sqrt{{\mathbf{p}}^2 + m^2} - m \approx {\mathbf{p}}^2/(2m) \,,
\end{equation}
%
\ie, just the kinetic energy $E_{\mathbf{p}}$ up to corrections of higher order in 
$1/m$.  In the nonrelativistic limit, $E_{\mathbf{p}} \ll m$, so we see that we can 
neglect the quadratic time derivative in Eq.~\eqref{eq:KG-phi-a} compared to the 
other two terms in Eq.~\eqref{eq:KG-phi-a}, and then recover the Schr\"odinger 
equation for $\phi_a$:
%
\begin{equation}
 \left(\ii\partial_t + \frac{\vNabla^2}{2m}\right)\phi_a(x) = 0 \,.
\end{equation}
%
This establishes the connection to our $\phi(t,\vx)$ in 
Eq.~\eqref{eq:L-phi-free-nonrel} when we insert an additional factor 
$\sqrt{2m}$ in the field redefinition to account for the otherwise different 
normalizations.  For the antiparticles, we can carry out an analogous procedure, 
except that we have to choose the opposite sign for the mass-dependent phase in 
the field redefinition analogous to Eq.~\eqref{eq:Phi-phi-a} because the 
antiparticle part of $\varphi(x)$ comes with a factor $\eex^{+\ii p\cdot x}$.

\subsubsection{Nonrelativistic fermions}
\label{sec:EFT-NonRelFerm}

For relativistic Dirac fermions, the nonrelativistic reduction can be carried 
out with the help of a so-called Foldy-Wouthuysen transformation.  The idea 
behind the approach is to decouple the particle and antiparticle modes 
contained together in a four-spinor $\psi$ through a sequence of unitary 
transformations.  In the following, we demonstrate this procedure, using an 
\emph{interacting} model theory to also illustrate what happens to interaction 
terms in the nonrelativistic limit.  Since it will be useful to motivate the 
pionless EFT discussed in Sec.~\ref{sec:EFT-Nucleons}, we start with a 
Lagrangian of the form
%
\begin{multline}
 \LL = \bar{\psi}\left(\ii\slashed{\partial} - \MN\right)\psi
 + \frac12(\partial^\mu\skvec{\pi}) \cdot (\partial_\mu\skvec{\pi})
 - \frac12\mpi^2 \skvec{\pi}^2
 + \frac12(\partial^\mu\sigma) \cdot (\partial_\mu\sigma)
 - \frac12\msigma^2 \sigma^2 \\
 \null - g\bar{\psi}(\sigma - \ii \gamma^5\skvec{\tau}\cdot\skvec{\pi})\psi \,,
\label{eq:L-pi-N-PS}
\end{multline}
%
where the nucleon field $\psi$ is an isospin doublet of Dirac spinors, 
$\vec{\pi}$ is an isospin triplet, and $\sigma$ is an isoscalar.  A Lagrangian 
of this form (plus additional interaction terms among $\sigma$ and 
$\vec{\pi}$), can be obtained from a linear sigma model after spontaneous 
symmetry breaking (see, for example, \cite{Donoghue:1992}, Chapter I) and 
augmented by an explicit mass term for $\vec{\pi}$.\footnote{We stress, 
however, that this really is a model and not a proper EFT describing QCD.}  We 
denote the Pauli matrices in spin and isospin space as $\vec{\sigma} = 
(\sigma^i)$ and $\vec{\tau} = (\tau^\lambda)$, respectively.  For the gamma 
matrices we use the standard (Dirac) representation:
%
\begin{equation}
 \gamma^0 = \twodmat{\one & 0 \\ 0 & {-}\one} \mathtext{,}
 \gamma^i = \twodmat{0 & \sigma^i \\ {-}\sigma^i & 0} \mathtext{,}
 \gamma^5 = \twodmat{0 & \one \\ \one & 0} \,.
\end{equation}

To perform the nonrelativistic reduction, we start by separating odd and even 
operators, which are two-by-two block matrices in Dirac space.  The result is
%
\begin{equation}
 \LL_\psi = \psi^\dagger\left(\hat{E} + \hat{O} - \gamma^0\MN\right)\psi \,,
\label{eq:L-E-O}
\end{equation}
%
where
%
\begin{equation}
 \hat{E} = \twodmat{
  \ii\partial_t - g\sigma & 0 \\
  0 & \ii\partial_t + g\sigma
 }
 \mathtext{and}
 \hat{O} = \twodmat{
   0 & {-}\ii\skvec{\sigma}\cdot\vNabla + \ii g\skvec{\tau}\cdot\skvec{\pi} \\
   {-}\ii\skvec{\sigma}\cdot\vNabla - \ii g\skvec{\tau}\cdot\skvec{\pi} & 0
 } \,.
\end{equation}
%
Rotating the phase of the fermion field,
%
\begin{equation}
 \psi \rightarrow \tilde{\psi} = \eex^{{-}\ii\MN t}\psi \,,
\end{equation}
%
just like we did for the bosonic field in Eq.~\eqref{eq:Phi-phi-a}, we can 
remove the mass term for the upper components:
%
\begin{equation}
 \LL_\psi
 = \tilde{\psi}^\dagger\left(\hat{E} + \hat{O}
 - (\gamma^0-\one)\MN\right)\tilde{\psi} \,.
\label{eq:L-E-O-tilde}
\end{equation}
%
The Foldy-Wouthuysen transformation is now constructed to (approximately) 
decouple the upper from the lower components, \ie, nucleons from their 
antiparticles.  To achieve this, we use a sequence of further unitary 
redefinitions of the fermion field.  The first of these is
%
\begin{equation}
 \tilde\psi \rightarrow \tilde{\psi}' = \eex^{{-}\ii\hat{S}}\tilde{\psi}
 \mathtext{with}
 \hat{S} = {-}\frac{\ii\gamma^0\hat{O}}{2\MN} \,.
\end{equation}
%
Let us consider this transformation up to quadratic order in $1/\MN$.  
Expanding the exponential, we have
%
\begin{equation}
 \tilde{\psi} = \eex^{\ii\hat{S}} \tilde{\psi}'
 = \left(1 + \frac{\gamma^0\hat{O}}{2\MN}
 + \frac{\big(\gamma^0\hat{O}\big)^2}{8\MN^2}
 + \OO\big(1/\MN^3\big)\right) \tilde{\psi}'
\end{equation}
%
and likewise
%
\begin{equation}
 \tilde{\psi}^\dagger = \tilde{\psi}'^\dagger\eex^{{-}\ii\hat{S}} 
 = \tilde{\psi}'^\dagger \left(1 - \frac{\gamma^0\hat{O}}{2\MN}
 + \frac{\big(\gamma^0\hat{O}\big)^2}{8\MN^2}
 + \OO\big(1/\MN^3\big)\right) \,.
\end{equation}
%
Inserting this into Eq.~\eqref{eq:L-E-O-tilde} and collecting contributions up 
to corrections which are $\OO(1/\MN^2)$, we get a number of terms:
%
\begin{subequations}
\begin{equation}
 {-}\frac{\gamma^0\hat{O}\hat{E}}{2\MN} + \frac{\hat{E}\gamma^0\hat{O}}{2\MN}
 = \frac{\gamma^0\commbig{\hat{O}}{\hat{E}}}{2\MN} \,
\end{equation}
%
\begin{equation}
 \frac{\hat{O}\gamma^0\hat{O}}{2\MN}
 - (\gamma^0-\one)\frac{\big(\gamma^0\hat{O}\big)^2}{8\MN}
 - \frac{\gamma^0\hat{O}^2}{2\MN}
 + \frac{\gamma^0\hat{O}}{2\MN}(\gamma^0-\one)\frac{\gamma^0\hat{O}}{2\MN}
 - \frac{\big(\gamma^0\hat{O}\big)^2}{8\MN}(\gamma^0-\one)
 = {-}\frac{\gamma^0\hat{O}^2}{2\MN} \,,
\end{equation}
%
\begin{equation}
 \frac12\gamma^0\hat{O}(\gamma^0-\one) - \frac12(\gamma^0-\one)\gamma^0\hat{O}
 = -\hat{O} \,.
\end{equation}
\end{subequations}
%
Above we have used that
%
\begin{equation}
 \commbig{\gamma_0}{\hat{E}} = 0
 \mathtext{,}
 \acommbig{\gamma_0}{\hat{O}} = 0 \,,
\end{equation}
%
and $(\gamma^0)^2=\one$.  Collecting everything, we get
%
\begin{equation}
 \LL_{\pi N}
 = \tilde{\psi}'^\dagger\left(
 \hat{E} - \frac{\gamma^0\hat{O}^2}{2\MN}
 + \frac{\gamma^0\commbig{\hat{O}}{\hat{E}}}{2\MN}
 - (\gamma^0-\one)\MN\right)\tilde{\psi}'
 + \OO(1/\MN^2) \,.
\label{eq:L-E-O-tilde-prime}
\end{equation}
%
The $\hat{O}^2$ term is even and we see that the original odd operator is 
canceled, but we have generated a new term $\sim\commbig{\hat{O}}{\hat{E}}$ 
If we neglect the interaction and consider
%
\begin{equation}
 \hat{E} = \hat{E}_{\text{free}} = \twodmat{\ii\partial_t & 0 \\ 0 & 
 \ii\partial_t}
 \mathtext{and}
 \hat{O} = \hat{O}_{\text{free}} = \twodmat{
   0 & {-}\ii\vec{\sigma}\cdot\vNabla \\
   {-}\ii\vec{\sigma}\cdot\vNabla & 0
 } \,,
\end{equation}
%
we find that $\commbig{\hat{O}_{\text{free}}}{\hat{E}_{\text{free}}}=0$ (partial 
derivatives commute) and thus the desired decoupling up to $\OO(1/\MN^2)$.  For 
the interacting case, on the other hand, the commutator does not vanish.  We 
see, however, that the new odd contribution is suppressed by a factor $1/\MN$.  
To push it to the next higher order, we need another rotation:
%
%
\begin{subequations}
\begin{equation}
 \tilde\psi' \rightarrow \tilde{\psi}'' = \eex^{{-}\ii\hat{S'}}\tilde{\psi} \,,
\end{equation}
%
with
%
\begin{equation}
 \hat{S'} = {-}\frac{\ii\gamma^0\hat{O}'}{2\MN}
 \mathtext{with}
 \hat{O}' = \frac{\gamma^0\commbig{\hat{O}}{\hat{E}}}{2\MN}
\end{equation}
\label{eq:S-prime}
\end{subequations}
%
After a couple of steps, we arrive at
%
\begin{equation}
 \LL_\psi
 = \tilde{\psi}''^\dagger\left(
 \hat{E} - \frac{\gamma^0\hat{O}^2}{2\MN}
 - (\gamma^0-\one)\MN\right)\tilde{\psi}''
 + \OO(1/\MN^2) \,,
\label{eq:L-E-O-tilde-prime-prime}
\end{equation}
%
\ie, up to $\OO(1/\MN^2)$ there are now no odd terms left and the upper and 
lower components of $\tilde{\psi}''$ are decoupled at this order.

\begin{prob}
\emph{Exercise:}  Carry out the steps that lead from the 
transformation~\eqref{eq:S-prime} to Eq.~\eqref{eq:L-E-O-tilde-prime-prime}.  
Note that it suffices to expand the exponentials up to first order.
\end{prob}

In this Lagrangian, we can now write
%
\begin{equation}
 \tilde{\psi}'' = \twodvec{N \\ n}
\end{equation}
%
and identify the upper (``large'') component $N$---a doublet in both spin and 
isospin space---with the particle and the lower (``small'') component with the 
antiparticle states.  The term $(\gamma^0-\one)\MN$ in 
Eq.~\eqref{eq:L-E-O-tilde-prime-prime} ensures that there is no explicit mass 
term for the field $N$, whereas that for $n$ comes with a factor two, 
corresponding to the Dirac mass gap between particles and antiparticles.  Let 
us now write down the Lagrangian obtained for $N$, omitting the decoupled small 
components:
%
\begin{equation}
 \LL_\psi = N^\dagger\left(
 \ii\partial_t 
 - g\sigma
 - \frac{1}{2\MN} \left[{-}\ii\skvec{\sigma}\cdot\vNabla
 + \ii g\skvec{\tau}\cdot\skvec{\pi}\right]
 \left[{-}\ii\skvec{\sigma}\cdot\vNabla
 - \ii g\skvec{\tau}\cdot\skvec{\pi}\right]
 \right)N + \cdots \,.
\end{equation}
%
To simplify this further, we use that\footnote{Note that 
Eq.~\eqref{eq:sigma-Nabla-simple} is very simple because we have not included a 
coupling of $\psi$ to the electromagnetic field.  If we had done that, the 
$\vNabla$ would be a covariant derivative, $\vec{D}=\vNabla + \ii e \vec{A}$, 
and Eq.~\eqref{eq:sigma-Nabla-simple} would generate, among other terms, the 
magnetic spin coupling $\skvec{\sigma}\cdot\vec{B}$.}
%
\begin{equation}
 (\skvec{\sigma}\cdot\vNabla)(\skvec{\sigma}\cdot\vNabla) = \Laplace
\label{eq:sigma-Nabla-simple}
\end{equation}
%
and, from the product rule,
%
\begin{equation}
 (\skvec{\sigma}\cdot\vNabla)(\skvec{\tau}\cdot\skvec{\pi})
 = \skvec{\sigma}\cdot(\skvec{\tau}\cdot\vNabla\skvec{\pi})
 + (\skvec{\tau}\cdot\skvec{\pi})(\skvec{\sigma}\cdot\vNabla)
 = \sigma^i\tau^a(\partial_i\pi^a) + \tau^a\pi^a \, \sigma^i\partial_i \,.
\end{equation}
%
In the last step we have written out all indices to clarify the meaning of the 
two dot products.  Collecting everything, we find that the 
$(\skvec{\tau}\cdot\skvec{\pi})(\skvec{\sigma}\cdot\vNabla)$ terms cancel out 
and arrive at
%
\begin{multline}
 \LL = N^\dagger\left(
 \ii\partial_t + \frac{\Laplace}{2\MN}\right)N
 - g\,\sigma N^\dagger N
 + N^\dagger\left(
 \frac{g}{2\MN}\skvec{\sigma}\cdot(\skvec{\tau}\cdot\vNabla\skvec{\pi})
 + \frac{g^2}{2\MN}(\skvec{\tau}\cdot\skvec{\pi})^2
 \right)N \\
 + \frac12(\partial^\mu\skvec{\pi}) \cdot (\partial_\mu\skvec{\pi})
 - \frac12\mpi^2 \skvec{\pi}^2
 + \frac12(\partial^\mu\sigma) \cdot (\partial_\mu\sigma)
 - \frac12\msigma^2 \sigma^2 + \cdots \,.
\label{eq:L-pi-N-nonrel}
\end{multline}
%
This includes the expected nonrelativistic kinetic term for the fermion field, 
as well as various interactions with $\sigma$ and $\skvec{\pi}$.  Note that the 
latter two particles are still relativistic and unchanged by the 
Foldy-Wouthuysen transformation, so that we could simply reinstate their 
kinetic terms as in Eq.~\eqref{eq:L-pi-N-PS}.

\subsubsection{Integrating out exchange particles: part II}
\label{sec:EFT-IntOut-2}

With Eq.~\eqref{eq:L-pi-N-PS} we are now also in a convention situation to 
illustrate how we end up with only contact interactions between the 
nonrelativistic fermions if we integrate out the $\sigma$ and $\skvec{\pi}$ 
fields.  Their equations of motion are
%
\begin{equation}
 \left(\dAlem + \msigma^2\right)\sigma = g\,N^\dagger N
\label{eq:sigma-N-EOM-NR}
\end{equation}
%
and
%
\begin{equation}
 \left(\dAlem + \mpi^2\right)\pi^\lambda
 = {-}\frac{g}{2\MN} 
 \vNabla\cdot \left[N^\dagger\skvec{\sigma}\tau^\lambda N\right]
 - \frac{g^2}{\MN}
 N^\dagger (\skvec{\tau}\cdot\skvec{\pi}) \tau^\lambda N \,.
\label{eq:pi-N-EOM-NR}
\end{equation}
%
The $\sigma$ part can be handled exactly as in Sec.~\ref{sec:EFT-IntOut-1}, 
giving a leading four-nucleon contact interaction $\sim g^2/\msigma^2$ plus a 
tower of operators with increasing number of derivatives.  The $\skvec{\pi}$ 
part is more interesting, but also more complicated due to the derivative in 
Eq.~\eqref{eq:pi-N-EOM-NR}.  We thus keep the following discussion rather 
qualitative and leave it as an exercise to work out the details.

In that spirit, we consider only the first term in Eq.~\eqref{eq:pi-N-EOM-NR},
corresponding to a one-$\skvec{\pi}$-exchange operator when substituted back 
into the Lagrangian.  With the propagator $D_\pi(x-y)$ defined in complete 
analogy to Eq.~\eqref{eq:D-chi-KG}, we can write
%
\begin{equation}
 \pi^\lambda(x) = {-}\frac{\ii g}{2\MN} \int\dd^4y \, D_\pi(x-y)
 \, \partial_j^y \left[N^\dagger(y)\,{\sigma^j}\tau^\lambda N(y)\right]
 + \cdots \,,
\end{equation}
%
and thus get
%
\begin{equation}
 \LL_\text{int}
 \sim \left[N^\dagger(x)\,\sigma^i\tau^\lambda N(x)\right]
 \partial_i^x \int\dd^4y \, D_\pi(x-y)
 \, \partial_j^y \left[N^\dagger(y)\,\sigma^j\tau^\lambda N(y)\right]
 + \cdots \,,
\end{equation}
%
where for the time being we omit the prefactor ${g^2}/{(4\MN^2)}$.  We 
integrate by parts to have $\partial_j^y$ act on $D_\pi(x-y)$.  The 
$\partial_i^x$ does this already, so, with all indices written out for clarity:
%
\begin{multline}
 \LL_\text{int} \sim \int\dd^4y \,
 N^\dagger_{\alpha a}(x)\, \idxx{\sigma^i}{\alpha}{\beta}
 \idxx{\tau^\lambda}{a}{b} \, N^{\beta b}(x)
 \left[\partial_x^i \partial_y^j D_\pi(x-y) \right]
 N^\dagger_{\gamma c}(y) \idxx{\sigma^j}{\gamma}{\delta}
 \idxx{\tau^\lambda}{c}{d} \, N^{\delta d}(y)
 + \cdots \,.
\label{eq:L-pi-N-Dpi}
\end{multline}
%
From the definition of the propagator we find that
%
\begin{equation}
 \partial^x_i \partial^y_j D_\pi(x-y)
 = \int\frac{\dd^4p}{(2\pi)^4} (\ii p_i)({-}\ii p_j) \eex^{{-}\ii p(x-y)}
 \frac{\ii}{p^2-\mpi^2 + \ii\eps} 
 = {-}\partial^x_i \partial^x_j D_\pi(x-y) \,,
\label{eq:D-pi-partial}
\end{equation}
%
so the partial derivatives can be written fully symmetric in $i$ and $j$.  The 
various fermion field operators can be rearranged with the help of
%
\begin{multline}
 N^{\beta b}(x) N^\dagger_{\gamma c}(y)
 = {-}\frac{1}{4}\Big[
 (N^\dagger(y) N(x))\,\delta^\beta_\gamma \delta^b_c
 + (N^\dagger(y) \tau^\kappa N(x))
 \,\delta^\beta_\gamma \idxx{\tau^\kappa}{b}{c} \\
 + (N^\dagger(y) \sigma^k N(x))\,\idxx{\sigma^k}{\beta}{\gamma} \delta^b_c
 + (N^\dagger(y)\,\tau^\kappa\sigma^k N(x)) \,
 \idxx{\sigma^k}{\beta}{\gamma}\idxx{\tau^\kappa}{b}{c} \Big] \,.
\label{eq:NN-Fierz}
\end{multline}
%
Using also
%
\begin{subalign}
 \sigma^i\sigma^j &= \delta^{ij}\one + \ii\leviciv^{ijk}\sigma^k \,, \\
 \sigma^i\sigma^k\sigma^j &= \delta^{kj}\sigma^i + \delta^{ki}\sigma^j
 - \delta^{ij}\sigma^k + \ii\leviciv^{ikj}\one \,, \\
  \sigma^i\sigma^j\sigma^i &= {-}\sigma^j \,,
\end{subalign}
%
we get four terms from Eq.~\eqref{eq:L-pi-N-Dpi} decomposed into contributions 
symmetric and antisymmetric in $i$ and $j$, with the latter all vanishing upon 
contraction with $\partial^x_i \partial^x_j$.  The simplest symmetric term 
comes with a $\delta^{ij}$, yielding $\Laplace D_\pi(x-y)$.  To see what this 
generates, we Taylor-expand the fermion fields that depend on $y$ about $x$, 
\eg, $N(y) = N(x) + (y-x)^\mu \partial_\mu N(x) + \cdots$.  This gives as the 
leading piece a combination of four fermion operators all evaluated at $x$, 
times
%
\begin{equation}
 \int\dd^4y \int\frac{\dd^4p}{(2\pi)^4} \eex^{{-}\ii p\cdot(x-y)}
 \frac{\ii{\mathbf{p}}^2}{p^2-\mpi^2 + \ii\eps} \\
 = \int\dd^4y \int\frac{\dd^3p}{(3\pi)^4} \int\frac{\dd p_0}{2\pi} \,
 \eex^{{-}\ii p(x-y)}
 \frac{\ii{\mathbf{p}}^2}{p_0^2 - {\mathbf{p}}^2 - \mpi^2 + \ii\eps} \,.
\end{equation}
%
The integral over $p_0$ can be solved via contour integration.  Defining
$\omega_{\mathbf{p}} = \sqrt{{\mathbf{p}}^2+\mpi^2}$, we get
%
\begin{equation}
 \int\frac{\dd p_0}{2\pi} \, \eex^{{-}\ii p_0(x_0-y_0)}
 \frac{\ii}{p_0^2 - {\mathbf{p}}^2 - \mpi^2 + \ii\eps}
 = \frac{\eex^{-\ii\omega'_{\mathbf{p}}\abs{x_0-y_0}}}{2\omega'_{\mathbf{p}}}
 \mathtext{with}
 \omega'_{\mathbf{p}} = \omega_{\mathbf{p}} - \frac{\ii\eps}{2\omega_{\mathbf{p}}}
 \equiv \omega_{\mathbf{p}} - \ii\eps' \,.
\end{equation}
%
It is important here to keep track of the small imaginary part, as it allows us 
to write
%
\begin{equation}
 \int_{{-}\infty}^\infty \dd y_0 \,
 \frac{\eex^{-\ii\omega_{\mathbf{p}}'\abs{x_0-y_0}}}{2\omega'_{\mathbf{p}}}
 = \int_{0}^\infty \dd y_0
 \frac{\eex^{-\ii\omega'_{\mathbf{p}}\abs{y_0}}}{\omega'_{\mathbf{p}}}
 = {-}\ii / {(\omega'_{\mathbf{p}})^2} \,.
\end{equation}
%
Collecting the results up to this point, we arrive 
at
%
\begin{equation}
 \int\dd^4y \int\frac{\dd^4p}{(2\pi)^4}
 \, \eex^{{-}\ii p(x-y)}
 \dfrac{\ii{\mathbf{p}}^2}{\strut p^2 - \mpi^2 + \ii\eps}
 = {-}\!\int\dd^3y \int\frac{\dd^3p}{(2\pi)^3}
 \, \eex^{\ii {\mathbf{p}}\cdot(\vx-\vy)}
 \frac{{\mathbf{p}}^2}{\strut{\mathbf{p}}^2+\mpi^2} \,.
\end{equation}
%
We finally obtain the desired contact interaction by expanding
%
\begin{equation}
 \frac{{\mathbf{p}}^2}{\strut{\mathbf{p}}^2+\mpi^2}
 = \frac{{\mathbf{p}}^2}{\mpi^2} \left(1 - \frac{{\mathbf{p}}^2}{\mpi^2} + \cdots\right) \,,
\end{equation}
%
with a leading term $\sim{\mathbf{p}}^2$, generating a contact interaction $\sim 
(N^\dagger N)\Laplace(N^\dagger N)$.  This is of course not surprising: after 
all, the original interaction term in Eq.~\eqref{eq:L-pi-N-nonrel} generating 
the contact operator had a single derivative $\vNabla$.  Considering other 
terms coming from Eq.~\eqref{eq:NN-Fierz}, one can also find operators like 
$(N^\dagger \skvec{\sigma}\cdot\vNabla N)(N^\dagger \skvec{\sigma}\cdot\vNabla 
N)$, and it is a useful exercise to work this out in detail.  But already from 
our qualitative discussion here we can infer that the resulting effective 
theory is an expansion in ${\mathbf{p}}^2/\mpi^2$, \ie, its range of validity is 
determined by three-momenta---rather than the energies---being small 
compared to $\mpi$.\footnote{This is assuming $\mpi<\msigma$.}

\subsubsection{The Schr\"odinger field}

We conclude this section by looking at the non-relativistic field theory from
a more general perspective, establishing its close connection to the ``second
quantized'' approach to (many-body) quantum mechanics that is used 
in several later chapters of this volume.

Recall from the beginning of this section that the 
Lagrangian~\eqref{eq:L-phi-free-nonrel} for the free Schr\"odinger field 
$\phi$ is
%
\begin{equation}
 \LL_{\phi,\text{free}} = \phi^\dagger
 \left(\ii\partial_t + \frac{\Laplace}{2m}\right)\phi \,.
\label{eq:L-phi-free-nonrel-hat}
\end{equation}
%
This trivially gives the equation of motion
%
\begin{equation}
 \left(\ii\partial_t + \frac{\Laplace}{2m}\right)\phi = 0 \,,
\label{eq:phi-nonrel-free-EOM}
\end{equation}
%
which is formally the same as the free Schr\"odinger equation.  However, recall 
that $\phi$ here is a field \emph{operator}, \ie, $\phi(x)$ creates a particle 
at $x=(t,\vx)$ from the vacuum, so to really get an ordinary Schr\"odinger 
equation, we have to act with both sides of Eq.~\eqref{eq:phi-nonrel-free-EOM} 
on $\ket{0}$, and define the quantum-mechanical one-body state
%
\begin{equation}
 \ket{\phi(t,\vx)} = \phi(t,\vx)\ket{0} \,.
\end{equation}
%
If we add to Eq.~\eqref{eq:L-phi-free-nonrel-hat} a term 
$V(x)\phi^\dagger(x)\phi(x)$, we obtain the Schr\"odinger equation for a 
particle in a potential $V(x)$.  Exactly as for a relativistic field we can 
define the propagator
%
\begin{equation}
 D_\phi(x-y)
 = \int\frac{\dd^4q}{(2\pi)^4} \, \eex^{{-}\ii p(x-y)}
 \frac{\ii}{p_0-\frac{{\mathbf{p}}^2}{2m} + \ii\eps} \,,
\label{eq:D-phi}
\end{equation}
%
satisfying
%
\begin{equation}
 \left(\ii\partial_t + \frac{\Laplace}{2m}\right)D_\phi(x-y)
 = {-}\ii \delta^{(4)}(x-y) \,.
\end{equation}
%
Up to a conventional factor $\ii$, this is precisely the (retarded) Green's 
function\footnote{Note that in the nonrelativistic case there is no ``Feynman 
propagator.''  Particles and particles are decoupled, and the denominator in 
Eq.~\eqref{eq:D-phi} only has a single pole at $p_0=\vec{{\mathbf{p}}^2}/(2m)-\ii\eps$.  
Flipping the sign of the $\ii\eps$ term gives the advanced Green's function.} 
familiar, for example, from non-relativistic scattering theory (then typically 
denoted $G_0$).  This will appear again when the Lippmann--Schwinger equation 
is derived using the field-theory language in Sec.~\ref{sec:EFT-Bosons}.

While it is nice and reassuring that we can go back to simple quantum mechanics 
from the one-body Schr\"odinger Lagrangian discussed so far, this feature is 
not very relevant in practice.  We can, however, straightforwardly generalize 
it to the many case.  To that end, consider a Lagrangian that includes a 
two-body interaction, written in terms of a general non-local 
potential:\footnote{A static (time-independent) potential, as it is more common 
in quantum mechanics, would be a function only of $\vx$ and $\vy$, and all 
fields in the interaction term would be evaluated at the same time $t$.}
%
\begin{equation}
 \LL_{\phi,\text{2-body}}(x) = \phi^\dagger(x)
 \left(\ii\partial_t + \frac{\Laplace}{2m}\right)\phi(x)
 + \int\dd^4 y \, \phi^\dagger(x) \phi(x)
 \, V(x,y) \, \phi^\dagger(y) \phi(y) \,.
\label{eq:L-phi-V2-nonrel-hat}
\end{equation}
%
Note that this has exactly the structure that we found when we integrated out 
particles in the preceding sections, before expanding the propagators to get 
simple contact interactions.  Such a Lagrangian (possibly including also 
higher-body forces) is a convenient starting point for example for many-body 
perturbation theory used to study quantum systems at finite density.

Coming back to effective field theories, we stress that these are \emph{not} 
defined by putting a given potential into a Lagrangian; in doing that, 
one merely gets a model written in a convenient way.  The EFT instead makes no 
assumptions on the interaction (besides symmetry constraints).  It is thus much 
more general and not a model, but to be predictive it requires a number of 
\apriori unknown parameters to be fixed and its various terms to be ordered 
systematically.  It is this that we turn to next.

\subsection{Symmetries and power counting}

So far, we have discussed how to obtain effective low-energy Lagrangians by 
integrating out "heavy" degrees of freedom, leaving only those that we want  
to describe at low energies or rather, as we showed explicitly with the 
pseudoscalar pion-nucleon model, low momenta.  We found the contact 
interactions generated this way to come with the integrated-out particle's mass 
in the denominator, and with an increasing number of derivatives as we keep 
more and more terms from the expansion.  These derivatives will turn into 
powers of momentum, which is a small scale for external states.  We furthermore 
showed how a nonrelativistic reduction generates a chain of operators with an 
increasing power of the particle's mass in the denominator, thus also giving a 
hierarchy of terms that eventually restore the original theory's relativistic 
structure with coupling between particles and antiparticles.

From these procedures it is clear that the terms in the effective Lagrangian 
should be ordered in a natural way, with the most important ones being those 
with the least number of large mass scales in the denominator and the least 
number of derivatives in the numerator.  It is also clear that they are 
restricted in their structure.  For example, if we start with a 
Lorentz-invariant relativistic theory, after the nonrelativistic reduction we 
will only get terms that are invariant under ``small'' Lorentz boosts.  More 
precisely, the nonrelativistic operators should be invariant under Galilean 
transformations (assuming the original theory had rotational invariance, this 
simply gets inherited by the effective one), and the form of so-called 
``relativistic corrections'' is determined by the expansion of the dispersion 
relation:
%
\begin{equation}
 E^2 = m^2 + p^2
 \implies E = m + \frac{p^2}{2m} - \frac{p^4}{8m^3} + \cdots \,.
\end{equation}

We now turn to discussing the bottom-up approach guided by these principles.  
To that end, consider the effective Lagrangian for a nonrelativistic bosonic 
field with contact interactions:
%
\begin{multline}
 \LL = \phi^\dagger \left(\ii\partial_t + \frac{\Laplace}{2m}\right)\phi
 + \phi^\dagger \frac{\vNabla^4}{8m^3} \phi + \cdots \\
 + g_{2}^{(0)} (\phi^\dagger \phi)^2
 + g_{2}^{(2s/p)} \left(
  (\phi^\dagger\vNablaLR\phi)^2
   - (\phi^\dagger\phi)(\phi^\dagger(\vNablaLR)^2\phi)
   \mp 2 (\phi^\dagger\phi)\vNabla^2(\phi^\dagger\phi)
  \right) + \cdots \\
 + g_{3}^{(0)} (\phi^\dagger \phi)^3 + \cdots \,.
\label{eq:L-phi-generic}
\end{multline}
%
Here we have used the definition
%
\begin{equation}
 f \vNablaLR g = f(\vNabla)g - (\vNabla f)g
\end{equation}
%
and conveniently separated the two-body terms with two derivatives into 
those which contribute to S-wave ($\sim g_{2}^{(2s)}$) and P-wave ($\sim 
g_{2}^{(2p)}$) interactions, respectively.  One can of course choose different 
linear combinations, but a separation by partial waves is typically a good 
choice for systems with rotational invariance.  It is a useful exercise to work 
out how the structure for the derivative interactions gives the desired result, 
working in momentum space and considering contractions with external in and out 
states that have center-of-mass momenta $\pm\vk^2$ and $\pm{\mathbf{p}}/2$, respectively.
The structure of the individual terms is determined by the requirement of 
Galilean invariance,\footnote{See for example Ref.~\cite{Hagen:2014}, 
Sec.~2.1.1 for a rigorous discussion of the required transformation 
properties.}, and the EFT paradigm tells us to write down all possible terms 
with a given number of derivatives (with odd numbers excluded by parity 
invariance).

\subsubsection{The breakdown scale}

As mentioned in the introduction, the most important requirement to construct 
an EFT is the identification of---at least two, but possibly more---separated 
scales, ratios of which are used to extract a small expansion parameter.  The 
better the scale separation, the smaller this parameter becomes, and 
consequently the better the more precise (and, provided all contributions have 
been identified correctly, accurate) the theory becomes at any given order in 
the expansion.  In the simplest case, there is one low scale $Q$ 
associated with the typical momentum of the physical system that we want to 
describe, and a single large scale $\Mhi$, the ``breakdown scale'' associated 
with the physics that our EFT does not take into account---in other words: 
resolve---explicitly.  This is exactly the situation that we constructed when we 
integrated out exchange particles from a given theory in 
Secs.~\ref{sec:EFT-IntOut-1} and~\ref{sec:EFT-IntOut-2}.  By construction, the 
EFT is not appropriate to describe processes with momenta of the order of or 
large than the breakdown scale.  To emphasize this meaning, it is sometimes also 
denoted by the letter $\Lambda$ (with or possibly without some qualifying 
subscript).\footnote{We alert the reader that in the literature this is 
sometimes referred to as the ``cutoff of the EFT.''  We do not use that language 
to avoid confusion with an (arbitrary) momentum cutoff introduced to regularize 
divergent loop integrals (discussed .}

As already mentioned, integrating out degrees of freedom from a given more 
fundamental theory will naturally give a breakdown scale set by that particle's 
mass.  But it can also be something more general.  For example, although in the 
situations discussed here so far the particles we were ultimately interested in 
were already present as degrees of freedom in the original theory, such a 
scenario is merely a special case.  The first step in writing down an effective 
field theory is to identify what the appropriate---literally: 
effective---degrees of freedom are for the processes one wants to describe, and 
they can be different from those of the fundamental theory.  This is exactly 
the case in nuclear physics: while the degrees of freedom in quantum 
chromodynamics (QCD) are quarks and gluons, describing the binding of nuclei 
with these is, although possible with state-of-the-art lattice QCD 
calculations, largely inefficient to say the least.  It is much more economical 
to work with nucleons directly as degrees of freedom, as done in most chapters 
of this volume, because a detailed knowledge of the internal structure of 
protons and neutrons is not necessary to describe their binding into nuclei; it 
is only resolved at much higher energies, for example in deep inelastic 
scattering.  The reason for this is color confinement: the low-energy degrees 
of freedom of QCD are not quarks and gluons, but color-neutral hadrons.  Chiral 
effective field theory, which we will come back to in 
Sec.~\ref{sec:EFT-Chiral}, is designed to work at momenta of the order of pion 
mass, breaking down at the scale of chiral-symmetry breaking (estimated to be 
roughly a \GeV, but possibly lower).

Other examples are halo EFT, constructed to describe nuclear systems that have 
the structure of a few nucleons weakly bound to a tight core, which can then 
effective be treated as a structureless particle.  Clearly, such a theory will 
break down at momenta large enough to probe the core's internal structure.  
Similarly, one can construct an effective theory for systems of ultracold 
atomic gases, the constituents of which can be treated as pointlike degrees of 
freedom without using QED to describe their individual structure, and much less 
QCD to describe their atomic nuclei.

Whatever the breakdown scale is, once identified it can be used to 
systematically order terms in the effective Lagrangian by powers of 
$Q/\Mhi \ll 1$, and we now turn to discussing how this ordering can be set 
up.

\subsubsection{Na\"ive dimensional analysis}

In our units with $\hbar = c = 1$, the action
%
\begin{equation}
 S = \int \dd^4 x\, \LL(x)
\end{equation}
%
has to be a dimensionless quantity.  This, in turn, fixes the dimensions for 
the individual building blocks in the Lagrangian.  In a relativistic theory,  
mass and energy are equivalent and one would simply express everything in terms 
of a generic mass dimension.  For our nonrelativistic framework, on the other 
hand, energies are \emph{kinetic} energies because the time dependence 
associated with the rest mass has been absorbed into the field (\cf 
Secs.~\ref{sec:EFT-NonRelBos} and~\ref{sec:EFT-NonRelFerm}).  This implies 
that energy and mass scales---as well as time and space---should be counted 
separately.\footnote{This separation would be quite clear if we had not set 
$c=1$, which would in fact be more appropriate for a nonrelativistic system.  
The reason we still do it that it allows us to still energies and momenta in the 
same units, \eg, in $\MeV$, following the standard convenient in nuclear 
physics.}  In fact, it is more natural to consider powers of momentum.  To 
understand what this means, let us start with the kinetic term in 
Eq.~\eqref{eq:L-phi-generic}: $[\vNabla^2/(2m)] = 
\text{momentum}^2/\text{mass}$.  The time derivative has to scale in the same 
way, implying that for time itself we have $[t] = 
\text{mass}/\text{momentum}^2$, whereas $[x] = \text{momentum}^{-1}$.  
Consequently, the integration measure scales like $[\dd^4 x = \dd t\,\dd^3x] = 
\text{mass}/\text{momentum}^5$ (to compare, in the relativistic theory one 
would simply count $[\dd^4 x] = \text{mass}^{-4} = \text{energy}^{-4}$).

Since the dimension of $\LL$ has to cancel that of the measure to give a 
dimensionless action,we can now infer that our field has to satisfy $[\phi] = 
\text{momentum}^{3/2}$, \ie, even though it is a scalar field it scales with a 
fractional dimension (recall that in the relativistic case a scalar would have 
dimension $\text{energy}^1$).  Knowing the scaling of the field and the 
measure, we can now proceed and deduce that of the various coupling constants.

The basic idea is very simple: each term (operator) in the 
Lagrangian~\eqref{eq:L-phi-generic} has $2n$ fields and $2m$ derivatives, 
giving it a total dimension of $\text{momentum}^{3n+2m}$.  For example, the 
$(\phi^\dagger \phi)^2$ term with $2n=4$ and $m=0$ has dimension 
$\text{momentum}^6$.  Hence, to get the correct overall dimension 
$\text{momentum}^5/\text{mass}$ for $\LL$, the coupling constant $g_{2}^{(0)}$ 
has to be $\sim 1/(\text{momentum}\times\text{mass})$.  Since it is supposed to 
describe unresolved short-distance details, the momentum scale in the 
denominator is should be the breakdown scale, whereas the mass scale, which 
as we mentioned is a feature of the nonrelativistic framework and common to 
all operators, is simply associated with $m_\phi$.  Of course, counting a 
single operator does not tell us much: it is the relative order of terms that 
matters, so we proceed to the $g_2^{(2)}$ interactions.  These all come with 
two derivatives, which are associated with the external (small) momentum scale 
$Q$. Hence, we have $2n=4$ and $2m=2$, and we need to compensate the two 
additional powers of momentum in the numerator with two more powers of 
$\Mhi$ in the denominator, finding that the $g_2^{(2)}$ interactions are 
down compared to the $g_2^{(0)}$ term by a factor $(Q/\Mhi)^2$.  This is 
exactly in line with our picture of the contact terms gradually building up the 
an unresolved particle exchange through a derivative expansion.  For 
higher-body interactions, it is the larger number of fields that gives a 
suppression by inverse powers of $\Mhi$ compared to operators with fewer 
fields.

This kind of analysis can be much improved if something is known about which 
unresolved physics is supposed to be represented by which operator, and it is 
generally more complex if the theory involves different fields.  For example, in 
the EFT for halo nuclei there are contact interactions associated with 
unresolved pion exchange, as well as those systematically accounting for the 
internal structure of the core field.  Instead of merely putting generic powers 
of $\Mhi$ in every denominator, it can be necessary to keep track of several 
high scales separately to figure out the ordering of terms.  Also, it is 
possible that the external momentum is not the only relevant low-momentum scale 
in the problem.

This rather abstract discussion will become clearer when we finally discuss 
concrete EFTs in the following sections~\ref{sec:EFT-Bosons} 
and~\ref{sec:EFT-Nucleons}.  In that context, we will use the scaling of 
the various terms in the Lagrangian to power-count diagrams as a 
whole, \ie to estimate the size of individual contributions composed of 
vertices and loops to a given physical amplitude of interest.  We will then 
also discuss how the actual so-called scaling dimension of a field in the 
Lagrangian can turn out to deviate from what we estimated here based purely on 
dimensional grounds.

\subsubsection{Fine tuning}
\label{sec:EFT-FineTuning}

In connection with the previous comment is another point worth stressing already 
here: n\"ive dimensional analysis resides at the beginning of EFT wisdom, not at 
the end, and in quite a few cases it turns out to be exactly what the name says: 
na\"ive.  In other words, the actual scaling of a coupling constant can be quite 
different from what one would infer by counting dimensions, a scenario that is 
commonly referred to as ``fine tuning.''  To understand why that is consider, 
for example, our bosonic toy model from Sec.~\ref{sec:EFT-IntOut-1}, but now 
assume that there already is a four-$\phi$ contact interaction present prior to 
integrating out the $\chi$ field:
%
\begin{equation}
 \LL = {-}\phi^\dagger\left(\dAlem + m_\phi^2\right)\phi
 - \chi^\dagger\left(\dAlem + m_\chi^2\right)\chi
 + g\left(\phi^\dagger\phi^\dagger\chi + \hc\right)
 + h (\phi^\dagger\phi)^2 \,.
\label{eq:L-phi-chi-rel-contact}
\end{equation}
%
This could, for example, come from unknown (or integrated out) short-distance 
physics at a yet higher scale.  When we now integrate out the $\chi$, the 
generated non-derivative contact term will combine with the existing one, 
giving a single operator in the effective low-energy Lagrangian (recall that on 
dimensional grounds $h$ has to have dimensions of inverse mass squared):
%
\begin{equation}
 \LL = {-}\phi^\dagger\left(\dAlem + m_\phi^2\right)\phi
 + \left(h-\frac{g^2}{m_\chi^2}\right) (\phi^\dagger\phi)^2 + \cdots \,.
\label{eq:L-phi-chi-rel-contact-FT}
\end{equation}
%
Now suppose we had started in the bottom-up approach and simply written down 
the four-$\phi$ contact operator with some coefficient $c$ to be determined.  
According to NDA, we would assume that its scale is set by two powers of the 
breakdown scale in the denominator, and assuming we actually know about the 
more fundamental theory, we might have estimated that breakdown scale to be of 
the order $m_\chi$.  From Eq.~\eqref{eq:L-phi-chi-rel-contact-FT} we see 
that depending on what values $g$ and $h$ take in the underlying theory, the 
actual size of $c$ might deviate strongly from the na\"ive expectation, and it 
could even be set by a low-energy scale of the effective theory.  But for this 
to happen, there would have to be a delicate cancellation between $h$ and 
$g^2/m_\chi^2$, which is typically deemed unlikely given the \apriori vast 
range of possible values these parameters could take; thus the term ``fine 
tuning.''  The fact that coupling constants are in fact not simple fixed 
numbers but get renormalized by loop effects (\ie, depend on a regularization 
scale with a behavior determined by the renormalization group) justifies this 
language even more.

\subsubsection{Loops and renormalization}
\label{sec:Loops}

It is indeed high time we talk about loops.  Our considerations in this section 
so far have been limited to tree level, which is always only a first 
approximation in a quantum field theory.  In a perturbative theory, loop 
contributions from virtual intermediate states are added to improve the 
accuracy of the result.  To treat a nonperturbative system such as a bound 
nucleus, on the other hand, they are absolutely crucial: recall that any finite 
sum of diagrams in perturbation can never produce a bound state (for example, 
think about poles in the S-matrix, which cannot be generated through a finite 
sum of terms).  In the field-theory language, this means that an infinite 
number of diagrams with increasing number of loops has to be summed to get the 
amplitude with the desired physical properties.

This situation is in fact familiar already from the Schr\"odinger equation 
written in the form
%
\begin{equation}
 \ket{\psi} = \hat{G}_0(E) \hat{V} \ket{\psi} \mathtext{,}
 \hat{G}_0 = \hat{G}_0(E) = (E - \hat{H}_0)^{-1} \,,
\end{equation}
%
which can be iterated to get $\ket{\psi} = \hat{G}_0\hat{V}
\hat{G}_0\hat{V} \ket{\psi} = \cdots$.  When these operators are written 
out in momentum space, each propagator $\hat{G}_0$ corresponds to a loop.
More closely related to the amplitude written down in a nonrelativistic field 
theory, this exercise can be repeated with the Lippmann-Schwinger equation for 
the T-matrix and its formal solution, the infinite Born series.  Exactly this 
will be recovered in Sec.~\ref{sec:EFT-Bosons}.

Of course, even in a nonperturbative theory we do not expect that \emph{all} 
loop diagrams should be summed up to infinity.  Generally, we want the power 
counting to tell us how to estimate the contribution from a given diagram, 
including loop diagrams.  To do that, we need to know not only what which 
factors we pick up from vertices, but also need an estimate for the integration 
measure $\dd^4q = \dd q_0\,\dd^3 q$.  Any loop diagram contributing to an 
amplitude with external momenta of the order $Q$ will have this scale running 
through it a whole.  It is thus natural to count the contribution from the 
three-momentum as $\dd^3 q \sim Q^3$ and, recalling that in the nonrelativistic 
theory $q_0$ is a kinetic energy, $\dd q_0 \sim Q^2/m$.  For each Schr\"odinger 
propagator we get, conversely, a factor $m/Q^2$, as can be seen from 
Eq.~\eqref{eq:D-phi}.  These simple rules combined with those for the vertices 
give an estimate for any diagram in the theory determined by 
Eq.~\eqref{eq:L-phi-generic}.

What this discussion does not cover is the fact that loops in a quantum field 
theory can be---and mostly are---divergent.  Compared to the loops one gets 
from integrating the Schr\"odinger or Lippmann-Schwinger equation in quantum 
mechanics with a potential $\hat{V}$, which are typically all finite, this is 
different in the EFT simply because our delta-function (contact) interactions 
are too singular to make direct sense beyond tree level.  Of course, this is no 
different than in any other quantum field theory, and it just means that 
divergent loops have to be regularized (for example, by imposing a momentum 
cutoff or with dimensional regularization), and then suitable renormalization 
conditions have to be imposed to fix the various coupling constants in the 
effective Lagrangian.  These then become functions of the renormalization scale, 
with a behavior governed by the renormalization group (RG).  In 
Sec.~\ref{sec:EFT-Bosons} this will be discussed in detail for a bosonic EFT 
that describes, for example, ultracold atomic systems.

\paragraph{The cutoff}

What the regularization of loop integrals does is most transparent with a 
momentum cutoff.  We denote this by $\Lambda$ and stress again that it has to 
be distinguished from the EFT breakdown scale $\Mhi$.  The latter determines 
the scale beyond which we know our EFT not to be valid.  In other words, 
short-range dynamics corresponding to momenta larger than $\Mhi$ is, in 
general, not correctly described by the EFT.  Yet from loop integrals we get 
contributions from states up to the UV cutoff $\Lambda$.  Renormalization means 
to adjust the coupling constants in such a way that they compensate the wrong 
high-momentum loop contributions in such a way that the physics the EFT is 
supposed to describe comes out correctly.  For momenta up to $\Mhi$, we 
trust the EFT, so it makes sense to keep such states in loops.  Hence, one 
should typically choose $\Lambda > \Mhi$.  Choosing it lower that the 
breakdown scale is possible, but this can induce corrections of the order 
$Q/\Lambda > Q/\Mhi$, which is not desirable for the power counting.  In the 
renormalized EFT, any cutoff in the interval $[\Mhi,\infty)$ is thus an 
equally good choice---it does not have to be ``taken to infinity.''  Instead, 
that phrase should be understood to mean adjusting the couplings at any given 
finite cutoff.  If this procedure is carried out numerically, it can be 
desirable to keep the cutoff small, but one has to make sure that in principle 
in \emph{can} be varied arbitrarily.

\subsection{Matching}

The determination of the couplings (``low-energy constants'') in the effective 
Lagrangian is done by expressing a given physical quantity (\eg, a scattering 
amplitude or related ) in terms of the couplings and then adjusting them to 
reproduce a known result.  This can be done using experimental input or, when 
working top-down, by calculating the same amplitude in the more fundamental 
theory.  Generally, this procedure is referred to as ``matching.''  At tree 
level, this is again exactly what we did by integrating out particles and found 
the coefficients of the generated contact interactions in terms of the original 
coupling and mass denominators.  Once loop diagrams are involved, the process 
becomes somewhat more complicated because (a) one has to make sure to use 
compatible regularization schemes and renormalization scales and (b) loop 
diagrams with lower-order vertices typically mix with higher-order tree-level 
diagrams.  The latter is a general feature of combined loop and derivative 
expansions and is thus also important when matching to experimental input.  
While these comments may sound a bit cryptic here, they will become much clearer 
in the next section when we finally work with a concrete EFT.

%\input{chap4_bosons}
\section{Effective field theory for strongly interacting bosons}
\label{sec:EFT-Bosons}

We will now use the insights from the previous sections to
construct a local effective field theory for 
identical, spinless bosons with short-range S-wave 
interactions.\footnote{See Ref.~\cite{Braaten:2004rn} for a similar 
discussion with a focus on applications in ultracold atoms.}
For the treatment of higher partial wave interactions the 
reader is referred to the 
literature~\cite{Beane:2000fx,Bertulani:2002sz,Bedaque:2003wa}. 
The most general effective Lagrangian consistent with 
Galilei invariance can be written as
%----------------------
\begin{equation}
 {\mathcal L} = \phi^\dagger \left(\ii\partial_t + \frac{\nabla^2}{2m} 
 \right)\phi - \frac{C_0}{4} \left(\phi^\dagger \phi
 \right)^2 - \frac{C_2}{4} \left(\nabla (\phi^\dagger \phi)\right)^2
 + \frac{D_0}{36}  \left(\phi^\dagger \phi \right)^3+\cdots \,.
\label{L-2body}
\end{equation}
%----------------------
where $m$ is the mass of the particles and
the ellipses denote higher-derivative and/or higher-body
interactions. The leading two- and three-body interactions are 
explicitly written out. The scaling of the coefficients $C_0$, $C_2$,
$D_0$, $\ldots$ depends on the scales of the considered
system. Two explicit examples, corresponding natural and unnaturally
large scattering length, are discussed below.

\subsection{EFT for short-range interactions}

We start by considering natural system where all interactions 
are characterized by only one mass scale $\Mhi$ that we identify with the 
formal breakdown scale of the EFT introduced in the previous section.  We will 
see below that this is indeed justified.  Since the nonrelativistic boson fields 
have dimension 3/2, the coupling constants must scale
as
%----------------------
\begin{equation}
 C_0 \sim \frac{1}{m\Mhi}\,,\quad C_2 \sim \frac{1}{m\Mhi^3}\,,
 \quad\text{and}\quad
 D_0 \sim \frac{1}{m\Mhi^4}\,,
\label{eq:cscale}
\end{equation}
%----------------------
such that higher dimension operators are strongly suppressed for small 
momenta $k \ll \Mhi$.\footnote{Note that coupling constants scale 
with the particle mass as 
$1/m$ in nonrelativistic theories. This can be seen by rescaling all 
energies as $q_0 \to \tilde{q_0}/m$ and all time coordinates as 
$t\to \tilde{t}m$, so that dimensionful quantities are measured in 
units of momentum. Demanding that the action is independent of $m$,
it follows that the coupling constants must scale as $1/m$.}
We first focus on the two-body system and
calculate the contribution of the interaction terms in Eq.~(\ref{L-2body}) 
to the scattering amplitude of two particles in perturbation theory. 
After renormalization, the result reproduces
the low-energy expansion of the 
scattering amplitude for particles with relative momentum $k$ and 
total energy $E=k^2/m$:
%----------------------
\begin{equation}
 T_2(E)=\frac{8\pi}{m}\frac{1}{k\cot\delta_0(k)-\ii k}=
 {-}\frac{8\pi a}{m}\left(1-\ii ak+ (ar_e/2-a^2)k^2 +{\mathcal O}(k^3)\right)\,,
\label{eq-ere}
\end{equation}
%----------------------
where the effective range expansion for short-range interactions
$k\cot\delta_0(k)={-}1/a+r_e k^2/2 +{\mathcal O}(k^4)$ has been used.

Since all coefficients of the effective Lagrangian are natural (scaling with 
inverse powers of $\Mhi$), it is sufficient to count the powers of small momenta 
$Q$ in scattering amplitudes to determine the scaling of the amplitudes with 
$\Mlo$. The correct dimensions are made up with appropriate factors of $\Mhi$ 
contained in the coupling constants (\cf~Eq.~(\ref{eq:cscale})).
For a general two-body amplitude with $L$ loops and $V_{2i}$ interaction
vertices with $2i$ derivatives, we thus have $T_2 \sim Q^\nu$ where the
power $\nu$ is given by
%----------------------
\begin{equation}
  \nu=3L+2+\sum_i (2i-2) V_{2i} \geq 0\,.
\label{nu-pert}
\end{equation}
%----------------------
Here we have used that loop integrations contribute a factor $k^5$
and propagators a factor $k^{-2}$ in nonrelativistic theories.
The values of the coupling constants $C_0$ and $C_2$ can be determined by 
matching to Eq.~(\ref{eq-ere}). In the lowest two orders only $C_0$ 
contributes.

\begin{prob}
{\emph Exercise:}  Derive Eq.~(\ref{nu-pert}) using the topological
identity for Feynman diagrams:
%----------------------
\begin{equation}
 L=I-V+1\,,
\end{equation}
%----------------------
with $L$, $V$, and $I$ the total number of loops, vertices, and internal lines
respectively. 
\end{prob}

The contact interactions in Eq.~(\ref{L-2body}) are ill-defined unless an
ultraviolet cutoff is imposed on the momenta in loop diagrams.
This can be seen by writing down the off-shell amplitude for two-body
scattering at second order in perturbation theory:
%----------------------
\begin{equation}
 T_2 (E) \approx {-}C_0 - \frac\ii2
 C_0^2 \int \frac{\dd^3q}{(2 \pi)^3}
 \int\frac{\dd q_0}{2 \pi} \frac1{q_0 - q^2 /2m + \ii \epsilon}
 \frac1{E - q_0 - q^2 /2m + \ii\epsilon} + \cdots \,.
\nonumber\\
\label{A-pert}
\end{equation}
%----------------------
The two terms correspond to the first two diagrams in 
Fig.~\ref{fig:amp2}. 
%%%%%%%%%%%%%%%%%%%%%%%%%%%%%%%%%%%%%%%%%%%%%%%%%%
\begin{figure}[htb]
\bigskip
\centerline{\includegraphics*[width=11cm,angle=0]{fig_2bdypert.pdf}}
\medskip
\caption
{Diagrammatic expression for the two-body scattering amplitude $T_2$.
The circle (square) denotes a $C_0$ ($C_2$) interaction, respectively.}
\label{fig:amp2}
\end{figure}
%%%%%%%%%%%%%%%%%%%%%%%%%%%%%%%%%%%%%%%%%%%%%%%%%%
The intermediate lines have momenta $\pm {\bm q}$.
The integral over $q_0$ in Eq.~(\ref{A-pert}) is easily
evaluated using contour integration:
%----------------------
\begin{equation}
 T_2 (E) \approx {-}C_0 - \frac12 C_0^2 
 \int \frac{\dd^3q}{(2 \pi)^3} \frac1{E - q^2/m +\ii \epsilon} + \cdots \,.
\end{equation}
%----------------------
The integral over ${\bm q}$ diverges. It can be regularized by imposing
an ultraviolet
cutoff $|{\bm q}| < \Lambda$. Taking the limit $\Lambda \gg |E|^{1/2}$,
the amplitude reduces to \footnote{
If the calculation was carried out in a
frame in which the total momentum of the two scattering particles was
nonzero, the simple cutoff $|{\bm q}| < \Lambda$ would give a result that does
not respect Galilean invariance. To obtain a Galilean-invariant result 
requires either using a more sophisticated cutoff or else imposing
the cutoff $|{\bm q}| < \Lambda$ only after an
appropriate shift in the integration variable ${\bm q}$.}
%----------------------
\begin{equation}
 T_2 (E) \approx - C_0 + \frac{mC_0^2}{4 \pi^2} 
 \left(\Lambda - \frac{\pi}{2} \sqrt {-mE -\ii \epsilon} \right) + \cdots \,.
\label{amp2-2nd}
\end{equation}
%----------------------

The dependence on the ultraviolet cutoff $\Lambda$ can be consistently
eliminated by a perturbative renormalization procedure. A simple choice
is to eliminate the parameter $C_0$ in favor of the scattering length
$a$, which is given by Eq.~(\ref{eq-ere}):
%----------------------
\begin{equation}
 a \approx \frac{mC_0}{8 \pi} \left( 1- \frac{m C_0 \Lambda}{4\pi^2}
 + \cdots\right) \,.
\end{equation}
%----------------------
Inverting this expression to obtain $C_0$ as a function of $a$ we obtain
%----------------------
\begin{equation}
 C_0  \approx \frac{8 \pi a}{m} \left( 1+ \frac{2 a \Lambda}{\pi}
 +  \cdots \right) \,,
\end{equation}
%----------------------
where we have truncated at second order in $a$. Inserting the expression
for $C_0$ into Eq.~(\ref{amp2-2nd}) and expanding to second order in $a$,
we obtain the renormalized expression for the amplitude:
%----------------------
\begin{equation}
 T_2 (E) \approx {-}\frac{8 \pi a}{m} \left( 1 + a \sqrt{-mE - \ii 
 \epsilon} + \cdots \right) = {-}\frac{8 \pi a}{m} \left( 1 - \ii ak +
 \cdots \right) \,.
\label{A2pert}
\end{equation}
%----------------------
If we evaluate this at the on-shell point $E=k^2/m$ and insert it into
Eq.~(\ref{amp2-2nd}), we find that it reproduces the first two terms in the
expansion of the universal scattering amplitude in Eq.~(\ref{eq-ere}) 
in powers of $ka$. By calculating $T_2 (E)$ to higher order 
in perturbation theory, we can reproduce the low-momentum 
expansion of Eq.~(\ref{eq-ere}) to higher order in $ka$. At the next order,
the $C_2$ term will contribute at tree level while $C_0$ will contribute
at the two-loop level. Thus a perturbative
treatment of the EFT reproduces the low-momentum
expansion of the two-body scattering amplitude.
The perturbative approximation is valid only if the momentum satisfies
$k \ll 1/a$.

A more interesting case occurs when the scattering length
is large, but all other effective range coefficients are still
determined by the scale $\Mhi$: $k\sim 1/|a|\sim\Mlo \ll \Mhi \sim 1/r_e$.
This scenario
is able to support shallow bound states with binding momentum of order $1/a$
and is relevant to ultracold atoms close to a
Feshbach resonance and very low-energy nucleons.
The scaling of the operators is then modified to:
%----------------------
\begin{equation}
 C_0 \sim \frac{1}{m\Mlo}\,,\quad C_2 \sim \frac{1}{m\Mlo^2\Mhi} \,,
 \quad\text{and}\quad
 D_0 \sim \frac{1}{m\Mlo^4}\,.
\label{eq:scalingMlo}
\end{equation}
%----------------------
The factors of $\Mlo$ in amplitudes can now come from small momenta and from the
coupling constants.  Above we adjusted the scaling of the three-body coupling 
$D_0$ as well, foreclosing a result discussed below Eq.~\eqref{BhvK:general}.

With the scaling as in Eq.~\eqref{eq:scalingMlo}, the power counting expression 
in Eq.~(\ref{nu-pert}) is therefore modified to
%----------------------
\begin{equation}
 \nu=3L+2+\sum_i (i-3) V_{2i} \geq -1\,.
\end{equation}
%----------------------
If we are interested in two-body observables involving energy $E \sim 1/a^2$,
such as shallow bound or virtual states,
we must resum the diagrams involving only $C_0$ interactions
to all orders~\cite{vanKolck:1998bw,Kaplan:1998we}.  Without this resummation, 
our EFT would break down not at $\Mhi$, but already at the much smaller scale 
$1/|a|\sim\Mlo$.  In the scenario assumed here, all higher-derivative two-body 
interactions ($C_2$ and beyond) still involve inverse powers of $\Mhi$ and are 
thus perturbative.


The resummation of $C_0$ interactions is most easily accomplished by
realizing that
the corresponding Feynman diagrams in Fig.~\ref{fig:amp2}
form a geometric series. 
Summing the geometric series, the exact expression
for the amplitude is
%----------------------
\begin{equation}
 T_2(E) = {-}C_0 \left[ 1 + \frac{mC_0}{4 \pi^2}
 \left( \Lambda - \frac{\pi}{2} \sqrt {-mE -\ii\epsilon} \right) 
 \right]^{-1} \,.
\label{A-nonpert}
\end{equation}
%----------------------
Alternatively, we can use the fact that summing the $C_0$ diagrams in
Fig.~\ref{fig:amp2} is equivalent to solving the following
integral equation:
%----------------------
\begin{equation}
 T_2(E) = {-}C_0 - \frac{\ii}{2} C_0 \int \frac{\dd^3q}{(2\pi)^3}
 \int\frac{\dd q_0}{2\pi} \frac1{q_0 - q^2/2m + \ii \epsilon}
 \frac1{ E - q_0 - q^2/2m + \ii \epsilon} \, T_2 (E) \,.
\label{inteq-2}
\end{equation}
%----------------------
The integral equation is expressed diagrammatically in Fig.~\ref{fig:amp2b}.
%%%%%%%%%%%%%%%%%%%%%%%%%%%%%%%%%%%%%%%%%%%%%%%%%%
\begin{figure}[htb]
\bigskip
\centerline{\includegraphics*[width=6.5cm,angle=0]{fig_feynC0int.pdf}}
\medskip
\caption
    {Integral equation  for the two-body scattering amplitude $T_2$
      at leading order in the case of large scattering length.
      Notation as in Fig.~\ref{fig:amp2}.}
\label{fig:amp2b}
\end{figure}
%%%%%%%%%%%%%%%%%%%%%%%%%%%%%%%%%%%%%%%%%%%%%%%%%%
Since the function $T_2(E)$ is independent of ${\bm q}$ and $q_0$,
it can be pulled
outside of the integral in  Eq.~(\ref{inteq-2}). The integral can be
regularized by imposing an ultraviolet cutoff $\Lambda$.
The integral equation is now trivial to solve and the
solution is given in Eq.~(\ref{A-nonpert}).

The expression for the resummed two-body
amplitude in Eq.~(\ref{A-nonpert}) depends on the
parameter $C_0$ in the Lagrangian and
on the ultraviolet cutoff $\Lambda$. As in the perturbative case,
renormalization can be implemented
by eliminating $C_0$ in favor of a low-energy observable, such as the
scattering length $a$. 
Matching the resummed expression to the effective range expansion
for $T_2$, we obtain
%----------------------
\begin{equation}
 C_0 = \frac{8\pi a}{m} \left( 1 - \frac{2 a \Lambda}{\pi} \right)^{-1}  \,.
\label{g2-tune}
\end{equation}
%----------------------
Given a fixed ultraviolet cutoff $\Lambda$, this equation prescribes how
the parameter $C_0$ must be tuned in order to give the 
scattering length $a$.  Note that for $\Lambda \gg 1/|a|$,
the coupling constant $C_0$ is always negative 
regardless of the sign of $a$.
Eliminating $C_0$ in Eq.~(\ref{A-nonpert}) in favor of $a$, 
we find that the resummed amplitude reduces to
%----------------------
\begin{equation}
 T_2 (E) = \frac{8\pi}{m}\frac1{{-}1/a + \sqrt {{-}mE -\ii \epsilon}} \,,
\label{a-npren}
\end{equation}
%----------------------
which reproduces the effective range expansion of the scattering
amplitude by construction. 
In this simple case, we find that our renormalization prescription 
eliminates the dependence on $\Lambda$ completely. In general, 
we should expect it to only be suppressed by powers of
$1/(a \Lambda)$ or $mE/ \Lambda^2$.  A final step of taking the limit 
$\Lambda \to \infty$ would then be required to obtain results 
that are completely independent of  $\Lambda$.
The first correction to Eq.~(\ref{a-npren}) is given by the
$C_2$ interaction. The corresponding diagrams are shown in
Fig.~\ref{fig:amp2c}.
%%%%%%%%%%%%%%%%%%%%%%%%%%%%%%%%%%%%%%%%%%%%%%%%%%
\begin{figure}[htb]
\bigskip
\centerline{\includegraphics*[width=12cm,angle=0]{fig_feynC2int.pdf}}
\medskip
\caption
    {Next-to-leading order correction to $T_2$.
      Notation as in Fig.~\ref{fig:amp2}.}
\label{fig:amp2c}
\end{figure}
%%%%%%%%%%%%%%%%%%%%%%%%%%%%%%%%%%%%%%%%%%%%%%%%%%
After the matching, the final result for the scattering amplitude
at next-to-leading order is
%----------------------
\begin{equation}
 T_2 (E) = \frac{8\pi}{m}\left(
 \frac1{{-}1/a + \sqrt {{-}mE -\ii\epsilon}}
 + \frac{r_e mE/2}{({-}1/a + \sqrt {-mE - \ii \epsilon})^2}\right) \,,
\label{a-npren2}
\end{equation}
%----------------------
where $r_e$ is the effective range. 
The derivation of this expression will be left as an exercise.

\begin{prob}
{\emph Exercise:} 
Derive the  next-to-leading order correction in Eq.~(\ref{a-npren2})
by calculating the loop diagrams in Fig.~\ref{fig:amp2c}. Neglect 
all terms that vanish as $\Lambda \to \infty$. Introduce a 
next-to-leading order piece of $C_0$ to cancel the cubic divergence. 
\end{prob}

\subsection{Dimer field formalism}

In applications to systems with more than two particles, it is often useful to 
rewrite the EFT for short-range interactions specified by the 
Lagrangian~(\ref{L-2body}) using so-called dimer fields
$d$~\cite{Kaplan:1996nv}:
%----------------------
\begin{spliteq}
 {\mathcal L} &= \phi^\dagger \left( \ii \partial_t + \frac{\nabla^2}{2m} 
 \right) \phi +{g_0} d^\dagger d
 +g_2  d^\dagger\left( \ii \partial_t + \frac{\nabla^2}{4m} 
 \right)d+\cdots \\
 &\phantom{=}- {y} \left( d^\dagger \phi^2 + {\phi^\dagger}^2 d \right)
 -{d_0} d^\dagger d \phi^\dagger \phi +\cdots\,.
\label{L-BHvK}
\end{spliteq}
%----------------------
One important feature of this Lagrangian is that there is no direct two-body 
contact interaction term $(\phi^\dagger \phi)^2$.  All interactions between 
$\phi$ particles are mediated via exchange of a dimer field $d$, \ie, we have 
effectively performed a Hubbard--Stratonovich transformation.  Eliminating 
the dimer field $d$ by using its equations of motion, it can be shown that the 
physics of this EFT is equivalent to the Lagrangian~(\ref{L-2body}).

Note that the Lagrangian~(\ref{L-BHvK}) contains one more free parameter than 
the Lagrangian (\ref{L-2body}).  Thus some parameters are redundant.  For the 
leading-order case ($g_n=0$ for $n\geq 2$), \eg, we find explicitly,
%----------------------
\begin{equation}
 C_0=\frac{4y^2}{g_0}\,,
\label{eq-relation}
\end{equation}
%----------------------
such that $y$ and $g_0$ are not independent.  Higher-order corrections can be 
obtained by including a kinetic-energy term for the dimer field.  The constants 
$g_0$ and $y$ then become independent and can be related to combinations of 
$C_0$ and $C_2$ in the theory without dimers.  Here, we only discuss the 
leading-order case. 

\begin{prob}
{\emph Exercise:} Derive Eq.~(\ref{eq-relation}) using the classical 
equation of motion for $d$.
\end{prob}

%%%%%%%%%%%%%%%%%%%%%%%%%%%%%%%%%%%%%%%%%%%%%%%%%%
\begin{figure}[htb]
\bigskip
\centerline{\includegraphics*[width=12cm,angle=0]{fig_feynDimer.pdf}}
\medskip
\caption
{Diagrammatic equations for the full dimer propagator
$i D(P_0,P)$. Thin (thick)
solid lines represent particle (full dimer) propagators.
Double lines indicate bare dimer propagators.
(i) perturbative expansion in powers of $y$,
(ii) integral equation summing the geometric series in (i).}
\label{fig:bubbles}
\end{figure}
%%%%%%%%%%%%%%%%%%%%%%%%%%%%%%%%%%%%%%%%%%%%%%%%%%

The bare propagator for the dimer field is simply the constant $\ii/g_0$,
which corresponds to no propagation in space or time.  However, there are 
corrections to the dimer propagator from the diagrams in 
Fig.~\ref{fig:bubbles}(i) which allow the dimer to propagate.  This is 
completely analogous to the geometric series we found we had to sum to obtain 
the leading-order scattering amplitude.  In Feynman diagrams, we represent the 
full dimer propagator $\ii D(P_0,P)$ by a thick solid line.  We can 
calculate the full dimer propagator by solving the simple integral equation 
shown in Fig.~\ref{fig:bubbles}(ii). The loop on the right side is just the 
integral in Eq.~(\ref{A-pert}), with $E$ replaced by $P_0-P^2/(4m)$, where $P_0$ 
and ${\bm P}$ are the energy and momentum of the dimer.  The solution for the 
full dimer propagator is
%----------------------
\begin{equation}
 \ii D(P_0,P) = \frac{2\pi i}{y^2 m}
 \left[ \frac{2\pi g_0}{y^2 m} + \frac{2}{\pi} \Lambda -
  \sqrt{{-}mP_0 + P^2/4 - \ii \epsilon} \right]^{-1} \,,
\label{diprop}
\end{equation}
%----------------------
where as before $\Lambda$ is a cutoff on the loop momentum in the bubbles.
Using Eq.~(\ref{eq-relation}) and making the substitution given in 
Eq.~(\ref{g2-tune}), the expression for the complete dimer propagator is
%----------------------
\begin{equation}
 \ii D(P_0, P) = -\frac{2\pi \ii}{y^2 m}
 \left[ {-}1/a + \sqrt{{-}mP_0 + P^2/4 -\ii \epsilon}
 \right]^{-1} .
\label{propdiatom}
\end{equation}
%----------------------
Note that all the dependence on the ultraviolet cutoff is now in the
multiplicative factor $1/y^2$.  The complete dimer propagator differs from the 
off-shell two-body amplitude $T_2$ in Eq.~(\ref{a-npren}) only by a 
multiplicative constant.  For $a>0$, it has a pole at $P_0 = -1/(ma^2) + P^2/4$ 
corresponding to a dimer of momentum ${\bm P}$ and binding energy 
$B_2=1/(ma^2)$.  As $P_0$ approaches the dimer pole, the limiting behavior of 
the propagator is
%----------------------
\begin{equation}
 D(P_0, P) \longrightarrow
 \frac{Z_D}{P_0 - ({-}1/(ma^2)+ P^2/4) + \ii \epsilon} \,,
\label{dimer-pole}
\end{equation}
%----------------------
where the residue factor is
%----------------------
\begin{equation}
 Z_D= \frac{4\pi}{a m^2 y^2} \,.
\label{dimer-Z}
\end{equation}
%----------------------
If we regard the composite operator $d$ as a quantum field that annihilates and 
creates dimers, then $Z_D$ is the wave function renormalization constant for 
that field.  The renormalized propagator $Z_D^{-1} D(P_0,P)$ is completely 
independent of the ultraviolet cutoff.

\subsection{Three-body system}
\label{sec:EFT-ThreeBosons}

We now study the amplitude for particle-dimer scattering $T_3$.  The simplest 
diagram we can write down involving only two-body interactions is the exchange 
of a particle between in- and outgoing dimers.  With the scaling of low-energy 
constants as in Eq.~\eqref{eq:scalingMlo}, the power counting implies that all 
diagrams that are chains of such exchanges are equally important, \ie, they have 
to be summed up nonperturbatively.  Just like in the two-body case, this can be 
written as an integral equation.  Also including the three-body interaction 
(note $D_0 \to d_0$ in the Lagrangian with dimer fields), we get the result that 
is shown diagrammatically in Fig.~\ref{fig:inteq12}.\footnote{Note that this 
amplitude is well defined even if $a<0$ and there is no two-body bound state. 
In this case particle lines must be attached to the external dimer propagators 
to obtain the 3-particle scattering amplitude.}

Omitting the three-body interaction, 
this is exactly the well-known Skorniakov-Ter-Martirosian (STM)
integral equation~\cite{Skorniakov:1957aa}, which the EFT with
Lagrangian~(\ref{L-BHvK}) reproduces by construction.  In addition, EFT 
provides a clear method to renormalize this equation with a three-body 
interaction and thus remove its pathologies (discussed below).

In Fig.~\ref{fig:inteq12}, all external lines are understood to be amputated.
It simply gives the non-perturbative solution of the three-body problem
for the interaction terms proportional to $g_0$, $y$, and $d_0$ in 
Eq.~(\ref{L-BHvK}).

The two tree diagrams on the right side of Fig.~\ref{fig:inteq12} constitute the 
inhomogeneous term in the integral equation.  An iterative ansatz for the 
solution of this equation shows that all diagrams with the $g_0$, $y$, and 
$d_0$ interactions are generated by the iteration.  Note also that the thick 
black lines in Fig.~\ref{fig:inteq12} represent the full dimer propagator given 
in Eq.~(\ref{propdiatom}).

%%%%%%%%%%%%%%%%%%%%%%%%%%%%%%%%%%%%%%%%%%%%%%%%%%
\begin{figure}[htb]
\bigskip
\centerline{\includegraphics*[width=9cm,angle=0]{fig_feyn8c.pdf}}
\medskip
\caption
{The integral equation for the three-body amplitude $T_3$. Thin (thick)
solid lines represent particle (full dimer) propagators. External lines
are amputated.}
\label{fig:inteq12}
\end{figure}
%%%%%%%%%%%%%%%%%%%%%%%%%%%%%%%%%%%%%%%%%%%%%%%%%%

In the center-of-mass frame, we can take the external momenta of the
particle and dimer to be $-{\bm p}$ and $+{\bm p}$ for the incoming lines
and $-{\bm k}$ and $+{\bm k}$ for the outgoing lines.  We take their
energies to be $E_A$ and $E-E_A$ for the incoming lines and $E_A'$ and
$E-E_A'$ for the outgoing lines. The amplitude $T_3$ is then a
function of the momenta ${\bm p}$ and ${\bm k}$ and the energies $E$,
$E_A$ and $E_A'$. The integral equation involves a loop over the
momentum $-{\bm q}$ and energy ${q_0}$ of a virtual particle. Using the
Feynman rules encoded in the Lagrangian (\ref{L-BHvK}), we obtain
%----------------------
\begin{multline}
 T_3 ({\bm p}, {\bm k}; E, E_A, E_A')
 = {-}\left[ \frac{4y^2}{ E-E_A-E_A' - ({\bm p} + {\bm k})^2 /(2m) + \ii
 \epsilon}
 + {d_0} \right] \\
 \null + \frac{2\pi \ii}{m y^2} \int \frac{\dd q_0}{2 \pi}
 \int\frac{\dd^3q}{(2 \pi)^3}
 \left[ \frac{4 y^2}{E-E_A - q_0 - ({\bm p} + {\bm q})^2/(2m) + \ii\epsilon}
 + {d_0} \right] \\
 \times \frac1{q_0 - q^2/(2m) + \ii \epsilon}\;
 \frac{T_3 ({\bm q}, {\bm k}; E, q_0, E_A')}
 {1/a - \sqrt{-m(E-q_0) + q^2 /4 -\ii \epsilon} } \,.
\end{multline}
%----------------------
The integral over $q_0$ can be evaluated by contour integration. This
sets $q_0 = q^2/(2m)$, so the amplitude $T_3$ inside the integral has the
incoming particle on-shell. 

We obtain a simpler integral
equation if we also set the energies of both the initial and final particles in
$T_3$ on-shell: $E_A = p^2/(2m)$, $E_A' = k^2/(2m)$.
Thus only the dimer lines have energies that are off-shell.
The resulting integral equation is
%----------------------
\begin{multline}
 T_3 \left({\bm p}, {\bm k}; E, \frac{p^2}{2m}, \frac{k^2}{2m}\right)
 = {-}{4my^2} 
 \left[ \frac1{mE - (p^2 + {\bm p} \cdot {\bm k} + k^2) + \ii\epsilon} 
 + \frac{d_0}{4 my^2} \right] \\
 \null - 8 \pi \int\frac{\dd^3q}{(2 \pi)^3}
 \left[ \frac1{mE - (p^2 + {\bm p} \cdot {\bm q} + q^2) + \ii \epsilon}
 + \frac{d_0}{4my^2} \right] \\
 \null\times
 \frac{T_3 ({\bm q}, {\bm k}; E, q^2/(2m), k^2/(2m))}
 {{-}1/a + \sqrt{{-}mE + 3q^2 /4 -\ii \epsilon} } \,.
\label{BhvK:general}
\end{multline}
%----------------------
This is an integral equation with three integration variables 
for an amplitude $T_3$ that depends explicitly 
on seven independent variables.  There is also an additional implicit variable
provided by an  ultraviolet cutoff $|{\bm q}| < \Lambda$
on the loop momentum.

If we set $d_0 = 0$ and ignore the ultraviolet cutoff, 
the integral equation in Eq.~(\ref{BhvK:general}) is equivalent to the 
Skorniakov-Ter-Martirosian (STM) equation, 
an integral equation for three particles interacting via
zero-range two-body forces derived by Skorniakov and Ter-Martirosian 
in 1957~\cite{Skorniakov:1957aa}.  
It was shown by Danilov that the STM equation has no unique 
solution in the case of identical bosons~\cite{Danilov:1961aa}. He also
pointed out that a unique solution could be obtained if one 
three-body binding energy is fixed. 
Kharchenko was the first to solve the STM equation with a finite 
ultraviolet cutoff that was tuned to fit observed three-body data.  
Thus the cutoff was treated 
as an additional parameter~\cite{Kharchenko:1973aa}. 
When we discuss the running of $d_0$, we will see
that this \adhoc procedure is indeed justified and emerges
naturally when the three-body equation is renormalized~\cite{Bedaque:1998kg}.


Here we restrict our attention to the sector of the three-body problem with 
total orbital angular momentum $L=0$ where the three-body interaction
contributes. For higher $L$, the original STM equation has a unique solution
and can be solved numerically without complication.

The projection onto $L=0$ can be accomplished
by averaging the integral equation over the cosine
of the angle between ${\bm p}$ and ${\bm k}$: $x={\bm p}\cdot{\bm k}/
(pk)$. It is also convenient to multiply the amplitude $T_3$
by the wave function renormalization factor $Z_D$ given in 
Eq.~(\ref{dimer-Z}).
We will denote the resulting amplitude by $T_3^0$:
%----------------------
\begin{equation}
 T_3^0(p, k; E) \equiv Z_D 
 \int_{-1}^1 \! \frac{\dd x}{2}\,
 T_3 \left({\bm p}, {\bm k}; E, p^2/(2m), k^2/(2m)\right) .
\label{A-def}
\end{equation}
%----------------------
Furthermore, it is convenient to express the three-body coupling constant 
in the form
%----------------------
\begin{eqnarray}
 d_0 = {-}\frac{4my^2}{\Lambda^2} H(\Lambda) \,.
\label{g3g2}
\end{eqnarray}
%----------------------
Since $H$ is dimensionless, it can only  be a
function of the dimensionless variables $a \Lambda$ and 
$\Lambda/ \Lambda_*$, where $\Lambda_*$ is a three-body 
parameter defined below. We will find that $H$ is a function of
$\Lambda/ \Lambda_*$ only.  

The resulting integral equation is:
%----------------------
\begin{multline}
 T_3^0 (p, k; E)  = \frac{16 \pi}{m a} 
 \left[ \frac{1}{2pk} \ln \left(\frac{p^2 + pk + k^2 -mE - \ii \epsilon}
 {p^2 - pk + k^2 - mE - \ii \epsilon}\right) + \frac{H(\Lambda)}{\Lambda^2} 
 \right] \\
 + \frac{4}{\pi} \int_0^\Lambda \dd q \, q^2
 \left[\frac{1}{2pq} \ln\left( \frac{p^2 +pq + q^2 - mE - \ii \epsilon}
 {p^2 - pq + q^2 -mE -\ii \epsilon}\right)
 + \frac{H(\Lambda)}{\Lambda^2} \right] \\
 \null \times \frac{ T_3^0 (q, k; E)}
 {{-}1/a + \sqrt{3q^2/4 -mE - \ii\epsilon}} \,.
\label{BHvK}
\end{multline}
%----------------------
Note that the ultraviolet cutoff $\Lambda$ on the
integral over $q$ has been made explicit. 
A change in the endpoint $\Lambda$ of the loop integral
should be compensated by the $\Lambda$-dependence of the function 
$H$ in Eq.~(\ref{BHvK}).
More specifically, $H$ must be tuned as a function of $\Lambda$ 
so that the cutoff dependence of the solution $T_3^0 (p, k; E)$
of Eq.~(\ref{BHvK}) decreases as a power of $\Lambda$.  This will 
guarantee that $T_3^0 (p, k; E)$ has a well-behaved limit 
as $\Lambda \to \infty$. The renormalization group behavior of
$H$ will be discussed in detail below. In the next subsection, we 
show how
different three-body observables can be obtained from the 
solution $T_3^0 (p, k; E)$ of Eq.~(\ref{BHvK}).

\begin{prob}
Fill in the gaps in the above derivation of Eq.~(\ref{BHvK}) and
generalize the derivation to general angular momentum $L$.
\end{prob}

%%%%%%%%%%%%%%%%%%%%%%%%%%%%%%%%%%%%%%%%%%%%%%%%%%
%    Three-body observables
%%%%%%%%%%%%%%%%%%%%%%%%%%%%%%%%%%%%%%%%%%%%%%%%%%

\subsection{Three-body observables}
\label{sec:EFT-3BObs}

The solution $T_3^0 (p, k; E)$ to the three-body
integral equation~(\ref{BHvK})
encodes all information about three-body observables in the sector with total
orbital angular momentum quantum number $L=0$.
In particular, it contains information about the 
binding energies $B_3^{(n)}$ of the three-body bound 
states~\cite{Efimov:1970aa}.
For a given ultraviolet cutoff $\Lambda$, the amplitude 
$T_3^0 (p, k; E)$ has a finite number of poles in $E$
corresponding to the bound states whose binding energies
are less than about $\Lambda^2$.  As $\Lambda$ increases,
new poles emerge corresponding to deeper bound states.
In the limit $\Lambda \to \infty$, the locations of these poles 
approach the energies $-B_3^{(n)}$ of the three-body bound states.
The residues of the poles of $T_3^0 (p, k; E)$
factor into functions of $p$ and functions of $k$:
%----------------------
\begin{equation}
 T_3^0 (p, k; E) \longrightarrow 
 \frac{ {\mathcal B}^{(n)}(p) {\mathcal B}^{(n)}(k)}{ E + B_3^{(n)} } \,,
 \qquad \text{as}\ E \to {-}B_3^{(n)} \,.
\end{equation}
%----------------------
Matching the residues of the poles on both sides of Eq.~(\ref{BHvK}),
we obtain the bound-state equation
%----------------------
\begin{spliteq}
 {\mathcal B}^{(n)}(p) &=  \frac{4}{\pi} \int_0^\Lambda \! \dd q \, q^2
 \left[\frac{1}{2pq} \ln \frac{p^2 +pq + q^2 - mE - \ii \epsilon}
 {p^2 - pq + q^2 -mE -\ii \epsilon} + \frac{H(\Lambda)}{\Lambda^2} \right] \\
 &\phantom{=}\times \left[{-}1/a + \sqrt{3q^2/4 -mE
 -\ii \epsilon} \right]^{-1} {\mathcal B}^{(n)}(q) \,.
\label{BHvK-homo}  
\end{spliteq}
%----------------------
The values of $E$ for which this homogeneous integral equation has 
solutions are the energies ${-}B_3^{(n)}$ of the three-body states.
For a finite ultraviolet cutoff $\Lambda$,
the spectrum of $B_3^{(n)}$ is cut off around $\Lambda^2$. 


%%%%%%%%%%%%%%%%%%%%%%%%%%%%%%%%%%%%%%%%%%%%%%%%%%
\begin{figure}[htb]
\bigskip
\centerline{\includegraphics*[width=11cm,angle=0]{fig_feyn9c.pdf}}
\medskip
\caption
{Amplitudes for (i) particle-dimer scattering, (ii) three-body recombination, 
and (iii) three-body breakup. 
Diagrams (ii) [(iii)] should be summed over the three pairs of 
particles that can interact first [last].  
Notation as in Fig.~\ref{fig:inteq12}.}
\label{fig:3br}
\end{figure}
%%%%%%%%%%%%%%%%%%%%%%%%%%%%%%%%%%%%%%%%%%%%%%%%%%

The S-wave phase shifts for particle-dimer scattering can be determined
from the solution $T_3^0 (p, k; E)$ to the 
integral equation~(\ref{BHvK}).
The T-matrix element for the elastic scattering of an particle and a dimer 
with momenta $k$ is given by the amplitude $T_3^0$ 
evaluated at the on-shell point $p=k$ and $E= {-}B_2 + 3k^2/(4m)$
and multiplied by a wave function renormalization factor $Z_D^{1/2}$ 
for each dimer in the initial or final state.
It can be represented by the Feynman diagram in 
Fig.~\ref{fig:3br}(i).  The blob represents the amplitude $T_3$ or
equivalently $Z_D^{-1} T_3^0$.
The external double lines are amputated and correspond 
to asymptotic dimers and are associated with factors $Z_D^{1/2}$

The S-wave contribution to the T-matrix element is
%----------------------
\begin{equation}
 T_{PD \to PD}^{0} = 
 T_3^0 (k, k; 3k^2/(4m)-1/(ma^2)) \,,
\end{equation}
%----------------------
where $B_2=1/(ma^2)$ has been used.
Note that the factors of $Z_D$ multiplying $T_3^0$ cancel.
The differential cross section for elastic particle-dimer scattering is
%----------------------
\begin{equation}
 \dd \sigma_{PD \to PD}
 = \frac{2 m}{3 k} \left| T_{PD \to PD}(k) \right|^2
 \frac{k m}{6 \pi^2} d \Omega \,.
\label{T-AD}
\end{equation}
%----------------------
The flux factor $2m/(3 k)$ is the inverse of the relative velocity 
of the particle and the dimer.  The phase space factor 
$k md \Omega/(6 \pi^2)$ takes into account energy and momentum 
conservation and the standard normalization of momentum eigenstates:
%----------------------
\begin{equation}
 \int \frac{\dd^3 p_A}{(2 \pi)^3} \frac{\dd^3 p_D}{(2 \pi)^3} 
 (2 \pi)^4 \delta^3({\bm p}_A + {\bm p}_D)
 \delta(p_A^2/(2m) + p_D^2/(4m) - E)
 = \frac{m}{6 \pi^2} (4 mE/3)^{1/2} \int \dd\Omega \,.
\end{equation}
%----------------------
The S-wave phase shift for particle-dimer scattering
is related to the T-matrix element via
%----------------------
\begin{equation}
 \frac1{k\cot\delta^{PD}_0 (k) -\ii k}
 = \frac{m}{3 \pi} T_3^0 (k, k; 3k^2/(4m)-1/(ma^2)) \,.
\label{T12}
\end{equation}
%----------------------
In particular, the particle-dimer scattering length is given by
%----------------------
\begin{equation}
 a_{PD} = {-}\frac{m}{3 \pi} T_3^0 (0, 0; {-}1/(ma^2)) \,.
\end{equation}
%----------------------


The threshold rate for three-body recombination can also be obtained 
from the solution $T_3^0 (p, k; E)$ to the three-body integral 
equation in Eq.~(\ref{BHvK}). This is possible only at threshold, 
because a 3-particle scattering state becomes pure $L=0$ only
in the limit that the energies of the particles go to zero. 
The T-matrix element for the recombination process can be represented 
by the Feynman diagram in Fig.~\ref{fig:3br}(ii)
summed over the three pairs of particle lines that can attach 
to the dimer line.
The blob represents the amplitude $Z_D^{-1} T_3^0$ evaluated 
at the on-shell point $p=0$ $k=2/(\sqrt{3}\, a)$, and $E=0$.
The solid line represents the dimer propagator $\ii D(0,0)$
evaluated at zero energy and momentum $2/(\sqrt{3}\, a)$, which is given by 
Eq.~(\ref{diprop}).  The factor for the particle-dimer vertex is $-\ii 2y$.
The wave function renormalization factor $Z_D^{1/2}$ for the
final-state dimer is given by Eq.~(\ref{dimer-Z}).
In the product of factors multiplying $T_3^0$, 
the dependence on $y$ and $\Lambda$ can be eliminated in favor 
of the scattering length $a$.
Taking into account a factor of 3 from the three Feynman diagrams,
the T-matrix element is
%----------------------
\begin{equation}
 T_{PD \to PPP} = 6\sqrt{\pi a^3}\, 
 T_3^0 (0, 2/(\sqrt{3}a);0) \,.
\end{equation}
%----------------------
The differential rate $dR$ for the recombination of three particles
with energies small compared to the dimer binding energy
can be expressed as
%----------------------
\begin{equation}
 \dd R = \left| T_{PPP \to PD} \right|^2
 \frac{k m}{6 \pi^2}\, \dd \Omega \,,
\label{dR-T}
\end{equation}
%----------------------
where $k = 2/(\sqrt{3} a)$.
The threshold rate for three-body breakup can be obtained
in a similar way from  the Feynman diagram in Fig.~\ref{fig:3br}(iii)

The inhomogeneous integral equation for the off-shell particle dimer 
amplitude, Eq.~(\ref{BHvK}), and the homogeneous equation, 
Eq.~(\ref{BHvK-homo}), for the three-body binding energies afford
no analytical solution. They are usually solved by 
discretizing the integrals involved and solving the resulting
matrix problems numerically.

\subsection{Renormalization group limit cycle}
\label{sec:RGlc}

The form of the full renormalized dimer propagator in
Eq.~(\ref{propdiatom}) is consistent with the continuous scaling symmetry
%----------------------
\begin{equation}
 a \longrightarrow \lambda a \,,
 \qquad
 E \longrightarrow \lambda^{-2} E \,,
\label{scaling-1}
\end{equation}
%----------------------
for any positive real number $\lambda$.
In the integral equation~(\ref{BHvK}), this scaling
symmetry is broken by the ultraviolet cutoff on the integral and by the
three-body terms proportional to $H/ \Lambda^2$. To see that the cutoff and
the three-body terms are essential, we set $H=0$ and take
$\Lambda \rightarrow \infty$. The resulting integral equation has exact
scaling symmetry. We should therefore expect its solution 
$T_3^0 (p,k; E)$ to behave asymptotically as $p \rightarrow \infty$ 
like a pure power of $p$. Neglecting the inhomogeneous term,
neglecting $E$ and $1/a^2$ compared to $q^2$,
and setting $T_3^0 \approx p^{s-1}$, 
the integral equation reduces to~\cite{Danilov:1961aa}
%----------------------
\begin{equation}
 p^{s-1} = \frac{4}{\sqrt{3} \pi p} \int_0^\infty \dd q \, q^{s-1} 
 \ln \frac{p^2 + pq + q^2}{p^2 -pq + q^2} \,.
\end{equation}
%----------------------
Making the change of variables
$q = xp$, the dependence on $p$ drops out, and we obtain
%----------------------
\begin{equation}
 1 = \frac{4}{\sqrt{3} \pi} \int_0^\infty \dd x \, x^{s-1} 
 \ln \frac{1 + x + x^2}{1 -x + x^2} \,.
\end{equation}
%----------------------
The integral is a Mellin transform that can be evaluated
analytically.  The resulting equation for $s$ is
%----------------------
\begin{equation}
 1 = \frac{8}{\sqrt{3} s} \frac{\sin (\pi s/6)}{\cos (\pi s/2)} \,.
\end{equation}
%----------------------
The solutions with the lowest values of $|s|$ are
purely imaginary: $s = \pm \ii s_0$, where $s_0 \approx 1.00624$. The most
general asymptotic solution therefore has two arbitrary constants:
%----------------------
\begin{equation}
 T_3^0 (p, k; E) \longrightarrow A_+ \, 
 p^{-1+\ii s_0} + A_- \, p^{-1-\ii s_0} 
 \,,\qquad \text{as}\ p \to \infty \,.
\end{equation}
%----------------------
The inhomogeneous term in the integral equation~(\ref{BHvK}) will
determine one of the constants. The role of the three-body term in the
integral equation is to determine the other constant, thus giving the
integral equation a unique solution.

By demanding that the solution of the integral equation~(\ref{BHvK}) 
has a well-defined limit as $\Lambda \to \infty$, 
Bedaque~\etal deduced the $\Lambda$-dependence 
of $H$ and therefore of $d_0$~\cite{Bedaque:1998kg}. 
The leading dependence on $\Lambda$ on the right side of the
three-body integral equation in Eq.~(\ref{BHvK}) as $\Lambda \to \infty$ 
is a log-periodic term of order $\Lambda^0$ that comes from the 
region $q \sim \Lambda$.  
There are also contributions of order $1/\Lambda$
from the region $|a|^{-1},k,|E|^{1/2} \ll q \ll \Lambda$,
which have the form
%----------------------
\begin{equation}
 \frac{8}{\pi\sqrt{3}} \int^\Lambda \dd q \, 
 \left( \frac{1}{q^2}+ \frac{H(\Lambda)}{\Lambda^2}\right)
 (A_+ \, q^{+\ii s_0} + A_- \, q^{-\ii s_0}) \,.
\label{1overLambda}
\end{equation}
%----------------------
The sum of the two terms will decrease even faster as $1/\Lambda^2$ 
if we choose the function $H$ to have the form
%----------------------
\begin{equation}
 H(\Lambda) =
 \frac{ A_+ \Lambda^{\ii s_0}/(1-\ii s_0) + A_- \Lambda^{-\ii s_0}/(1+\ii s_0)}
 {A_+ \Lambda^{\ii s_0}/(1+\ii s_0)  + A_- \Lambda^{-\ii s_0}/(1-\ii s_0)} \,.
\label{H-tune}
\end{equation}
%----------------------
The tuning of $H$ 
that makes the term in Eq.~(\ref{1overLambda}) decrease like
$1/\Lambda^2$ also suppresses the contribution from the region
$q \sim \Lambda$ by a power of $1/\Lambda$  so that it goes to 0 
in the limit $\Lambda \to \infty$.
By choosing $A_\pm = (1 + s_0^2)^{1/2} \Lambda_*^{\mp \ii s_0}/2$
in Eq.~(\ref{H-tune}), we obtain~\cite{Bedaque:1998kg}
%----------------------
\begin{equation}
 H (\Lambda) \approx \frac{\cos [s_0 \ln (\Lambda/ \Lambda_*) + \arctan s_0]}
 {\cos [s_0 \ln (\Lambda/ \Lambda_*) - \arctan s_0]} \,.
\label{H-Lambda}
\end{equation}
%----------------------
This equation defines a three-body scaling-violation parameter 
$\Lambda_*$ with dimensions of momentum. The value of $\Lambda_*$
can be fixed from a three-body datum. All other three-body
observables can then be predicted. If Eq.~(\ref{H-Lambda}) is 
substituted back into the three-body equation~(\ref{BHvK})
for numerical calculations, it must be multiplied by a normalization factor
$b\approx 1$ whose precise value
depends on the details of the regularization~\cite{Braaten:2011sz}.


Note that $H$ is a $\pi$-periodic function of 
$s_0\ln(\Lambda/\Lambda_*)$, so $\Lambda_*$ is defined only up to a 
multiplicative factor of $(\eex^{\pi/s_0})^n$, where $n$ is an integer.
Thus the scaling symmetry of Eq.~(\ref{scaling-1}) is broken to
the discrete subgroup of scaling transformations with multiples of
the preferred scaling factor $\lambda=\eex^{\pi/s_0}$. This discrete 
scaling symmetry is, \eg, evident in the geometric 
spectrum of three-body Efimov states~\cite{Efimov:1970aa} 
in the unitary limit ($1/a=0$) that naturally emerge in this EFT:
%----------------------
\begin{equation}
 B_3^{(n)}\approx 0.15 \lambda^{2(n_*-n)}\,\frac{\Lambda_*^2}{m}\,,
\end{equation}
%----------------------
where $n_*$ an integer labeling the state
with binding energy  
$0.15\,\Lambda_*^2/m$.\footnote{For a detailed discussion of the Efimov effect
for finite scattering length and applications to ultracold
atoms, see Ref.~\cite{Braaten:2004rn}.}.
The discrete scaling symmetry becomes also manifest in the log-periodic
dependence of three-body observables on the scattering length.
This log-periodic behavior is the hallmark signature of a
renormalization group limit cycle. 
It has been observed experimentally 
in the three-body recombination spectra of ultracold atomic gases 
close to a Feshbach resonance~\cite{Ferlaino:2010viw,PhysRevA.93.022707}.

%%%%%%%%%%%%%%%%%%%%%%%%%%%%%%%%%%%%%%%%%%%%%%%%%%%%%%%%%%%%%%%%%%%%%%%%
\begin{figure}[t]
\begin{center}
\includegraphics[width=8cm,clip=]{B3LecNote.pdf}
\end{center}
\caption{Unrenormalized three-body energies $B_3$ as a function of
the momentum cutoff $\Lambda$ (solid lines).  
The dotted line indicates the cutoff where a new three-body state appears 
at the particle-dimer threshold (dash-dotted line). The dashed line 
shows a hypothetical renormalized energy. The inset shows the running
of the three-body force $d_0(\Lambda) \sim - H(\Lambda)$ with $\Lambda$.}
\label{fig:B3lambda}
\end{figure}
%%%%%%%%%%%%%%%%%%%%%%%%%%%%%%%%%%%%%%%%%%%%%%%%%%%%%%%%%%%%%%%%%%%%%%%%
The physics of the renormalization procedure is illustrated in 
Fig.~\ref{fig:B3lambda} where we show the unrenormalized 
three-body binding energies $B_3$ in the case of positive scattering length
as a function of the cutoff $\Lambda$ (solid line).
As the cutoff $\Lambda$ is increased, $B_3$ increases
asymptotically as $\Lambda^2$. At a certain cutoff 
(indicated by the dotted line), a new bound state appears at the 
boson-dimer threshold. This pattern repeats every time the cutoff
increases by the discrete scaling factor $\exp(\pi/s_0)$. 
Now assume that we adopt 
the renormalization condition that the shallowest state should have a constant 
energy given by the dashed line. At small values of the cutoff,
we need an attractive three-body force to increase the binding energy 
of the shallowest state as indicated by the arrow. As the cutoff is increased
further, the required attractive contribution becomes smaller and around
$\Lambda a =1.1$ a repulsive three-body force is required (downward arrow). 
Around $\Lambda a=4.25$, a new three-body state appears at threshold
and we cannot satisfy the renormalization condition by keeping the first 
state at the required energy anymore. The number of bound states has changed
and there is a new shallow state in the system. At this
point the three-body force 
turns from repulsive to attractive to move the
new state to the required energy.  The corresponding running of the 
three-body force with the cutoff $\Lambda$ is shown in the inset.
After renormalization, the first state is still present as a deep state 
with large binding energy, but for
threshold physics its presence can be ignored. This pattern goes on 
further and further as the cutoff is increased~\cite{Bedaque:1998km}.


%\input{chap4_nucleons}
\section{Effective field theory for nuclear few-body systems}
\label{sec:EFT-Nucleons}

\subsection{Overview}

Depending on the physics one wishes to describe, there are several effective 
field theories for low-energy nuclear physics to choose from.  They differ in 
the set of effective degrees of freedom, their expansion point (typical 
low-energy scale) and range of applicability.  Chiral effective field theory 
includes nucleons and pions and is designed as an expansion about the so-called 
``chiral limit,'' \ie, the scenario where the quark masses are exactly zero 
such that the pions emerge as exactly massless Goldstone bosons from the 
spontaneous breaking of chiral symmetry.  In reality, the quark-masses are 
nonzero such that the pions become ``pseudo-Goldstone'' bosons with a small 
(compared to typical QCD scales like $\mrho$ or $\MN$) mass $\mpi$.  Chiral EFT 
takes this as a typical low scale so that its power counting is designed for 
momenta of the order $Q\sim\mpi$; we come back to this in 
Sec.~\ref{sec:EFT-Chiral}.

For momenta much smaller than $\mpi$, explicit pion 
exchange cannot be resolved such that these can be regarded as integrated out, 
much like we did explicitly for the pseudoscalar toy model in 
Sec.~\ref{sec:EFT-IntOut-2}.  The resulting ``pionless'' theory has, up to 
long-range forces that we consider in Sec.~\ref{sec:EFT-EM}) only contact 
interactions between nucleons left.  These contact interactions parameterize not 
only unresolved pion exchange, but also that of heavier mesons, for which 
contact terms already exist in Chiral EFT.  Pionless EFT is formally very 
similar to the few-boson EFT discussed in Sec.~\ref{sec:EFT-Bosons}, and it 
despite its simplicity it gives rise to surprisingly rich physics, as we will 
show in this section.

\subsection{Pionless effective field theory}

The neutron-proton S-wave scattering lengths are experimentally determined to 
be about $5.4~\fm$ in the \ThreeSOne channel, and ${-}23.7~\fm$ in the \OneSNot 
channel.  What is special about these numbers is that they are large compared 
to the typical nuclear length scale determined by the pion Compton wavelength, 
$\mpi^{-1}\sim1.4~\fm$.  This estimate comes from the long-range component of 
the nuclear interaction being determined by one-pion exchange.  Na\"ively, if 
we consider the low-energy limit ($NN$ center-of-mass momentum going to zero) 
we expect that we can integrate out the pions and end up with a contact 
interaction scaling with the inverse pion mass, and thus a perturbative EFT 
reproducing natural-sized scattering lengths.  The fact that this is not the 
case is typically interpreted as nature ``choosing'' the fine-tuned scenario 
outlined in Sec.~\ref{sec:EFT-FineTuning}.  In this case, pion 
exchange\footnote{As discussed in Sec.~\ref{sec:EFT-Chiral} this actually has to 
be the exchange of two or more pions, as one-pion exchange does not contribute 
to S-wave scattering at zero energy.} combines with shorter range interactions 
to yield the large S-wave scattering lengths (and the deuteron as an 
unnaturally shallow bound state), implying that nuclear physics is a strongly 
coupled and thus nonperturbative system at low energies.  This is what allows 
us to write down an EFT that is closely related to the one that describes 
strongly interacting bosons.  What governs the physics of low-energy 
observables is to a good appropriation just the fact that the scattering 
lengths are large, so we end up with a short-range EFT much like the one for 
bosons encountered in Sec.~\ref{sec:EFT-Bosons}.  Some rather technical new 
features arise from the fact that nucleons are fermions with spin and isospin.

\subsection{The two-nucleon S-wave system}

The leading-order Lagrangian for pionless EFT can be written as

\begin{equation}
 \LL
 = N^\dagger \left(\ii\partial_0 + \frac{\Laplace}{2\MN} + \cdots\right)N \\
 - C_{0,s} (N^T \hat{P}_s N)^\dagger(N^T \hat{P}_s N) \\
 - C_{0,t} (N^T \hat{P}_t N)^\dagger(N^T \hat{P}_t N)
 + \cdots \,,
\label{eq:L-NN}
\end{equation}
%
with projectors
%
\begin{equation}
 (\hat{P}_t)^i = \sigma^2\sigma^i\tau^2 / \sqrt8 \mathtext{,}
 (\hat{P}_s)^\lambda = \sigma^2\tau^2\tau^\lambda/\sqrt8
\label{eq:P-t-s}
\end{equation}
%
such that $C_{0,s}$ and $C_{0,t}$ refer to the \OneSNot and \ThreeSOne $NN$ 
channels, respectively.  As in previous sections of this chapter, we use 
$\sigma^i$ to denote the Pauli matrices in spin space, and write 
$\idxx{\sigma^i}\alpha\beta$ to refer to their individual entries (with the
upper index referring to the row).  Conversely, we use the notation 
$\idxx{\tau^\lambda}ab$ in isospin space.

With the usual cartesian indices $i,\lambda = 1,2,3$ the projectors for given 
$i$ or $\lambda$ give somewhat unusual combinations of individual states.  For 
example, $np$ configurations are completely 
contained in the $\lambda=3$ isospin component, whereas $nn$ and $pp$ are 
obtained from linear combinations $1\pm\ii2$,  In other words, in order to 
separate the physical states (and likewise to get spin-1 states with $m=0,\pm1$ 
quantum numbers) one should work instead with a spherical basis.  For example, 
if one wants to include isospin-breaking terms, it is convenient to work with 
the projectors
%
\begin{equation}
 (\tilde{P}_s)^{-1}
 = \frac{1}{\sqrt{2}}\left[(\hat{P}_s)^1-\ii (\hat{P}_s)^2\right]
 \mathtext{,}
 (\tilde{P}_s)^{0}
 = (\hat{P}_s)^3
 \mathtext{,}
 (\tilde{P}_s)^{+1}
 = {-}\frac{1}{\sqrt{2}}\left[(\hat{P}_t)^1+\ii (\hat{P}_s)^2\right] \,.
\label{eq:P-s-spherical}
\end{equation}
%
Otherwise, since the difference is a unitary rotation, the choice of basis is 
arbitrary.

\subsubsection{Spin and isospin projection}

To understand why the projectors have been defined as in Eq.~\eqref{eq:P-t-s}, 
it is instructive to calculate the tree-level contribution to the amplitude in a 
given channel.  With all spin and isospin vertices written out, the Feynman rule 
for the four-nucleon vertex in the \ThreeSOne channel is
%
\begin{equation}
 \parbox{7.5em}{\centering\includegraphics[width=6.5em]{C0-Vertex}}
 \sim \; \ii \frac{C_{0,t}}{8}
 \idxx{\sigma^i\sigma^2}\alpha\beta
 \idxx{\tau^2}ab
 \idxx{\sigma^2\sigma^i}\gamma\delta
 \idxx{\tau^2}cd \,,
\label{eq:C0-Vertex}
\end{equation}
%
which is obtained by simply writing out the $(\hat{P}_t)^i$ from 
Eq.~\eqref{eq:P-t-s}.  Furthermore, this diagram has an associated combinatorial 
factor $4$ because there are two possibilities each to contract the in- and 
outgoing legs with external nucleon fields.

In order to calculate the T-matrix, we have to write out the Lippmann--Schwinger 
equation with all indices (and symmetry factors) and then apply the appropriate 
projectors.  For ${}^3S_1$ and isospin $0$, the result should have two free 
spin-$1$ indices, which we label $k$ and $j$ for the in- and outgoing side, 
respectively.  The inhomogeneous term is just the vertex~\eqref{eq:C0-Vertex} 
with an additional factor $4$.  Applying the projectors, we get
%
\begin{multline}
 \frac{1}{\sqrt8} \idxx{\sigma^2\sigma^j}\beta\alpha \idxx{\tau^2}ba
 \times 4
 \times \ii\frac{C_{0,t}}{8}\idxx{\sigma^i\sigma^2}\alpha\beta \idxx{\tau^2}ab
 \idxx{\sigma^2\sigma^i}\gamma\delta \idxx{\tau^2}cd
 \times \frac{1}{\sqrt8} \idxx{\tau^2}dc
 \idxx{\sigma^k\sigma^2}\delta\gamma \\
 = \ii\frac{C_{0,t}}{16} \, \Tr\big(\sigma^j\sigma^i\big)
 \, \Tr\big(\tau^2\tau^2\big)
 \, \Tr\big(\sigma^i\sigma^k\big) \, \Tr\big(\tau^2\tau^2\big)
 = \ii C_{0,t} \, \delta^{jk} \,,
\end{multline}
%
where we have used the well-known of the Pauli matrices.  This is 
exactly the expected result: the projectors~\eqref{eq:P-t-s} have been 
constructed to give this.  The projection of other, more complicated diagram 
works in the same way.  Albeit somewhat tedious, it is technically 
straightforward.  For higher partial waves, one of course has to take into 
account the coupling of spin and orbital angular momentum.

\subsubsection{Dibaryon fields}
\label{sec:EFT-Dibaryons}

Just like for bosons, it is convenient to introduce auxiliary dimer---now 
called dibaryon---field for each of the $NN$ S-wave states.  This is done by 
rewriting the Lagrangian~\eqref{eq:L-NN} as
%
\begin{multline}
 \LL
 = N^\dagger \left(\ii\partial_0 + \frac{\Laplace}{2\MN} + \cdots\right)
 - t^{i\dagger}\left[\sigt
 + \left(\ii \partial_0+\frac{\vNabla^2}{4\MN}\right)\right]t^i
 + \yt\left[t^{i\dagger}\left(N^T P^i_t N\right)+\hc\right] \\
 - s^{\lambda\dagger}\left[\sigs
 \left(\ii \partial_0+\frac{\vNabla^2}{4\MN}\right)\right]s^\lambda
 + \ys\left[s^{\lambda\dagger}\left(N^T P^\lambda_s N\right)+\hc\right]
 + \cdots \,,
\label{eq:L-NN-d}
\end{multline}
%
where $t$ ($s$) denotes a \ThreeSOne (\OneSNot) dibaryon field and the 
projection operators $P_{s,t}$ are as defined in Eq.~\eqref{eq:P-t-s}.  The 
``bare'' dibaryon propagators are just $\ii/\sigst$, while the full 
leading-order expressions are obtained by summing up all nucleon bubble 
insertions.  This resummation, which without dibaryon fields gives the $NN$ 
T-matrix as a bubble chain, reflects the fact we need to generate shallow 
states (the bound deuteron and the virtual \OneSNot state) to account for the 
unnaturally large $NN$ scattering lengths.  Pionless EFT is designed to capture 
this feature.

Omitting spin-isospin factors for simplicity, the resummed propagators are
%
\begin{equation}
 \ii\Delta_{\st}(p_0,{\mathbf{p}})
 = \frac{-\ii}{\sigst + \yst^2 I_0(p_0,{\mathbf{p}})} \,,
\end{equation}
%
where
%
\begin{multline}
 I_0(p_0,{\mathbf{p}}) = \MN\int^\Lambda \frac{\dd^3q}{(2\pi)^3}
 \frac{1}{\MN p_0 - {\mathbf{p}}^2/4 - \vq^2 + \ii\eps}
 = {-}\frac{\MN}{4\pi}
 \left(\frac{2\Lambda}{\pi} - \sqrt{\frac{{\mathbf{p}}^2}{4}-\MN p_0-\ii\eps}\right)
 +\OO(1/\Lambda)
\label{eq:I0-cutoff}
\end{multline}
%
is the familiar bubble integral regularized with a momentum cutoff.  The cutoff 
dependence is absorbed into the parameters $\yst$ and $\sigst$ to obtain the 
renormalized propagators.  Attaching external nucleon legs gives the $NN$ 
T-matrix,
%
\begin{equation}
 \ii T_\st(k)
 = (\ii \yst)^2\, \ii\Delta_\st\!\left(p_0=k^2/\MN,{\mathbf{p}}=\vZero\right)
 = \frac{4\pi}{M_N}\frac{\ii}{k\cot\delta_\st(k) - \ii k}\,,
\label{eq:Tnd}
\end{equation}
%
so we can match to the effective range expansions for $k\delta_\st(k)$.  At 
leading order, the renormalization condition is to reproduce
$k\cot\delta_\st(k) = {-}1/{a_\st} + \OO(k^2)$, which gives
%
\begin{equation}
 \frac{4\pi\sigst}{\MN\yst^2} = {-}\frac{1}{a_\st} + \frac{2\Lambda}{\pi} \,.
\label{eq:NN-renorm-0}
\end{equation}
%
Instead of this standard choice of the expansion around the zero-energy 
threshold, it is convenient to expand the \ThreeSOne channel around the 
deuteron pole.\footnote{The shallow deuteron bound-state pole is within the 
radius of convergence of the effective range expansion. The deuteron binding 
momentum is $\gamt=1\at + \cdots$, where the ellipses include corrections from 
the effective range (and higher-order shape parameters).}  This is
%
\begin{equation}
 k\cot\delta_t(k) = \gamt + \dfrac{\rhot}2\big(k^2+\gamt^2\big) + \cdots \,,
\end{equation}
%
where $\gamt = \sqrt{\mathstrut\MN \Bd}\simeq 45.7$ is the deuteron binding 
momentum and $\rhot\simeq 1.765$ is the ``deuteron effective range.''  This 
choice, which sets
%
\begin{equation}
 \frac{4\pi\sigt}{\MN\yt^2} = {-}\gamt + \frac{2\Lambda}{\pi} \,,
\label{eq:NN-renorm-deut}
\end{equation}
%
gets the exact deuteron pole position right at leading order, but is equivalent 
to the choice in Eq.~\eqref{eq:NN-renorm-0} up to range corrections.

\paragraph{Wavefunction renormalization}

The residue at the deuteron pole gives the deuteron wavefunction 
renormalization.  We find
%
\begin{equation}
 Z_t^{-1} = \ii\frac{\partial}{\partial p_0}
 \left.\frac{1}{\ii\Delta_t(p_0,{\mathbf{p}})}\right|_{p_0=-\frac{\gamt^2}{\MN},\,{\mathbf{p}}=0}
 = \frac{\MN^2 \yt^2}{8\pi\gamt}
 \;\Rightarrow\;
 \sqrt{Z_t} = \frac{1}{\yt} \frac{\sqrt{8\pi\gamt\mathstrut}}{\MN}
\label{eq:Zd}
\end{equation}
%
for the renormalization as in Eq.~\eqref{eq:NN-renorm-deut}.  If we 
directly consider the (off-shell) T-matrix near the pole, we find
%
\begin{spliteq}
 T(E) &= \frac{4\pi}{\MN}
 \frac{1}{-\gamt+\sqrt{-\MN E-\ii\eps\mathstrut}}
 = {-}\frac{4\pi}{\MN}
 \frac{\sqrt{-\MN E-\ii\eps\mathstrut}+\gamt}{-\MN E-\gamd^2}
 \mathtext{for $\eps\to0$} \\
 &\sim
 {-}\frac{8\pi\gamt}{\MN^2} \frac{1}{E+\dfrac{\gamt^2}\MN}
 \mathtext{as $E\to-\gamt^2/\MN$} \,.
\label{eq:T-deut-pole}
\end{spliteq}
%
Comparing to the standard factorization at the pole,\footnote{The minus sign in 
Eq.~\eqref{eq:T-factorization} is a consequence of the convention we use here 
for the T-matrix.}
%
\begin{equation}
 T(k,p;E) = {-}\frac{B^\dagger(k)B(p)}{E+E_B}
 + \text{terms regular at $E=-E_B$} \,,
\label{eq:T-factorization}
\end{equation}
%
we can read off from Eq.~\eqref{eq:T-deut-pole} that
%
\begin{equation}
 B(p) = \sqrt{\frac{8\pi\gamt}{\MN^2}} = \yt\sqrt{Z_t} \,,
\label{eq:B-deut}
\end{equation}
%
independent of momentum at this order.

\subsection{Three nucleons: scattering and bound states}

As done in Sec.~\ref{sec:EFT-ThreeBosons} for bosons, the dimer/dibaryon 
formalism allows for a particularly intuitive and simple description of the 
three-body system.  Looking at nucleon-deuteron S-wave scattering\footnote{We 
work in the isospin-symmetric theory here, but in the absence of 
electromagnetic interactions (discussed in Sec.~\ref{sec:EFT-EM}), the nucleon 
here should be thought of as a neutron.}, we find that the spin $1$ of the 
deuteron can couple with the spin $1/2$ of the nucleon to a total spin of 
either $3/2$ or $1/2$.  These two cases are referred to as the quartet and 
doublet channel, respectively.

\subsubsection{Quartet channel}
\label{sec:EFT-NdQuartet}

%%%%%%%%%%%%%%%%%%%%%%%%%%%%%%%%%%%%%%%%%%%%%%%%%%%%%%%%%%%%%%%%%%%%%%%%%%%%%%
\begin{figure}[htbp]
\centering
\includegraphics[clip,width=0.5\textwidth]{nd-IntEq-Q}
\caption{$Nd$ quartet-channel integral equation.  Nucleons and deuterons are 
represented as single and double lines, respectively.  The blob represents the 
T-matrix.}
\label{fig:nd-IntEq-Q}
\end{figure}
%%%%%%%%%%%%%%%%%%%%%%%%%%%%%%%%%%%%%%%%%%%%%%%%%%%%%%%%%%%%%%%%%%%%%%%%%%%%%%

In the quartet-channel, the spins of all three nucleons have to be aligned to 
produce the total spin $3/2$.  This means that only the deuteron field can 
appear in intermediate states, and the Pauli principle excludes a three-body 
contact interaction without derivatives.  The resulting integral equation for 
the $Nd$ T-matrix is shown diagrammatically in Fig.~\ref{fig:nd-IntEq-Q}.  
Compared to Fig.~\ref{fig:inteq12}, we use a different convention here were the 
T-matrix blob is drawn to the left of the nucleon exchange, and we denote in- 
and outgoing momenta (in the $Nd$ center-of-mass frame) by $\pm\vk$ and 
$\pm{\mathbf{p}}$, respectively.\footnote{Unlike what is done in 
Sec.~\ref{sec:EFT-Bosons}, we also read diagrams from left (incoming particles) 
to right (outgoing particles).  Both conventions can be found in the 
literature.}  The energy and momentum dependence are exactly the same as for 
bosons, but we have to include the additional spin-isospin structure from the 
vertices.  Doing this, the result in its full glory reads:
%
\begin{multline}
 \idxx{\ii T_q^{ij}}{\beta b}{\alpha a}(\vk,{\mathbf{p}};E)
 = -\frac{\ii\MN\yt^2}{2}\,\idxx{\sigma^j\sigma^i}\beta\alpha\delta^b_a
 \,\frac{1}{\vk^2+\vk\cdot{\mathbf{p}}+{\mathbf{p}}^2-\MN E-\ii\eps}\\
 +\int\frac{\dd^3 q}{(2\pi)^3} \,
 \Delta_t\left(E-\frac{\vq^2}{2\MN},\vq\right)
 \idxx{\ii T_q^{ik}}{\gamma c}{\alpha a}(E;\vk,\vq)\\
 \times\frac{\MN\yt^2}{2}
 \frac{\idxx{\sigma^j\sigma^k}\beta\gamma\delta^b_c}
 { \vq^2+\vq\cdot{\mathbf{p}}+{\mathbf{p}}^2-\MN E-\ii\eps} \,.
\label{eq:nd-IntEq-Q-raw}
\end{multline}
%
This unprojected amplitude carries spin and isospin indices for the various 
fields in the initial and final states.  To select the overall spin $3/2$ 
contribution, we take linear combinations as in Eq.~\eqref{eq:P-s-spherical} to 
select the maximal projections for the in- and outgoing deuterons and 
$\alpha=\beta=1$ to get nucleons with spin orientation ${+}1/2$.  We also set 
$a=b=2$ to select neutrons.  Altogether, this gives
%
\begin{equation}
 \idxx{\sigma^j\sigma^i}\beta\alpha\delta^b_a \rightarrow 2 \,,
\end{equation}
%
in the inhomogeneous term, and the same factor for the integral part.  The 
fully projected quartet-channel amplitude is
%
\begin{equation}
 T_q = \frac{1}{2}\idxxx{T_q^{11} + \ii\left(T_q^{12} - T_q^{21}\right)
 + T_q^{22}}{12}{12} \,.
\label{eq:T-q}
\end{equation}

\begin{prob}
\emph{Exercise:} Work out the details leading to Eq.~\eqref{eq:T-q}.
\end{prob}

Finally, the S-wave projection of Eq.~\eqref{eq:nd-IntEq-Q-raw} is done by 
applying the operator $\frac12\int_{-1}^{1}\dd\cos\theta$, where $\theta$ is 
the angle between $\vk$ and ${\mathbf{p}}$.  Introducing a momentum cutoff $\Lambda$, 
the resulting equation can be solved numerically by discretizing the 
remaining one-dimensional integral.  From the result, which we denote by 
$T_q^0$, we can calculate observables like scattering phase shifts,
%
\begin{equation}
 \delta_{q}(k) = \frac{1}{2\ii}
 \ln\!\left(1+\frac{2\ii k\MN}{3\pi} Z_t T_q^0(k,k;E_k)\right)
 \mathtext{,} E_k = \frac{3k^2}{4\MN}-\frac{\gamt^2}{\MN} \,,
\label{eq:delta-q}
\end{equation}
%
or the $Nd$ scattering length:
%
\begin{equation}
 a_q = \frac{\MN}{3\pi}\lim\nolimits_{k\to0} Z_t T_q^0(k,k;E_k) \,.
\label{eq:and-2}
\end{equation}
%
Note that we have not absorbed the wavefunction renormalization $Z_t$ into 
$T_q^0$ but instead chose to keep it explicit in Eqs.~\eqref{eq:delta-q} 
and~\eqref{eq:and-2}

\subsubsection{Doublet channel}
\label{sec:EFT-NdDoublet}

%%%%%%%%%%%%%%%%%%%%%%%%%%%%%%%%%%%%%%%%%%%%%%%%%%%%%%%%%%%%%%%%%%%%%%%%%%%%%%
\begin{figure}[htbp]
\centering
\includegraphics[clip,width=0.65\textwidth]{nd-IntEq-D}
\caption{$Nd$ doublet-channel integral equation.  As in 
Fig.~\ref{fig:nd-IntEq-Q}, nucleons and deuterons are drawn as single and 
double lines, respectively.  Additionally, we represent the \OneSNot dibaryon 
as a thick line.  The hatched and shaded blobs are the two components of the 
doublet-channel $Nd \to Nd$ T-matrix.}
\label{fig:nd-IntEq-D}
\end{figure}
%%%%%%%%%%%%%%%%%%%%%%%%%%%%%%%%%%%%%%%%%%%%%%%%%%%%%%%%%%%%%%%%%%%%%%%%%%%%%%

The doublet channel (total spin $1/2$) has a richer structure, since now also 
the \OneSNot dibaryon can appear as an intermediate state.  The result is a 
coupled channel integral equation, shown diagrammatically in 
Fig.~\ref{fig:nd-IntEq-D}.  We skip here the technicalities of the spin-isospin 
projection (for details, see Ref.~\cite{Konig:2014ufa} and earlier references 
therein) and merely quote the result in a compact notation:
%
\begin{equation}
 \twodvec{T_{d,1}^0 \\ T_{d,2}^0}
 = \twodvec{\MN\yt^2/2 \\ {-}3\MN\yt\ys/2} K
 + \twodmat{D_t & 0 \\ 0 & D_s}
 \twodmat{{-}\MN\yt^2/2 & 3\MN\yt\ys/2 \\ 3\MN\ys\yt/2 & {-}\MN\ys^2/2} K
 \otimes \twodvec{T_{d,1}^0 \\ T_{d,2}^0} \,,
\label{eq:nd-IntEq-noH}
\end{equation}
%
where $K$ is the ``kernel'' function 
%
\begin{equation}
 K(k,p;E) = \frac{1}{2kp}\ln\!\left(
 \frac{k^2+p^2+kp-\MN E-\ii\eps}{k^2+p^2-kp-\MN E-\ii\eps}\right) \,,
\label{eq:KS}
\end{equation}
%
\begin{equation}
 D_\st(q;E) = \Delta_\st\!\left(E-\frac{q^2}{2\MN};q\right) \,,
\label{eq:D-st}
\end{equation}
%
and the integral operation
%
\begin{equation}
 A \otimes B \equiv \frac1{2\pi^2}
 \int_0^\Lambda\dd q\,q^2\,A(\ldots,q)B(q,\ldots)
\label{eq:SH-int}
\end{equation}
%
has to be applied within each block.  Just like the quartet-channel equation, 
Eq.~\eqref{eq:nd-IntEq-noH} can be solved numerically by discretizing the 
integrals, with the additional complication that we now have a $2\times2$ block 
matrix.  The T-matrix likewise becomes a $2$-block vector, the upper part 
of which is gives the physical $Nd \to Nd$ amplitude.\footnote{Note that this 
vector is one part of the more general full off-shell amplitude, which is 
a $2\times2$ block matrix including the two combinations of dibaryon legs that 
do not appear in Fig.~\ref{fig:nd-IntEq-D}.}

\begin{prob}
\emph{Exercise:} Express the fully projected integral equation for the quartet 
channel amplitude $T_q^0$ using the compact notation based on 
Eqs.~\eqref{eq:KS}, \eqref{eq:D-st}, and~\eqref{eq:SH-int}.
\end{prob}

\paragraph{Leading-order three-nucleon force}

Studying the doublet-channel solution as a function of increasing UV cutoff 
$\Lambda$, one finds that there is no stable limit as $\Lambda\to\infty$.  
Instead, the amplitude changes wildly as $\Lambda$ is varied.  This is much 
unlike the quartet-channel case, which shows a rapid convergence with $\Lambda$.

The origin of this behavior was explained by 
Bedaque~\etal~\cite{Bedaque:1999ve}.  The behavior for large $\Lambda$ is 
governed by large momenta, which means that infrared scales like the scattering 
lengths do not matter.  Indeed, one finds that $D_t(E;q)$ and $D_s(E;q)$ have 
the same leading behavior as $q\to\infty$.  To analyze the asymptotic behavior 
of the amplitude we can thus go to the $SU(4)$ spin-isospin symmetric limit and 
set $D_t = D_t \equiv D$ as well as $\yt = \ys \equiv \y$.  In this limit, the 
two integral equations in Eq.~\eqref{eq:nd-IntEq-noH} can be decoupled by 
defining $T_{d,\pm} = T_{d,1} \pm T_{d,2}$.  For $T_{d,+}$ we find the integral 
equation
%
\begin{equation}
 T_{d,+}^0(k,p;E) = {-}\MN\y^2 K(k,p;E)
 + \MN\y^2 \int_0^\Lambda \frac{\dd q}{2\pi^2}
 \, q^2 \, K(k,q;E) D(q;E) T_{d,+}^0(q,p;E) \,,
\end{equation}
%
which is formally exactly the same as the three-boson integral equation, 
Eq.~\eqref{BHvK}, in the absence of a three-body force.  As discussed in 
Sec.~\ref{sec:EFT-ThreeBosons}, this equation does not have a unique solution 
in the limit $\Lambda\to\infty$.  Since $T_{d,1/2}$ are linear combinations of 
involving $T_{d,+}$, they inherit the same behavior.  But the cure is now 
obvious: a three-nucleon force, which by naïve counting would only enter at 
higher orders, has to be promoted to leading order in order to make 
Eq.~\eqref{eq:nd-IntEq-noH} well defined.  This three-nucleon force, like the 
asymptotic amplitudes, is $SU(4)$ symmetric (invariant under arbitrary 
spin-isospin rotations) and can be written as
%
\begin{equation}
 \mathcal{L}_3 = \frac{h_0}{3}N^\dagger\left[\yt^2\,
 t^{i\dagger} t^j \sigma^i \sigma^j+\ys^2\,s^{A\dagger} s^B \tau^A\tau^B
 - \yt\ys\left(t^{i\dagger} s^A \sigma^i \tau^A + \hc\right) \right]N \,,
\label{eq:L-3}
\end{equation}
%
where the cutoff-running of the three-nucleon coupling is analogous to 
what we derived for bosons:
%
\begin{equation}
 h_0 = \frac{\MN H(\Lambda)}{\Lambda^2} \,.
\end{equation}
%
Including this, the coupled doublet-channel integral equation becomes
%
\begin{equation}
 \twodvec{T_{d,1}^0 \\ T_{d,2}^0}
 = \twodvec{\MN\yt^2/2 \\ {-}3\MN\yt\ys/2} K
 + \twodmat{D_t & 0 \\ 0 & D_s}
 \twodmat{{-}\MN\yt^2/2(K + h_0) & 3\MN\yt\ys/2 (K + h_0) \\
 3\MN\ys\yt/2 (K + h_0) & {-}\MN\ys^2/2 (K + h_0)}
 \otimes \twodvec{T_{d,1}^0 \\ T_{d,2}^0} \,.
\label{eq:nd-IntEq}
\end{equation}
%
We stress that the requirement to include this three-body force at 
leading-order is a feature of the nonperturbative physics that can be traced 
back to the large $NN$ scattering lengths.  All loop diagrams obtained by 
iterating the integral equation are individually finite as $\Lambda\to\infty$, 
yet their infinite sum does not exist in that limit unless the $h_0$ contact 
interaction is included.

\paragraph{The Phillips line}

The form of $H(\Lambda)$ is given by Eq.~\eqref{H-Lambda}, but since there is a 
prefactor that depends on the details of the regularization scheme, in practice 
one has to determine the appropriate value numerically after choosing a cutoff 
$\Lambda$.  This requires a three-body datum as input, which is conveniently 
chosen to be the triton binding energy---the \ThreeH bound state corresponds to 
a pole in $T_d$ at $E = {-}E_B(\ThreeH)$, \cf~Sec.~\ref{sec:EFT-3BObs}---or the 
doublet-channel $nd$ scattering length (or, in principle, a phase shift at some 
fixed energy).  Once one of these is fixed, the rest can be predicted.  In 
particular, this means that the existence of a single three-body parameter in 
pionless EFT at leading order provides a natural explanation of the Phillips 
line, \ie, the observation that various phenomenological potentials, which 
were all tuned to produce the same $NN$ phase shifts, gave different results 
for the triton binding energy and doublet S-wave scattering length, which 
however are strongly correlated.  In our framework, we can obtain the line 
shown in Fig.~\ref{fig:Phillips} by fitting $H(\Lambda)$ to different values of 
the scattering length and then calculating $E_B(\ThreeH)$ (or vice versa).  
Alternatively, one can find the same curve by setting $h_0$ to zero and varying 
$\Lambda$ to move along the line.

%%%%%%%%%%%%%%%%%%%%%%%%%%%%%%%%%%%%%%%%%%%%%%%%%%%%%%%%%%%%%%%%%%%%%%%%%%%%%%
\begin{figure}[tb]
\centering
\includegraphics[width=0.65\textwidth,clip]{Phillips}
\caption{Correlation (Phillips line) between \Triton binding energy (in \MeV)
and doublet $nd$ scattering length (in \fm) in leading-order pionless EFT.}
\label{fig:Phillips}
\end{figure}
%%%%%%%%%%%%%%%%%%%%%%%%%%%%%%%%%%%%%%%%%%%%%%%%%%%%%%%%%%%%%%%%%%%%%%%%%%%%%%

\medskip
With this, we conclude our discussion the three-nucleon system.  For more 
details, like calculations beyond leading order, we encourage the reader to 
encourage the literature.

%\input{chap4_beyond}
\section{Beyond short-range interactions: adding photons and pions}
\label{sec:EFT-Beyond}

\subsection{Electromagnetic interactions}
\label{sec:EFT-EM}

So far, we have studied only effective theories where the particles (atoms of 
nucleons) interact solely via short-range force (regulated contact 
interactions).  While this is, as we have argued, sufficient to describe the 
strong nuclear interactions at (sufficiently low) energies where the EFT is 
valid, the electromagnetic interaction does not fit in this scheme.  
Nevertheless, since almost all systems of interest in low-energy nuclear 
physics involve more than one proton, the inclusion of such effects 
is of course important.

Any coupling of photons to charged particles has to be written down in a 
gauge-invariant way.  A natural way to ensure gauge invariance is to replace 
all derivatives in the effective Lagrangian with covariant ones, \ie,
%
\begin{equation}
 \partial_\mu \rightarrow D_\mu = \partial_\mu + \ii eA_\mu \hat{Q} \,,
\label{eq:D-mu}
\end{equation}
%
where $\hat{Q}$ is the charge operator (for nucleons, for example, 
$\hat{Q}_N=(\one+\tau_3)/2$).  Moreover $e^2 = 4\pi\alpha$ defines the electric 
unit charge in terms of the fine structure constant $\alpha\approx1/137$.

Since in the EFT there is an infinite tower of contact interactions with an 
increasing number of derivatives, we also get an infinite number of photon 
vertices.  Still, merely plugging in the covariant derivative is not enough 
(after all, this is called \emph{minimal} substitution); it is possible to write 
down terms which are gauge invariant by themselves, and for the EFT to be 
complete these have to be included as well.  Before we come back to this in 
Sec.~\ref{sec:EFT-ExtCurrents}, let us first see what we get from gauging the 
derivatives in the nucleon kinetic term:
%
\begin{equation}
 N^\dagger \left(\ii\partial_t + \frac{\Laplace}{2\MN} + \cdots\right)N
 \rightarrow
 N^\dagger \left(\ii D_t + \frac{\vD^2}{2\MN} + \cdots\right)N \,.
\end{equation}
%
In addition to this, we also have to include the photon 
kinetic and gauge fixing terms to complete the electromagnetic sector.  A 
convenient choice for the our nonrelativistic framework is Coulomb gauge, \ie, 
demanding that $\vNabla\cdot\vA=0$.  A covariant way to write is condition is
%
\begin{equation}
 \partial_\mu A^\mu-\eta_\mu\eta_\nu\partial^\nu A^\mu = 0
\label{eq:Coulomb-Gauge}
\end{equation}
%
with the timelike unit vector $\eta^\mu=(1,0,0,0)$.  Hence, we add to our 
effective Lagrangian the term
%
\begin{equation}
 \mathcal{L}_\mathrm{photon} = -\frac14 F_{\mu\nu}F^{\mu\nu} -\frac{1}{2\xi}
 \left(\partial_\mu A^\mu-\eta_\mu\eta_\nu\partial^\nu A^\mu\right)^2 \,.
\label{eq:L-photon}
\end{equation}

\subsubsection{The Coulomb force}

From Eq.~\eqref{eq:L-photon} we get the equation of motion for the photon field 
as
%
\begin{equation}
 \left[\dAlem g^{\mu\nu} - \partial^\mu\partial^\nu
 + \frac{1}{\xi}\left(\partial^\mu\partial^\nu
 - \eta^\nu\eta_\kappa\partial^\mu\partial^\kappa 
 - \eta^\mu\eta_\lambda\partial^\lambda\partial^\nu
 + \eta^\mu\eta^\nu\eta_\lambda\eta_\kappa\partial^\lambda\partial^\kappa
 \right)\right] A_\nu(x) = 0 \,.
\end{equation}
%
The photon propagator is defined as the Feynman Green's function for the 
differential operator acting on $A_\nu(x)$.  Writing down the general solution 
in momentum space and choosing $\xi = 0$ at the end (recall that it is an 
artificial parameter introduced through enforcing the gauge condition in the  
path integral~\cite{Peskin:1995}), we get
%
\begin{equation}
 D_\gamma^{\mu\nu}(k) = \frac{-\ii}{k^2+\ii\eps}
 \left(g^{\mu\nu}+\frac{k^\mu k^\nu + k^2\eta^\mu\eta^\nu
 - (k\cdot\eta)(k^\mu\eta^\nu+\eta^\mu k^\nu)}{(k\cdot\eta)^2-k^2}
 \right) \,.
\label{eq:D-gamma-gen}
\end{equation}
%
A simple inspection of which shows that it vanishes if $\mu$ or $\nu=0$.
In other words, $A_0$ photons do not propagate.  Correspondingly, their equation 
of motion becomes time independent and we can use it to remove $A_0$ from the 
effective Lagrangian (the nuclear part plus $\mathcal{L}_\mathrm{photon}$).  
We find
%
\begin{equation}
 \Laplace A_0 = {-}e N^\dagger \hat{Q} N \,,
\label{eq:Poisson}
\end{equation}
%
which is just the Poisson equation with the nucleon charge density on the 
right-hand side.  Solving this in Fourier space, where $\vNabla^2$ turns into 
the squared three-momentum, we eventually get a term
%
\begin{equation}
 \LL_{\text{Coulomb}}(x_0,\vx)
 = -e^2 \int\dd^3y \,
 N^\dagger(x_0,\vx)N(x_0,\vx) \,
 \frac{\eex^{-\ii\vq\cdot(\vx-\vy)}}{\vq^2}
 \left(\frac{\one + \tau^3}{2}\right)
 N^\dagger(x_0,\vy)N(x_0,\vy) \,,
\label{eq:L-Coulomb}
\end{equation}
%
\ie, static Coulomb potential ($\sim 1/r$ in configuration space) 
between charged nucleons.  This is a non-local interaction that really should 
be kept in the Lagrangian as a whole.  Still, to calculate Feynman diagrams it 
is convenient to split it up into a vertex
%
\begin{equation}
 \parbox{12em}{\includegraphics[width=13em]{Photon-Vertex-0}}
 \sim \; {-}{\ii e} \idxxx{\frac{\one + \tau^3}{2}}ab
 \delta\idx\alpha\beta \,,
\label{eq:Photon-Vertex-0}
\end{equation}
%
and a factor $\ii/\vk^2$ for each ``Coulomb-photon propagator,'' which is 
really just an expression for the static potential in momentum space.  Note 
that the sign in Eq.~\eqref{eq:Photon-Vertex-0} is arbitrary because these 
vertices really only come in pairs; we have chosen it here to coincide with 
what one would na\"ively read off from the $N^\dagger \ii D_0 N$ term.

Finding that $A_0$ photons only appear as static internal exchanges goes well in 
line that physical photons in external states have to be transverse.  Still, 
one might wonder how to treat diagrams with virtual photons coupled to a 
nuclear system (\eg, electrodisintegration), which should have $A_0$ 
contributions.  The answer is that the proper way to treat these is to add 
appropriate external currents to the Lagrangian, which then, in turn, appear on 
the right-hand side of Eq.~\eqref{eq:Poisson}.

\subsubsection{Coulomb enhancement and divergences}

The Coulomb force is a long-range interaction: it only falls off like a power 
law ($\sim 1/r$) in configuration space.  In momentum space, it correspondingly 
has a pole at vanishing momentum transfer ($\vq^2=0$), \ie, it gives rise to an 
infrared divergence.  There are standard techniques for dealing with this, 
for example defining so-called Coulomb-modified scattering phase shifts and 
effective-range expansions well-known from quantum-mechanical scattering 
theory.  These are based on treating Coulomb effects fully nonperturbatively, 
\ie, resumming to exchange of Coulomb photons to all orders.  In the EFT power 
counting, the need for this resummation is reflected in the fact dressing a 
given two-body scattering diagram with external momenta of order $Q$ by an 
additional Coulomb photon gives a factor $\alpha\MN/Q$, \ie, an 
\emph{enhancement} if $Q \lsim \alpha\MN$.  In Figure~\ref{fig:OneTwoPhoton} we 
show this for two-photon exchange compared to the single-photon diagram.

%%%%%%%%%%%%%%%%%%%%%%%%%%%%%%%%%%%%%%%%%%%%%%%%%%%%%%%%%%%%%%%%%%%%%%%%%%%%%%
\begin{figure}[htbp]
\centering
\begin{minipage}{0.25\textwidth}
 \centering
 \vcenteredhbox{\includegraphics[height=6em,clip]{OnePhotonExchange}}
 \vcenteredhbox{\scalebox{1.5}{$\;\sim\;\frac{\alpha}{Q^2}$}}
\end{minipage}
\hspace{0.5em}
\begin{minipage}{0.65\textwidth}
 \centering
 \vcenteredhbox{\includegraphics[height=6em,clip]{TwoPhotonExchange}}
 \vcenteredhbox{\scalebox{1.5}{$\;\sim\;
 \frac{Q^5}{\MN}
 \left(\frac{\alpha}{Q^2}\right)^{\!2}
 \left(\frac{\MN}{Q^2}\right)^{\!2}
 = \frac{\alpha}{Q^2}\times\frac{\alpha\MN}{Q}$}}
\end{minipage}
\caption{Infrared enhancement of Coulomb-photon exchange.}
\label{fig:OneTwoPhoton}
\end{figure}
%%%%%%%%%%%%%%%%%%%%%%%%%%%%%%%%%%%%%%%%%%%%%%%%%%%%%%%%%%%%%%%%%%%%%%%%%%%%%%

If a problem with Coulomb interactions is solved numerically, the 
IR divergence has to be regularized in some way to have all quantities well 
defined.  One way to do this ``screening'' the potential with a photon mass 
$\lambda$, \ie, replacing $\vq^2$ with $\vq^2 + \lambda^2$ in the 
Coulomb-photon propagator.  If this is done, $\lambda$ should be kept as small 
as possible and be extrapolated to zero for all physical observables at the end 
of the calculation.

%%%%%%%%%%%%%%%%%%%%%%%%%%%%%%%%%%%%%%%%%%%%%%%%%%%%%%%%%%%%%%%%%%%%%%%%%%%%%%
\begin{figure}[htbp]
\centering
\includegraphics[clip,width=0.25\textwidth]{SinglePhotonBubble}
\caption{Single photon insertion in a proton-proton bubble diagram.}
\label{fig:SinglePhotonBubble}
\end{figure}
%%%%%%%%%%%%%%%%%%%%%%%%%%%%%%%%%%%%%%%%%%%%%%%%%%%%%%%%%%%%%%%%%%%%%%%%%%%%%%

In contrast to what one might na\"ively expect, Coulomb contributions can 
also modify the UV behavior of diagrams.  Consider, for example the insertion 
of a single photon within a bubble contributing to proton-proton scattering, as 
shown in Fig.~\ref{fig:SinglePhotonBubble}.  Power counting momenta in this 
diagram (with $C_0$ vertices) gives a factor $(Q^5/\MN)^2$ from the two loops, 
$(\MN/Q^2)^4$ from the nucleon propagators, and an obvious $\alpha/Q^2$ from 
the photon, meaning that overall we have $Q^0$, corresponding to a superficial 
logarithmic divergence of this diagram (which one can confirm with an explicit 
calculation).  This is a new feature compared to the theory with only 
short-range interactions, where this particular divergence is absent (all 
divergences are of power-law type there).  Without going into any details, we 
stress that this divergence has to be accounted for when the theory with 
Coulomb contributions is renormalized.  In particular, this example teaches us 
that in the EFT the Coulomb force is not merely an ``add-on potential'' that 
slightly shifts results, but that it has to be treated consistently along with 
the short-range interactions.

\subsubsection{Transverse photons}

Transverse photons come from the quadratic spatial derivative:
%
\begin{equation}
 \vec{D}^2 = (\partial^i + \ii e A^i \hat{Q})(\partial^i + \ii e A^i \hat{Q})
 = \Laplace
 + \ii e (A^i\hat{Q}\,\partial^i + \partial^i A^i \hat{Q})
 - e^2 A^2 \hat{Q} \,.
\end{equation}
%
We can rewrite this using
%
\begin{equation}
 \partial^i \left(A^i \hat{Q} N\right)
 = (\partial^i A^i) \hat{Q} N + A^i \hat{Q} \, \partial^i N \,,
\end{equation}
%
where the first term vanishes in Coulomb gauge.  Hence, we get the vertex
%
\begin{equation}
 \parbox{12em}{\includegraphics[width=13em]{Photon-Vertex-trans}}
 \sim \; {-}\frac{e}{\MN} \idxxx{\frac{\one + \tau^3}{2}}ab
 \delta\idx\alpha\beta \, (\ii\vq)^n \,,
\label{eq:Photon-Vertex-gauge}
\end{equation}
%
with the momentum dependence coming from the derivative.  We leave it as an 
exercise to write down the Feynman rule for the two-photon term $\sim A^2 
\hat{Q}$.

Comparing Eq.~\eqref{eq:Photon-Vertex-gauge} with Eq.~\eqref{eq:Photon-Vertex-0}
that a diagram with the exchange of a transverse photon is suppressed by a 
factor $Q^2/\MN^2$ compared to the same topology with a Coulomb photon.
Transverse photons also have a more complicated propagator than the simple
$\ii/\vk^2$ that we found for Coulomb photons.  From Eq.~\eqref{eq:D-gamma-gen} 
we find that
%
\begin{equation}
 \parbox{8.25em}{\includegraphics[width=7.75em]{Photon-Prop-gauge}}
 \sim \; \frac{\ii}{k_0^2 - \vk^2 + \ii\eps}
 \left(\delta^{mn} - \frac{k^m k^n}{\vk^2}\right) \,.
\label{eq:Photon-Prop-gauge}
\end{equation}
%
which now depends on the energy $k_0$ and thus gives rise to poles in Feynman 
diagrams.  Note that the structure is somewhat different from what one 
typically sees in QED textbooks that use Lorenz/Feynman gauge.

\subsubsection{Other external currents}
\label{sec:EFT-ExtCurrents}

The covariant derivative alone only gives us photons coupled to the proton's 
charge.  However, as mentioned previously, minimal substitution only gives a 
subset of all possible electromagnetic terms.  For example, the magnetic 
coupling of photons to the nucleons (both protons and neutrons in this case) is 
given by
%
\begin{equation}
 \LL_\mathrm{mag} = \frac{e}{2\MN} N^\dagger
 (\kappa_0 + \kappa_1\tau^3)\,\skvec{\sigma}\cdot\vec{B}\,N \,.
\label{eq:L-mag}
\end{equation}
%
where $\kappa_0$ and $\kappa_1$ are the isoscalar and isovector nucleon 
magnetic moments, respectively.  That is, the low-energy constant of this 
operator has been fixed directly to the associated single-nucleon observable 
(similarly to how the nucleon mass is fixed immediately in the nonrelativistic 
theory).  With $\vec{B} = \vNabla\times\vec{A}$ and all indices written out, 
this is
%
\begin{equation}
 \LL_\mathrm{mag} = \frac{e}{2\MN} N^\dagger_{\alpha a}
 \idxx{\kappa_0\one + \kappa_1\tau^3}ab \idxx{\sigma^i}\alpha\beta
 \leviciv_{ijm} \partial^j A^m N^{\beta b} \,.
\label{eq:L-mag}
\end{equation}
%
and it gives rise to the Feynman rule
%
\begin{equation}
 \parbox{8em}{\includegraphics[width=8em]{Photon-Vertex-mag}}
 \sim \; \frac{\ii e}{2\MN} \idxx{\kappa_0\one + \kappa_1\tau^3}ab \,
 \leviciv_{ijm} \idxx{\sigma^i}\alpha\beta (\ii\vk)^j \,.
\label{eq:Photon-Vertex-mag}
\end{equation}
%
We point out that Eq.~\eqref{eq:L-mag} only gives the leading magnetic 
coupling.  In traditional nuclear physics language, it corresponds to a 
one-body operator.  At higher orders in the EFT, there are additional 
operators, like a four-nucleon contact interaction with an additional photon.  
Such many-body terms correspond to what phenomenological approaches model as 
``meson exchange currents.''

\subsection{Example: deuteron breakup}

As an application of the things discussed in the previous sections, we now 
consider the low-energy reaction $d\gamma \leftrightarrow np$.  By 
time-reversal symmetry, the amplitudes for the processes corresponding to the 
two possible directions of the arrow are the same.  For definiteness, we show 
the simplest diagram for the breakup reaction in Fig.~\ref{fig:DeutDis}.

%%%%%%%%%%%%%%%%%%%%%%%%%%%%%%%%%%%%%%%%%%%%%%%%%%%%%%%%%%%%%%%%%%%%%%%%%%%%%%
\begin{figure}[htbp]
\centering
\begin{minipage}{0.45\textwidth}
 \centering
 \includegraphics[height=12em,clip]{DeutBreakup-IA-kin-Dcm}
\end{minipage}
\hspace{0.5em}
\begin{minipage}{0.45\textwidth}
 \centering
 \includegraphics[height=12em,clip]{DeutBreakup-IA-kin-NNcm}
\end{minipage}
\caption{Deuteron breakup diagram in two different kinematic frames.
 \textbf{Left:} lab frame, deuteron at rest.
 \textbf{Right:} center-of-mass frame of the outgoing nucleons.
}
\label{fig:DeutDis}
\end{figure}
%%%%%%%%%%%%%%%%%%%%%%%%%%%%%%%%%%%%%%%%%%%%%%%%%%%%%%%%%%%%%%%%%%%%%%%%%%%%%%

The reaction can be considered in different reference frames.  In the lab 
frame (left panel of Fig.~\ref{fig:DeutDis}), the deuteron is initially at rest 
and then gets hits by a photon with four-momentum $(k_0,\vk)$.  For our 
theoretical discussion here it is more convenient to take the two outgoing 
nucleons in their center-of-mass frame, as shown in the right panel of 
Fig.~\ref{fig:DeutDis}).  To first order, we can translate between the two 
frames by boosting all nucleons lines with a momentum $\vk/2$.  A more careful 
analysis would keep track of relativistic kinematics (the external photon is 
never nonrelativistic), but this is not essential for the illustration here, so 
we can get away with interpreting $(k_0,\vk)$ to mean different things in the 
two frames.

In the $NN$ center-of-mass frame, the initial and final-state energies are
%
\begin{subalign}
 E_i &= k_0 + \frac{k^2}{4\MN} - \frac{\gamd^2}{\MN} \,, \\
 E_f &= \frac{p^2}{2\MN} + \frac{p^2}{2\MN} = \frac{p^2}{\MN} \,,
\end{subalign}
%
where we have neglected the small deuteron binding energy by setting $\Md=2\MN$.
Conservation of energy implies that
%
\begin{equation}
 p = \sqrt{\MN k_0 - \gamd^2 + k^2/4}
 \;\Leftrightarrow\;
 k_0 = \frac{p^2}{\MN} - \frac{k^2}{4\MN} + \frac{\gamd^2}{\MN} \,,
\end{equation}
%
and the energy assigned to the internal nucleon propagator has to be 
${-}\frac{k_0}{2}-\frac{\gamd^2}{2\MN}+\frac{k^2}{8\MN}$.

The blob in Fig.~\ref{fig:DeutDis} represents the deuteron vertex function 
calculated in Sec.~\ref{sec:Dibaryons}.  Taking the result found there and 
adding the spin-isospin structure, we find
%
\begin{equation}
\parbox{7.0em}{\centering\includegraphics[width=6.0em]{Deuteron-Vertex}}
 \sim \; \ii\frac{1}{\sqrt8} \frac{\sqrt{8\pi\gamt}}{\MN}
 \idxx{\sigma^i\sigma^2}\alpha\beta \idxx{\tau_2}ab \,.
\label{eq:Deuteron-Vertex}
\end{equation}
%
With this, we can finally write down the amplitude.  For an E1 transition, the 
photon couples to the nucleon charge via the Feynman rule given in 
Eq.~\eqref{eq:Photon-Vertex-gauge}.  Combining this with the ingredients above, 
we get
%
\begin{multline}
 \ii\mathcal{M}_{\text{E1}} \\
 = 2 \times \frac{{-}e}{\MN}
 \idxxx{\frac{\one + \tau^3}{2}}ca
 \delta^\gamma_\alpha\,\ii({\mathbf{p}}-\vk)^j
 \dfrac{\ii}{{-}\dfrac{k_0}{2}-\dfrac{\gamd^2}{2\MN}+\dfrac{k^2}{8\MN}
 -\dfrac{({\mathbf{p}}-\vk)^2}{2\MN}+\ii\eps} \,
 \, \frac{\ii}{\sqrt8}\frac{\sqrt{8\pi\gamd}}{\MN}
 \idxx{\sigma^2\sigma^i}\beta\gamma
 \idxx{\tau_2}bc \\
 \times (\epsgammaout)^j \, (\epsdout)^i
 \, (N_{s_p})^{\alpha a} \, (N_{s_n})_{\beta b} \,.
\label{eq:np-Capture-E1-full}
\end{multline}
%
We have included a factor $2$ to account for the fact that in drawing the 
diagram, the photon could be coupled to either nucleon (the isospin projection 
operator ensures of course that only the proton charge gives a contribution).  
Moreover, in the second line we have included polarization vectors and spinors 
for all external particles, the spins of which we denote as $s_\gamma$, $s_d$, 
$s_p$, and $s_n$, respectively.  From the amplitude in 
Eq.~\ref{eq:np-Capture-E1-full} one can proceed to calculate the corresponding 
cross section by summing/averaging over the various initial and final states 
and integrating over the available phase space.  We skip these details and 
close by noting that the isospin part is of course completely fixed 
(``polarized'') by the experimental setup.  Hence, the spinor-isospinors are
%
\begin{equation}
 (N_{s_p})^{\alpha a}
 = N_{s_p}^\alpha \twodvec{1 \\ 0}^{\!a}
 \mathtext{,}
 (N_{s_n})_{\beta b}
 = N_{s_p}^\alpha \twodvec{1 \\ 0}_{\!b} \,,
\end{equation}
%
\ie, we can directly set $a=1$ (isospin ``up'', proton) and $b=2$ (isospin 
``down'', neutron) in the amplitude.

\begin{prob}
\emph{Exercise:} Write down the analog of Eq.~\eqref{eq:np-Capture-E1-full} 
with a magnetically-coupled (M1) photon.
\end{prob}


\subsection{Chiral effective field theory}
\label{sec:EFT-Chiral}

Pionless EFT provides a simple yet powerful framework to describe few-nucleon 
systems at very low energies, but its name implies its natural limitation, \ie, 
the inability to describe physics at energy scales where pion exchange can be 
resolved.  Certainly this becomes important for scattering calculations at 
momentum scales larger than the pion mass.  But nuclear binding also 
generally increases with increasing number $A$ of bound nucleons, which 
translates to larger typical momentum scales within the nucleus.  There are 
indications that pionless EFT still converges for $A=4$, but the question is 
not fully settled.

The construction of an effective field theory of nucleons and pions was 
pioneered by Weinberg in the early 1990s~\cite{Weinberg:1990rz,Weinberg:1991um},
proposing a scheme to construct a nuclear potential based on Feynman diagrams 
from chiral perturbation theory.  This theory, which is constructed as an 
expansion around the so-called ``chiral limit,'' in which the pions are exactly 
massless Goldstone bosons.  The resulting theory, which has been applied with 
great success in the purely pionic and single-nucleon sector, treats the pion 
mass as a \emph{small} scale and thus has a power counting designed for typical 
momenta $Q\sim\mpi$.

For two or more nucleons, the theory is highly nonperturbative, which motivated 
Weinberg to develop a scheme where the power counting is applied to the 
\emph{potential} instead of the amplitude, as we have otherwise done throughout 
this section.  Kaplan~\etal~\cite{Kaplan:1996xu,Kaplan:1998we,Kaplan:1998tg} 
proposed a different scheme where pions are included perturbatively on top of a 
leading order given by pionless EFT.  This approach has, however, been found not 
to converge in channels where pion exchange generates a singular attractive 
interaction~\cite{Fleming:1999ee}.  It is thus understood today that pions in
general have to be treated nonperturbatively, in a framework generally referred 
to as ``chiral effective field theory.''  How exactly this should be 
implemented, however, is still a matter of debate.  Instead of summarizing this 
here, we refer the reader to the 
literature (see, e.g., Refs.~\cite{Beane:2000fx,Bedaque:2002mn,Nogga:2005hy,
Birse:2005um,Epelbaum:2008ga,Machleidt:2011zz,Epelbaum:2006pt,Hammer:2012id,
Long:2016vnq}).

\subsubsection{Leading-order pion-nucleon Lagrangian}

For an thorough introduction to the field of chiral perturbation theory, we 
recommend the reader to study the lecture notes of Scherer and 
Schindler~\cite{Scherer:2012xha} as well as the vast original literature cited 
therein.  It uses an elaborate formalism to construct the most general 
pion-nucleon Lagrangian that is invariant under chiral transformations 
(individual rotations of left- and right-handed nucleon fields) plus terms 
implementing the explicit breaking of chiral symmetry due to finite quark 
masses.  In the conventions of Ref.~\cite{Scherer:2012xha}, the leading-order 
pion-nucleon Lagrangian is
%
\begin{equation}
 \LL_{\pi N}^{(1)} + \LL_2^\pi
 = \bar{\psi}\left(\ii\slashed{D} - \MN
 +\frac{g_A}{2}\gamma^\mu\gamma^5 u_\mu\right)\psi
 + \frac{\fpi^2}{4}\Tr\left[(\partial^\mu U)^\dagger (\partial_\mu U)
 + \chi U^\dagger + U \chi^\dagger\right] \,.
\label{eq:L-pi-N-chiral}
\end{equation}
%
Here, $\psi$ is the nucleon Dirac field, and the matrix-valued field
%
\begin{equation}
 U \equiv = \exp\!\left(\ii\frac{\skvec{\tau}\cdot\skvec{\pi}}{\fpi}\right)
\label{eq:U-pi}
\end{equation}
%
collects the pion fields in an exponential representation.  $D_\mu$ here is the 
so-called chiral covariant derivative that couples the pion field to the 
nucleons.  The matrix
%
\begin{equation}
 \chi = 2 B_0 \, \diag(m_q,m_q) \,,
\end{equation}
%
where $m_q$ is the light quark mass (in the exact isospin limit, $m_u=m_d=m_q$) 
contains the effect from explicit chiral symmetry breaking.  Via the 
Gell-Mann--Oakes--Renner relation,
%
\begin{equation}
 \mpi^2 = 2B_0 m_q \,,
\label{eq:GMOR}
\end{equation}
%
one can show that $\chi$ generates a mass term for the pion field.

\begin{prob}
\emph{Exercise:} Expand the exponential in Eq.~\eqref{eq:U-pi} to show that 
$\frac{\fpi^2}{4} \Tr\left[(\partial^\mu U)^\dagger (\partial_\mu U)\right]$ 
generates a kinetic term for the pion field $\skvec{\pi}$ plus higher-order 
pion self interactions.
\end{prob}

After a couple of steps, which we encourage the reader to follow 
based on the definitions given in Ref.~\cite{Scherer:2012xha}, the leading 
terms in the Lagrangian are found to be
%
\begin{multline}
 \LL_{\pi N}^{(1)} + \LL_2^\pi
 = \bar{\psi}\left(\ii\slashed{\partial}-\MN\right)\psi
 + \frac{1}{2}(\partial^\mu\skvec{\pi})\cdot(\partial_\mu\skvec{\pi})
 - \frac{1}{2}\mpi^2 \skvec{\pi}^2\\
 - \frac{g_A}{2\fpi}\bar{\psi}\gamma^\mu\gamma^5
 (\skvec{\tau}\cdot\partial_\mu\skvec{\pi})\psi
 + \frac{\ii}{8\fpi^2}\bar{\psi}\gamma^\mu
 (\skvec{\tau}\cdot\skvec{\pi})(\skvec{\tau}\cdot\partial_\mu\skvec{\pi})\psi
 + \cdots \,.
\label{eq:L-pi-N-chiral-LO}
\end{multline}
%
Comparing this to our pseudoscalar model~\eqref{eq:L-pi-N-PS} in 
Sec.~\ref{sec:EFT-NonRelFerm}, we see that the pion-nucleon coupling now comes 
with explicit derivatives, as required by chiral symmetry.  After a 
Foldy-Wouthuysen transformation, however, one-pion-exchange in the 
nonrelativistic limit gives the same structure as in 
Eq.~\eqref{eq:pi-N-EOM-NR}.  There is no explicit $\sigma$ field in 
Eq.~\eqref{eq:L-pi-N-chiral-LO}; in the chiral theory, this particle only 
appears as a resonance generated by two-pion exchange.

\begin{acknowledgement}
We thank Dick Furnstahl and Bira van Kolck and for various stimulating 
discussions.  Moreover, SK is grateful to Martin Hoferichter for the insights 
into Coulomb-gauge quantization presented in Sec.~\ref{sec:EFT-EM}.
\end{acknowledgement}

\bibliographystyle{spphys}
\bibliography{lnplib}
\label{chapp:chapter4}
\label{chap:chapter5}
\newcommand\mpi {m_{\pi}}
%\newcommand\qslash {\slashed{q}}
%\newcommand{\note}[1]{{\bf \textcolor{red}{\uppercase{#1}}}}
\newcommand\pslash {\slashed{p}}
\newcommand\dslash {\slashed{\partial}}
%\newcommand\Dslash {\slashed{D}}
\newcommand\psidag{\psi^{\dagger}}
\newcommand\tautoinfty{\underset{\tau \to\infty}{\longrightarrow}}
\newcommand\Eq[1]{Eq.~(\ref{eq:#1})}
\newcommand\Eqs[2]{Eqs.~(\ref{eq:#1},\ref{eq:#2})}
%\newcommand\Eqs[2]{Eqs.~(\ref{eq:#1})-(\ref{eq:#2})}
\newcommand\Fig[1]{Fig.~\ref{fig:#1}}
\newcommand\Figtwo[2]{Figs.~\ref{fig:#1} and \ref{fig:#2}}
\newcommand\Figs[2]{Figs.~\ref{fig:#1}-\ref{fig:#2}}
\newcommand\Sec[1]{Sec.~\ref{sec:#1}}
\newcommand\Sect[1]{Section~\ref{sec:#1}}
\newcommand\Tab[1]{Table~\ref{tab:#1}}
%\newcommand{\be}{\begin{equation}}
%\newcommand{\ee}{\end{equation}}
%\newcommand\beq{\begin{eqnarray}}
%\newcommand\eeq{\end{eqnarray}} 
\newcommand\eqn[1]{\label{eq:#1}} 
\newcommand\eq[1]{eq.~(\ref{eq:#1})} 
\newcommand\eqstwo[2]{eqs. (\ref{eq:#1},\ref{eq:#2})} 
\newcommand\eqs[2]{eqs. (\ref{eq:#1}-\ref{eq:#2})} 
\newcommand{\vev}[1]{\langle #1 \rangle}
\newcommand{\lfb}{\bigskip\noindent}
\newcommand{\bfn}{{\mathbf n}}
\newcommand{\bfx}{{\mathbf x}}
\newcommand{\bfr}{{\mathbf r}}
\newcommand{\bfk}{{\mathbf k}}
\newcommand{\bfq}{{\mathbf q}}
\newcommand{\bfp}{{\mathbf p}}
\newcommand{\bfP}{{\mathbf P}}
\newcommand\Ncfg{N_{\mbox{\tiny cfg}}}
\newcommand{\eV}{{\rm ~eV }}
\newcommand{\keV}{{\rm ~keV }}
\newcommand{\GeV}{{\rm ~GeV }}
\newcommand{\TeV}{{\rm ~TeV }}
\newcommand{\MeV}{{\rm ~MeV }}
\newcommand{\calA}{{\mathcal{ A}}}
\newcommand{\calB}{{\mathcal{ B}}}
\newcommand{\calC}{{\mathcal{ C}}}
\newcommand{\calD}{{\mathcal{ D}}}
\newcommand{\calE}{{\mathcal{ E}}}
\newcommand{\calF}{{\mathcal{ F}}}
\newcommand{\calI}{{\mathcal{ I}}}
\newcommand{\calH}{{\mathcal{ H}}}
\newcommand{\calO}{{\mathcal{ O}}}
\newcommand{\calM}{{\mathcal{ M}}}
\newcommand{\calN}{{\mathcal{ N}}}
\newcommand{\calP}{{\mathcal{ P}}}
\newcommand{\calR}{{\mathcal{ R}}}
\newcommand{\calG}{{\mathcal{ G}}}
\newcommand{\calQ}{{\mathcal{ Q}}}
\newcommand{\calS}{{\mathcal{ S}}}
\newcommand{\calT}{{\mathcal{ T}}}
\newcommand{\calU}{{\mathcal{ U}}}
\newcommand{\calV}{{\mathcal{ V}}}
\newcommand{\calW}{{\mathcal{ W}}}
\newcommand{\calZ}{{\mathcal{ Z}}}
\newcommand{\calL}{{\mathcal{ L}}}
\newcommand\rf[1]{{\bf [REF #1]}}
\newcommand{\Tr}{{\rm Tr\,}}
\newcommand\half{{\textstyle{\frac{1}{2}}}} 
\newcommand\fourth{{\textstyle{\frac{1}{4}}}} 
\newcommand\expect[3]{\langle #1|#2|#3\rangle}

%
% Young Tableaux:
%
\newcommand{\mybar}[1]%
        {\kern 0.6pt\overline{\kern -0.6pt#1\kern -0.6pt}\kern 0.6pt}
\newcommand{\drawsquare}[2]{\hbox{%
\rule{#2pt}{#1pt}\hskip-#2pt%  left vertical
\rule{#1pt}{#2pt}\hskip-#1pt%  lower horizontal
\rule[#1pt]{#1pt}{#2pt}}\rule[#1pt]{#2pt}{#2pt}\hskip-#2pt%  upper horizontal
\rule{#2pt}{#1pt}}% right vertical
\newcommand{\Yfund}{\raisebox{-.5pt}{\drawsquare{6.5}{0.4}}}%  fund
\newcommand{\Ybarfund}{\mybar{\raisebox{-.5pt}{\drawsquare{6.5}{0.4}}}}%

%\begin{document}
\title{Lattice methods and effective field theory}
\author{Amy Nicholson}
\institute{Amy Nicholson \at Department of Physics, University of California, Berkeley, Berkeley CA 94720, USA, \email{anicholson@berkeley.edu}}
\maketitle
\abstract{Lattice field theory is a non-perturbative tool for studying properties of strongly interacting field theories, which is particularly amenable to numerical calculations and has quantifiable systematic errors. In these lectures we apply these techniques to nuclear Effective Field Theory (EFT), a non-relativistic theory for nuclei involving the nucleons as the basic degrees of freedom. The lattice formulation of \cite{EKLN1,EKLN4} for so-called pionless EFT is discussed in detail, with portions of code included to aid the reader in code development. Systematic and statistical uncertainties of these methods are discussed at length, and extensions beyond pionless EFT are introduced in the final Section.}
\unitlength = 1mm


\section{\label{sec:intro}Introduction}
Quantitative understanding of nuclear physics at low energies from first principles remains one of the most challenging programs in contemporary theoretical physics research. While physicists have for decades used models combined with powerful numerical techniques to successfully reproduce known nuclear structure data and make new predictions, currently the only tools available for tackling this problem that have direct connections to the underlying theory, Quantum Chromodynamics (QCD), as well as quantifiable systematic errors, are Lattice QCD and Effective Field Theory (EFT). In principle, when combined these techniques may be used to not only quantify any bias introduced when altering QCD in order to make it computationally tractable, but also to better understand the connection between QCD and nuclear physics.

The lattice is a tool for discretizing a field theory in order to reduce the path integral, having an infinite number of degrees of freedom, to a finite-dimensional ordinary integral. After rendering the dimension finite (though extremely large), the integral may then be estimated on a computer using Monte Carlo methods. Errors introduced through discretization and truncation of the region of spacetime sampled are controlled through the spatial and temporal lattice spacings, $b_s,b_{\tau}$, and the number of spatial and temporal points, $L,N_{\tau}$. Thus, these errors may be quantified through the lattice spacing dependence of the observables, and often may be removed through extrapolation to the continuum and infinite volume limits.

LQCD is a powerful and advanced tool for directly calculating low-energy properties of QCD. However, severe computational issues exist when calculating properties of systems with nucleons. Unfortunately, these problems grow rapidly with the number of nucleons in the system. 

The first issue is the large number of degrees of freedom involved when using quark fields to create nucleons. In order to calculate a correlation function for a single nucleon in LQCD using quarks (each of which has twelve internal degrees of freedom given by spin and color), one has to perform all possible Wick contractions of the fields in order to build in fermion antisymmetrization. For example, to create a proton using three valence quark operators requires the calculation of two different terms corresponding to interchanging the two up quark sources. The number of contractions involved for a nuclear correlation function grows with atomic number $Z$ and mass number $A$ as $(A+Z)!(2A-Z)!$. For He$_4$ this corresponds to $\sim 5 \times 10^5$ terms\footnote{This is a very na\"ive estimate; far more sophisticated algorithms exist with power-law scaling.}!

The second major problem occurs when performing a stochastic estimate of the path integral. A single quark propagator calculated on a given gauge field configuration may be a part of either a light meson or a heavy nucleon. However, the difference cannot be determined until correlations with the other quark fields present are built in by summing over a sufficiently large number of these field configurations\footnote{This interpretation of the signal-to-noise problem has been provided by David B. Kaplan.}. This leads to large fluctuations from configuration to configuration, and a stochastic signal-to-noise ratio, $\calR$, which degrades exponentially with the number of nucleons in the system,
\beq
\calR \sim e^{-A(M-3/2 m_{\pi})\tau} \ ,
\eeq
where $M$ is the nucleon mass and $m_{\pi}$ is the pion mass \cite{Lepage:1989hd}. This is currently the major limiting factor for the size of nuclear which can be probed using LQCD. The best calculations we have from LQCD using multiple nucleons to date are in the two-nucleon sector \cite{Berkowitz:2015eaa,Kurth:2015cvl,Nicholson:2015pys,Orginos:2015aya,Detmold:2015daa,Chang:2015qxa,Beane:2015yha,Beane:2014sda,Beane:2014ora,Beane:2013br,Beane:2012vq,Beane:2011iw,Beane:2009py,Yamazaki:2015vjn,Yamazaki:2015asa,Yamazaki:2013rna,Yamazaki:2012fn,Yamazaki:2012hi,Doi:2015uvd,Doi:2015oha,Ishii:2006ec,Murano:2013xxa,Aoki:2014mia,Murano:2013gta,HALQCD:2012aa,Inoue:2010hs}, while fewer calculations have been performed for three and four nucleon systems \cite{Beane:2009gs,Beane:2012vq,Beane:2014ora,Beane:2014sda,Chang:2015qxa,Yamazaki:2015vjn,Yamazaki:2015asa,Yamazaki:2013rna,Yamazaki:2012fn,Yamazaki:2012hi,Doi:2011gq}; however, even for two nucleon systems unphysically large pion masses must be used in order to reduce the noise problem. We will discuss signal-to-noise problems in more detail in \Sec{SNR}. 

Starting from an EFT using nucleons as the fundamental degrees of freedom greatly reduces the consequences from both of these issues. EFTs also enjoy the same benefit as the lattice over traditional model techniques of having quantifiable systematic errors, this time controlled by the cutoff of the EFT compared to the energy regime studied. For chiral EFTs this scale is generally $\Lambda_{\chi} \sim m_{\rho} \sim 700$ MeV. Systematic errors can be reduced by going to higher orders in an expansion of $p/\Lambda_{\chi}$, where $p$ is the momentum scale probed, with the remaining error given by the size of the first order which is not included. In a potential model there is no controlled expansion, and it is generally unknown how much the results will be affected by leaving out any given operator. In addition, field theories provide a rigorous mathematical framework for calculating physical processes, and can be directly translated into a lattice scheme.

In these lecture notes we will explore the use of lattice methods for calculating properties of many-body systems starting from nuclear EFT, rather than QCD. Our discussion will begin with understanding a very basic nuclear EFT, pionless EFT, at leading order. We will then proceed to discretize this theory and set up a framework for performing Monte Carlo calculations of our lattice theory. We will then discuss how to calculate observables using the lattice theory, and how to understand their associated statistical uncertainties. Next we will discuss quantifying and reducing systematic errors. Then we will begin to add terms to our theory going beyond leading order pionless EFT. Finally, we will discuss remaining issues and highlight some successes of the application of these methods by several different groups.

\section{Basics of Effective Field Theory and Lattice Effective Field Theory}
\subsection{\label{sec:EFT}Pionless Effective Field Theory}
To develop an EFT we will first write down all possible operators involving the relevant degrees of freedom within some energy range (determined by the cutoff) that are consistent with the symmetries of the underlying theory. Each operator will be multiplied by an unknown low-energy constant which may be fixed by comparing an observable with experiment or lattice QCD. In order to reduce this, in principle, infinite number of operators to a finite number we must also establish a power-counting rule for neglecting operators that do not contribute within some desired accuracy. This is a notoriously difficult problem for nuclear physics, and is in general observable and renormalization scheme dependent. Here, we will only briefly touch upon two common power-counting schemes, the so-called Weinberg and KSW expansions \cite{Weinberg:1990rz,Weinberg:1991um,Kaplan:1996xu,Kaplan:1998tg,Kaplan:1998we}. For reviews of these and other power-counting schemes, see \cite{Epelbaum:2008ga,Epelbaum:2010nr,Machleidt:2011zz}.

The simplest possible nuclear EFT involves non-relativistic nucleon fields interacting via delta functions. This is known as a pionless EFT, and is only relevant for energy scales up to a cutoff $\Lambda \sim m_{\pi}$. Below this scale, the finite range of pion exchange cannot be resolved, and all interactions appear to be point-like. In this discussion we will closely follow that of Ref.~\cite{Kaplan:2005es}. For the moment, let's just consider a theory of two-component (spin up/down) fermion fields, $\psi$, with the following Lagrangian,
\beq
\label{eq:leff}
\mathcal{L}_{\mbox{\tiny eff}} = \psi^{\dagger}\left( i \partial_{\tau} + \frac{\nabla^2}{2M}\right) \psi + g_0 \left(\psi^{\dagger}\psi\right)^2 + \frac{g_2}{8}\left[ \left(\psi\psi\right)^{\dagger}\left(\psi\overleftrightarrow{\nabla}^2\psi\right)+ \mbox{\tiny h.c.}\right]+\cdots \ ,
\eeq
where
\beq
\overleftrightarrow{\nabla}^2 \equiv \overleftarrow{\nabla}^2-2\overleftarrow{\nabla} \cdot \overrightarrow{\nabla}+\overrightarrow{\nabla}^2 \ ,
\eeq
$M$ represents the nucleon mass, $g_0, g_2, \ldots$ are unknown, low-energy constants (LECs) which may be fixed by comparing to experimental or LQCD results, and all spin indices are suppressed. Because the effective theory involves dynamical degrees of freedom that are only relevant up to a certain scale, we must define a cutoff, $\Lambda$, above which the theory breaks down. In general, the LECs scale as $\Lambda^{-\mbox{\tiny dim}(\mathcal{O})}$, where dim$(\mathcal{O})$ represents the dimension of the operator associated with the LEC. According to na\"ive power counting, the $g_2$ term in \Eq{leff} should be suppressed relative to the $g_0$ term, because adding a derivative to an operator increases its dimension. One should be careful in practice, however, because na\"ive power counting does not always hold, as we will see several times throughout these lectures. 

\subsubsection{\label{sec:scatamp}Two particle scattering amplitude}
In order to set the coefficients $g_0, g_2, \ldots$, we may look to experimental scattering data. In particular, if we wish to set the $g_0$ coefficient we should consider two-particle $s$-wave scattering because the operator associated with $g_0$ contains no derivatives. $g_2$ and other LECs may be set using $p$- and higher-wave scattering data. Recall that the S-matrix for non-relativistic scattering takes the following form:
\beq
S=1+\frac{iMp}{2\pi}A \ ,
\eeq
where $p$ is the scattering momentum and $A$ is the scattering amplitude. For $s$-wave scattering the amplitude may be written as,
\beq
\label{eq:Apcotd}
A=\frac{4\pi}{M} \frac{1}{p\cot\delta - ip} \ ,
\eeq
where $\delta$ is the $s$-wave scattering phase shift. Given a short-range two-body potential, the scattering phase shift has a well-known expansion for low momenta, called the effective range expansion,
\beq
\label{eq:ere}
p\cot\delta = -\frac{1}{a}+\frac{1}{2}r_0p^2+r_1p^4+\cdots \ ,
\eeq
where $a$ is the scattering length, $r_0$ is the effective range, and $r_1$ and higher order terms are referred to as shape parameters. The effective range and shape parameters describe the short-range details of the potential, and are generally of order of the appropriate power of the cutoff in a naturally tuned scenario. 

The scattering length may be used to describe the asymptotic behavior of the radial wavefunction. In particular, consider two-particles interacting via an attractive square-well potential. If the square-well is sufficiently strongly attractive, the wavefunction turns over and goes to zero at some finite characteristic length. This means the system is bound and the size of the bound state is given by the scattering length, $a$. On the other hand, if the wavefunction extends over infinite space, then the system is in a scattering state and the scattering length may be determined as the distance from the origin where the asymptote of the wavefunction intersects the horizontal axis (see \Fig{a0}). This implies that the scattering length in the case of a scattering state is negative. If the potential is tuned to give a system which is arbitrarily close to the crossover point from a bound state to a scattering state, corresponding to infinite scattering length, the state is described as being near unitarity, because the unitarity bound on the scattering cross section is saturated at this point. Note that this implies that the scattering length may be any size and is not necessarily associated with the scale set by the cutoff. However, such a scenario requires fine-tuning of the potential. Such fine-tuning is well-known to occur in nuclear physics, with the deuteron and neutron-neutron $s$-wave scattering being notable examples.

\begin{figure}
\begin{center}
\includegraphics[width=0.3\linewidth]{Chapter5-figures/al0.png}
\includegraphics[width=0.3\linewidth]{Chapter5-figures/ae0.png}
\includegraphics[width=0.3\linewidth]{Chapter5-figures/ag0.png}
\caption{\label{fig:a0}Sketches of two-body radial wavefunctions vs. $r$ corresponding to various scattering lengths. From left to right: $a<0$, $a\to\infty$,$a>0$.}
\end{center}
\end{figure}

A many-body system composed of two-component fermions with an attractive interaction is known to undergo pairing between the species (higher $N$-body interactions are prohibited by the Pauli exclusion principle), such as in neutron matter, found in the cores of neutron stars, which is composed of spin up and spin down neutrons. At low temperature, these bosonic pairs condense into a coherent state. If the interaction is only weakly attractive, the system will form a BCS state composed of widely separated Cooper pairs, where the average pair size is much larger than the average interparticle spacing. On the other hand, if the interaction is strongly attractive then the pairs form bosonic bound states which condense into a Bose-Einstein condensate. The crossover between these two states corresponds to the unitary regime, and has been studied extensively in ultracold atom experiments, where the interaction between atoms may be tuned using a Feshbach resonance. In this regime, the average pair size is equal to the interparticle spacing (given by the inverse density), which defines the only scale for the system. Thus, all dimensionful observables one wishes to calculate for this system are determined by the appropriate power of the density times some dimensionless constant.  For a review of fermions in the unitary regime, see e.g., \cite{AtomsReview1,AtomsReview2}.

\subsubsection{\label{sec:couplings}Two-body LECs}
Returning to our task of setting the couplings using scattering parameters as input, we might consider comparing \Eq{leff} and \Eq{ere}, to determine the LEC $g_0$ using the scattering length, $g_2$ using the effective range, and so forth. To see how this is done in practice we may compute the scattering amplitude $A$ in the effective theory, and match the coefficients to the effective range expansion. Let's begin using only the first interaction term in the effective theory, corresponding to $g_0$. Diagrammatically, the scattering amplitude may be written as the sum of all possible bubble diagrams (see \Fig{bubblesum}). Because the scattering length may take on any value, as mentioned previously, we cannot assume that the coupling $g_0$ is small, so we should sum all diagrams non-perturbatively. The first diagram in the sum is given by the tree level result, $g_0$. If we assume that the system carries energy $E=p^2/M$, then the second diagram may be labeled as in \Fig{loop}, and gives rise to the loop integral,
\beq
\label{eq:loop}
I_0 = i\int \frac{d^4q}{(2\pi)^4}\frac{1}{\left(E/2+q_0-\frac{q^2}{2M}-i\epsilon\right)\left(E/2-q_0-\frac{q^2}{2M}+i\epsilon\right)} \ .
\eeq 
Performing the integral over $q_0$ and the solid angle gives
\beq
\label{eq:loopEFT}
I_0 &=& \frac{1}{2\pi^2}\int^{\pi\Lambda/2} dq\frac{q^2}{\left(E-\frac{q^2}{M}\right)} \\
&=& \frac{M}{2\pi^2}\left[\frac{\pi\Lambda}{2}-\sqrt{ME}\tanh^{-1}\left(\frac{\Lambda}{\sqrt{ME}}\right)\right] \ ,
\eeq
where I have introduced a hard momentum cutoff, $\Lambda$. Removing the cutoff by taking it to infinity results in
\beq
I_0 \underset{\Lambda\to\infty}{\longrightarrow}\frac{M}{4\pi}\left[\Lambda+ip\right] \ .
\eeq
Because the interaction is separable, the $n$th bubble diagram is given by $n$ products of this loop function. Thus, the scattering amplitude is factorizable, and may be written
\beq
\label{eq:Abubble}
A&=&g_0\left[1+\sum_n\left(g_0I_0\right)^n\right] \\
&=& \frac{g_0}{1-g_0I_0} \ .
\eeq
We may now compare \Eqs{Apcotd}{ere} and \Eq{Abubble} to relate the coupling $g_0$ to the scattering phase shift. This is easiest to do by equating the inverse scattering amplitudes,
\beq
\frac{1}{A} &=& \frac{1}{g_0}-\frac{M}{4\pi}\Lambda -\frac{iMp}{4\pi} = -\frac{M}{4\pi a}-\frac{iMp}{4\pi} \ ,
\eeq
where I have used \Eq{ere} cut off at leading order. We now have the relation
\beq
g_0 = \frac{4\pi}{M} \frac{1}{\Lambda-1/a} 
\eeq
between the coupling and the physical scattering length. 

\begin{figure}
\begin{center}
\includegraphics[width=\linewidth]{Chapter5-figures/simple1}
\caption{\label{fig:bubblesum}Two-body scattering amplitude represented as a sum of bubble diagrams corresponding to a single contact interaction with coupling $g_0$.}
\end{center}
\end{figure}
\begin{figure}
\begin{center}
\includegraphics[width=\linewidth]{Chapter5-figures/loop1}
\caption{\label{fig:loop}Feynman diagram for a single bubble in \Fig{bubblesum}, giving rise to the loop integral \Eq{loop}.}
\end{center}
\end{figure}

Note that the coupling runs with the scale $\Lambda$; the particular dependence is determined by the regularization and renormalization scheme chosen. In order to understand the running of the coupling we may examine the beta function. To do so we first define a dimensionless coupling,
\beq
\hat{g}_0 \equiv -\frac{M\Lambda}{4\pi}g_0 \ ,
\eeq
then calculate
\beq
\label{eq:beta}
\beta\left(\hat{g}_0\right) \equiv \Lambda \frac{\partial \hat{g}_0}{\partial \Lambda} = -\frac{a\Lambda}{\left(a\Lambda-1\right)^2} = -\hat{g}_0\left(\hat{g}_0-1\right) \ .
\eeq
This function is a simple quadratic that is plotted in \Fig{beta}. The beta function has two zeroes, $\hat{g}_0 = 0,1$, corresponding to fixed points of the theory. At a fixed point, the coupling no longer runs with the scale $\Lambda$, and the theory is said to be scale-invariant (or conformal, given some additional conditions). This means that there is no intrinsic scale associated with the theory. The fixed point at $\hat{g}_0=0$ is a trivial fixed point, and corresponds to a non-interacting, free field theory (zero scattering length). The other, non-trivial fixed point at $\hat{g}_0=1$ corresponds to a strongly interacting theory with infinite scattering length; this is the unitary regime mentioned previously. Here, not only does the scattering length go to infinity, as does the size of the radial wavefunction, but the energy of the bound state (as approached from $\hat{g}_0 > 1$) goes to zero and all relevant scales have vanished. Note that this is an unstable fixed point; the potential must be finely tuned to this point or else the theory flows away from unitarity as $\Lambda \to 0$ (IR limit).

\begin{figure}
\begin{center}
\includegraphics[width=0.5\linewidth]{Chapter5-figures/beta.pdf}
\end{center}
\caption{\label{fig:beta}Beta function (\Eq{beta}) for the two-body contact interaction. Arrows represent the direction of flow toward the IR.}
\end{figure}

Generally perturbation theory is an expansion around free field theory, corresponding to a weak coupling expansion. This is the approach used as part of the Weinberg power counting scheme for nuclear EFT \cite{Weinberg:1990rz,Weinberg:1991um}. However, in some scattering channels of interest for nuclear theory the scattering length is indeed anomalously large, such as the $^1S_0$ and $^3S_1$ nucleon-nucleon scattering channels, where
\beq
a_{^1S_0} &\sim& -24 \mbox{ fm} \ , \\
a_{^3S_1} &\sim& 5 \mbox{ fm} \ .
\eeq
Such large scattering lengths suggest that an expansion around the strongly coupled fixed-point of unitarity may be a better starting point and lead to better convergence. This approach was taken by Kaplan, Savage, and Wise and led to the KSW power-counting scheme \cite{Kaplan:1998we,Kaplan:1998tg,Kaplan:1996xu}. Unfortunately, nuclear physics consists of many scales of different sizes and a consistent power-counting framework with good convergence for all observables has yet to be developed; in general the convergence of a given scheme depends on the scattering channels involved. 

Because nuclear physics is not weakly coupled in all channels, non-perturbative methods, such as lattice formulations, will be favorable for studying few- and many-body systems, where two-body pairs may interact through any combination of channels simultaneously. Due to the scale-invariant nature of the unitary regime, it provides a far simpler testbed for numerical calculations of strongly-interacting theories, so we will often use it as our starting point for understanding lattice EFT methods. 

\subsection{\label{sec:LEFT}Lattice Effective Field Theory}

Our starting point for building a lattice EFT will be the path integral formulation of quantum field theory in Euclidean spacetime. The use of Euclidean time allows the exponent of the path integral to be real (in certain cases), a property which will be essential to our later use of stochastic methods for its evaluation. Given a general theory for particles $\psi,\psi^{\dagger}$ obeying a Lagrangian density \beq 
\mathcal{L}(\psi^{\dagger},\psi) = \psi^{\dagger}\left( \partial_{\tau}-\mu \right)\psi+ \mathcal{H}\left[\psi^{\dagger}, \psi\right] \ ,
\eeq
where $\tau$ is the Euclidean time, $\mu$ the chemical potential, and $\mathcal{H}$ is the Hamiltonian density, the Euclidean path integral is given by
\beq
Z=\int \mathcal{D} \psi^{\dagger}\mathcal{D} \psi e^{-\int d\tau d^3x\left[\mathcal{L}(\psi^{\dagger},\psi)\right]} \ .
\eeq
If the integral over Euclidean time is compact, then the finite time extent $\beta$ acts as an inverse temperature, and we may draw an analogy with the partition function in statistical mechanics, $Z = tr\left[e^{-H\beta}\right]$. This analogy is often useful when discussing lattice formulations of the path integral. In this work we will generally consider $\mu=0$ and create non-zero particle density by introducing sources and sinks for particles and calculating correlation functions. 

We discretize this theory on a square lattice consisting of $L^3 \times N_{\tau}$ points, where $L$ is the number of points in all spatial directions, and $N_{\tau}$ is the number of temporal points. We will focus on zero temperature physics, corresponding to large $N_{\tau} ~$\footnote{The explicit condition on $N_{\tau}$ required for extracting zero temperature observables will be discussed in \Sec{observables}}. We must also define the physical distance between points, the lattice spacings $b_s, b_{\tau}$, where $b_{\tau} = b_s^2/M$ by dimensional analysis for non-relativistic theories. The fields are now labeled by discrete points, $\psi(\vec{x},\tau) \to \psi_{\vec{n},\tau}$, and continuous integrals are replaced by discrete sums, $\int d^3x \to \sum_{\vec{n},\tau}^{L,N_{\tau}}$.

\subsubsection{Free field theory}
To discretize a free field theory, we must discuss discretization of derivatives. The simplest operator which behaves as a single derivative in the continuum limit is a finite difference operator,
\beq
\partial_{\hat{k}}^{(L)} f_j = \frac{1}{b_s}\left[f_{j+\hat{k}}-f_j\right] \ ,
\eeq
where $\hat{k}$ is a unit vector in the $k$-direction. The discretized second derivative operator must involve two hops, and should be a symmetric operator to behave like the Laplacian. A simple possibility is
\beq
\nabla_L^2 f_j = \sum_{k} \frac{1}{b_s^2}\left[ f_{j+\hat{k}}+f_{j-\hat{k}}-2 f_j \right] \ .
\eeq
We can check the continuum limit by inspecting the corresponding kinetic term in the action,
\beq
S_{\mbox{\tiny KE}} \propto \sum_j \psi_j^{\dagger} \nabla_L^2 \psi_j \ .
\eeq
The fields may be expanded in a plane wave basis,
\beq
\psi_j = \sum_{k=-L/2}^{L/2} \psi_k e^{-\frac{2\pi i}{L} j \cdot k} \ ,
\eeq
for spatial indices, $j$, leading to
\beq
\sum_j \psi_j^{\dagger} \nabla_L^2 \psi_j = \frac{1}{b_s^2}\sum_j \sum_{k'}\sum_k \psi_{k'}^{\dagger} \psi_k \left[ e^{\frac{2\pi i}{L}j \cdot k'} e^{\frac{-2\pi i}{L} j\cdot k} \right] \left[e^{\frac{-2\pi i}{L}k}+e^{\frac{2\pi i}{L} k}-2\right] \ .
\eeq
After performing the sum over $j$ the first piece in brackets gives $\delta_{kk'}$, while the second is proportional to $\sin^2(k\pi/L)$, resulting in,
\beq
\sum_j \psi_j^{\dagger} \nabla_L^2 \psi_j = -\frac{4}{b_s^2}\sum_k \psi_k^{\dagger}\psi_k \sin^2\left(\frac{k\pi}{L}\right) \ .
\eeq 
Finally, expanding the sine function for small $k/L$ gives,
\beq
\label{eq:kinetic}
\sum_j \psi_j^{\dagger} \nabla_L^2 \psi_j = \sum_k \psi_k^{\dagger}\psi_k &&\left[\underbrace{-\left(\frac{2\pi k}{b_s L}\right)^2+\frac{b_s^2}{12}\left(\frac{2\pi k}{b_s L}\right)^4+ \cdots }\right] \ ,\cr
&& \hspace{6mm} -p^2 + \frac{b_s^2}{12} p^4 + \cdots \underset{b_s\to 0}{\longrightarrow} -p^2 
\eeq
where I've used the finite volume momentum $p = \frac{2\pi k}{b_s L}$ to rewrite the expression in square brackets. Thus, we have the correct continuum limit for the kinetic operator. Note that for larger momenta, approaching the continuum limit requires smaller $b_s$. However, this is only one possibility for a kinetic term. We can always add higher dimension operators (terms with powers of $b_s$ in front of them), in order to cancel leading order terms in the expansion \Eq{kinetic}. This is a form of what's called improvement of the action, and will be discussed in more detail in \Sec{systematic}.

Adding a temporal derivative term, 
\beq
\partial_{\tau}^{(L)} \psi_{\vec{x},\tau} = \frac{1}{b_{\tau}}\left[\psi_{\vec{n},\tau}- \psi_{\vec{n},\tau-1}\right] \ ,
\eeq
we can now write down a simple action for a non-relativistic free-field theory,
\beq
S_{\mbox{\tiny free}} = \sum_{\tau,\tau'} \frac{1}{b_{\tau}}\psi_{\tau'}^{\dagger}\left[K_0\right]_{\tau,\tau'}\psi_{\tau} \ ,
\eeq
where I've defined a matrix $K_0$ whose entries are $L^3 \times L^3$ blocks,
\beq
K_0 \equiv \left(\begin{array}{ccccccc}
D & -1 & 0 & 0 & . & . & . \\
0 & D & -1 & 0 & . & . &.  \\
0 & 0 & D & -1 &  .&.  & . \\
. & .&. &. & . & & \\ 
. & .& .& .& & . & \\ 
1 & .&. & .& & & . \\ 
\end{array} \right)
\eeq
where $D \equiv 1-\frac{b_s^2 \nabla_L^2}{2}$ contains the spatial Laplacian, and therefore connects fields on the same time slice (corresponding to diagonal entries of the matrix $K_0$), while the temporal derivative contributes the off-diagonal pieces. Note that the choice of ``1" in the lower left corner corresponds to anti-periodic boundary conditions, appropriate for fermionic fields. For zero temperature calculations the temporal boundary conditions are irrelevant, and it will often be useful to choose different temporal boundary conditions for computational or theoretical ease. 

\subsubsection{Interactions}
Now let's discuss adding interactions to the theory. We'll focus on the first term in a nuclear EFT expansion, the four-fermion interaction:
\beq
\mathcal{L}_{\mbox{\tiny int}} = \sum_n g_0 \psi_{n,\uparrow} \psi_{n,\uparrow} \psi_{n,\downarrow} \psi_{n,\downarrow} \ ,
\eeq
where $(\uparrow,\downarrow)$ now explicitly label the particles' spins (or alternatively, flavors). Because anti-commuting fields cannot easily be accommodated on a computer, they must be integrated out analytically. The only Grassmann integral we know how to perform analytically is a Gaussian, so the action must be bilinear in the fields. One trick for doing this is called a Hubbard-Stratonovich (HS) transformation, in which auxiliary fields are introduced to mediate the interaction. The key is to use the identity,
\beq
e^{b_{\tau}g_0\psi_{\uparrow}^{\dagger}\psi_{\uparrow}\psi_{\downarrow}^{\dagger}\psi_{\downarrow}} = \frac{1}{\sqrt{2\pi}}\int_{-\infty}^{\infty}d\phi^{-\phi^2/2-\phi\sqrt{b_{\tau}g_0}\left(\psi_{\uparrow}^{\dagger}\psi_{\uparrow}+\psi_{\downarrow}^{\dagger}\psi_{\downarrow}\right)} \ ,
\eeq
where I have dropped the spacetime indices for brevity. This identity may be verified by completing the square in the exponent on the right hand side and performing the Gaussian integral over the auxiliary field $\phi$. This form of HS transformation has the auxiliary field acting in what is called the density channel $\left(\psi_{\uparrow}^{\dagger}\psi_{\uparrow}+\psi_{\downarrow}^{\dagger}\psi_{\downarrow}\right)$. It is also possible to choose the so-called BCS channel, $\left(\psi_{\uparrow}^{\dagger}\psi_{\downarrow}^{\dagger}+\psi_{\uparrow}\psi_{\downarrow}\right)$, the usual formulation used in BCS models, however this causes a so-called sign problem when performing Monte Carlo sampling, as will be discussed in detail in \Sec{sign}. Transformations involving non-Gaussian auxiliary fields may also be used, such as
\beq
Z_2 \mbox{ field: } &&\frac{1}{2} \sum_{\phi=\pm 1}e^{-\phi\sqrt{b_{\tau}g_0}\left(\psi_{\uparrow}^{\dagger}\psi_{\uparrow}+\psi_{\downarrow}^{\dagger}\psi_{\downarrow}\right)} \cr
\mbox{compact continuous: } && \frac{1}{2\pi} \int_{-\pi}^{\pi}e^{-\sin \phi\sqrt{b_{\tau}g_0}\left(\psi_{\uparrow}^{\dagger}\psi_{\uparrow}+\psi_{\downarrow}^{\dagger}\psi_{\downarrow} \right)}  \ .
\eeq
These formulations may have different pros and cons in terms of computational and theoretical ease for a given problem, and should be chosen accordingly. For example, the $Z_2$ interaction is conceptually and computationally the simplest interaction, however, it also induces explicit $4-$ and higher-body interactions in systems involving more than two-components which may not be desired. 

\subsubsection{Importance sampling}
The action may now be written with both kinetic and interaction terms,
\beq
\label{eq:actiongen}
S=\frac{1}{b_{\tau}}\sum_{\tau,\tau'}\psi_{\tau'}^{\dagger}\left[K(\phi)\right]_{\tau'\tau}\psi_{\tau} \ ,
\eeq
where the matrix $K$ includes blocks which depend on the auxiliary field $\phi$, and also contains non-trivial spin structure that has been suppressed. The partition function can be written
\beq
Z = \int \mathcal{D}\phi \mathcal{D}\psi^{\dagger}\mathcal{D}\psi \rho[\phi]e^{-S[\phi,\psi^{\dagger}\psi]} \ ,
\eeq
where the integration measure for the $\phi$ field, $\rho[\phi]$, depends on the formulation chosen,
\beq
\rho[\phi] = \left\{ \begin{array}{cc}
\prod_n e^{-\phi_n^2/2} & \mbox{Gaussian}\\
\prod_n \frac{1}{2}\left(\delta_{\phi_{n,1}}+\delta_{\phi_{n,-1}}\right) & Z_2 \\
\prod_n \left(\theta(-\pi+\phi_n)\theta(\pi-\phi_n)\right) & \mbox{compact continuous} 
\end{array}\right. \ .
\eeq

With the action in the bilinear form of \Eq{actiongen}, the $\psi$ fields can be integrated out analytically, resulting in
\beq
\label{eq:prob}
Z_{\phi}=\int \calD \phi P[\phi] \qquad P[\phi] \equiv \rho[\phi]\det K[\phi] \ .
\eeq
Observables take the form
\beq
\langle \calO \rangle = \frac{1}{Z} \int \calD \phi P[\phi]\calO[\phi] \ .
\eeq

Through the use of discretization and a finite volume, the path integral has been converted into a standard integral with finite dimension. However, the dimension is still much too large to imagine calculating it on any conceivable computer, so we must resort to Monte Carlo methods for approximation. The basic idea is to generate a finite set of $\phi$ field configurations of size $\Ncfg$ according to the probability measure $P[\phi]$, calculate the observable on each of these configurations, then take the mean as an approximation of the full integral,
\beq
\langle \calO \rangle \approx \frac{1}{\Ncfg}\sum_n^{\Ncfg}\calO(\phi_n) \ .
\eeq
Assuming the central limit theorem holds, for $\Ncfg$ large enough (a non-trivial condition, as will be discussed in \Sec{overlap}), the distribution of the mean approaches a Gaussian, and the error on the mean falls off with the square root of the sample size. 

There are several algorithms on the market for generating field configurations according to a given probability distribution, and I will only briefly mention a few. Lattice calculations are particularly tricky due to the presence of the determinant in \Eq{prob}, which is a highly non-local object and is very costly to compute. One possible algorithm to deal with this is called determinantal Monte  Carlo, which implements local changes in $\phi$, followed by a simple Metropolis accept/reject step. This process can be rather inefficient due to the local updates. An alternative possibility is Hybrid Monte Carlo, commonly used for lattice QCD calculations, in which global updates of the field are produced using molecular dynamics as a guiding principle. Note that the field $\phi$ must be continuous in order to use this algorithm due to the use of classical differential equations when generating changes in the field. Also common in lattice QCD calculations is the use of pseudofermion fields as a means for estimating the fermion determinant. Here the determinant is rewritten in terms of a Gaussian integral over bosonic fields, $\chi$,
\beq
\det K[\phi] \propto \int \calD \chi^{\dagger}\calD \chi e^{-\chi^{\dagger}K^{-1}[\phi]\chi} \ .
\eeq
This integral is then evaluated stochastically. These are just a sample of the available algorithms. For more details on these and others in the context of non-relativistic lattice field theory, see \cite{Drut:2012md}.

\subsubsection{Example formulation}
Now that we have developed a general framework for lattice EFT, let's be explicit and make a few choices in order to further our understanding and make calculations simpler. The first choice I'm going to make is to use a $Z_2$ $\phi$ field, so that $\rho[\phi]$ is trivial. The next simplification I'm going to make is to allow the $\phi$ fields to live only on temporal links,
\beq
\label{eq:pointint}
\calL_{\mbox{\tiny int}} = \sum_{\bfx}\sqrt{b_{\tau}g_0}\phi_{\bfx,\tau}\psidag_{\bfx,\tau}\psi_{\bfx,\tau-1} \ .
\eeq
Note that we are free to make this choice, so long as the proper four-fermion interaction is regained in the continuum limit. This choice renders the interaction separable, as it was in our continuum effective theory. This means we may analytically sum two-body bubble chain diagrams as we did previously in order to set the coupling $g_0$ using some physical observable (see \Fig{dimer}). 

With this choice we can now write the $K$-matrix explicitly as
\beq
K[\phi,N_{\tau}] \equiv \left(\begin{array}{ccccccc}
D & -X(\phi_{N_{\tau}-1}) & 0 & 0 & . & . & . \\
0 & D & -X(\phi_{N_{\tau}-2}) & 0 & . & . & . \\
. & .& .& .&  & & \\ 
. & .&. & &. &  & \\ 
. &. & .& & & D & X(\phi_0)\\ 
X(\phi_{N_{\tau}}) & .&. & & &0 & D \\ 
\end{array} \right) \ ,
\eeq
where $X(\phi_\tau) \equiv 1-\sqrt{g_0}\phi_\tau$. Now the $\phi$-dependence exists only on the upper diagonal, as well as the lower left due to the boundary condition. This block will be eliminated through our final choice: open boundary conditions in time for the $\psi$ fields, $X(\phi_{N_{\tau}})=0$. As mentioned previously, we are free to choose the temporal boundary conditions as we please, so long as we only consider zero temperature (and zero chemical potential) observables. 

\begin{figure}
\begin{center}
\includegraphics[width=\linewidth]{Chapter5-figures/dimer}
\end{center}
\caption{\label{fig:dimer}Two-body scattering amplitude of \Fig{bubblesum}, where the contact interaction has been replaced in the second line by exchange of a dimer auxiliary field via a Hubbard-Stratonovich transformation.}
\end{figure}

With this set of choices the matrix $K$ consists purely of diagonal elements, $D$, and upper diagonal elements, $X(\phi_\tau)$. One property of such a matrix is that the determinant, which is part of the probability distribution, is simply the product of diagonal elements, $\det K = \prod_{\tau} D$. Note that $D$ is completely independent of the field $\phi$. This means that the determinant in this formulation has no impact on the probability distribution $P[\phi]$, and therefore never needs to be explicitly computed, greatly reducing the computational burden. Thus in all of our calculations, performing the path integral over $\phi$ simply amounts to summing over $\phi = \pm 1$ at each lattice site. 

Finally, this form of $K$ also makes the calculation of propagators very simple. The propagator from time 0 to $\tau$ may be written,
\beq
K^{-1}(\tau,0) &=& D^{-1} X(\phi_{\tau-1})D^{-1}X(\phi_{\tau-2})D^{-1} \cdots X(\phi_0)D^{-1} \cr
&=& D^{-1}X(\phi_{\tau-1})K^{-1}(\tau-1,0) \ ,
\eeq
where $K^{-1}(0,0) = D^{-1}$, and all entries are $V\times V \ , (V=L^3)$ matrices which may be projected onto the desired state. This form suggests a simple iterative approach to calculating propagators: start with a source (a spatial vector projecting onto some desired quantum numbers and interpolating wavefunction), hit it with the kinetic energy operator corresponding to free propagation on the time slice, then hit it with the $\phi$ field operator on the next time link, then another free kinetic energy operator, and so on, finally projecting onto a chosen sink vector. 

As will be discussed further in Sec{systematic}, it is often preferable to calculate the kinetic energy operator in momentum space, while the auxiliary field in $X(\phi)$ must be generated in position space. Thus, Fast Fourier Transforms (FFTs) may be used between each operation to quickly translate between the bases. Example code for generating source vectors, kinetic operators, and interaction operators will be provided in later Sections.

A cartoon of this process on the lattice is shown in \Fig{lat}. The choice of $Z_2$ auxiliary fields also simplifies the understanding of how four-fermion interactions are generated. On every time link, imagine performing the sum over $\phi = \pm 1$. If there is only a single fermion propagator on a given link this gives zero contribution because the term is proportional to $\sum_{\phi=\pm 1} \sqrt{g_0} \phi = 0$. However, on time slices where two propagators overlap, we have instead $\sum_{\phi = \pm 1} g_0 \phi^2 = 2 g_0$. In sum, anywhere two fermions exist at the same spacetime point a factor of $g_0$ contributes, corresponding to an interaction.

\begin{figure}
\begin{center}
\includegraphics[width=0.4\linewidth]{Chapter5-figures/lattice.png}
\end{center}
\caption{\label{fig:lat}Schematic of a lattice calculation for a two-particle correlation function. The two particles (red and blue lines) propagate through the lattice between source $\psi(0)$ and sink $\psi(\tau)$, seeing particular values of the auxiliary field, $\phi$, on each time link. If two particles occupy the same temporal link, then upon summation over all possible values of $\phi$ at each link, a non-zero contribution is generated by the interaction term because $\langle \phi^2 \rangle \neq 0$.}
\end{figure}

\subsubsection{\label{sec:tuning}Tuning the two-body interaction}

There are several ways to set the two-body coupling. Here we will explore two methods, using different two-body observables. The first involves calculating the two-particle scattering amplitude, and tuning the coupling to reproduce known scattering parameters, to make a connection with our previous calculation for the effective theory. The second method uses instead the energy spectrum of a two-particle system in a box. This powerful method will be useful later when we begin to improve the theory in order to reduce systematic errors.

We have calculated the scattering amplitude previously for our effective theory using a momentum cutoff. For the first method for tuning the coupling, we will calculate it again using our lattice theory with the lattice cutoff as a regulator. First we need the single particle free propagator:
\beq
\label{eq:oneprop}
G_0(\tau,\vec{p}) &=& \langle \vec{p},\tau | \left(D^{-1}\right)^{\tau+1}|\vec{p},0\rangle = \left(1+\frac{\Delta(p)}{M}\right)^{-(\tau+1)} \ , \cr
\Delta(p) &\equiv& -\frac{1}{2}\langle \vec{p}|\nabla_L^2 | \vec{p}\rangle \cr
&=& \sum_i \sin^2 \frac{p_i}{2} \ ,
\eeq
where I've set $b_s=1$ (we will use this convention from now on until we begin to discuss systematic errors), and have used the previously defined discretized Laplacian operator. I've written the propagator in a mixed $\vec{p},\tau$ representation, as this is often useful in lattice calculations for calculating correlation functions in time when the kinetic operator, $D$, is diagonal in momentum space. 

The diagrammatic two-particle scattering amplitude is shown on the bottom line in \Fig{dimer}. Because we have chosen the interaction to be separable, the amplitude can be factorized:
\beq
\label{eq:aint}
A=g_0\left[1+\sum_n(g_0\hat{L})^n\right] = \frac{g_0}{1-g_0\hat{L}} \ ,
\eeq
where the one loop integral, $\hat{L}$, will be defined below. As before, in order to set a single coupling we need one observable, so we use the effective range expansion for the scattering phase shift to leading order,
\beq
\label{eq:aERE}
A = \frac{4\pi}{M} \frac{1}{p\cot\delta-i p} \approx -\frac{4\pi a}{M} \ .
\eeq
Relating \Eqs{aint}{aERE}, we find
\beq
\label{eq:eigeqscat}
\frac{1}{g_0} = -\frac{M}{4\pi a} + \hat{L} \ .
\eeq

We will now evaluate the loop integral using the free single particle propagators, \Eq{oneprop},
\beq
\hat{L} &=& \frac{1}{V} \sum_{\vec{p}}\sum_{\tau=0}^{\infty} \left[G_0(\tau,\vec{p})\right]^2 \cr
&=& \frac{1}{V} \sum_{\vec{p}}\sum_{\tau=0}^{\infty} \frac{1}{\left(1+\frac{\Delta(p)}{M}\right)^{2\tau+2}} \cr
&=& \frac{1}{V} \sum_{\vec{p}} \frac{1}{\left(1+\frac{\Delta(p)}{M}\right)^2}\left[1+\sum_{\tau=0}^{\infty}\frac{1}{\left[\left(1+\frac{\Delta(p)}{M}\right)^2\right]^\tau}\right] \cr
&=&\frac{1}{V} \sum_{\vec{p}}\frac{M}{2}\frac{1}{\Delta(p)\left(1+\frac{\Delta(p)}{2M}\right)} \ .
\eeq
This final sum may be calculated numerically for a given $M$ and $L$ (governing the values of momenta included in the sum), as well as for different possible definitions of the derivative operators contained in $\Delta$, giving the desired coupling, $g_0$, via \Eq{eigeqscat}.

The second method for setting the coupling utilizes the calculation of the ground state energy of two particles. We start with the two-particle correlation function,
\beq
C_2(\tau) = \frac{1}{Z}\int \calD \phi \calD \psidag \calD \psi e^{-S[\psidag, \psi, \phi]} \Psi^{\dagger}_{\mbox{\tiny src,2}} \Psi_{\mbox{\tiny snk,2}} \ ,
\eeq
where $\Psi_{\mbox{\tiny src,2(snk,2)}}$ is a source (sink) wavefunction involving one spin up and one spin down particle. Integrating out the fermion fields gives,
\beq
C_2(\tau) &=& \frac{1}{Z_{\phi}}\int \calD \phi P[\phi] \langle \Psi_{\mbox{\tiny snk,2}}|K^{-1}(\tau,0) \otimes K^{-1}(\tau,0) | \Psi_{\mbox{\tiny src,2}}\rangle \cr
&=& \frac{1}{4\tau}\sum_{\phi=\pm 1}\langle \Psi_{\mbox{\tiny snk,2}} | D^{-1} \otimes D^{-1} X(\phi_\tau)\otimes X(\phi_{\tau})D^{-1}\otimes D^{-1} X(\phi_{\tau-1})\otimes X(\phi_{\tau-1}) \cdots |\Psi_{\mbox{\tiny src,2}} \rangle \ . \cr
\eeq
I will now write out the components of the matrices explicitly:
\beq
\label{eq:c2xspace}
C_2(\tau) &=& \frac{1}{4\tau}\sum_{x_1 x_2 x_1' x_2' \cdots y_1y_2}\sum_{\phi_{x_1} \phi_{x_1'}\cdots = \pm 1} \langle \Psi_{\mbox{\tiny snk,2}}|x_1 x_2\rangle D^{-1}_{x_1 x_1'}D^{-1}_{x_2 x_2'}(\delta_{x_1 x_1'} + \sqrt{g_0}\phi_{x_1}\delta_{x_1 x_1'})(\delta_{x_2 x_2'} + \sqrt{g_0}\phi_{x_2}\delta_{x_2 x_2'}) \cr
&& \times D^{-1}_{x_1' x_1''}D^{-1}_{x_2 x_2''} \cdots \langle y_1 y_2|\Psi_{\mbox{\tiny src,2}}\rangle  \ .
\eeq
The first (last) piece in angle brackets represents the position space wavefunction created by the sink (source). All $\phi$ fields in \Eq{c2xspace} are uncorrelated, so we can perform the sum for each time slice independently. One such sum is given by,
\beq
&&\frac{1}{4}\sum_{x_1x_1'x_2x_2'}\sum_{\phi_{x_1}\phi_{x_2}} \delta_{x_1x_1'}\delta_{x_2x_2'}(1+\sqrt{g_0}\phi_{x_1}+\sqrt{g_0}\phi_{x_2}+g_0 \phi_{x_1}\phi_{x_2}) \cr
&=&\sum_{x_1x_2}(1+g_0\delta_{x_1x_2}) \ ,
\eeq
where the cross terms vanish upon performing the sum. If we make the following definitions,
\beq
\langle x_1 x_1' | \calD ^{-1} | x_2 x_2' \rangle \equiv D^{-1}_{x_1 x_1'} D^{-1}_{x_2 x_2'} \ ,  \qquad \langle x_1 x_2 | \calV | x_1' x_2' \rangle \equiv g_0 \delta_{x_1x_1'}\delta_{x_2x_2'}\delta_{x_1x_2}  \ ,
\eeq
then we can write the two-particle correlation function as,
\beq
C_2(\tau) &=& \langle \Psi_{\mbox{\tiny snk,2}}| \calD^{-1}(1+\calV)\calD^{-1}(1+\calV) \cdots \calD^{-1}(1+\calV)\calD^{-1} | \Psi_{\mbox{\tiny src}} \rangle \cr
&=& \langle \Psi_{\mbox{\tiny snk}} | \calD^{-1/2} \calT \calD^{-1/2} | \Psi_{\mbox{\tiny src,2}} \rangle \ ,
\eeq
where I have made the definition
\beq
\label{eq:transmat}
\calT \equiv \calD^{-1/2}(1+\calV) \calD^{-1/2} \ .
\eeq
Recall from statistical mechanics that correlation functions may be written as $\tau$ insertions of the transfer matrix, $e^{-H}$, acting between two states,
\beq
C(\tau) &=& \langle \Psi_{\mbox{\tiny snk,2}} | e^{-H\tau} | \Psi_{\mbox{\tiny src,2}} \rangle \cr 
&=& \langle \Psi_{\mbox{\tiny snk,2}} | \left[e^{-H}\right]^{\tau} | \Psi_{\mbox{\tiny src,2}} \rangle \ .
\eeq 
Then we may identify $\calT$ in \Eq{transmat} as the transfer matrix of the theory, $\calT = e^{-H}$. This in turn implies that the logarithm of the eigenvalues of $\calT$ give the energies of the two-particle system.

We will now evaluate the transfer matrix in momentum space:
\beq
\label{eq:transexplicit}
\langle p q| \calT | p'q'\rangle &=& \sum_{kk'll'}\langle pq|\calD^{-1/2}|kl\rangle \langle kl | 1+\calV | k'l' \rangle \langle k'l' |\calD^{-1/2}|p'q'\rangle \cr
&=& \sum_{kk'll'} \delta_{k'p'}\delta_{l'q'}\delta_{pk}\delta_{ql} \left(\delta_{kk'}\delta_{ll'}+\delta_{k+l,k'+l'} \frac{g_0}{V}\right) \cr
&\times& \left[\frac{1}{\left(1+\frac{\Delta(p)}{M}\right)\left(1+\frac{\Delta(q)}{M}\right)\left(1+\frac{\Delta(p')}{M}\right)\left(1+\frac{\Delta(q')}{M}\right)}\right]^{1/2} \cr
&=&\frac{\delta_{pp'}\delta_{qq'}+\frac{g_0}{V}\delta_{p+q,p'+q'}}{\sqrt{\xi(p)\xi(q)\xi(q')\xi(p')}} \ ,
\eeq
where I have made the definition,
\beq
\xi(p) \equiv 1+\frac{\Delta(q)}{M} \ .
\eeq
The eigenvalues of the matrix $\calT$ may be evaluated numerically to reproduce the entire two-particle spectrum. However, for the moment we only need to set a single coupling, $g_0$, so one eigenvalue will be sufficient. The largest eigenvalue of the transfer matrix, corresponding to the ground state, may be found using a simple variational analysis\footnote{Many thanks to Michael Endres for the following variational argument.}. Choosing a simple trial state wavefunction,
\beq
\langle pq| \Psi \rangle = \frac{\psi(p)}{\sqrt{V}}\delta_{p,-q} \ ,
\eeq
subject to the normalization constraint,
\beq
\frac{1}{V}\sum_p |\psi(p)|^2 = 1 \ ,
\eeq
we now need to maximize the following functional:
\beq
\langle \Psi |\calT | \Psi \rangle = \left[\frac{1}{V} \sum_p \frac{|\psi(p)|^2}{\xi^2(p)} + \frac{g_0}{V^2}\left| \sum_p \frac{\psi(p)}{\xi(p)} \right|^2 + \lambda \left(1-\frac{1}{V}\sum_p |\psi(p)|^2\right)\right] \ ,
\eeq
where $\lambda$ is a Lagrange multiplier enforcing the normalization constraint, and I have used the fact that $\xi(p)$ is symmetric in $p$ to simplify the expression. Taking a functional derivative with respect to $\psidag(q)$ on both sides gives
\beq
-\lambda \psi(q) + \frac{\psi(q)}{\xi^2(q)} + \frac{g_0}{V}\sum_p \frac{\psi(p)}{\xi(p)\xi(q)} = 0 \ ,
\eeq
where I have set the expression equal to zero in order to locate the extrema. Rearranging this equation, then taking a sum over $q$ on both sides gives
\beq
\sum_q\frac{\psi(q)}{\xi(q)} &=& \sum_q \frac{g_0}{V} \frac{1}{\lambda \xi^2(q)-1} \sum_p \frac{\psi(p)}{\xi(p)} \ ,
\eeq
finally resulting in
\beq
\label{eq:eigeqlambda}
1 = \frac{g_0}{V}\sum_q \frac{1}{\lambda \xi^2(q)-1} \ .
\eeq
We now have an equation involving two unknowns, $\lambda$ and $g_0$. We need a second equation in order to determine these two parameters. We may use the constraint equation to solve for $\psi(p)$, giving
\beq
\label{eq:varpsi}
\psi(p) = \calN \frac{\xi(p)}{\lambda \xi^2(p)-1} \ , \qquad \frac{1}{\calN^2} = \frac{1}{V} \sum_p \frac{\xi^2(p)}{\left[\lambda \xi^2(p)-1\right]^2} \ .
\eeq
Plugging this back in to our transfer matrix we find,
\beq
\langle \Psi | \calT | \Psi \rangle = \lambda \ .
\eeq
This tells us that $\lambda$ is equivalent to the eigenvalue we sought, $E_0 = -\ln \lambda(g_0)$. As a check, we can compare \Eqs{eigeqscat}{eigeqlambda} in the unitary limit: $a \to \infty, \lambda \to 1$, giving
\beq
\frac{1}{g_0} = \frac{M}{2V}\sum_p \frac{1}{\Delta\left(1+\frac{\Delta}{2M}\right)}
\eeq
for both Equations.

In \Sec{tuning} we will discuss a simple formalism for determining the exact two particle spectrum in a box for any given scattering phase shift. This will allow us to eliminate certain finite volume systematic errors automatically. The transfer matrix method is also powerful because it gives us access to the entire two particle, finite-volume spectrum. When we discuss improvement in \Sec{improve}, we will add more operators and couplings to the interaction in order to match not only the ground state energy we desire, but higher eigenvalues as well. This will allow us to control the interaction between particles with non-zero relative momentum. To gain access to higher eigenvalues, the transfer matrix must be solved numerically, however, this may be accomplished quickly and easily for a finite volume system. 

\section{\label{sec:observables}Calculating observables}
Perhaps the simplest observable to calculate using lattice (or any imaginary time) methods is the ground-state energy. While the two-body system may be solved exactly and used to set the couplings for two-body interactions, correlation functions for $N$-body systems can then be used to make predictions. However, the transfer matrix for $N\gtrsim 4$ cannot in general be solved exactly, because the dimension of the matrix increases with particle number. For this reason we form instead $N$-body correlation functions,
\beq
C_N(\tau)=\frac{1}{Z}\int \calD\phi\calD\psidag\calD\psi e^{-S[\psidag,\psi,\phi]}\Psi_{b_1 \cdots b_N}^{(b)}(\tau)\Psi_{a_1 \cdots a_N}^{\dagger (a)}(0) \ ,
\eeq
where 
\beq
\Psi_{a_1 \cdots a_N}^{(a)\dagger}(\tau) = \int dx_1\cdots dx_N A^{(a)}(x_1\cdots x_N)\psi_{a_1}(x_1,\tau) \cdots \psi_{a_N}(x_N,\tau)
\eeq
is a source for $N$ particles with spin/flavor indices $a_1 \cdots a_N$, and a spatial wavefunction $A^{(a)}(x_1 \cdots x_N)$. For the moment the only requirement we will make of the wavefunction is that it has non-zero overlap with the ground-state wavefunction (i.e. it must have the correct quantum numbers for the state of interest).

Recall that a correlation function consists of $\tau$ insertions of the transfer matrix between source and sink. We can then expand the correlation function in a basis of eigenstates,
\beq
C_N(\tau) &=& \frac{1}{Z} \langle \tilde{\Psi}_{a_1\cdots a_N}^{(a)} | e^{-H\tau}|\tilde{\Psi}_{b_1\cdots b_N}^{(b)} \rangle = \frac{1}{Z} \sum_{m,n} \langle \tilde{\Psi}_{a_1\cdots a_N}^{(a)} |m\rangle \langle m | e^{-H\tau} | n \rangle \langle n | \tilde{\Psi}_{b_1\cdots b_N}^{(b)} \rangle \cr
&=&\sum_m Z_m^{(a)} Z_m^{*(b)} e^{-E_n\tau} \ ,
\eeq
where $Z_m^{(a)}$ is the overlap of wavefunction $a$ with the energy eigenstate $m$, and $E_n$ is the $n$th eigenvalue of the Hamiltonian. In the limit of large Euclidean time (zero temperature), the ground state dominates,
\beq
C_N(\tau) \tautoinfty Z_0^{(a)} Z_0^{*(b)}e^{-E_0 \tau} \ ,
\eeq
with higher excited states exponentially suppressed by $\sim e^{-\Delta_{n0}\tau}$, where $\Delta_{n0} \equiv E_n - E_0$ is the energy splitting between the $n$th state and the ground state. It should be noted that for a non-relativistic theory the rest masses of the particles do not contribute to these energies, so the ground state energy of a single particle at rest is $E_0=0$, in contrast to lattice QCD formulations.

In this way, we can think of the transfer matrix as acting as a filter for the ground state, removing more excited state contamination with each application in time. A common method for determining the ground state energy from a correlation function is to construct the so-called effective mass function,
\beq
M_{\mbox{\tiny eff}}(\tau) \equiv \ln \frac{C(\tau)}{C(\tau+1)} \tautoinfty E_0 \ ,
\eeq
and look for a plateau at long times, whose value corresponds to the ground-state energy.

Once the ground state has been isolated, we can calculate matrix elements with the ground state as follows,
\beq
\langle \Psi_{a_1\cdots a_N}^{(a)} | A(\tau')|\Psi_{b_1\cdots b_N}^{(b)} \rangle &=& \sum_{lmnq} \langle \Psi_{a_1\cdots a_N}^{(a)} | l \rangle \langle l | e^{-H(\tau-\tau')} | m \rangle \langle m | A | n \rangle \langle n | e^{-H\tau'} | q \rangle \langle q | \Psi_{b_1\cdots b_N}^{(b)} \rangle \cr
&=& \sum_{ln} Z_l^{(a)} Z_n^{*(b)} e^{-E_l(\tau-\tau')}e^{-E_n\tau'} \langle m | A | n \rangle \ .
\eeq
To filter out the ground state, the matrix element insertion $A$ must be placed sufficiently far in time from both source and sink, $\{ \Delta_{l0}(\tau-\tau'), \Delta_{n0} \tau' \} \gg 1$,
\beq
\underset{\tau,\tau'\to\infty}{\longrightarrow} Z_0^{(a)} Z_0^{*(b)} e^{-E_0 \tau} \langle 0 | A | 0 \rangle \ .
\eeq
In order to isolate the matrix element and remove unknown $Z$ factors and ground state energies, ratios may be formed with correlation functions at various times, similar to the effective mass function.

Another observable one may calculate using lattice methods is the scattering phase shift between interacting particles. Because all lattice calculations are performed in a finite volume, which cannot accommodate true asymptotic scattering states, direct scattering measurements are not possible. However, a method has been devised by L\"uscher which uses finite volume energy shifts to infer the interaction, and therefore, the infinite volume scattering phase shift. The L\"uscher method will be discussed further in \Sec{Luscher}. Because the inputs into the L\"uscher formalism are simply energies, correlation functions may be used in the same way as described above to produce this data.

\subsection{\label{sec:SNR}Signal-to-noise}
Recall that we must use Monte Carlo methods to approximate the partition function using importance sampling,
\beq
C(\tau) \approx \frac{1}{\Ncfg} \sum_{i=1}^{\Ncfg} C(\phi_i,\tau) \tautoinfty Z_0 e^{-E_0\tau}\ ,
\eeq
where $C(\phi_i,\tau)$ is the operator for some correlation function of interest evaluated on a single configuration $\phi_i$, and the set of all fields, $\phi$, are generated according to the appropriate probability distribution. In the long Euclidean time limit we expect that this quantity will give us an accurate value for the ground state energy. As stated previously, if the ensemble is large enough for the central limit theorem to hold, then the error on the mean (noise) will be governed by the sample standard deviation,
\beq
\sigma_C^2(\tau) = \frac{1}{\Ncfg}\left[ \sum_{i=1}^{\Ncfg}|C(\phi_i,\tau)|^2 - \left|\sum_{i=1}^{\Ncfg}C(\phi_i,\tau)\right|^2\right] \ .
\eeq

As an example of how to estimate the size of the fluctuations relative to the signal, let's consider a single particle correlation function, consisting of a single propagator,
\beq
\frac{1}{Z_{\phi}}\int \calD \phi P(\phi) \langle \Psi_a | K^{-1}(\phi,\tau) | \Psi_b \rangle \approx \frac{1}{\Ncfg}\sum_{i=1}^{\Ncfg} K_{ab}^{-1}(\phi_i,\tau) \ ,
\eeq 
where the indices $\{ ab\}$ indicate projection onto the states specified by the source/sink. In the large Euclidean time limit, this object will approach a constant, $Z_0$, because the ground state energy for a single particle is $E_0=0$. For the non-relativistic theory as we have set it up, the matrix $K$ is real so long as $g_0>0$ (attractive interaction). The standard deviation is then given by
\beq
\label{eq:sig1part}
\sigma_{C_1}^2(\tau) = \frac{1}{\Ncfg}\left[ \sum_{i=1}^{\Ncfg} \left(K_{ab}^{-1}(\phi_i,\tau)\right)^2 - \left( \sum_{i=1}^{\Ncfg} K_{ab}^{-1}(\phi_i,\tau)\right)^2 \right] \ .
\eeq
The second term on the right hand side of the above equation is simply the square of the single particle correlation function, and will therefore also go to a constant, $Z_0^2$, for large Euclidean time. To gain an idea of how large the first term of $\sigma_{C_1}^2$ is, let's take a look at a correlation function for one spin up and one spin down particle,
\beq
C_2(\tau)=\frac{1}{Z}\int \calD\phi\calD\psidag\calD\psi e^{-S[\psidag,\psi,\phi]}\psi_{\uparrow}^{(b)}(\tau)\psi_{\downarrow}^{(b)}(\tau)\psi_{\uparrow}^{\dagger (a)}(0)\psi_{\downarrow}^{\dagger (a)}(0) \ ,
\eeq
where I have chosen the same single particle source (sink), $\psi^{(a)}$ ($\psi^{(b)}$), for both particles (this is only allowed for bosons or for fermions with different spin/flavor labels). After integrating out the $\psi$ fields we have
\beq
C_2(\tau) = \frac{1}{Z_{\phi}}\int \calD\phi P(\phi) K_{ab}^{-1}(\phi,\tau) K_{ab}^{-1}(\phi,\tau) \ ,
\eeq
which is approximately given by
\beq
C_2(\tau) \approx \frac{1}{\Ncfg}\sum_{i=1}^{\Ncfg}\left[ K_{ab}^{-1}(\phi_i,\tau)\right]^2 \ .
\eeq
This is precisely what we have for the first term on the right hand side of \Eq{sig1part}. Therefore, this term should be considered a two-particle correlation function, whose long Euclidean time behavior is known. Note that we must interpret this quantity as a two-particle correlation function whose particles are either bosons or fermions with different spin/flavor labels due to the lack of anti-symmetrization. 

We may now write the long-time dependence of the variance of the single particle correlator as
\beq
\sigma_{C_1}^2(\tau) \approx C_2(\tau) - \left( C_1(\tau) \right)^2 \tautoinfty Z_2e^{-E_0^{(2)} \tau} - Z_1^2 \ ,
\eeq
where $E_0^{(2)}$ is the ground state energy of the two-particle system. For a two-body system with an attractive interaction in a finite volume, $E_0^{(2)}< 0$, and we may write
\beq
\sigma_{C_1}^2(\tau) \tautoinfty Z_2e^{E_B^{(2)} \tau} - Z_1^2 \ ,
\eeq
where I've defined $E_B^{(2)} \equiv -E_0^{(2)}$. This tells us that $\sigma_{C_1}^2$, and therefore the noise, grows exponentially with time. We can write the signal-to-noise ratio $\calR_{C_1}(\tau)$ as
\beq
\calR_{C_1}(\tau) \equiv \frac{C_1(\tau)}{\frac{1}{\sqrt{\Ncfg}} \sigma_{C_1}(\tau)} \tautoinfty \sqrt{\Ncfg} \frac{Z_1}{\sqrt{Z_2} e^{E_B^{(2)}\tau/2}} = \sqrt{\Ncfg}\frac{Z_1}{\sqrt{Z_2}}e^{-E_B^{(2)} \tau/2} \ ,
\eeq
where I've dropped the constant term in $\sigma_{C_1}^2$, because it is suppressed in time relative to the exponentially growing term. This expression indicates that the signal-to-noise ratio itself grows exponentially with time, and therefore an exponentially large $\Ncfg$ will be necessary to extract a signal at large Euclidean time. Unfortunately, large Euclidean time is necessary in order to isolate the ground state. 

This exponential signal-to-noise problem is currently the limiting factor in system size for the use of any lattice method for nuclear physics. Here, we will discuss it in some detail because in many cases understanding the physical basis behind the problem can lead to methods for alleviation. One method we can use is to employ knowledge of the wavefunction of the signal and/or the wavefunction of the undesired noise in order to maximize the ratio of $Z$-factors, $Z_1/\sqrt{Z_2}$. For example, choosing a plane wave source for our single particle correlator gives perfect overlap with the desired signal, but will give poor overlap with the bound state expected in the noise. This leads to what has been referred to as a ``golden window" in time where the ground-state dominates before the noise begins to turn on \cite{Beane:2010em}. In general, choosing a perfect source for the signal is not possible, however, a proposal for simultaneously maximizing the overlap with the desired state as well as reducing the overlap with the noise using a variational principle has been proposed in \cite{Detmold:2014hla,Detmold:2014rfa}. We will discuss other methods for choosing good interpolating fields in \Sec{interp}, in order to allow us to extract a signal at earlier times where the signal-to-noise problem is less severe.

Another situation where understanding of the noise may allow us to reduce the noise is when the auxiliary fields and couplings used to generate the interactions can often be introduced in different ways, for instance, via the density channel vs. the BCS channel as mentioned previously. While different formulations can give the same effective interaction, they may lead to different sizes of the fluctuations. Understanding what types of interactions generate the most noise is therefore crucial. This will become particularly relevant when we discuss adding interactions beyond leading order to our EFT in \Sec{NLO}, where different combinations of interactions can be tuned to give the same physical observables.

Let's now discuss what happens to $\sigma_{C_1}^2$ if we have a repulsive interaction ($g_0<0$). Because nuclear potentials have repulsive cores, such a scenario occurs for interactions at large energy. Since the auxiliary-field-mediated interaction is given by $\sqrt{g_0}\phi\psidag\psi$, this implies that the interaction is complex. Our noise is now given by
\beq
\sigma_{C_1}^2(\tau) = \frac{1}{\Ncfg} \sum_{i=1}^{\Ncfg} K_{ab}^{-1}(\phi_i,\tau) \left[K_{ab}^{-1}(\phi_i,\tau)\right]^{\dagger} - |C_1(\tau)|^2 \ .
\eeq
Recall that the single particle propagator can be written
\beq
K^{-1}(\phi_i,\tau) = D^{-1} X(\phi_{i,\tau})D^{-1} X(\phi_{i,\tau-1}) \cdots \qquad X(\phi_{i,\tau}) = 1+\sqrt{g_0}\phi_{i,\tau} \ .
\eeq
The complex conjugate of the propagator then corresponds to taking $\phi \to -\phi$,
\beq
\left[ K^{-1}(\phi_i,\tau)\right]^{\dagger} = D^{-1} X(-\phi_{i,\tau})D^{-1} X(-\phi_{i,\tau-1})  \ .
\eeq
Again, $\phi$ fields on different time slices are independent, so we may perform each sum over $\phi = \pm 1$ separately. Each sum that we will encounter in the two-particle correlator consists of the product of $X(\phi_{\tau})X(-\phi_{\tau})$,
\beq
\sum_{\phi}(1+\sqrt{g_0}\phi)(1-\sqrt{g_0}\phi) = 1-g_0^2 = 1+|g_0|^2 \ ,
\eeq
which is exactly the same as we had for the attractive interaction. This implies that even though the interaction in the theory we're using to calculate the correlation function is repulsive, the noise is controlled by the energy of two particles with an attractive interaction, which we have already investigated. In this particular case for a single particle propagator, the signal-to-noise ratio is the same regardless of the sign of the interaction\footnote{This argument is somewhat simplified by our particular lattice setup in which we have no fermion determinant as part of the probability measure. For cases where there is a fermion determinant, there will be a mismatch between the interaction that the particles created by the operators see (attractive) and the interaction specified by the determinant used in the probability measure (repulsive). This is known as a partially quenched theory, and is unphysical. However, one may calculate a spectrum using an effective theory in which valence (operator) and sea (determinant) particles are treated differently. Often it is sufficient to ignore the effects from partial quenching because any differences contribute only to loop diagrams and may be suppressed.}. 

In general, however, signal-to-noise problems for systems with repulsive interactions are exponentially worse than those for attractive interactions. This is because generically the signal-to-noise ratio falls off as,
\beq
\calR \sim e^{-\left(E_{\mathcal{S}} - E_{\mathcal{N}}/2\right)\tau} \ ,
\eeq
where $E_{\mathcal{S}(\mathcal{N})}$ is the ground-state energy associated with the signal (noise). Because the signal corresponds to a repulsive system while the noise corresponds to an attractive system, the energy difference in the exponential will be greater than for a signal corresponding to an attractive system. 

\subsubsection{\label{sec:sign}Sign Problems}
A related but generally more insidious problem can occur in formulations having fermion determinants in the probability measure, known as a sign problem. A sign problem occurs when the determinant is complex, for example, in our case of a repulsive interaction. While we were able to eliminate the fermion determinant in one particular formulation, there are situations when having a fermion determinant in the probability measure may be beneficial, for example, when using forms of favorable reweighting, as will be discussed later on, or may be necessary, such as for non-zero chemical potential or finite temperature, when the boundary conditions in time may not be altered. For these reasons, we will now briefly discuss sign problems. 

The basic issue behind a sign problem is that a probability measure, by definition, must be real and positive. Therefore, a complex determinant cannot be used for importance sampling. Methods to get around the sign problem often result in exponentially large fluctuations of the observable when calculated on a finite sample, similar to the signal-to-noise problem (the two usually result from the same physical mechanism). One particular method is called reweighting, in which a reshuffling occurs between what is considered the ``observable" and what is considered the ``probability measure". For example, when calculating an observable,
\beq
\langle \calO \rangle = \frac{1}{Z_{\phi}}\int \calD\phi P(\phi)\calO(\phi) \ ,
\eeq
when $P(\phi)$ is complex, we can multiply and divide by the magnitude of $P(\phi)$ in both numerator and denominator,
\beq
\langle \calO \rangle = \frac{\int \calD\phi |P(\phi)| \frac{P(\phi)\calO(\phi)}{|P(\phi)|}}{\int \calD\phi |P(\phi)| \frac{P(\phi)}{|P(\phi)|}} \ ,
\eeq
as well as multiply and divide by $\tilde{Z}_{\phi} \equiv \int \calD \phi |P(\phi)|$,
\beq
\label{eq:reweight}
\langle \calO \rangle = \frac{\int \calD\phi |P(\phi)| \frac{P(\phi)\calO(\phi)}{|P(\phi)|}}{\tilde{Z}_{\phi}}\left/\frac{\int \calD\phi |P(\phi)| \frac{P(\phi)}{|P(\phi)|}} {\tilde{Z}_{\phi}} = \langle \calO' \rangle_{|P|}\left/\langle \calO''\rangle_{|P|}\right.\right. \ ,
\eeq
where
\beq
\calO' \equiv \frac{P(\phi)\calO(\phi)}{|P(\phi)|} \ , \qquad \calO'' \equiv \frac{P(\phi)}{|P(\phi)|} \ ,
\eeq
and $\langle \cdots \rangle_{|P|}$ implies that the path integrals in the expectation values use the measure $|P(\phi)|$. The advantage is that now the probability measure used for sampling is real and positive, at the cost of having to calculate two observables, $\calO', \calO''$. The real disadvantage, however, is that the second observable, $\calO''$ corresponds to the complex phase of the original measure, $P(\phi)$, which is highly oscillatory from field configuration to field configuration. 

We can measure the size of the fluctuations of the phase of $P(\phi)= \left[\det K(\phi)\right]^2$, corresponding to a two-spin (or flavor) theory with a repulsive interaction,
\beq
\langle \calO'' \rangle_{|P|} = \frac{\int \calD \phi \det K(\phi)\det K^*(\phi)}{\int \calD \phi \left[\det K(\phi)\right]^2} \ .
\eeq
The denominator of the above ratio corresponds to the partition function of the original theory which has two spins of particles interacting via a repulsive interaction. The numerator also corresponds to the partition function of a two-spin theory. However, recall that $K^*(\phi)$ corresponds to a propagator with the opposite sign on the interaction term. Because fermions of the same spin don't interact (Pauli principle), the only interaction in this theory is that between two particles of opposite spin, which we established previously will be an attractive interaction due to the sign flip on $K^*(\phi)$. Thus, the numerator corresponds to the partition function of a two-spin theory with an attractive interaction. 

A partition function is simply the logarithm of the free energy, $Z=e^{-\beta F}$. For a system in a finite volume at zero temperature this becomes $Z=e^{-V \calE_0}$, where $\calE_0$ is the energy density of the ground state of the theory. This implies that 
\beq
\label{eq:expectationsign}
\langle \calO'' \rangle_{|P|} \underset{\tau\to\infty}{\sim} e^{-V(\calE_0^{(\mbox{\tiny rep})}-\calE_0^{(\mbox{\tiny att})})} \ ,
\eeq
where $\calE_0^{(\mbox{\tiny rep})}$ ($\calE_0^{(\mbox{\tiny att})}$) is the energy density of the ground state of the repulsive (attractive) theory. Generically, $\calE_0^{(\mbox{\tiny att})} \leq \calE_0^{(\mbox{\tiny rep})} $, for theories which are identical up to the sign of their interaction. This may be shown using the Cauchy-Schwarz theorem,
\beq
\langle | \det K(\phi)| \rangle \leq | \langle \det K(\phi) \rangle | \ .
\eeq
Therefore, $\langle \calO'' \rangle_{|P|}$ will be exponentially small for large Euclidean times so long as $\calE_0^{(\mbox{\tiny rep})} \neq \calE_0^{(\mbox{\tiny att})}$. The variance, on the other hand, is
\beq
\langle |\calO'' |^2\rangle_{|P|} - |\langle \calO'' \rangle_{|P|}|^2 = \langle 1 \rangle - |\langle \calO'' \rangle_{|P|}|^2 \underset{\tau\to\infty}{\sim} 1-e^{-2V(\calE_0^{(\mbox{\tiny rep})}-\calE_0^{(\mbox{\tiny att})})} \sim 1 \ .
\eeq

So again, we have an exponentially small signal-to-noise ratio at large Euclidean time for the observable $\calO''$. This argument is very similar to our signal-to-noise argument for correlation functions. In general, if a theory has a sign problem there will be a corresponding signal-to-noise problem for correlation functions. The reverse is not always true, however, because reweighting is only necessary when the integration measure is complex, so even if there is a signal-to-noise problem in calculating correlation functions (as there is for an attractive interaction), a sign problem may not arise. Sign problems are in general far more problematic due to the exponential scaling with the volume, and because correlation functions give us the additional freedom of choosing interpolating fields in order to try to minimize the noise. In some cases, however, it may be possible to use knowledge learned from signal-to-noise problems in order to solve or reduce sign problems, and vice-versa \cite{Grabowska:2012ik,Nicholson:2012xt,EKLN5}.

\subsubsection{Noise in Many-Body Systems}

Let us now discuss signal-to-noise ratios for $N$-body correlation functions. First, we'll look at the two-particle case. We have already defined the correlation function for two particles with different spin/flavor labels,
\beq
C_2(\tau) = \langle\left[K_{ab}^{-1}(\phi_i,\tau)\right]^2\rangle \ .
\eeq
The variance is given by
\beq
\sigma_{C_2}^2(\tau) = \langle \left[K_{ab}^{-1}(\phi_i,\tau)\right]^4\rangle - \left(C_2(\tau)\right)^2 \ .
\eeq
It is simple to see that the first term in this expression corresponds to a four-particle correlation function, where each particle has a different flavor/spin index (because there is no anti-symmetrization of the fermion fields). Thus, we can write,
\beq
\sigma_{C_2}^2(\tau) = C_4(\tau) - \left(C_2(\tau)\right)^2 \ ,
\eeq
where $C_4(\tau)$ corresponds to a correlator with four particles having different flavors. This is much like a correlator for an alpha particle in the spin/flavor $SU(4)$ limit, thus, it will be dominated at large times by the binding energy, $E_B^{(4)}$, of a state with a large amount of binding energy per particle. Our signal-to-noise ratio is then,
\beq
\calR_{C_2}(\tau) \underset{\tau\to\infty}{\sim} \frac{e^{E_B^{(2)} \tau}}{e^{E_B^{(4)}\tau/2}} \ ,
\eeq
where, $E_B^{(4)}/2 > E_B^{(2)}$. Therefore, the signal-to-noise ratio is again falling off exponentially in time; this problem clearly becomes worse as the coupling becomes stronger. Finally, we can consider a many-body correlator composed of a Slater determinant over $N$ single-particle states in a two spin/flavor theory,
\beq
\label{2Ncorr}
C_{2N}(\tau) = \langle \left[ \det K^{-1}(\phi_i,\tau)\right]^2 \rangle \ .
\eeq
The ground state of this correlator will be either a BEC or BCS state, as discussed earlier in \Sec{scatamp}. The noise, on the other hand, will be dominated by a system of alpha-like clusters, since the number of flavors in the noise is always double that of the signal, which can bind to form nuclei. The ground-state energy of this bound state will clearly be much lower than that of a dilute BEC/BCS state, and our signal-to-noise ratio will be exponentially small in the large time limit. 

In general this pattern continues for fermion correlators with any number of particles, spins, and flavors. This is because doubling the number of flavors reduces the amount of Pauli repulsion in the resulting expression for the variance. Even for bosonic systems signal-to-noise can be a problem, simply as a result of the Cauchy-Schwarz triangle inequality, which tells you that, at best, your signal-to-noise ratio can be $1$, corresponding to a non-interacting system. Turning on interactions then generally leads to exponential decay of the signal-to-noise ratio. Signal-to-noise problems also generally scale exponentially with the system size, leading to limitations on system size based on computational resources. Thus, understanding and combatting signal-to-noise problems is paramount to further development in the field.

\subsection{\label{sec:overlap}Statistical Overlap}

For the lattice formulations we have thus far explored one generates configurations according to the probability distribution associated with the vacuum. One then introduces sources to create particles, which are considered part of the ``observable". However, the configurations which are the most important for creating the vacuum may not necessarily be the most important for the observable one wishes to calculate. 

We can look to lattice QCD for a pedagogical example. In QCD, the fermion determinant encodes vacuum bubbles created by quark/anti-quark pairs. According to the tenets of confinement, bubbles with large spacetime area require a large energy to produce, and are therefore highly suppressed in the partition function. When doing importance sampling, small vacuum bubbles will dominate. On the other hand, if we now calculate an observable which introduces particle sources, a configuration involving a large vacuum bubble may become very important to the calculation. This is because the total relevant spacetime area of the given configuration, taking into account the particles created by the sources, can in fact be small (see \Fig{QCDbubble}). However, by sampling according to the vacuum probability, this configuration will be missed, skewing the calculation in an unknown manner. The farther the observable takes us from the vacuum, the worse this problem becomes, making this a particularly troublesome issue for many-body calculations.

\begin{figure}
\begin{center}
\includegraphics[width=0.3\linewidth]{Chapter5-figures/QCDbubble.pdf}
\end{center}
\caption{\label{fig:QCDbubble}A schematic of an example configuration in LQCD which may lead to a statistical overlap problem. Red propagators correspond to valence quarks (quarks created by the sources/sinks in the operator), while blue corresponds to sea quarks (vacuum bubbles generated via Monte Carlo). Due to confinement, large bubbles (determined by the area enclosed by the blue propagator) are suppressed in the QCD vacuum and thus will likely be thrown out during importance sampling. In the presence of quark sources, however, these configurations are very important in the calculation of the observable (due to the small area enclosed between the red and blue propagators).}
\end{figure}

Such problems are referred to as statistical overlap problems. Another situation where these overlap problems can often occur is when doing reweighting to evade a sign problem, as discussed in \Sec{sign}. For example, if the distribution being sampled corresponds to a theory with an attractive interaction, but the desired observable has a repulsive interaction, the Monte Carlo sampling will be unlikely to pick up the most relevant configurations, affecting the numerator of \Eq{reweight}.

We can understand the problem further by studying probability distributions of observables. While the distribution of the sampled field, $\phi$ in our case, may be peaked around the mean value of $\phi$, the distribution of the observable as calculated over the sample may not be peaked near the true mean of the observable. Such a distribution necessarily has a long tail. Plotting histograms of the values of the observable as calculated over the sample, $\{ C(\phi_1), C(\phi_2), \cdots C(\phi_{\Ncfg}) \}$, can allow us to gain an idea of the shape of the distribution for that observable. An example of a distribution with a statistical overlap problem is plotted in \Fig{overlap}. In this case, the peak of the distribution is far from the true mean. Values in the tail of the distribution have small weight, and are likely to be thrown out during importance sampling, skewing the sample mean without a corresponding increase in the error bar. The error bar is instead largely set by the width of the distribution near the peak. One way to determine whether there is an overlap problem is to recalculate the observable on a different sample size; if the mean value fluctuates significantly outside the original error bar this indicates an overlap problem.

\begin{figure}
\begin{center}
\includegraphics[width=0.5\linewidth]{Chapter5-figures/overlap.pdf}
\end{center}
\caption{\label{fig:overlap}Schematic drawing of a long-tailed probability distribution (blue) which leads to an overlap problem. Monte Carlo sampling leads to a sample distribution which is centered around the peak of the underlying distribution (red), far from the mean. The ideal probability distribution one would like to sample is narrow and centered around the mean (green).}
\end{figure}

The central limit theorem tells us that regardless of the initial distribution we pull from, the distribution of the mean should approach a Gaussian for a large enough sample size, so in principle we should be able to combat an overlap problem by brute force. However, what constitutes a ``large enough" sample size is dictated by the shape of the original distribution. The Berry-Esseen theorem \cite{BerryEsseen1,BerryEsseen2} can be used to determine that the number of configurations necessary to assume the central limit theorem applies is governed by
\beq
\sqrt{\Ncfg} \sim \frac{\langle \mathcal{X}^3\rangle}{\langle \mathcal{X}^2\rangle^{3/2}} \ ,
\eeq
where $\langle \mathcal{X}^n \rangle$ is the $n$th moment of the distribution of an observable, $\mathcal{X}$. Thus, a large skewness, or long tail, increases the number of configurations necessary before the central limit theorem applies, and therefore, to trust an error bar determined by the standard deviation of the distribution of the mean.

One could imagine repeating an argument similar to that made for estimating the variance of our correlation functions in order to estimate the third moment. For example, if our observable is the two-particle correlation function, $C_2(\tau)$, then the third moment will be
\beq
\langle \mathcal{X}^3 \rangle \sim \langle \left[K_{ab}(\phi_i,\tau)\right]^6 \rangle \ ,
\eeq
corresponding to a correlation function containing six particles of different flavors. Again, increasing the number of flavors generally increases the binding energy per particle of the system, leading to a third moment which is exponentially large compared to the appropriately scaled second moment. This implies that an exponentially large number of configurations will be necessary before the central limit theorem applies to the distribution of the mean of correlation functions calculated using this formulation. 

While we mentioned that using reweighting to avoid a sign problem is one situation where overlap problems often occur, it is also possible to use reverse reweighting in order to lessen an overlap problem. Here instead we would like to reweight in order to make the distribution of $\phi$ have \underline{more} overlap with the configurations that are important for the observable. An example that is commonly used is to include the desired correlation function itself, calculated at some fixed time, to be part of the probability measure. This may be accomplished using ratios of correlators at different times,
\beq
\frac{C_N(\tau'+\tau)}{C_N(\tau')} = \frac{\int \calD\phi \tilde{P}(\phi)\tilde{\calO}(\phi,\tau)}{\int\calD\phi \tilde{P}(\phi)} \ ,
\eeq
where
\beq
\tilde{P}(\phi) \equiv P(\phi)C_N(\tau',\phi) \ , \qquad \tilde{\calO}(\phi,\tau) \equiv \frac{C_N(\tau'+\tau,\phi)}{C_N(\tau',\phi)} \ .
\eeq
Now the probability distribution incorporates an $N$-body correlator at one time, $\tau'$, and will therefore do a much better job of generating configurations relevant for the $N$-body correlator at different times. A drawback of this method is that it is much more computationally expensive to require the calculation of propagators for the generation of eaach configuration. Furthermore, the configurations that are generated will be operator-dependent, so that calculating the correlator $C_{N+1}$ will require the generation of a whole new set of field configurations.

Another method for overcoming a statistical overlap problem is to try to get a more faithful estimate of the mean from the long-tailed distribution itself. To try to better understand the distribution, let's use our signal-to-noise argument to estimate higher moments of the distribution. We can easily estimate the $N$th moment of the correlation function for a single particle,
\beq
\calM_N \sim C_{N} \underset{\tau\to\infty}{\sim} e^{-E_0^{(N)}\tau} \ ,
\eeq
where $E_0^{(N)}$ is the ground-state energy of $N$ particles with different flavors. Let's consider the theory to be weakly coupled (small scattering length, $a/L \ll 1$). In this case the two-body interaction dominates and we can use perturbation theory to estimate the energy of two particles in a box: $E_0^{(2)} \approx \frac{4\pi a}{ML^3}$. A weakly coupled system of $N$ particles interacting via the two-body interaction is given by simply counting the number of possible pairs of interacting particles, $E_0^{(N)} \approx N(N-1) \frac{4\pi a}{ML^3}$, leading to the following expression for the moments \cite{DeGrand:2012ik}:
\beq
\label{eq:lnmoments}
\calM_N \sim e^{-N(N-1) \frac{4\pi a}{ML^3}} \ .
\eeq
Distributions with the particular $N$ dependence seen in \Eq{lnmoments} are called log-normal distributions, so named because the distribution of the logarithm of a log-normally distributed quantity is normal. While we derived this expression for theories near weak coupling, there is also evidence that the log-normal distribution occurs for correlators near unitarity as well \cite{Nicholson:2012zp,Nicholson:2015zxa}. 

The central limit theorem implies that normal distributions occur generically for large sums of random numbers; the same argument leads to the conclusion that log-normal distributions occur for large products of random numbers. Let's think about how correlation functions are calculated on the lattice: particles are created, then propagate through random fields from one time slice to the next until reaching a sink. Each application of the random field is multiplied by the previous one,
\beq
K^{-1}(\tau) = D^{-1}X(\tau)D^{-1}X(\tau-1) \cdots \ ,
\eeq
and then products of these propagators may be used to form correlation functions for multiple particles. Thus, one might expect that in the $\tau\to\infty$ limit (or for large numbers of particles), the distributions of these correlation functions might flow toward the log-normal distribution. More precisely though, each block $X(\tau)$ is actually a matrix of random numbers, and products of random matrices are far less well understand than products of random numbers. Nonetheless, products of random link variables are used to form most observables in nearly all lattice calculations, and approximately log-normal distributions appear to be ubiquitous as well, including in lattice QCD calculations.

If it is $\ln C$ that is nearly Gaussian rather than $C$, then it may be better to sample $\ln C$ as our observable instead. Without asserting any assumptions about the actual form of the distribution, we can expand around the log-normal distribution using what is known as a cumulant expansion,
\beq
\label{eq:cumulantexp}
\ln \langle \calO \rangle = \sum_{n=1}^{\infty} \frac{1}{n!} \kappa_n(\ln \calO) \ ,
\eeq
where $\kappa_n$ is the $n$th cumulant, or connected moment. The cumulants may be calculated using the following recursion relation:
\beq
\kappa_n(\mathcal{X}) = \langle \mathcal{X}^n \rangle -\sum_{m=1}^{n-1}\left(\begin{array}{c}
n-1 \\
m-1 
\end{array} \right)\kappa_m(\mathcal{X}) \langle \mathcal{X}^{n-m}\rangle \ .
\eeq
Note that the expansion in \Eq{cumulantexp} is an exact equality for an observable obeying any distribution. We may now expand the correlation function as
\beq
\ln \langle C \rangle \tautoinfty -E_0 \tau = \langle \ln C\rangle + \frac{1}{2}\left(\langle ( \ln C)^2 \rangle -\langle \ln C \rangle^2 \right) + \frac{1}{6} \kappa_3(\ln C) + \cdots \ .
\eeq
Again, this expansion is true for a correlation function obeying any distribution. However, if the distribution of $\ln C$ is exactly log-normal, then $\kappa_{n\geq 3}(\ln C) = 0$. If the distribution is approximately log-normal, then the third and higher cumulants are small corrections, further suppressed in the cumulant expansion by $1/n!$. This suggests that we may cut off the expansion after including a finite number of cumulants without significantly affecting the result (see \Fig{cumulant}). We may also include the next higher order cumulant in order to estimate any systematic error associated with our cutoff.

\begin{figure}
\begin{center}
\includegraphics[width=0.5\linewidth]{Chapter5-figures/cumulant}
\end{center}
\caption{\label{fig:cumulant} Results for the energy of 50 two-component fermions at unitarity using the cumulant expansion (\Eq{cumulantexp}) cut off at $\calO(N_k)$. Figure from \cite{EKLN4}.}
\end{figure}

The benefit of using the cumulant expansion to estimate the mean rather than using the standard method is that for a finite sample size, high-order cumulants of $\ln C$ are poorly measured, which is the culprit behind the overlap problem. However, for approximately log-normal distributions these high-order cumulants should be small in the infinite statistics limit. Thus, by not including them in the expansion we do a better job at estimating the true mean on a finite sample size. In other words, by sampling $\ln C$ rather than $C$, we have shifted the overlap problem into high, irrelevant moments which we may neglect.

The cumulant expansion avoids some of the drawbacks of reweighting, such as greatly increased computational effort in importance sampling. However, the farther the distribution is from log-normal, the higher one must go in the cumulant expansion, which can be particularly difficult to do with noisy data. Thus, for some observables it may be difficult to show convergence of the series on a small sample. Which method is best given the competition between the computational effort used in generating samples via the reweighting method versus the large number of samples which may be required to show convergence of the cumulant expansion is unclear and probably observable dependent. 

\subsection{\label{sec:interp}Interpolating Fields}

The previous section highlights the importance of gaining access to the ground state as early in time as possible, since the number of configurations required grows exponentially with time. Returning to our expression for the expansion of a correlation function in terms of energy eigenstates,
\beq
C(\tau) &=& Z_0 e^{-E_0 \tau}+Z_1 e^{-E_1 \tau} + \cdots \cr
&=& Z_0 e^{-E_0\tau}\left[1+\frac{Z_1}{Z_0}e^{-(E_1-E_0)\tau} + \cdots \right] \ ,
\eeq 
we see that the condition that must be met in order to successfully suppress the leading contribution from excited state contamination is
\beq
\label{eq:taucond}
\tau \gg \frac{\ln \left(\frac{Z_1}{Z_0 E_0}\right)}{E_1-E_0} \ ,
\eeq
where $E_0,Z_0$ ($E_1,Z_1$) are the ground (first excited) state energy and wavefunction overlap factor, respectively. Assuming we have properly eliminated excited states corresponding to unwanted quantum numbers through the choice of our source/sink, we have no further control over the energy difference $E_1 - E_0$ in the denominator, because this is set by the theory. Unfortunately, this makes the calculation of many-body observables extremely difficult as this energy splitting can become arbitrarily small due to collective excitations. Therefore, our only recourse is to choose excellent interpolating fields in order to reduce the numerator of \Eq{taucond}.

The simplest possible choice for a many-body interpolating field is composed of non-interacting single particle states. A Slater determinant over the included states takes care of fermion antisymmetrization. For example, a correlation function for $N_{\uparrow}$ ($N_{\downarrow}$) spin up (spin down) particles can be written,
\beq
\label{eq:slaterdet}
C_{N_{\uparrow},N_{\downarrow}}(\tau) = \langle \det S^{\downarrow}(\tau) \det S^{\uparrow}(\tau) \rangle \ ,
\eeq
where
\beq
S_{ij}^{\sigma} (\phi,\tau) \equiv \langle \alpha_i^{\sigma} | K^{-1}(\phi,\tau) | \alpha_j^{\sigma} \rangle \ ,
\eeq
and $\langle \alpha_j^{\sigma}|$ corresponds to single particle state $i$ with spin $\sigma$. As an example, we may use a plane wave basis for the single particle states,
\beq
| \alpha_j^{\uparrow}\rangle = |\vec{p}_j \rangle \ , \qquad | \alpha_j^{\downarrow}\rangle = |-\vec{p}_j \rangle \ ,
\eeq
where I've chosen equal and opposite momenta for the different spin labels in order to enforce zero total momentum (this condition may be relaxed to attain boosted systems). 

Though the interpolating field chosen in \Eq{slaterdet} has non-zero overlap with the ground state of interest, if the overlap is small it may take an inordinately long time to remove excited state contributions. Consider a system involving only two-particle correlations, as in our two-spin fermion system, and make the simplification that the ground state consists of non-interacting two-body pairs having wavefunction $\Psi_{\mbox{\tiny 2-body}}$, and overlap with a product of two non-interacting single particle states given by
\beq
\langle \Psi_{\mbox{\tiny 2-body}} | \left( |\vec{p}\rangle \otimes | - \vec{p} \rangle \right) = \epsilon < 1.
\eeq
Then the corresponding overlap of the Slater determinant in \Eq{slaterdet} with the ground state wavefunction scales as
\beq
\left( \langle \Psi_{\mbox{\tiny 2-body}} | \otimes \cdots \otimes \langle \Psi_{\mbox{\tiny 2-body}} | \right) \left( |\vec{p}_1 \rangle \otimes | -\vec{p}_1 \rangle \otimes \cdots \otimes |\vec{p}_N \rangle \otimes | -\vec{p}_N \rangle \right) \sim \epsilon^N \ .
\eeq
Thus the overlap of single-particle states with an interacting $2N$-body state is exponentially small with $N$. This condition worsens for systems with $3$- and higher-body correlations.

In order to do a better job we can incorporate two-body correlations into the sinks as follows: first, we construct a two particle propagator,
\beq
S_{ij}^{\uparrow}{\downarrow}(\phi,\tau) &=& \langle \Psi_2|K^{-1}(\phi,\tau) \otimes K^{-1}(\phi,\tau)\left( | \alpha_i^{\uparrow}\rangle \otimes | \alpha_j^{\downarrow} \rangle \right)\cr
&=& \sum_{\vec{p}} \Psi(\vec{p}) \langle \vec{p}| K^{-1}(\phi,\tau) | \alpha_i^{\uparrow} \rangle \langle -\vec{p}| K^{-1}(\phi,\tau) | \alpha_j^{\downarrow} \rangle \ ,
\eeq
where $\Psi_2(\vec{p})$ is some two-body wavefunction (this process could equally well be performed in position space). As an example, to incorporate BCS pairing, we may use a wavefunction of the form:
\beq
\label{eq:pairing}
\Psi_2(\vec{p}) \sim \frac{e^{-b|\vec{p}|}}{|\vec{p}|^2} \ ,
\eeq 
where $b$ is some parameter which may be tuned to maximize the overlap of the wavefunction. We may also use the wavefunction derived in \Eq{varpsi} for a lattice version of such a wavefunction. An example code fragment for implementing such wavefunctions is given in \Fig{wfcode}.

\begin{figure}
\begin{center}
\includegraphics[width=\linewidth]{Chapter5-figures/twobody}
\end{center}
\caption{\label{fig:wfcode}Portion of c++ code for implementing two types of two-body source vector: \Eq{varpsi} (GND) and \Eq{pairing} (PAIR2). Note that these vectors are computed in momentum space. The first operator applied to a source is the kinetic operator, $D^{-1}$, which is also computed in momentum space.}
\end{figure}

To ensure Pauli exclusion, it is sufficient to antisymmetrize only the sources, $|\alpha_i \rangle$, leading to the following many-body correlation function,
\beq
C_{N_{\uparrow},N_{\downarrow}}(\tau) = \langle \det S^{\uparrow \downarrow}(\tau) \rangle \ ,
\eeq 
where the determinant runs over the two sink indices. For correlation functions having an odd number of particles, one may replace a row $i$ of $S^{\uparrow\downarrow}$ with the corresponding row of the single particle object, $S^{\uparrow}$. The benefit of folding the wavefunction in at the sinks only is an $\calO(V^2)$ savings in computational cost: to fold a two-body wavefunction in at both source and sink requires the calculation of propagators from all possible spatial points on the lattice to all possible spatial points in order to perform the resulting double sum. 

Higher-body correlations may also be important and can be incorporated using similar methods. However, these will lead to further $\calO(V)$ increases in computation time. Finally, the entire system should be projected onto the desired parity, lattice cubic irreducible representation (which we will now briefly discuss), etc. in order to eliminate any contamination from excited states having different quantum numbers. 

\subsubsection{Angular momentum in a box}

The projection onto the cubic irreps is the lattice equivalent of a partial wave decomposition in infinite volume (and the continuum limit). The cubic group is finite, and therefore has a finite number of irreps, reflecting the reduced rotational symmetry of the box. The eigenstates of the systems calculated on the lattice will have good quantum numbers corresponding to the cubic irreps. When mapping these states onto angular momenta associated with infinite volume, there will necessarily be copies of the same irrep corresponding to the same angular momentum due to the reduced symmetry. This means that the box mixes angular momenta, as displayed in \Tab{cubicirreps}. For example, an energy level calculated in a finite volume that has been projected onto the positive parity $A_1$ irrep will have overlap with $j=0,4,\cdots$. For low energies it may be possible to argue that contributions from high partial waves are kinematically suppressed, since the scattering amplitude scales with $p^{2l+1}$, but in general the different partial wave contributions must be disentangled using multiple data points from different cubic irreps. 

\begin{table}[h!]
\label{tab:cubicirreps}
\begin{center}
\begin{tabular}{cc}
 j \hspace{1mm} & cubic irreps \\
 \hline
 0  \hspace{1mm}& $A_1$ \\
 1 \hspace{1mm} & $T_1$ \\
 2 \hspace{1mm} & $E+T_2$ \\
 3 \hspace{1mm} & $A_2 + T_1 + T_2$ \\
 4 \hspace{1mm} & $A_1 + E + T_1 + T_2$ \\
 \end{tabular}
 \end{center}
 \caption{Decomposition of the cubic group onto total angular momentum, $j$.}
 \end{table}
 
 A pedagogical method for projecting two-particle states onto the desired cubic irrep involves first projecting the system onto a particular spin state: for example, a two nucleon system may be projected onto either a spin singlet (symmetric) or spin triplet (anti-symmetric) state. The wavefunctions may then be given an ``orbital angular momentum" label by performing a partial projection using spherical harmonics confined to only the allowed rotations in the box. For example, we could fix the position of one of the particles at the origin $(0,0,0)$, then displace the second particle to a position $(x_0,y_0,z_0)$. This configuration will be labeled by the wavefunction $\psi_{s,m_s}\left[(x_0,y_0,z_0)\right]$, where $s,m_s$ are the total and $z$-component of the spin. We can then perform the partial projection, 
 \beq
 \tilde{\psi}_{l,m_l;s,m_s} = \sum_i Y_{l,m_l}\left[R_i(x_0,y_0,z_0)\right]\psi_{s,m_s}\left[R_i(x_0,y_0,z_0)\right] \ ,
 \eeq
 where the $R_i$ are cubic rotation matrices. Essentially, the set $R_i(x,y,z)$ correspond to all possible lattice vectors of the same magnitude. For example, if our original vector was $(1,0,0)$, then we would sum over the set of displacements $\{ (\pm 1,0,0),(0,\pm1,0),(0,0,\pm1)\}$. I want to emphasize that the $l,m_l$ are only wavefunction labels and do not correspond to good quantum numbers due to the reduced rotational symmetry.
 
 Now that the wavefunctions have spin and orbital momentum labels, these may be combined into total angular momentum labels $j,m_j$ using the usual Clebsch-Gordan coefficients. Finally, these wavefunctions are projected onto cubic irreps using so-called subduction matrices \cite{Dudek:2010wm}. As an example, a wavefunction labeled with $j=2$ (having five possible $m_j$ labels) will have overlap with two cubic irreps, $T_2,E$. The subduction matrices are:
 \beq
 T_2: 
\begin{array}{c}
 \overbrace{\rule{3.2cm}{0pt}}^{m_j=-2,-1,0 ,1,2} \\
 \left(\begin{array}{ccccc}
 0 & 1 & 0 & 0 & 0 \\
 1/\sqrt{2} & 0 & 0 & 0 & -1/\sqrt{2} \\
 0 & 0 & 0 & 1 & 0 \\
 \end{array} \right)  \end{array} \ , \qquad E: \left( \begin{array}{ccccc}
 0 & 0 & 1 & 0 & 0 \\
 1/\sqrt{2} & 0 & 0 & 0 & 1/\sqrt{2} \\
 \end{array}\right) \ .
 \eeq
 Note that the $T_2$ irrep has three degenerate states, while the $E$ irrep has two, matching the total of five degenerate states for $j=2$ in infinite volume. 
 
 Using this method for projection onto the cubic irreps has several benefits, including ease of bookkeeping and extension to higher-body systems using pairwise combinations onto a given $j,m_j$, followed by subduction of the total resulting wavefunction. Furthermore, in cases where more than one partial wave has overlap onto the chosen cubic irrep, wavefunctions with different partial wave labels may have different overlap onto the ground- and excited states of the system. Therefore, they can be used as a handle for determining the best source for the state of interest. We will discuss methods for using multiple sources for disentangling low-lying states and allowing for measurements at earlier times in the next subsection.
 
 \subsection{\label{sec:analysis}Analysis methods}

Having done our best to come up with interpolating wavefunctions, we can attempt to extract the ground state energy (and possibly excited state energies) earlier in time by performing multiple exponential fits to take into account any remaining excited state contamination. Using the known functional form for the correlator,
\beq
\label{eq:funcformC}
y(\tau) = \sum_n^{\Lambda}Z_n e^{-E_n \tau} \ ,
\eeq
where $\Lambda$ is a cutoff in the number of exponentials included in the fit, we may perform a correlated $\chi^2$ minimization,
\beq
\chi_{\Lambda}^2 = \sum_{\tau,\tau'}\left[ C(\tau)-y(\tau)\right] \left(\calC^{-1}\right)_{\tau\tau'}\left[ C(\tau')-y(\tau')\right]  \ ,
\eeq
where $\calC$ is the covariance matrix taking into account the correlation between different time steps. Because the correlation function at a given time is built directly upon the correlation function for the previous time step, there is large correlation between times that must be taken into account. 

We can go further by noting that correlation functions formed using different sources, but having the same quantum numbers, will lead to the same spectrum in \Eq{funcformC}, but with different overlap factors, $Z_n$. Thus, the $\chi^2$ minimization can be expanded to include different sources $s$, with only a modest increase in the number of parameters to be fit. Different sources may be produced, for example, by varying some parameter in the wavefunction, such as $b$ in \Eq{pairing}, through a different basis of non-interacting single particle states, such as plane waves vs. harmonic oscillator states, or through different constructions of the same cubic irrep, as discussed in the previous subsection. The resulting $\chi^2$ minimization is
\beq
y_s(\tau) = \sum_n^{\Lambda}Z_n^{(s)} e^{-E_n \tau} \ , \qquad \chi_{\Lambda}^2 = \sum_{\tau,\tau',s,s'}\left[ C_s(\tau)-y_s(\tau)\right] \left(\calC^{-1}\right)_{\tau\tau'}^{ss'}\left[ C_{s'}(\tau')-y_{s'}(\tau')\right]  \ ,
\eeq
where the covariance matrix now takes into account the correlation between different sources calculated on the same ensembles. 

In general, multiple parameter fits require high precision from the data in order to extract several parameters. The use of priors through Bayesian analysis techniques may be beneficial in some circumstances when performing multi-exponential fits to noisy data.

A more elegant approach using a set of correlation functions created using different operators is based on a variational principle \cite{MICHAEL1983433,Luscher:1990ck}. A basic variational argument proceeds as follows \cite{Blossier:2009kd}: starting with some set of operators $\calO_i$ which produce states $|\phi_i\rangle = \calO_i |0\rangle$ from the vacuum, we can evolve the state to some time $\tau_0$, $|\tilde{\phi}_i\rangle = e^{-\tau_0H/2}|\phi_i\rangle$ in order to eliminate the highest excited states, but leaving a finite set of states contributing to the correlation function. We would like to find some wavefunction $|\psi \rangle = \sum_{i=1}^N \alpha_i | \tilde{\phi}_i \rangle$ which is a linear combination of our set of operators parameterized by $\{\alpha_i\}$, that maximizes the following quantity for $\tau>\tau_0$:
\beq
\lambda_0(\tau,\tau_0) = \underset{\{\alpha_i\}}{\mbox{Max}}\frac{\langle \psi|e^{-(\tau-\tau_0)H}|\psi \rangle}{\langle\psi | \psi \rangle} \ ,
\eeq
so that 
\beq
\lambda_0(\tau,\tau_0) \approx e^{-E_0(\tau-\tau_0)} \ .
\eeq

A powerful method for finding the appropriate linear combination of states satisfying the variational principle uses a generalized eigenvalue problem (GEVP). For this method we form a matrix of correlation functions using all combinations of sources and sinks formed from a set of operators,
\beq
C_{ij}(\tau) = \langle \calO_i(\tau) \calO^*_j(0)\rangle = \sum_n e^{-E_n \tau}Z_i^{(n)}Z_j^{(n)} \ .
\eeq
The GEVP may be stated as:
\beq
\label{eq:GEVP}
C(\tau)v_n(\tau,\tau_0) = \lambda_n(\tau,\tau_0)C(\tau_0)v_n(\tau,\tau_0) \ ,
\eeq
where $v_n$ ($\lambda_n$) are a set of eigenvectors (eigenvalues) to be determined as follows: assume we choose $\tau_0$ to be far out enough in time such that only $N$ states contribute to the correlation function,
\beq
C_{ij}(\tau) = \sum_n^N e^{-E_n \tau}Z_i^{(n)}Z_j^{(n)} \ .
\eeq
Let's introduce a set of dual vectors $u_i^{(n)}$ such that
\beq
\sum_i u_i^{(n)} Z_i^{(m)} = \delta_{mn} \ .
\eeq
Applying $u_i$ to $C_{ij}$ gives
\beq
\sum_jC_{ij}(\tau)u_j^{(m)} = \sum_j \sum_n e^{-E_n \tau} Z_i^{(n)}Z_j^{(n)}u_j^{(m)} = e^{-E_m \tau}Z_i^{(m)} \ .
\eeq
Going back to our original GEVP, \Eq{GEVP},
\beq
C(\tau) u^{(m)} = \lambda_m(\tau,\tau_0) C(\tau_0)u^{(m)} \ ,
\eeq
we can now identify,
\beq
\lambda_m(\tau,\tau_0) = e^{-E_m(\tau-\tau_0)} \ .
\eeq
Thus, the energies may be found from the eigenvalues of the matrix, $C^{-1}(\tau_0)C(\tau)$. Solving this GEVP gives us access to not only the ground state, but some of the lowest excited states as well. 

Any remaining contributions from states corresponding to $E_n, n>N$ can be shown to be exponentially suppressed as $e^{-(E_{N+1}-E_n)\tau_0}$, where $E_{N+1}$ is the first state neglected in the analysis. We should define a new effective mass function to study the time dependence of each of the extracted states,
\beq
E_n^{(\mbox{eff})}(\tau,\tau_0) \equiv \ln \frac{\lambda_n(\tau,\tau_0)}{\lambda_n(\tau+1,\tau_0)} \ ,
\eeq
and look for a plateau,
\beq
\underset{\tau\to\infty}{\mbox{lim}}E_n^{(\mbox{eff})}(\tau,\tau_0) = E_n \ ,
\eeq
to indicate convergence to the desired state. The reference time $\tau_0$ may be chosen to optimize this convergence, and should generally be close to the beginning of the plateau of the standard effective mass. 

The GEVP method works very well in many situations and has been used extensively for LQCD spectroscopy. The main determining factor on the applicability of the method is whether one is able to construct a basis of operators which encapsulates the full low-lying spectrum sufficiently well. One major drawback is that the GEVP assumes a symmetric correlator matrix, meaning that the same set of operators must be used at both source and sink. As discussed in \Sec{interp}, this may be difficult to do numerically due to increases in computational time which scale with the volume when projecting onto a given wavefunction (unless the wavefunction is simply a delta function; however, this operator generally has extremely poor overlap with any physical states of interest). This is particularly a problem for noisy systems where large amounts of statistics are necessary.

There are a few alternatives to the GEVP which do not require a symmetric correlator matrix, such as the generalized pencil of functions (GPof) method \cite{Aubin:2011zz,GPOF1,GPOF2}, and the matrix Prony method \cite{Beane:2009kya,Fleming:2009wb}. We will now briefly discuss the latter, following the discussion of \cite{Beane:2009kya}. 

The Prony method uses the idea of a generalized effective mass,
\beq
M_{\tau_0}^{(\mbox{eff})}(\tau) = \frac{1}{\tau_0}\ln \frac{C(\tau)}{C(\tau+\tau_0)} \tautoinfty E_0 \ ,
\eeq
for some, in principle arbitrary, offset $\tau_0$. Because the correlator $C(\tau)$ is a sum of exponentials, it follows certain recursion relations. As an example, for times where only a single exponential contributes we have,
\beq
C(\tau+\tau_0)+\alpha C(\tau) &=& 0 \ .
\eeq
Plugging in our single exponential for the correlator we can solve for $\alpha$, then plug it back in to our original expression,
\beq
e^{-E_0\tau_0} + \alpha &=& 0 \cr
\longrightarrow C(\tau-\tau_0)-e^{E_0\tau_0}C(\tau) &=& 0 \ .
\eeq
Solving for the ground state energy gives us the same expression as the generalized effective mass at large times,
\beq
E_0 = \frac{1}{\tau_0}\ln \frac{C(\tau)}{C(\tau+\tau_0)} \ .
\eeq
This recursion relation may be generalized for times with contributions from multiple states using the correlation function at different time separations,
\beq
C(\tau+\tau_0k) + \alpha_k C(\tau+\tau_0(k-1))+ \cdots + \alpha_1 C(\tau) = 0 \ .
\eeq

We can now generalize this method for a set of correlation functions produced using different operators. Let $C_i(\tau)$ be an $N$-component vector of correlation functions corresponding to different sources and/or sinks. The correlators then obey the following matrix recursion relation,
\beq
\label{eq:MProny}
M C(\tau+\tau_0)-V C(\tau) = 0 \ ,
\eeq
for some matrices, $M,V$, to be determined. Assume the correlator has contributions from $\Lambda$ states,
\beq
C(\tau) = \sum_n^{\Lambda} \alpha_n u_n \lambda_n^{-\tau} \ ,
\eeq
where $\lambda_n = e^{E_n}$, and $u_n$ is a normalized vector, then we have the following modified GEVP,
\beq
Mu=\lambda^{\tau_0}Vu \ .
\eeq
A solution for $M$ and $V$ may be found by applying $\sum_{t=\tau}^{\tau+t_W}C(t)^T$ to both sides of \Eq{MProny},
\beq
M\sum_{t=\tau}^{\tau+t_W}C(t+\tau_0)C(t)^T - V\sum_{t=\tau}^{\tau+t_W}C(t)C(t)^T = 0 \ ,
\eeq
leading to the solution,
\beq
M = \left[\sum_{t=\tau}^{\tau+t_W}C(t+\tau_0)C(t)^T\right]^{-1} \ , \qquad V=\left[\sum_{t=\tau}^{\tau+t_W}C(t)C(t)^T\right] \ .
\eeq
The parameter $t_W$ is essentially free and may be tuned for optimization, but must obey $t_W \geq \Lambda-1$ in order to ensure that the matrices are full rank. The $\lambda_n$ may then be found from the eigenvalues of $V^{-1}M$. 

Here we have only used a single recursion relation, which is useful for finding the ground state at earlier times than traditional methods. However, this method is generally less effective for calculating excited states than the symmetric GEVP described previously. It may be possible to construct higher order recursion relations for the matrix Prony method in order to get more reliable access to excited states.

\section{\label{sec:systematic}Systematic errors and improvement}
\subsection{Improving the kinetic energy operator}
The first systematic effect we will examine comes from the discretization of the kinetic operator, first discussed in \Sec{LEFT}. In this section I will show the lattice spacing dependence explicitly so that we may see how discretization errors scale. The kinetic term depends on the definition of the Laplacian operator, which we originally defined to be, 
\beq
\nabla_{L}^2 f_j = \sum_{k=1,2,3} \frac{1}{b_s^2} \left[ f_{j+\hat{k}}+f_{j-\hat{k}}-2f_j \right] \ ,
\eeq
leading to the following kinetic term in momentum space,
\beq
\label{eq:Deltasin}
\Delta(p) = \frac{1}{b_s^2}\sum_i \sin^2\frac{b_s p_i}{2} \approx -\frac{p^2}{2}  + \frac{p^4}{24}b_s^2 + \cdots \ .
\eeq
The transfer matrix for the non-interacting system is given by
\beq
\calT = e^{-b_{\tau}H} = 1+b_{\tau} \frac{\Delta({p)}}{M}  \ ,
\eeq
leading to the energy, 
\beq
E=\frac{p^2}{2M} + \calO\left(\frac{p^4}{M} b_s^2 \right) \ .
\eeq
Therefore, discretization errors in this observable appear at $\calO\left(b_s^2\right)$ using this particular discretization. To be more precise, the errors scale with the dimensionless combination $(pb_s)^2$, reflecting the fact that the errors grow as higher momentum scales are probed. As we will discuss in \Sec{NLO}, small lattice spacings can lead to computational difficulties beyond the obvious scaling with the number of lattice sites, and taking the continuum limit may prove to be quite difficult. Therefore, it would be beneficial to have an improved operator whose discretization errors come in at a higher order in $pb_s$. One way to determine such an operator is to examine the relation between the finite difference and the continuum derivative in more detail using a Taylor expansion of the finite difference operator acting on a generic function, $f(x)$,
\beq
f(x+b_s)-f(x) = b_s f'(x) + \frac{b_s^2}{2}f''(x) + \frac{b_s^3}{6} f'''(x) + \frac{b_s^4}{24}f''''(x) + \cdots  \ .
\eeq
Using this expansion, the expression we used previously for the discretized Laplacian can be written,
\beq
\nabla_L^2 f(x) = \frac{1}{b_s^2} \left( f(x+b_s)+f(x-b_s)-2f(x) \right)= f''(x) + \frac{b_s^2}{12} f''''(x)+ \cdots \ .
\eeq
We see that the leading error comes in at $\calO(b_s^2)$, as expected. One method for eliminating the leading error is to add terms involving multiple hops,
\beq
\label{eq:improvkinetic}
\tilde{\nabla}_L^2 f(x) = \frac{1}{b_s^2} \left( f(x+b_s)+f(x-b_s)-2f(x) +c_1 f(x+2b_s) + c_2 f(x-2b_s) \right) \ ,
\eeq
where $c_1,c_2$ must be fixed in such a way as to eliminate the leading error. From symmetry, we must have $c_1=c_2$. We can then Taylor expand these new terms in our action, and determine the resulting energy as a function of $c_1$,
\beq
E(c_1) = \frac{p^2}{2M} + h(c_1) \frac{p^4}{M} b_s^2 + \cdots \ .
\eeq
By solving $h(c_1)=0$ for $c_1$, discretization errors will only enter at $\calO(b_s^4)$, implying a faster approach to the continuum as $b_s$ is decreased. Perhaps more importantly, in cases where decreasing the lattice spacing is difficult or impossible, the resulting systematic errors at finite lattice spacing will be significantly reduced.

This is our first, very simple, example of improvement. A more general method for improving the action in order to reduce discretization effects utilizes an EFT-like approach \cite{Symanzik1,Symanzik2,Symanzik3,Symanzik4,EKLN4}: we add higher dimension operators consistent with the symmetries of the theory and having unknown coefficients. The coefficients are then fixed by matching onto known physical quantities. The dimension of the operator added determines the order at which discretization errors have been eliminated.

In principle, one would need an infinite number of operators in order to eliminate all discretization errors. We are, of course, limited in the number of displacements we can add, as in \Eq{improvkinetic}, by the number of lattice sites. Therefore, the best possible kinetic operator, utilizing all possible spatial hops allowed by the lattice, may still only exactly reproduce the non-interacting spectrum up to the momentum cutoff set by the edge of the first Brillouin zone. Because the kinetic operator $\Delta$ is diagonal in momentum space, we may determine this ``perfect" operator directly by setting the transfer matrix,
\beq
\calT= 1+\frac{b_{\tau}\Delta(p)}{M} = e^{-\frac{b_{\tau}p^2}{2M}} \ ,
\eeq
up to a cutoff, leading to the operator,
\beq
\label{eq:perfect}
\Delta_{\mbox{\tiny perf}}(p) = M\left(e^{\frac{b_{\tau}p^2}{2M}}-1\right) \ , \qquad p< \frac{\pi}{b_s} \ .
\eeq

While this operator is simple in momentum space, it is highly non-local in position space, as expected, and would be unwieldy to use in a typical lattice calculation. However, another benefit of having a non-relativistic formulation with a separable interaction is that the form of the propagator, 
\beq
K^{-1}(\tau) &=& D^{-1}X(\tau)D^{-1}X(\tau-1)\cdots D^{-1} \cr
&=& D^{-1}X(\tau)K^{-1}(\tau-1) \ ,
\eeq
suggests that the kinetic ($D^{-1}$) and interaction ($X$) operators may each be applied separately in whatever basis is most convenient. So, we may choose to start with a source in momentum space (which is often preferable), then apply an exact kinetic operator, $D^{-1}$, also in momentum space, perform a FFT to position space, hit the resulting vector with the $X$ operator, which is most easily specified in position space, FFT again back to momentum space to perform a kinetic operation, and so on until finally the sink is applied. Example code for calculating various forms of inverse kinetic operator in momentum space is shown in \Fig{dispersion}.

\begin{figure}
\begin{center}
\includegraphics[width=\linewidth]{Chapter5-figures/dispersion}
\end{center}
\caption{\label{fig:dispersion}Example c++ code fragment for computing various lattice Laplacian operators: \Eq{Deltasin} (STANDARD), \Eq{perfect} (PERFECT), as well as a simple quadratic in momentum (QUADRATIC). Note that these are computed in momentum space, and they may be used to calculate the kinetic operator $D^{-1}$, then directly applied to the momentum space vectors computed in \Fig{wfcode}.}
\end{figure}

The benefit to using the FFT repeatedly rather than simply converting the kinetic operator into position space is that modern FFT libraries are highly optimized and cheap to use. For comparison, if we used the ``perfect" kinetic operator in position space it would be a dense $V\times V$ matrix. The operation of applying such an object to a $V$-dimensional vector,
\beq
D^{-1}(x) | \psi(x) \rangle \ ,
\eeq
scales like $V^2$. On the other hand, using the FFT to convert the $V$-dimensional vector to momentum space, then applying a diagonal matrix to it,
\beq
D^{-1}(p) \left( \mbox{FFT} | \psi(x) \rangle = | \tilde{\psi}(p) \rangle \right) \ ,
\eeq
scales like $V \log V$. This is a method referred to as ``Fourier acceleration" (see e.g. \cite{Batrouni,Daviesetal1,Daviesetal2,Katzetal1}). 

For formulations lacking separability of the kinetic and interaction operations, this method cannot generally be applied. In such cases, the kinetic operator should be kept relatively sparse in position space. Such a condition disfavors the use of \Eq{perfect} for a more modestly improved operator, composed of only a few spatial displacements, using the method outlined in the beginning of this Section.

\subsection{\label{sec:improve}Improving the interaction}
To discuss systematic errors and improvement of the interaction, we will focus on systems tuned to unitarity. Because unitarity corresponds to a conformal fixed-point, the systems we will study only depend on a single scale, the density, $n$. The finite lattice spacing necessarily breaks this conformal symmetry, and we can consider dependence on any new scales to stem from systematic errors. Systems having multiple intrinsic scales contain more complicated dependences of systematic errors, and will be discussed later on. 

Recall that the scattering phase shift for two particles at unitarity is,
\beq
p\cot\delta = 0 \ , 
\eeq
implying that the inverse scattering length, effective range, and all other shape parameters vanish. In \Sec{tuning}, we discussed how to tune the two-particle coupling in order to reproduce infinite scattering length. The lattice, however, naturally induces an effective range for the interactions, which have been generated via auxiliary fields extending across a lattice link, of size $b_s$. In order to improve the interaction and eliminate the unwanted effective range contribution stemming from discretization, we may add a higher-order interaction operator,
\beq
\sum_{\bfx}\sqrt{g_2} \phi \psidag_{\bfx} \nabla_L^2 \psi_{\bfx} \ ,
\eeq
recalculate the scattering amplitude, $A$, as a function of $g_0, g_2$, and tune $g_2$ to eliminate the $r_0$ term in the effective range expansion. In principle, one may further generalize the interaction operator,
\beq
\calL_{\mbox{\tiny int}} = \sum_n \sqrt{g_{2n}}  \phi \psidag \nabla_L^{2n} \psi \ ,
\eeq
where we will now suppress spacetime indices, and use the $g_{2n}$ to tune away successive terms in the effective range expansion. In practice this may be difficult because the interaction is generally no longer separable, so that loops can't be summed analytically. An easier method may be to use the transfer matrix, as we did in \Sec{LEFT}, to determine the two particle energy spectrum in a box, then tune the couplings in order to reproduce the desired energies. The target energies may be determined for systems obeying any known physical scattering phase shift using an approach known as the L\"uscher method, which we will now briefly review.

\subsubsection{\label{sec:Luscher}L\"uscher's method}

L\"uscher's method (\cite{Luscher:1986pf,Luscher:1990ux}) was originally developed as a tool for extracting physical scattering phase shifts from finite volume, Euclidean space observables produced by lattice QCD. The concept of asymptotic ``in" and ``out" scattering states does not exist in a finite volume, making direct scattering ``experiments" impossible on the lattice. Furthermore, the issue of analytic continuation from Euclidean to Minkowski time is a tricky one, particularly when utilizing stochastic techniques. Thus, L\"uscher proposed utilizing a different observable, finite volume energy shifts, and inferring the infinite volume scattering phase shift that would lead to the observed finite volume spectrum. In this section, we will largely follow the discussion in \cite{Beane:2003da}.

First let's recap how to calculate the infinite volume $s$-wave scattering phase shift in our effective theory assuming the following generic tree-level interaction: $\calL_2 = \sum_n g_{2n}p^{2n}$. The scattering amplitude is given by,
\beq
\label{eq:Aluscher}
A_{\infty} = \frac{\sum_{n}g_{2n} p^{2n}}{1-\sum_n g_{2n}p^{2n}I_0^{\infty}} = \frac{4\pi}{M}\frac{1}{p\cot\delta-ip} \ ,
\eeq
where I will now include the super/subscript ``$\infty$" to indicate infinite volume quantities, and $I_0^{\infty}$ is defined as,
\beq
I_0^{\infty}=\int \frac{d^3q}{(2\pi)^3} \frac{1}{E-q^2/M} \ .
\eeq
Note that I have assumed that the interaction is separable in deriving \Eq{Aluscher}. This would not be possible using a momentum cutoff as a regulator, so we will use dimensional regularization for this integral. By investigating the inverse scattering amplitude,
\beq
A^{-1}_{\infty} = \frac{1}{\sum_n g_{2n}p^{2n}} - I_0^{\infty} = \frac{M}{4\pi}(p\cot\delta - i p) \ ,
\eeq 
we can identify
\beq
\label{eq:C2n}
\sum_ng_{2n}p^{2n} = \left[I_0^{\infty} + \frac{M}{4\pi}(p\cot\delta - ip) \right]^{-1} \ .
\eeq
the quantity on the right can be expanded using the effective range expansion; the couplings are then determined by the scattering parameters, as we have seen previously.

Now that we have a relation between the couplings and the physical scattering parameters, let's now use this same effective theory to determine its finite volume spectrum. In a finite volume, there is no continuum of scattering states, but rather a discrete spectrum corresponding to poles in the finite volume analogue of the scattering amplitude, $A_{\mbox{\tiny FV}}$,
\beq
\label{eq:luschereig}
\mbox{Re}\left[A_{\mbox{\tiny FV}}^{-1}\right]=0 \ .
\eeq
Because the imposition of a finite volume can affect only the IR behavior of the theory, the interactions, and therefore the couplings, $g_{2n}$, remain unchanged. Any differences come from loops, where intermediate particles may go on shell and explore the finite boundary. Therefore, our finite volume analogue of the scattering amplitude may be written,
where
\beq
A_{\mbox{\tiny FV}}^{-1} = \frac{1}{\sum_ng_{2n}p^{2n}} - I_0^{\mbox{\tiny FV}} \ ,
\eeq
where the loop integral has been replaced by a finite volume sum over the allowed quantized momenta in a box,
\beq
I_0^{\mbox{\tiny FV}} = \frac{1}{L^3} \sum_{\vec{n}}^{\Lambda} \frac{1}{E-\left(\frac{2\pi n}{L}\right)^2/M} \ .
\eeq

Again, because the couplings are unchanged by the finite volume we are free to use \Eq{C2n} to replace them with the physical infinite volume phase shift, resulting in,
\beq
A_{\mbox{\tiny FV}}^{-1} = \frac{M}{4\pi}(p\cot\delta - ip) + I_0^{\infty} -I_0^{\mbox{\tiny FV}} \ .
\eeq
This leads to the eigenvalue equation,
\beq
\mbox{Re}\left[A_{\mbox{\tiny FV}}^{-1}\right]= \frac{M}{4\pi}p\cot\delta + \mbox{Re}\left[I_0^{\infty} - I_0^{\mbox{\tiny FV}}\right] = 0 \ .
\eeq
I have specified taking the real part of the inverse amplitude merely for calculational simplicity; this quantity is, in fact, already purely real because there are no integrals, and therefore, no $i\epsilon$ prescription. Furthermore, the difference between the infinite volume integral and the finite volume sum must be finite because the two encode the same UV behavior. Finally, we have the result,
\beq
\label{eq:pcotdeltaeig}
p\cot\delta = \frac{4\pi}{M}\left[-\frac{M}{4\pi^2L} \sum_{\vec{n}}^{\Lambda}\frac{1}{\left(\frac{pL}{2\pi}\right)^2 - n^2} - \frac{M\Lambda}{\pi L}\right] = \frac{1}{\pi L}S(\eta) \ ,
\eeq
where $\eta \equiv \left(\frac{pL}{2\pi}\right)^2$, and
\beq
S(\eta) \equiv \sum_{\vec{n}}^{\Lambda} \frac{1}{n^2 - \eta} -4\pi \Lambda \ ,
\eeq
is related to the Riemann zeta function. The cutoff on the sum, $\Lambda$, may be interpreted as an upper limit on the allowed momenta due to the finite lattice spacing, however, in practice it is taken to $\infty$ so that discretization and finite volume effects may be separately accounted for (note that we haven't used our lattice propagators in this derivation, which would be necessary for a proper treatment of discretization effects). Values of momenta which solve this eigenvalue equation for a given phase shift and volume correspond to the predicted finite volume spectrum. This is illustrated in \Fig{luscher}, where the function $S(\eta)$ has been plotted, along with several representative phase shifts, corresponding to positive and negative scattering lengths. The locations of the intersections give the energy eigenvalues for that volume. The poles of the $S$ function give the locations of the energies of a non-interacting system in a box, while the zeroes give the energies for systems at unitarity. 

Many extensions of L\"uscher's method exist for more complicated systems, such as multi-channel processes \cite{Briceno:2012yi,Hansen:2012tf,Li:2014wga,Briceno:2014oea,Briceno:2015tza,Briceno:2015csa,Briceno:2015axa,Briceno:2014uqa}, higher partial waves \cite{Luu:2011ep,Konig:2011nz,Koenig:2011ti}, moving frames \cite{Rummukainen:1995vs,Kim:2005gf}, moving bound states \cite{Bour:2011ef,Davoudi:2011md}, asymmetric boxes \cite{Li:2003jn,Feng:2004ua}, and three-body systems \cite{Hansen:2016fzj,Hansen:2015zga,Hansen:2014eka,Briceno:2012rv}, as well as perturbative expansions for many-boson systems \cite{Beane:2007qr,Detmold:2008gh,Smigielski:2008pa}. Formulations for general systems involving two nucleons may be found in \cite{Briceno:2013lba,Briceno:2013bda}. These formulations have been successfully applied in Lattice QCD for the determination of scattering phase shifts of nucleon-nucleon \cite{Beane:2006mx,Beane:2011iw,Beane:2012vq,Detmold:2015daa,Beane:2015yha,Chang:2015qxa,Orginos:2015aya,Yamazaki:2012hi,Yamazaki:2015asa,Berkowitz:2015eaa,Murano:2013xxa}, meson-meson \cite{Wilson:2014cna,Wilson:2015dqa,Dudek:2012gj,Dudek:2012xn,Dudek:2014qha,Beane:2011sc,Aoki:2007rd,Aoki:2011yj,Pelissier:2012pi,Feng:2010es,Torres:2014vna,Bolton:2015psa,Briceno:2015dca,Lang:2012sv,Prelovsek:2013ela,Lang:2014yfa,Lang:2015hza,Lang:2011mn,Briceno:2016kkp}, meson-baryon \cite{Verduci:2014csa,Lang:2012db,Torok:2009dg,Detmold:2015qwf}, and hyperon-nucleon \cite{Beane:2012ey,Beane:2009py,Beane:2006gf} systems.

\begin{figure}
\begin{center}
\includegraphics[width=0.5\linewidth]{Chapter5-figures/luscher}
\end{center}
\caption{\label{fig:luscher}$S(\eta)$ (solid red) and $\pi L p \cot \delta$ (dashed) as a function of $\eta \equiv \left(\frac{p L}{2\pi}\right)^2$. The $\pi L p \cot \delta$ correspond to $r_0/a=-0.1$, for the following volumes: $L/|a|=2$ (blue), $L/|a|=4$ (pink), $L/|a|=8$ (yellow), $L/|a|=10$ (green).  The energy eigenstates for the corresponding volumes are given by the intercepts of $S(\eta)$ with the dashed lines. Figure from \cite{Drut:2012md}.}
\end{figure}

\subsubsection{Applying L\"uscher's method to tune the two-body couplings}
The prescription for a lattice QCD calculation of nucleon-nucleon phase shifts is to start with quark interpolating fields to create a two nucleon correlation function, measure a set of finite volume energies, then use the eigenvalue equation, \Eq{pcotdeltaeig}, to infer the infinite volume two nucleon phase shift that produces those energies. For our lattice EFT, however, two nucleon phase shifts are used as input into the coefficients in the Lagrangian. Thus, we can use the L\"uscher method in reverse to calculate what we expect the two nucleon energies in a box to be given a known phase shift, then tune the couplings to reproduce those same energies in our lattice calculations. Having tuned the two-body sector, we can then make predictions about 3- and higher-body systems.

Our prescription for tuning the coefficients will be to construct the two-body transfer matrix with some set of operators,
\beq
\label{eq:tuningcoef}
\mathcal{G}(\vec{p}) = \sum_n^{\Lambda_n} g_{2n}\calO_{2n}(\vec{p}) \ ,
\eeq
which satisfy the low energy expansion $\calO_{2n}(\vec{p}) = \vec{p}^{2n} \left[ 1 + \calO(\vec{p}^2) \right]$ at low momenta, and should be chosen to depend only on the relative momentum of the two particle system in order to ensure Galilean invariance. This is important so that once the interaction is tuned boosted pairs of particles will see the same interaction. A convenient choice for the operators is given by,
\beq
\label{eq:Ofunc}
\calO_{2n}(\vec{p}) =  M^n \left(1-e^{-\hat{\vec{p}}^2 /M} \right)^n\ ,
\eeq
where $\hat{\vec{p}}$ is taken to be a periodic function of $\vec{p}$ and satisfies the relation $\hat{\vec{p}}^2 = \vec{p}^2 \theta(\Lambda-|\vec{p}|) + \Lambda^2  \theta(|\vec{p}| - \Lambda)$ for $\vec{p}$ in the first Brillouin zone. Sample code for calculating this interaction operator is shown in \Fig{interaction}.

\begin{figure}
\begin{center}
\includegraphics[width=\linewidth]{Chapter5-figures/interaction}
\end{center}
\caption{\label{fig:interaction}C++ code fragment for calculating the interaction given in \Eq{tuningcoef}, using the operators \Eq{Ofunc}, given some set of input coefficients interaction\_arg.couplings$[\Lambda_n]$. Note that this operator is calculated in momentum space. It may be applied directly to the momentum space vector resulting from the first operation of the kinetic operator, $D^{-1}$. A FFT must then be performed before applying the random auxiliary field, $\phi_x$. A final FFT must then be performed to return to momentum space before applying the next operation of $D^{-1}$ in order to propagate the system forward in time.}
\end{figure}

The transfer matrix may then be diagonalized numerically to determine the energy eigenvalues. The $g_{2n}$ should then be tuned until the energies match the first $\Lambda_n$ eigenvalues given by the L\"uscher method. This process serves a dual purpose: tuning multiple couplings helps reduce lattice spacing effects like the effective range, as we discussed previously, and also takes into account finite volume effects by correctly translating the exact infinite volume phase shifts into a finite volume. The process of tuning for the case of unitarity is illustrated in \Fig{tuning}. Here, $N_{\calO}$ coefficients have been tuned to correctly reproduce the first $N_{\calO}$ L\"uscher eigenvalues. The entire two-body spectrum is then calculated using these coefficients, and the resulting energies are plugged back into \Eq{pcotdeltaeig} to determine the effective phase shift seen by pairs of particles with different momenta. To be truly at unitarity, we should have $p\cot\delta = 0$ for all momenta. Clearly, tuning more coefficients brings us closer to unitarity for larger and larger momenta. This is particularly important for calculations involving many-body systems, where the average momentum grows with the density, $\langle p \rangle \sim n^{1/3}$.

\begin{figure}
\begin{center}
\includegraphics[width=0.5\linewidth]{Chapter5-figures/tuning.png}
\end{center}
\caption{\label{fig:tuning}Effective scattering phase shifts $p \cot \delta$ vs. $\eta$ produced by a set of contact interactions of the form in \Eq{tuningcoef}, with $N_{\calO}$ coefficients tuned to unitarity. Figure from \cite{EKLN1}.}
\end{figure}

A quantitative prediction can be made for the error remaining in higher, untuned two-body energy levels \cite{EKLN1}. Assuming $N_{\calO}$ terms in the effective range expansion have been tuned to zero,
\beq
p\cot\delta \sim r_{N_{\calO}-1}p^{2N_{\calO}} =\left(\frac{2\pi}{L}\right)^{2N_{\calO}} r_{N_{\calO}-1} \eta^{N_{\calO}} \ ,
\eeq
we can then use L\"uscher's relation for the first untuned eigenvalue $\eta_k$,
\beq
\left(\frac{2\pi}{L}\right)^{2N_{\calO}} r_{N_{\calO}-1} \eta_k^{N_{\calO}} = \frac{1}{\pi L}S(\eta_k) \ .
\eeq
Let's suppose $\eta_k^*$ is the eigenvalue one would expect in the true unitary limit. We can then Taylor expand the function $S(\eta_k)$ around $\eta_k^*$,
\beq
S(\eta_k) \approx c_k (\eta_k-\eta_k^*)\ ,
\eeq
where $c_k$ is the slope near $\eta_k^*$. The error is then estimated as,
\beq
\frac{\eta_k}{\eta_k^*}-1 \approx \frac{\pi L}{\eta_k^* c_k} \left(\frac{2\pi}{L}\right)^{2N_{\calO}}r_{N_{\calO}-1} \left(\eta_k^{*}\right)^{N_{\calO}} \sim \calO\left(L^{1-2N_{\calO}}\right) \sim \calO\left((b_s n^{1/3})^{2N_{\calO}-1}\right) \ ,
\eeq
where on the right I have rewritten the scaling with the volume as a scaling with the density to remind you that though the errors scale with the volume, these are not actually finite volume errors we are investigating, but discretization effects scaling with the dimensionless quantity $b_s n^{1/3} \sim b_s /L_{\mbox \tiny phys} = 1/L$ for systems at unitarity. The L\"uscher method takes into account finite volume effects automatically. 

\subsection{Scaling of discretization errors for many-body systems}
Having tuned our two-body interaction, we can now also predict the scaling of errors that we should expect to find in an $N$-body calculation. Let us suppose that the first untuned operator contains at most $2N_{\calO}$ derivatives,
\beq
\label{eq:errop}
\calO_{2N_{\calO}} \sim \left(\psi\psi\right)^{\dagger} \psi \nabla^{2N_{\calO}}\psi \ .
\eeq
The leading error results when any pair of particles interacts via this operator, and should scale with the dimension of this operator. 

To determine the operator dimension, first let me briefly recap how scaling dimensions are determined in a non-relativistic theory (see \cite{Kaplan:2005es} for more details). We expect the action, $S$, to be a dimensionless quantity, so we will consider the action for a non-interacting theory to determine how the fields and derivatives must scale,
\beq
S=\int d\tau d^3x \psidag \left( \partial_{\tau}-\frac{\nabla^2}{2M} \right) \psi \ .
\eeq
First, note that the mass, $M$, carries zero scaling dimension in a non-relativistic theory because it is considered to be much larger than any scale of interest. Then, from the expression in parentheses, we see that time and space must scale differently, $[\partial_{\tau}] = 2 [\nabla]$. Using the convention $[\nabla]=1$, we can then determine that the dimension of the fermion field must be $[\psi]=3/2$. 

Now let us return to the operator, \Eq{errop}, and determine its scaling dimension relative to the energy,
\beq
\left[ \left(\psi\psi\right)^{\dagger} \psi \nabla^{2N_{\calO}}\psi \right] - \left[\psidag \partial_{\tau} \psi \right] =( 6+2N_{\calO} ) - ( 5 ) = 1+2N_{\calO} \ .
\eeq
This indicates that the error from such an operator will scale as $\sim \calO(b_s p)^{1+2N_{\calO}}$, or $\sim \calO\left((b_s n^{1/3})^{1+2N_{\calO}}\right)$ for unitary fermions. This is similar scaling that we saw for higher two-body states, however, here the dependence on the number of particles is also important.

One may in principle tune as many operators as possible in order to perfect the interaction for higher energies. In practice, however, as more and more operators are tuned, the coefficients in front of higher dimensional operators which are still untuned can become very large. This can cause interactions seen by pairs of particles far in the tail of the momentum distribution to generate large errors. Thus, similar to the case of the kinetic operator, there is a limit to how ``perfect" the interaction can be made. 

On the other hand, these $s$-wave two-body interactions are not the only possible errors that are induced by the lattice, so we should not expect to see much improvement by tuning more operators corresponding to errors which are higher order than the leading operator which is not accounted for. For example, an unfortunate consequence of our tuning program is the introduction of interactions in the $p$-wave channel, as well as in higher partial waves. While a simple interaction which is point-like in space has no $p$-wave contribution, the introduction of spatial derivatives in our tuning operators gives rise to these new $p$-wave interactions. The leading $p$-wave operator has the form,
\beq
\calO_{p\mbox{\tiny -wave}} \sim \psidag \vec{\nabla} \psi \cdot \psidag \vec{\nabla} \psi \ ,
\eeq
and induces errors at $\calO\left(b_s n^{1/3})^{3}\right)$. In order to cancel this operator we could in principle add a $\phi$ field which carries momentum and carry out a similar program for tuning the coefficients as we used for the $s$-wave interaction. This destroys the separability of our interaction, however, and may be difficult to implement, in addition to introducing a new source of noise.

In general, we can determine all possible sources of discretization error as well as their scaling using a method referred to as the Symanzik effective action \cite{Symanzik1,Symanzik2,Symanzik3,Symanzik4,EKLN4}. The basic procedure begins through considering any possible operators (that have not been explicitly tuned) which are allowed by the symmetry of the theory. Because these operators may only be induced through discretization and must disappear in the continuum limit, they should be multiplied by the lattice spacing raised to the appropriate scaling dimension of the operator. We can then determine at what order in $b_s$, relative to the energy, we can expect systematic errors to arise.

Let's take a look another interesting operator which arises due to discretization, corresponding to a three-body interaction. While there can be no point-like 3-body interaction in the continuum limit for 2-component fermions due to the Pauli exclusion principle, three particles separated by a lattice spacing may interact via $\phi$-field exchange because they don't all lie on the same spacetime point. Thus, we should include in our Symanzik effective action an operator,
\beq
\calO_{\mbox{\tiny 3-body}} \sim \left(\psi \psi \psi \right)^{\dagger} \psi \psi \psi \ .
\eeq
Na\"ively, the dimension of this operator is 9, and therefore should contribute errors of $\calO\left((b_sn^{1/3})^4\right)$. So far, all of the operators we've discussed obey this simple scaling, corresponding to na\"ive dimensional analysis. However, our theory is strongly interacting, which can in general lead to large anomalous dimensions of certain operators. 

As an example, let's consider the scaling dimension of a very basic operator, the field $\phi$. The canonical (non-interacting) dimension for a generic bosonic field in a non-relativistic theory can be deduced by looking at the kinetic term in the action,
\beq
S_{\mbox{\tiny kin}} = \int d\tau d^3x \nabla^2 \phi^2 \ ,
\eeq
leading to a scaling dimension, $[\phi] = 3/2$. However, once interactions with the $\psi$ fields are included, the $\phi$ propagator is renormalized through loop diagrams (see \Fig{phiprop}). For a non-perturbative interaction, we must sum all possible loop diagrams. However, there is a simpler way to determine the scaling dimension of the strongly interacting $\phi$ field. The key is to recognize that near unitarity the $\phi$ field represents a bound state of two $\psi$ fields at threshold. We can therefore write $\phi$ as a local operator,
\beq
\phi(x) = \underset{x\to y}{\lim} |x-y| \psidag(x)\psi(y) \ ,
\eeq
where $|x-y|$ must be included to ensure that matrix elements of the operator are finite (the wavefunction for two particles at unitarity must scale as $|x-y|^{-1}$ at short distances \cite{NishidaSonConformal}). Using our previous analysis for the scaling dimension of the $\psi$ field, we find,
\beq
[\phi]_{\mbox{\tiny int}} = 2 \ ,
\eeq
which implies a very strong wavefunction renormalization.

\begin{figure}
\begin{center}
\includegraphics[width=\linewidth]{Chapter5-figures/dimer2}
\end{center}
\caption{\label{fig:phiprop}Propagator for the bosonic field $\phi$, dressed by fermionic loops.}
\end{figure}

In general it can be very difficult to calculate anomalous dimensions directly in a non-perturbative fashion. However, for non-relativistic conformal field theories (CFT), there exists an operator-state correspondence (similar to an ADS/CFT correspondence), which relates the scaling dimension of an operator in the CFT (e.g. for unitary fermions) to the energy of the corresponding state in a harmonic potential \cite{NishidaSonConformal}. For example, we have already determined the dimension of the field $\psi$ to be 3/2, and the energy of a single fermion in a harmonic potential with oscillator frequency $\omega$ is $3/2 \omega$. The energy of two unitary fermions in a harmonic potential is $2\omega$, corresponding to the dimension of the $\phi$ field, $[\phi]=2$. 

Returning now to our 3-body operator, we can use numerical results for the energy of three fermions in a total $l=0$ state in a harmonic potential \cite{2007PhRvL..99w3201B,2011CRPhy..12...86B} to determine that,
\beq
\left[\psi\psi\psi\right] = 4.67 \ .
\eeq
The error-inducing operator in the Symanzik effective action both creates and destroys this 3-body state, resulting in
\beq
\left[\left(\psi\psi\psi\right)^{\dagger}\psi\psi\psi\right] = 9.34 \ .
\eeq
The relative error in the energy will then be $\calO\left(L^{-(9.34-5)}\right) = \calO\left(L^{-4.34}\right)$.

It turns out that the ground state of three fermions in a harmonic potential is actually not the $s$-wave state, but a $p$-wave state with energy $\sim 4.27\omega$. Thus, we should expect an additional systematic error corresponding to a 3-body $p$-wave operator that contributes at $\calO\left(L^{-3.55}\right)$ \cite{2006PhRvL..97o0401W}. Finally, at approximately the same order as the 3-body $s$-wave there is a 2-body $d$-wave operator (four derivatives) with zero anomalous dimension, and therefore contributing at $\calO\left(L^{-5}\right)$.

While certainly only the leading error ($\calO\left(L^{-3}\right)$) will dominate very close to the continuum limit, at a finite lattice spacing we have just demonstrated that there are several sources of error scaling with very similar powers of the lattice spacing. If we wish to eliminate discretization errors through extrapolation to the continuum limit, we must include all possible non-negligible contributions in our extrapolation function. For example, we could employ the following function:
\beq
E(L) = E_0\left[1+a L^{-3} + b L^{-3.55} + c L^{-4.34} + d L^{-5} + \cdots \right] \ ,
\eeq
and fit the coefficients $\{a,b,c,d\}$ using data at several volumes, in order to extract the continuum energy, $E_0$ \cite{EKLN4}.

\subsection{Additional sources of systematic error}
It should be pretty clear by now that understanding and controlling systematic errors can be quite complicated, even for conformal systems! For more complex systems with contributions from multiple scales, such as nucleii, things become even messier. As a simple example of a system with more than one scale we can consider trapping our unitary fermions in a harmonic potential, which will allow us to discuss finite volume errors that are not accounted for by the L\"uscher method. This is clearly relevant for cold atom experiments, which utilize traps, but may also be useful for calculating the energies needed to use the operator-state correspondence discussed in the previous subsection.

The new characteristic length scale contributed by the introduction of the harmonic trap is given by the size of the trap, $L_0$. We now have two different dimensionless quantities which determine the scaling of systematic errors due to discretization, $b_s/L_0$, and finite volume, $L_0/L_{\mbox{\tiny phys}}$, individually. To determine the size of discretization errors we may use the Symanzik effective action method as previously described, with the average momentum scale replaced by $n^{1/3} \to N^{1/3}/L_0$. Finite volume errors may be estimated by examining the long distance behavior of the wavefunction of the system of interest, where distortions due to the finite boundary can occur. For a system in a harmonic trap with local interactions, wavefunctions behave as Gaussians at large distance, so we might consider using a function $E(L_{\mbox{\tiny phys}}) = E_0\left(1+a e^{-\left(L_0/L_{\mbox{\tiny phys}}\right)^2}\right)$ to extrapolate to the infinite volume limit. 

For the case of nuclei, which are bound states whose wavefunctions fall off exponentially at long distance, we might expect systematic errors to scale as $e^{-R/L_{\mbox{\tiny phys}}}$, where $R$ is the characteristic size of the bound state. In general, one may also need to consider effects from interactions between images produced due to the periodic boundary conditions. For example, if the interaction between images is mediated at long distances by the exchange of a light particle, such as a pion, then we might expect systematic errors to fall off exponentially with $\sim \left(m_{\pi} L_{\mbox{\tiny phys}}\right)$. Note that this type of finite volume effect is not accounted for by the L\"uscher formalism; this is because in order to derive \Eq{pcotdeltaeig} we had to assume that all interactions were point-like.

Finally, we should briefly discuss systematic errors associated with temporal discretization. These tend to be far less worrisome for zero temperature results for several reasons. The first is due to the relation $b_{\tau} = \frac{b_s^2}{M}$ for non-relativistic theories, indicating that temporal discretization errors are of lower order than spatial discretization errors. Furthermore, our tuning method for improving the kinetic and interaction operators also translates into an improved temporal derivative operator. The lattice temporal derivative is given by the finite difference,
\beq
\partial_{\tau}\psi \sim \psi_{\tau+1} - \psi_{\tau} \sim \left(\calT -1\right) \psi_{\tau} \ ,
\eeq
where on the right hand side I have used the knowledge that the transfer matrix $\calT$ is our time-translation operator. By perfecting the transfer matrix with our tuning method, we are in turn perfecting the single time hop operation, thereby reducing temporal discretization errors.

We also have the freedom to use the anisotropy parameter $M$ to tune the temporal lattice spacing to be intrinsically smaller than the spatial lattice spacing. However, it should be noted that because the temperature is controlled by the physical Euclidean time length, $1/\left(b_{\tau}N_{\tau}\right)$, increasing the anisotropy parameter $M$ will necessitate an increase in the number of temporal lattice points to reach the zero temperature limit. On the other hand, having a finer temporal lattice spacing may also help to better resolve plateaus occurring within a short ``golden window" before the noise begins to set in, due to the increase in the number of points available for fitting. For this reason, anisotropic lattices are sometimes used in lattice QCD for noisy systems. However, points corresponding to a finer temporal lattice spacing are also more correlated, so it is currently unclear whether anisotropic lattices are actually beneficial for resolving noisy signals.

\section{\label{sec:NLO}Beyond leading order EFT}

The first step away from unitarity and toward real nuclear physics that we can easily take is to introduce a four-component nucleon field, $N$, containing two flavors of spin up and spin down fermions. The nucleons have two allowed $s$-wave scattering channels, $^1S_0$ and $^3S_1$, which should be tuned independently (breaking the approximate $SU(4)$ symmetry between the nucleons) to give the physical nucleon-nucleon scattering lengths. One possible way to achieve this is to introduce two four-fermion interactions corresponding to,
\beq
\label{eq:STint}
\calL_{\mbox{\tiny int}} = -\frac{1}{2} g_S \left(N^{\dagger}N\right)^2 - \frac{1}{2} g_T \left(N^{\dagger}\vec{\sigma}N\right)^2 \ ,
\eeq
where $\sigma_i$ is a Pauli matrix acting on the spin indices, and $g_S,g_T$ are couplings for the spin singlet and spin triplet channel, respectively. The lattice version of this interaction requires the introduction of two independent auxiliary fields, $\phi_{S},\phi_{T}$. One possibility is,
\beq
\label{eq:noSU4}
\calL_{\mbox{\tiny int}}^{(L)} = \sqrt{g_S} \phi_S N^{\dagger}N+\sqrt{g_T} \phi_T \vec{\sigma} \cdot N^{\dagger} \vec{\sigma} N \ .
\eeq

There are, in fact, many ways to implement the same interactions, and the different implementations will affect the signal-to-noise ratios of observables. For example, one could imagine having one of the $\phi$ fields couple to both channels equally (the $SU(4)$ limit), tuned to give the scattering length of the more attractive channel, $^3S_1$, then adding a second auxiliary field coupling only to the $^1S_0$ channel and tuning this coupling to be repulsive, making this channel more weakly attractive as desired. As we learned in \Sec{SNR}, repulsive interactions cause severe sign and noise problems, so this would clearly be a poor choice of implementation.

Let's look at the signal-to-noise ratio for a two-particle correlator in the $^1S_0$ channel using the interaction shown above, \Eq{noSU4}, where neither interaction is repulsive, but their relative strengths are different. The signal goes like,
\beq
\langle K^{\uparrow -1}_n(\tau) K^{\downarrow -1}_n(\tau) \rangle \sim e^{-E_0^{(^1S_0)}\tau} \ .
\eeq
while the noise is given by,
\beq
\label{eq:noSU4noise}
\sigma^2 \sim \langle K^{\uparrow -1}_n(\tau) K^{\downarrow -1}_n(\tau) \rangle K^{\uparrow -1}_{n'}(\tau) K^{\downarrow -1}_{n'}(\tau) \rangle \sim e^{E_B^{(4)}\tau} \ ,
\eeq
where $n'$ denotes a particle of different flavor from $n$, and $E_B^{(4)}$ is the binding energy of a four particle, four flavor state. This causes a signal-to-noise problem which is similar to our original two-body correlator, however, in this case the problem is exacerbated by the fact that particles in \Eq{noSU4noise} having different flavor index interact through the most attractive channel, $^3S_1$. This results in a greater disparity between the energies governing the signal and the noise, leading to more severe exponential decay of the signal-to-noise ratio. Unequal interactions can also lead to problems with reweighting methods designed to alleviate an overlap problem if the desired reweighting factor is no longer real or positive.

One method, devised by the Bonn-Raleigh group (for a review, see e.g. \cite{Lee:2008fa}), for avoiding the extra noise caused by unequal interactions in the two $s$-wave channels, is to use an $SU(4)$ symmetric transfer matrix, $\calT_{SU(4)}$, to evolve the system for several time steps before applying the full asymmetric transfer matrix. This process may be thought of as utilizing several applications of $\calT_{SU(4)}$ in order to produce a better interpolating wavefunction from some initial guess wavefunctions, $\Psi_{i,f}$, which is then used as a source for the correlation function,
\beq
C(\tau) = \langle \Psi_f | \calT_{SU(4)}^{\tau'} \calT^{\tau} \calT_{SU(4)}^{\tau'}|\Psi_i \rangle = \langle \tilde{\Psi}_f | \calT^{\tau} | \tilde{\Psi}_i\rangle \ ,
\eeq
where $| \tilde{\Psi}_i\rangle  \equiv \calT_{SU(4)}^{\tau'}|\Psi_i \rangle$. Using this method reduces the number of times the noisier $\calT$ must be used because the system begins in a state that is already closer to the true ground state.

Another method used by the same group to reduce noise is to perform a Fierz transformation on the four-fermion interactions in order to define interactions with more symmetric couplings \cite{Borasoy:2006qn}. Using the identity,
\beq
\left(N^{\dagger}N\right)^2 = -\frac{1}{2} \left(N^{\dagger}\vec{\sigma}N\right)^2 - \frac{1}{2} \left(N^{\dagger}\vec{\tau}N\right)^2 \ ,
\eeq
we can rewrite the four-fermion interactions, \Eq{STint}, to give the following,
\beq
\tilde{\calL}_{\mbox{\tiny int}} = -\frac{1}{2} g_0 \left(N^{\dagger}N\right)^2 - \frac{1}{2} g_I \left(N^{\dagger}\vec{\tau}N\right)^2 \ ,
\eeq
where $\tau_i$ is a Pauli matrix acting on the flavor components of $N$, and the couplings $g_{0,I}$ are related to the original couplings by,
\beq
g_0 = g_S-2g_T \ , \qquad g_I = - g_T \ .
\eeq

\subsection{Tuning the effective range}
The method outlined in \Sec{tuning} was devised as a way to allow us to tune our couplings to reproduce any physical scattering phase shift using the L\"uscher finite volume method. We were able to successfully tune the system to unitarity, where the effective range and all higher shape parameters vanish. For nucleon scattering, the effective ranges in the $s$-wave channels are given roughly by the Compton wavelength of the pion, so the next logical step in our quest toward nuclear physics should be to try to tune our coefficients to give the physical effective ranges. Unfortunately, a problem arises for producing a non-zero effective range non-perturbatively using point-like interactions in combination with a lattice regulator. 

The choice of regulator is relevant when attempting to perform non-perturbative calculations because EFTs in general are non-renormalizable. However, they should be renormalizable order by order in perturbation theory, because at each order we introduce a new operator having the correct dimensions and symmetries to act as a counterterm, absorbing infinities from loops containing lower order interactions. Lattice methods incorporate the Lagrangian of the theory non-perturbatively, effectively summing the entire subset of diagrams for each interaction. In principle, such a formulation may also require the introduction of an infinite number of counterterms to absorb the divergences from all loop diagrams. 

In certain cases, however, this situation can be avoided. An example is our non-perturbative tuning of the scattering length. Recall that all bubble diagrams involving only the coupling $g_0$ were separable; this allowed us to write the non-perturbative scattering amplitude as a geometric sum, and we were able to absorb all loop divergences into the single coupling, $g_0$. The condition of separability for loop diagrams containing interactions which carry momenta is dependent on the choice of regulator. Our choice of a lattice regulator, which is similar to a momentum cutoff, leads to a bound, known as the Wigner bound, on the allowed effective ranges one can access non-perturbatively \cite{PhysRev.98.145,Phillips:1996ae,Cohen:1996my}.

Because the general tuning method introduced in \Sec{tuning} involves the numerical calculation of the transfer matrix, understanding the Wigner bound in this context is difficult. To better illustrate the issue, let's attempt to tune the effective range instead using the first method for tuning, outlined in \Sec{couplings}. This method involves calculating the scattering amplitude and tuning the couplings to match the desired scattering parameters directly from the effective range expansion. 

We will again calculate a sum of bubble diagrams, however, we must now include an interaction of the form $\calL_{\mbox{\tiny int}} \sim g_2 \psidag \nabla^2 \psi$, which we would like to use to tune the effective range. We will largely follow the discussion of \cite{Phillips:1997xu}. A generic integral from one of these diagrams will have the form,
\beq
I_{2n} = \frac{1}{2\pi^2}\int dq \frac{q^{2+2n}}{E-q^2/M} \ ,
\eeq
where $n=0,1,2$, depending on which of the two interactions  we have at the two vertices. Since we are interested in the renormalizability of the scattering amplitude, we will separate out the divergent pieces of such an integral by expanding around $q\to\infty$,
\beq
\label{eq:intrecursion}
I_{2n} = \frac{1}{2\pi^2}\int dq \left[M q^{2n}-EM\int dq \frac{q^{2n-2}}{E-q^2/M} \right] \ ,
\eeq
and investigate the integrals using different regularization schemes. The above relation may be iterated for a given $n$ until the remaining integral is finite. The lowest order integral that we will need is given by,
\beq
I_0 = -\frac{1}{2\pi^2}\int dq \frac{q^2}{E-q^2/M} \ .
\eeq
We evaluated this integral previously using a cutoff, $\pi\Lambda/2$, to find,
\beq
I_0 = \frac{M}{4\pi}\left[\Lambda+iME\right] \qquad \mbox{(cutoff)} \ .
\eeq
Using dimensional regularization (dim reg), on the other hand, eliminates power-law divergences, so the result becomes,
\beq
I_0 = \frac{M}{4\pi}iME  \qquad \mbox{(dim reg)}\ .
\eeq
The other two integrals we will need have two and four additional powers of the momentum. Using our relation, \Eq{intrecursion}, we can write,
\beq
I_2 = ME I_0 - \lambda_2 \ ,
\eeq
where
\beq
\lambda_2 = \frac{M}{2\pi^2} \int dq q^2 = \left\{ \begin{array}{cc}
-\frac{M\pi}{48} \Lambda^3 & \mbox{cutoff} \\
0 & \mbox{dim reg} \\
\end{array} \right. \ ,
\eeq
and
\beq
I_4 = ME I_2 - \lambda_4 \ ,
\eeq
where
\beq
\lambda_4 = \frac{M}{2\pi^2} \int dq q^4 = \left\{ \begin{array}{cc}
-\frac{M\pi^3}{320} \Lambda^5 & \mbox{cutoff} \\
0 & \mbox{dim reg} \\
\end{array} \right. \ .
\eeq
From these results we see that dim reg leads to a separable interaction because each of the integrals can be written in terms of $I_0$ times some overall factor. On the other hand, the cutoff introduces new terms which cannot be factorized. 

In order to evaluate the scattering amplitude more generally for a non-separable interaction we must solve a matrix equation. We will set this up by noting that the interaction can be written,
\beq
V(p,p') = \sum_{i,j=0}^1 p'^{2i} v_{ij} p^{2j} \ ,
\eeq
where 
\beq
v=\left( \begin{array}{cc}
g_0 & g_2 \\
g_2 & 0 \\
\end{array} \right) \ .
\eeq
The amplitude is then,
\beq
A = - \sum_{i,j=0}^1 \left(ME\right)^{i+j} a_{ij} \ ,
\eeq
where
\beq
a = v + v \calI a \ , \qquad \calI = \left( \begin{array}{cc}
I_0 & I_2 \\
I_2 & I_4 \\
\end{array} \right) \ .
\eeq
We can now solve for $a$,
\beq
a = \left[1-v\calI \right]^{-1} v = \frac{1}{\lambda} \left( \begin{array}{cc}
g_0+g_2^2 I_4 & g_2(1-g_2I_2) \\
g_2(1-g_2I_2) & g_2^2 I_0 \\
\end{array} \right) \ ,
\eeq
where
\beq
\lambda \equiv 1-g_0 I_0 -2g_2 I_2 + g_2^2 (I_2^2-I_0I_4) \ .
\eeq
Finally, we have
\beq
\frac{1}{A} &=& - \frac{(g_2\lambda_2 -1)^2}{g_0+g_2[ME(2-g_2 \lambda_2) +g_2 \lambda_4 ]}+I_0 \cr
&=& \frac{M}{4\pi}\left(-1/a + 1/2 r_0 ME - i\sqrt{ME} \right) \ ,
\eeq
where I have used the effective range expansion for the inverse scattering amplitude on the right hand side. 

This expression may be used to determine the couplings $g_{0,2}$ in terms of the effective range parameters, $a,r_0$, by expanding the left hand side in powers of $ME$, and comparing the resulting coefficients to the corresponding parameters in the effective range expansion. The leading order is,
\beq
\left. \frac{1}{A}\right|_{E=0} = - \frac{(g_2\lambda_2-1)^2}{g_0+g_2^2\lambda_4}+\left. I_0 \right|_{E=0} = -\frac{M}{4\pi a}  \ ,
\eeq
while the next order gives,
\beq
\left[ \frac{\partial}{\partial (ME)}\frac{1}{A}\right]_{E=0}\frac{g_2\left( \left. I_0 \right|_{E=0} + \frac{M}{4\pi a}\right)^2)(2-g_2\lambda_2)}{(g_2\lambda_2-1)^2} = \frac{M}{8\pi} r_0 \ .
\eeq
Using these two expressions and the above relations for  $\lambda_n$ and $I_0$, we can derive the following dependence of the effective range on the couplings for a theory regularized using dim reg,
\beq
r_0 = \frac{Mg_2}{\pi a^2} \ .
\eeq
Because the effective range is proportional to the coupling $g_2$, it can be tuned arbitrarily. Thus, as expected from the separability of the interaction, there are no issues with renormalizability when using dim reg. 

Let us now see what happens for the case of a cutoff. The relation becomes,
\beq
\label{eq:r0cutoff}
r_0 &=& \frac{8\pi}{M} \left(\frac{M}{4\pi a}+\left. I_0 \right|_{E=0}\right)^2 \left[\frac{1}{(g_2\lambda_2-1)^2\lambda_2}-\frac{1}{\lambda_2}\right] \cr
&=& \frac{M}{2\pi} \left(1/a+\Lambda\right)^2 \left[ -\frac{1}{\left(g_2 \frac{M\pi}{48}\Lambda^3-1\right)^2\frac{M\pi}{48}\Lambda^3} + \frac{48}{M\pi \Lambda^3} \right] \ .
\eeq
We should now attempt to remove the cutoff by taking, $\Lambda \to \infty$,
\beq
r_0 \underset{\Lambda\to\infty}{\longrightarrow} -\frac{\frac{M}{2\pi}\Lambda^2}{(g_2\frac{M\pi}{48}\Lambda^3-1)^2\frac{M\pi}{48}\Lambda^3} \ ,
\eeq
where I have kept the first term in square brackets in \Eq{r0cutoff} because there $g_2$ may be renormalized to absorb factors of $\Lambda$. Because $g_2$ must be real to ensure a Hermitian Hamiltonian, this expression shows that if we attempt to remove the cutoff of the theory, we are only allowed to tune $r_0 \leq 0$. 

More generally, Wigner showed that for any potential which obeys $V(r,r') \to 0$ for $r,r'>R$ sufficiently quickly for some characteristic radius $R$, then
\beq
r_0 \leq 2\left(R-\frac{R^2}{a}+\frac{R^3}{3a^2}\right) \ .
\eeq
For a potential generated using delta function interactions and a momentum cutoff, $R\sim 1/\Lambda$, and we arrive at our expression $r_0\leq 0$.

In our lattice formulation the interactions are generated by an auxiliary field extending across a single time link, so that $R \sim b_s$. Therefore, if we try to tune $r_0$ non-perturbatively via the inclusion of such interactions in the Lagrangian, we are limited to $r_0 \lesssim b_s$. This was not a problem when we considered unitarity, since at this point $r_0 =0$. For nuclear physics, this bound restricts us to tuning the effective range to be smaller than the lattice spacing, implying that there is no continuum limit to the theory. On the other hand, the theory we are attempting to simulate is only an effective theory of nucleons, valid up to a physical cutoff. Thus, so long as we do not attempt to probe physics beyond scales of order $\sim 1/r_0$ there will be no inconsistencies. This is clearly a limitation, however, and also restricts our ability to vary the lattice spacing when studying discretization effects.

One possibility for avoiding this restriction is to include the effective range contribution to observables perturbatively, keeping the renormalizability of the effective theory intact. Perturbative corrections may be added by expanding the transfer matrix,
\beq
\calT \approx e^{-H_0 b_{\tau}} - b_{\tau} \delta H e^{-H_0 b_{\tau}} \ ,
\eeq
where $H = H_0 + \delta H$ is the full Hamiltonian and $\delta H$ is the piece we wish to treat perturbatively. Multiple insertions of $\delta H$ may be included to reach higher orders in the effective theory.

\subsection{Including pions}
If we wish to probe energies of order the pion mass we must include pions explicitly into the effective theory. Unfortunately, pions are notoriously difficult to include in a consistent power counting scheme. Here, we will only briefly outline some of the issues related to power counting for pion contributions. 

The KSW expansion proposed that pion exchange be treated as a series of perturbative corrections to the leading order pionless EFT \cite{Kaplan:1996xu,Kaplan:1998tg,Kaplan:1998we}. In this case, a tree level one pion exchange (1PE) diagram may be given by \cite{Fleming:1999ee},
\beq
\label{eq:tree}
\begin{array}{cc}
\includegraphics[width=0.1\linewidth]{Chapter5-figures/tree.png} & \sim \frac{g_A^2}{2f_{\pi}^2} f\left(\frac{p}{m_{\pi}}\right) \\
\end{array} \ ,
\eeq
where $g_A$ is the axial coupling, $f_{\pi}$ is the pion decay constant, and $f(p/m_{\pi})$ is a dimensionless function. By comparison, at one loop there is a box diagram,
\beq
\begin{array}{cc}
\includegraphics[width=0.1\linewidth]{Chapter5-figures/box.png} & \sim \left(\frac{g_A^2}{2f_{\pi}^2}\right)^2 \frac{M m_{\pi}}{4\pi} \tilde{f}\left(\frac{p}{m_{\pi}}\right)\\
\end{array} \ .
\eeq
Note that the factor of the nucleon mass, a large energy scale for the effective theory, comes from diagrams in which intermediate nucleons can go on-shell. This implies that an expansion parameter for the set of ladder diagrams is approximately,
\beq
\frac{g_A^2Mm_{\pi}}{8\pi f_{\pi}^2} \sim 0.5 \ ,
\eeq
and that the expansion may converge very slowly. In practice, the convergence for this formulation might be acceptable in the $^1S_0$ scattering channel, but is poor in the spin triplet channel. This is likely due to the singular tensor force contribution to the two-nucleon potential in this channel, which we will discuss in a moment \cite{Fleming:1999ee}.

Weinberg's formulation for nuclear EFT involves summing a subset of diagrams non-perturbatively, then using the resulting nucleon-nucleon potential to solve the Schrodinger equation. In doing so we can take into account higher orders in a perturbative expansion that  breaks down or converges slowly. For the pions we can iterate all possible tree level pion exchange diagrams to give the following 1PE potential \cite{Epelbaum:2010nr},
\beq
V_{\mbox{\tiny 1PE}}(\vec{r} = \left(\frac{g_A}{2f_{\pi}}\right)^2 \vec{\tau}_1 \cdot \vec{\tau}_2 \left[m_{\pi}^2\frac{e^{-m_{\pi}r}}{12\pi r}\left(S_{12}(\hat{r}) \left(1+\frac{3}{m_{\pi}r} + \frac{3}{(m_{\pi}r)^2}\right)+\vec{\sigma}_1 \cdot \vec{\sigma}_2 \right) -\frac{1}{3}\vec{\sigma}_1 \cdot \vec{\sigma}_2 \delta^3(r)\right] \ , \cr
\eeq
where $S_{12} = 3\vec{\sigma}_1 \cdot \hat{r} \vec{\sigma}_2 \cdot \hat{r} - \vec{\sigma}_1 \cdot \vec{\sigma}_2$ . 

The most divergent part of this potential, scaling like $\sim 1/r^2$, comes from the tensor force in the spin triplet channel. Attractive potentials which scale as $r^{-n}$ for $n \geq 2$ are referred to as singular potentials. Particles sitting in a singular potential eventually fall toward the center with infinite velocity, which is clearly unphysical. Thus, singular potentials can only be defined with an explicit cutoff that cannot be removed. Particles generally sit near this cutoff, rendering the system sensitive to the short-range details of the choice of boundary condition. Therefore, systems involving singular potentials are generally model dependent and we can no longer have a true effective theory because the cutoff cannot be removed. 

The reason such a singular potential arises is similar to that which led to the Wigner bound in the previous section. Again, we are attempting to sum a subset of diagrams in an effective theory non-perturbatively, which cannot in general be assumed to be a renormalizable process. In practice, nuclear theorists using so-called chiral potentials are generally able to demonstrate that the cutoff dependence is small so long as the cutoff is only varied within a particular range, typically $\Lambda \sim 300-1000$ MeV. Therefore, if we wish to include pions non-perturbatively in our lattice theory we should keep this in mind as it implies a restriction on the allowed lattice spacings, just as we found for the non-perturbative inclusion of effective range contributions. 

Pion fields may be added directly to our lattice Lagrangian in a straightforward way. The incorporation of dynamical pions, however, will likely complicate importance sampling by introducing noise and/or sign problems, and adds complexity to the Monte Carlo algorithms. Fortunately fully dynamical pions are unnecessary; all we actually seek is the addition of a term in the Lagrangian which generates the tree level diagrams between a single pion and two nucleons. The lattice formulation then non-perturbatively accounts for all possible loop diagrams involving this pion-nucleon interaction. Diagrams involving vacuum pion loops, pion self-energies, etc. are higher order in our chiral expansion and can be included perturbatively if necessary. 

One possible implementation utilized by the Bonn-Raleigh group is to use static pion auxiliary fields, $\pi_{\vec{x},\tau}^{(I)}$, with isospin $I$, and the following action \cite{Lee:2008fa,Borasoy:2006qn}:
\beq
S_{\pi\pi} = \left( \frac{m_{\pi}^2}{2} + 3\right) \sum_{\vec{x},\tau,I} \pi_{\vec{x},\tau}^{(I)} \pi_{\vec{x},\tau}^{(I)} - \sum_{\vec{x},\tau,I,k} \pi_{\vec{x},\tau}^{(I)} \pi_{\vec{x}+\hat{k},\tau}^{(I)} \ .
\eeq
Because the pions are derivatively coupled to the nucleons, the interaction term should behave like,
\beq
S_{\pi NN} \sim \frac{g_A}{2f_{\pi}}\sum_{I,k} \left[\pi^{(I)}_{\vec{x}+\hat{k}}-\pi^{(I)}_{\vec{x}-\hat{k}}\right] \psidag_{\vec{x}} \psi_{\vec{x}} \ ,
\eeq
(see \cite{Lee:2008fa} for more details on the particular interaction chosen). The pions have been chosen to only couple to the nucleons through spatial displacements. This simplifies the analysis by eliminating the renormalization of the nucleon mass through nucleon self-energy diagrams such as:

\includegraphics[width=\linewidth]{Chapter5-figures/sunset.png}

Then we can simply utilize the physical value, $M \sim 938$ MeV, for the nucleon mass. These pions therefore act instantaneously, much the same way as they do in a pion potential picture. 

\subsection{3- and higher-body interactions}

Na\"ive dimensional analysis dictates that the leading three-body interaction should be suppressed relative to the two-body interaction by $\calO(L^3)$. We should be more cautious by this point, since we have seen dimensional analysis fail in previous cases for strongly interacting systems. For that reason, we will now inspect the three-body system more carefully. 

To begin, we will consider a system of three particles interacting via only the simplest, leading order two-body contact interaction. We will follow the discussion of \cite{Braaten:2004rn}. Let us assume that all three particles carry different quantum numbers, as they do for the triton and $^3$He, and that all pairs of particles interact via the same two-body coupling, $g_0$. To calculate the three-particle scattering amplitude for a strongly coupled system we must iterate this interaction non-perturbatively, as we did for the two-particle system. 

A useful trick for calculating this quantity is the addition of a bosonic dimer field, $\phi$, coupling to two fermion particles, $\psi$. This allows us to rewrite the three-particle scattering amplitude in the form of a two-particle scattering amplitude. The dimer propagator must be fully dressed by fermion loop bubbles and can be written diagrammatically as shown in \Fig{dimerprop}. This bubble sum is essentially the same as the one we have encountered several times before in these lectures. However, we must now allow external momentum, $(p_0,\vec{p})$ to flow through the diagrams, leading to the following dressed propagator for the dimer field,
\beq
D_0(p_0,\vec{p}) = \frac{1}{1-g_0 \left. I_0 \right|_{E=p_0-p^2/M}} = \frac{1/a-\Lambda}{1/a + i \sqrt{Mp_0-p^2-i\epsilon}} \ ,
\eeq 
where I've used the results from \Sec{couplings} to rewrite the coupling in terms of the scattering length, $a$, and the cutoff, $\Lambda$. We see that the dimer propagator has a pole at $p_0=\frac{p^2}{M}-\frac{1}{Ma^2}$, corresponding to a (virtual) bound state for (negative) positive scattering length with energy $E_B=\frac{1}{Ma^2}$. 

\begin{figure}
\begin{center}
\includegraphics[width=\linewidth]{Chapter5-figures/dresseddimer}
\end{center}
\caption{\label{fig:dimerprop}Dressed propagator for the bosonic dimer field, $\phi$.}
\end{figure}


Using this dimer field, we can write the full three-body scattering amplitude, $A_3$, as an integral equation, shown in \Fig{3bodyint}. To simplify the expression, we can set the $\psi$ fields to be on-shell, so that all off-shell properties are absorbed into the dimer propagator. The amplitude can then be written,
\beq
A_3(p,k;E,p^2/M) &=& -\frac{ g_0}{E-p^2/M-k^2/M -(p+k)^2/M+i\epsilon} \cr
&+& \frac{8\pi i}{g_0}\int\frac{d^4q}{(2\pi)^4} \left(\frac{g_0}{E-p^2/M - q_0-(p+q)^2/M+i\epsilon}\right) \cr
&\times &\left(\frac{1}{q_0-q^2/M+i\epsilon}\right)\left(\frac{A_3(q,k;E,q_0)}{1/a+i\sqrt{M(E-q_0)+q^2-i\epsilon}}\right) \ ,
\eeq
known as the Skorniakov-Ter-Martirosian (STM) integral equation. Integrating over $q_0$ and projecting the system onto the $s$-wave channel gives (see \cite{Braaten:2004rn} for more details),
\beq
\tilde{A}_3(p,k;E) &=& \frac{1}{apk}\ln \left(\frac{p^2+pk+k^2-ME-i\epsilon}{p^2-pk+k^2-ME-i\epsilon}\right) \cr
&+& \frac{1}{4\pi^2}\int^{\Lambda}dq \frac{q}{p}\ln \left( \frac{p^2+pq+q^2-ME-i\epsilon}{p^2-pq+q^2-ME-i\epsilon}\right)  \frac{\tilde{A}_3(q,k;E)}{-1/a + \sqrt{3q^2-ME-i\epsilon}} \ .
\eeq
For large scattering length (strong interaction) we have,
\beq
\tilde{A}_3(p,k;E) \underset{a\to\infty}{\longrightarrow} \frac{1}{4\pi^2}\int^{\Lambda}dq \frac{q}{p} \ln \left(\frac{p^2+pq+q^2-ME-i\epsilon}{p^2-pq+q^2-ME-i\epsilon}\right)\frac{\tilde{A}_3(q,k;E)}{\sqrt{3q^2-ME-i\epsilon}}
\eeq

\begin{figure}
\begin{center}
\includegraphics[width=\linewidth]{Chapter5-figures/3body}
\end{center}
\caption{\label{fig:3bodyint}Full three-particle scattering amplitude written in terms of a two-particle amplitude for a fermion scattering with a dimer field. Here we have only included two-body interactions, with no explicit three-body contact interaction.}
\end{figure}

This integral contains divergences, which may be renormalized by adding an explicit three-body coupling, $H$. To absorb the divergences, the coupling must have the following dependence on the momentum cutoff, $\Lambda$ \cite{Bedaque:1998km,Bedaque:1998kg,Beane:2000wh}:
\beq
H(\Lambda) = \frac{\cos \left[s_0 \ln(\Lambda/\Lambda_{*})+\tan^{-1}s_0\right]}{\cos \left[s_0 \ln(\Lambda/\Lambda_{*})-\tan^{-1}s_0\right]} \ ,
\eeq
where $s_0 \sim 1.006$ is a constant, and $\Lambda_{*}$ is some reference scale which may be set by a three-body observable, such as the triton binding energy, or the neutron-deuteron scattering length. 

There are two remarkable things to note here: the first is that this result for the scattering amplitude is only a leading order result, yet we had to introduce a three-body coupling in order to renormalize the theory. This illustrates another case where na\"ive dimensional analysis does not work, because the three-body coupling contributes at the same order as the two-body coupling. The second is the running of the coupling $H(\Lambda)$, plotted on a logarithmic scale in \Fig{HLambda}. We see that the coupling, and therefore also observables depending on the coupling, displays a log-periodic discrete scaling symmetry, related to the so-called Efimov effect. This property arises for systems obeying a potential at the threshold of singularity, $\sim 1/r^2$, as can be shown to occur for our three-body system using hyperspherical coordinates \cite{V1970563,Efimov:1971zz}.

\begin{figure}
\begin{center}
\includegraphics[width=0.5\linewidth]{Chapter5-figures/HLambda.pdf}
\end{center}
\caption{\label{fig:HLambda}Running of the three-body contact interaction $H(\Lambda)$ at unitarity vs. the momentum cutoff, $\Lambda$, showing log-periodicity.}
\end{figure}

Because the three-body interaction has been demonstrated to be relevant at leading order, we should in general include it non-perturbatively to our lattice theory by adding an interaction term to the Lagrangian such as,
\beq
C_3 \phi_3 \psidag_{\tau} \psi_{\tau+1} \ ,
\eeq
where $C_3$ is tuned to reproduce some three-body observable, and $\phi_3 \in Z_3$ (cube roots of 1). However, $\phi_3$ is necessarily a complex field, will induce severe noise and/or sign problems. The interaction may alternatively be introduced via multiple $Z_2$ interactions, but the noise problem remains.

\Fig{HLambda} is important for our discussion because it shows how the three-body coupling runs as we change the lattice spacing. The larger the coupling, the worse the noise/sign problem will be. The solution chosen by the Bonn-Raleigh group is to tune the ratio $b_{\tau}/b_s$ until a chosen three-body observable is sufficiently well-described by tuning only the two-body interactions. This implies that the three-body interaction is small at this point, and can then be regarded as a higher-order correction and included perturbatively. A drawback to this approach is that we can no longer use the anisotropy parameter as a knob for probing temporal discretization errors. Because the spatial lattice spacing may also already be restricted by the condition of renormalizability of any pion or effective range contributions to the Lagrangian, we have forfeited most of our ability to demonstrate that discretization errors are under control.

Another possibility for reducing the contribution from the three-body interaction might be to change the short-distance behavior of the two-body sector in another way. For example, tuning different numbers of two-body interaction coefficients (\Sec{tuning}) or changing the discretization of the kinetic operator will shift the reference scale $\Lambda^*$, giving us a different value for $H(\Lambda)$ at a fixed lattice spacing.

Finally, given that the three-body sector required a reshuffling of the orders in perturbation theory at strong coupling, should we expect the same for higher $N$-body interactions? Fortunately it has been fairly well established that four- and higher body operators are not necessary to renormalize the theory at leading order and are therefore irrelevant. This means that we may treat four- and higher-body interactions as perturbative corrections.

This is observed via the so-called Tjon line (see, e.g. \cite{Hammer:2010kp}). Recall that while the two-body system at unitarity has no intrinsic scale, in order to describe the three-body system we had to introduce a single scale, $\Lambda_{*}$, to be set by some three-body observable. Once this scale is set, all other three-body observables may then be predicted. If four- and higher-body operators appear only at higher orders, then this three-body scale remains the only relevant scale in the problem, and observables must be proportional to $\Lambda_{*}$ \footnote{This single scale is also critical for the appearance of the log-normal distribution in correlators near unitarity, where the moments are given by
\beq
\calM_N \sim e^{-E_{\mbox{\tiny N-body}}\tau} \sim e^{-f(N) \Lambda_{*} \tau} \ .
\eeq
Numerical evidence was shown in \cite{Nicholson:2012zp} that $f(N)$ has the expected form for the log-normal distribution.}. This implies that varying the three-body parameter $\Lambda_{*}$, in a plot of the binding energy for the four-body system versus the binding energy of the three-body system, will result in a straight line. Any non-linear dependence on higher-order $N$-body operators contributes only within the error band predicted at this order in perturbation theory.

\subsection{\label{conclusions}Final considerations}
Perhaps the most worrisome issue we have discussed is the inability to take the continuum limit due to interactions that are included non-pertubatively and which generate new non-zero scales beyond the scattering length. The lattice spacing must also be kept reasonably large for another reason mentioned previously, related to numerical stability: if the lattice spacing becomes too small, the system will begin to probe the repulsive core of the two-body potential, leading to sign and/or signal-to-noise problems. 

Though we may not have the ability to vary the lattice spacing by significant amounts, we must still prove that our results do not depend strongly on the short-distance details of the action. This can be demonstrated instead by changing the discretization of derivatives in the action, using more or less improvement of the interaction, etc., and showing that the results do not change significantly \cite{Lee:2008fa}. 

Showing convergence of the EFT for the lattice results is also a major concern, particularly since we have no single power-counting scheme that is known to converge in all channels even in the continuum theory. One possible indication of issues with convergence in the current Bonn-Raleigh method is the need for a significant repulsive four-body interaction in order to stabilize four- and higher-body systems, which seem prone to forming four-body clusters on a single lattice site. This is akin to the particles falling to the bottom of a singular potential, and may be related to the particular tuning of the three-body interaction. However, once this interaction has been set the convergence of the results appears to be relatively stable.

Possibly the biggest open issues to be resolved are the sign/noise problems and proving convergence to the ground (or desired excited) state. Noise problems have restricted most calculations of nuclear systems to nuclei in (or near) the alpha ladder, where approximate $SU(4)$ symmetry applies. New theories and/or algorithms would be enormously helpful in this arena. The engineering of better sources or methods for extracting the desired states might be particularly beneficial for both the reduction of noise and to eliminate the need for performing long temporal extrapolations.

Despite these limitations there have been enormous successes for lattice EFT for few- and many-body states both for systems at unitarity and nuclei. As an example, at unitarity the energies of up to 50 two-component fermions have been calculated with errors comparable to state-of-the-art Green's Function Monte Carlo calculations \cite{EKLN1,EKLN2,EKLN3,EKLN4,LEKN1,NEKL1}. The Raleigh-Bonn group has calculated properties of nuclei up to $A=28$ \cite{Epelbaum:2010xt,Epelbaum:2009pd,Epelbaum:2013paa,Lahde:2013kma,Lahde:2013uqa}. Particularly exciting is their investigation of the structure of the Hoyle state, a key component of the triple alpha process necessary for Carbon production in stars \cite{Epelbaum:2011md,Epelbaum:2012qn,Epelbaum:2012iu,Epelbaum:2013wla,Lahde:2014bna}. 

\section{Reading assignments and Exercises}

\begin{prob}
Much of these lecture notes follow this review: arXiv:1208.6556. There you will also find more information about algorithms. The following is an excellent pedagogical introduction to EFT's by David B. Kaplan: arXiv:nucl-th/0510023.
\end{prob}

\begin{prob}
Explore the cumulant expansion using a toy model \cite{EKLN2}: 
\beq
C(\tau,\phi) = \prod_{i=1}^{\tau}(1+g\phi_i) \ ,
\eeq
for $0\leq g \leq 1$ and $\phi \in [-1,1]$. The true mean of the correlator should be $\langle C(\tau,\phi)\rangle = 1$, corresponding to $E_0=0$. Compare the cumulant expansion cut off at various orders on a finite sample size to the mean calculated using standard methods as the sample size is varied.
\end{prob}
\begin{prob}
Reading:
D. Lee: arXiv:0804.3501 \cite{Lee:2008fa}
G.P. Lepage: Analysis of algorithms for lattice field theory \cite{Lepage:1989hd}.
\end{prob}
\begin{prob}
Add a term
\beq
c \psidag_{\tau} \nabla_{L}^2 \psi_{\tau-1}
\eeq
to the simple interaction, \Eq{pointint}, and derive an analytic expression for tuning the couplings, $g_0$ and $c$ in order to eliminate the effective range contribution. You may use either the scattering amplitude or the transfer matrix method.
\end{prob}
\begin{prob}
Write numerical code (Mathematica will suffice) to solve the transfer matrix for two particles for a chosen set of coefficients, $g_{2n}$ (\Eq{tuningcoef}), using $L=32$, $M=5$, and tune your coefficients to match the first few expected L\"uscher eigenvalues at unitarity. Compare your results with those in Table II of Ref.~\cite{EKLN1}.
\end{prob}

\section*{Appendix}
\addcontentsline{toc}{section}{Appendix}
\section{Compilation and running the code}
This code requires the use of the FFTW library, which you may download and install from fftw.org. The script ``create\_lib.sh" should be run first from the head directory. Once this script is successful, you may go into the production directory, modify the script ``create\_binary.sh" to reflect your path to the FFTW library, and compile by running this script. The executable created is called ``a.out", which should be run without specifying any additional parameters in the command line. Input parameters are specified in the files included in the ``arg" folder. The parameters for each file are described in the header ``arg.h". The codes can be downloaded from the link \url{https://github.com/ManyBodyPhysics/LectureNotesPhysics/blob/master/doc/src/Chapter5-programs/}.

Output is created in the folder ``results". The file gives a list of the values (real part listed first, imaginary second) of the two-particle correlation function calculated at different values of Euclidean time, on a set of auxiliary field configurations. The organization of the output is as follows: 
\beq
\begin{array}{ccccccc}
\mathrm{Re}\left[C(\phi_1,\tau_1)\right] & \mathrm{Im}\left[C(\phi_1,\tau_1)\right] & \mathrm{Re}\left[C(\phi_1,\tau_2)\right] & \mathrm{Im}\left[C(\phi_1,\tau_2)\right] & \cdots & \mathrm{Re}\left[C(\phi_1,\tau_{N_{\tau}})\right] & \mathrm{Im}\left[C(\phi_1,\tau_{N_{\tau}})\right] \\
\mathrm{Re}\left[C(\phi_2,\tau_1)\right] & \mathrm{Im}\left[C(\phi_2,\tau_1)\right] & \mathrm{Re}\left[C(\phi_2,\tau_2)\right] & \mathrm{Im}\left[C(\phi_2,\tau_2)\right] & \cdots & \mathrm{Re}\left[C(\phi_2,\tau_{N_{\tau}})\right] & \mathrm{Im}\left[C(\phi_2,\tau_{N_{\tau}})\right] \\
&&& \vdots &&& \\
\mathrm{Re}\left[C(\phi_{\Ncfg},\tau_1)\right] & \mathrm{Im}\left[C(\phi_{\Ncfg},\tau_1)\right] & \mathrm{Re}\left[C(\phi_{\Ncfg},\tau_2)\right] & \mathrm{Im}\left[C(\phi_{\Ncfg},\tau_2)\right] & \cdots & \mathrm{Re}\left[C(\phi_{\Ncfg},\tau_{N_{\tau}})\right] & \mathrm{Im}\left[C(\phi_{\Ncfg},\tau_{N_{\tau}})\right] \\
\end{array} \nonumber
\eeq
where $N_{\tau}$ and $\Ncfg$ are the total number of time steps, specified in ``do.arg", and total number of configurations, specified in ``evo.arg", respectively. To calculate the correlation function at a given time, $\tau$, average over all values: $C(\tau) = \sum_i \left(\mathrm{Re}\left[C(\phi_i,\tau)\right] + i \ \mathrm{Im}\left[C(\phi_i,\tau)\right]\right)$.

\subsection{Exercises}
\begin{prob}
Set the first value in the file ``interaction.arg" to a coupling of your choice, and the remaining couplings to $0$. Use the long time behavior of the effective mass function, $\ln \frac{C(\tau)}{C(\tau+1)} \tautoinfty E_0$ (see \Sec{observables}), to determine the ground state energy for your choice of coupling, $g$. Compare this with what you expect from \Eq{eigeqlambda}, using the relation $\lambda = e^{-E_0}$, as the number of lattice points is increased. You may test the improved interaction, \Sec{tuning}, using coefficients calculated from your code developed in Prob.~4 by setting multiple couplings in the ``interaction.arg" file. Be careful to set the dispersion relation in ``kinetic.arg" to match the one used in setting up your transfer matrix for the tuning.
\end{prob}

\begin{prob}
 Add a harmonic potential by setting the parameters in potential.arg. The three numerical values correspond to the spring constant, $\kappa$, for the $x,y,z$-directions. Set the interaction coefficients to correspond to unitarity, then find the energies of two unitary fermions in a harmonic trap, exploring and removing finite volume and discretization effects by varying the parameters, $L,L_0=\left(\kappa M\right)^{-1/4}$, and performing extrapolations in these quantities if necessary. Compare your result to the expected value of 2$\omega$, where $\omega = \sqrt{\kappa/M}$, and the mass $M$ is set in the file ``kinetic.arg". 
\end{prob}

\begin{prob}
Construct sources for three fermions in an $l=0$ and $l=1$ state and find the lowest energies corresponding to each state at unitarity. Which $l$ corresponds to the true ground state of this system?
\end{prob}

\begin{acknowledgement}
The author would like to thank Michael Endres, David B. Kaplan, and Jong-Wan Lee for extensive discussions, and especially M. Endres for the development of and permission to use this code. AN was supported in part by U.S. DOE grant No. DE-SC00046548. 
\end{acknowledgement}

\bibliographystyle{spphys}
\bibliography{lnplib}





\label{chap:chapter5}
\include{chapter6}\label{chap:chapter6}

\title{Ab initio methods for nuclear structure and reactions: from few to many nucleons}
\author{Giuseppina Orlandini} 
\institute{Giuseppina Orlandini \at Dipartimento di Fisica, Universit\`a di Trento, Via Sommarive, 14 I-38123 Trento, Italy,
Trento Institute for Fundamental Physics and Applications - I.N.F.N, Via Sommarive 14, I-38123 Trento, Italy, \email{giuseppina.orlandini@unitn.it}}
\maketitle

\abstract{These lecture notes intend to give a brief overview of some {\it ab initio} approaches currently used to study 
nuclear structure  properties and reactions.
In the first part particular attention is devoted to two methods useful to account for bound state properties. 
They are both based on the diagonalization of the full many-body Hamiltonian matrix, but share in addition 
the use of similarity transformations.
Transforming the bare potential into an effective one, the latter help in speeding up the convergence of the results.
In the second part {\it ab initio} methods for reaction cross sections involving 
the continuum part of the nuclear spectrum is described, with emphasis on perturbation induced reactions. 
They are based on integral transforms which make it possible to reduce 
the many-body scattering problem to a bound state problem, allowing to take advantage of any of the methods described in the first part.}


\section{Introduction: Theory, Model, Method}\label{sec:TMM}

The importance of studying nuclei lays in the fact that they are the most common manifestation of the strong 
interaction at {\it low}-energy (order of MeV), a regime where the fundamental theory, 
Quantum Chromo-Dynamics (QCD), is non-perturbative. 

Describing nuclei as an assembly of interacting  protons and neutrons corresponds to choosing the {\it effective} degrees 
of freedom (d.o.f) most relevant at that energy. This idea comes from observing that just protons and/or neutrons 
emerge, when energies of a few MeV are transferred to a nuclear system.
Since such degrees of freedom have a comparatively much larger mass (about one GeV), we are allowed 
to adopt a non relativistic quantum mechanical  framework to describe  nuclear  properties.
Then, any observable we would like to account for, will require solving the Schr\"odinger equation, 
governed by a many-nucleon Hamiltonian. In other words we will need to solve the so called {\it non relativistic quantum many-body problem}.

In this context we will adopt the word {\it Theory} referred excluvely to non relativistic quantum mechanics (NRQM), 
in Schr\"odinger, Heisenberg or Interaction representation. The word {\it \bf Model} will be used in connection to the choice
of the d.o.f. and of their mutual interaction, namely  the potential part of the Hamiltonian.
It is clear that any nuclear {\it  Model} (in the above acceptation) must have its roots in QCD, and that the Hamiltonian 
will have to share all its symmetries with QCD.  
These lectures will not deal with the problem of establishing the best {\it model} for a ``realistic'' potential 
(this subject is extensively treated elsewhere 
in this book), here we limit ourselves to consider it as an input for our problem. 

Summarizing, we want to calculate observables within our defined {\it Theory}, with an input nuclear {\it Model}, 
which is both rooted in QCD and {\it realistic} enough to accurately reproduce (at least) 
the nucleon-nucleon scattering data ($\chi$-square per datum close to 1).
However, in addition, we want to be able to control the degree of {\it accuracy} of the method used to solve 
the NRQM many-body problem, namely  we want to determine the theoretical accuracy on the value of an observable.
This is done by benchmarking different results obtained by different methods, using the same input.
All this is what characterizes an {\it ab initio} approach. Comparing the {\it ab initio} results to data we can 
learn about the degree of reliability of the nuclear {\it Model}. In this way we will be
able to predict new nuclear observables, as well as to give other fields (e.g. astrophysics) the needed nuclear information,
complemented by the degree of accuracy of the many-body method used.

\section{The Non-relativistic Quantum Mechanical Many-Nucleon Problem}\label{sec:NRQMP}

The non-relativistic quantum dynamics of a system of $A$ nucleons, supposed to have equal masses $m$,
is governed by the nuclear Hamiltonian $H$, 
which consists of kinetic energy $T$ and potential $V$:
\be
H = T + V  = \sum_{i=1}^A {\frac {\mathbf{p}^2_i} {2m} } + \sum_{i<j}^A V_{ij} + \sum_{i<j<k}^A V_{ijk} + ...\,.
\label{Hint}
\ee
In the equation above ${\p}_i$ is the momentum of the $i$-th nucleon in a general laboratory system,  
and $V_{ij}$ and $V_{ijk}$ denote the nucleon-nucleon (NN)
potential $V_{\rm NN}$, the three-nucleon potential $V_{\rm NNN}$, etc., respectively.
Notice that the reason why in general the nuclear Hamiltonian should contain many-body potentials is due to the fact 
the nucleons are {\it effective} degrees of freedom. The Chiral Effective Field Theory approach to the nuclear potential~\cite{EpM11}
 shows that nuclear forces obey a hierarchy: forces of
more and more many-body nature appear at higher and higher order in a perturbative expansion. 
In our discussion we will restrict to two- and three-body potentials, which appear to be the most relevant for common observables (in
general the inclusion of three-body potentials represents a technical challenge for most {\it ab initio} approaches).

Our problem consists in solving the Schr\"odinger equation,
\be
(H-E_n) |\Psi_n\rangle = 0 \,,
\ee
where $E_n$ and $|\Psi_n\rangle$ denote the eigenenergies and eigenfunctions of $H$, respectively.
The spectrum of $H$, represented by the infinite set of eigenenergies is discrete below, and continuous above, 
the first break-up threshold $E_{\rm th}$. 


In order to solve the Schr\"odinger equation  one has to supply proper
boundary conditions.  For $E < E_{\rm th}$ the wave function represents a bound state
and thus it is described by a square integrable (localized) function. This characteristic 
leads to major technical simplifications, compared
to the case  $E \ge E_{\rm th}$, where  the asymptotic boundary conditions pose   serious problems, especially when  $A>2$ 
(many-body scattering problem). 

There are different ways to tackle the NRQM problem for bound states or for the continuum. 
Very often a reformulation of the problem allows a practical solution, 
which seems impossible otherwise. It is this possibility that generates the richness of methods in many-body theory.



\subsection{Translation and Galileian Invariance}\label{sec:TGI}

A correct approach to the non relativistic many-nucleon problem  should fulfill 
two fundamental symmetries, namely those related to translational and Galileian invariance. One can easily show that the 
corresponding conserved quantities  are center of mass (CM) momentum 
$ {\P}_{CM} = \sum_i^A {\p}_i $ and CM position  $ {\R}_{CM} = \frac{1}{A}\sum_i^A {\r}_i $, respectively. 
Therefore the correct nuclear Hamiltonian must commute with those operators. This can be achieved if one rewrites  
Eq.~(\ref{Hint}) in terms of the Jacobi vectors, i.e. the $A$ independent (normalized) vectors  given by
$ {\R}_{CM}$ and  
%\be\label{jacobi}
%\bm{\eta}_{k-1}=\sqrt{\frac{k-1}{k}}\left({\r}_{k}-\frac{1}{k-1}({\r}_{1}+{\r}_{2}+...{\r}_{k-1})\right)\,\,\,\,\,\,\,k=2,3...A\,,
%\ee
\begin{eqnarray} \label{jacobi}
  \bm{\eta}_1 & = & \sqrt{\frac{A-1}{A}}\Big(\vec{ r}_1 
                - \frac{1}{A-1}(\vec{ r}_2 + \vec{ r}_3 + \cdots
                + \vec{ r}_{A} )\Big)  \nonumber \\
  \bm{\eta}_2 & = & \sqrt{\frac{A-2}{A-1}}\Big(\vec{ r}_{2} 
               - \frac{1}{A-2}(\vec{ r}_3 + \vec{ r}_4 + \cdots
               + \vec{ r}_{A} )\Big)  \nonumber \\
  &\ldots&  \nonumber \\ 
  \bm{\eta}_{N} & = & \sqrt{\frac{1}{2}}\Big( \vec{ r}_{A-1}
                        - \vec{ r}_{A} \Big) \,,\,\,\,\,\,\,\,\,\,\,\, N=A-1\,;
\end{eqnarray} 
together with their conjugate momenta ${\P}_{CM}$ and $\bm{\pi}_1, \bm{\pi}_2, ... \bm{\pi}_N$.

In terms of Jacobi vectors the Hamiltonian in (\ref{Hint}) becomes 
\be\label{Hlab}
H = H_{CM} + {\cal H} = \frac{\P_{CM}^2}{2 A m} + \sum_{i=1}^{A-1} \frac{\bm{\pi}_i^2}{2m} + V(\bm{\eta}_1,\bm{\eta}_2.... \bm{\eta}_{A-1})\,,
\ee
and the translation and Galileian invariant Hamiltonian of our interest is ${\cal H}$, which commutes both with $ {\P}_{CM}$ and 
$ {\R}_{CM}$.
An interesting remark is in order here:
any potential, even when it is limited to have a two-(or three-)body character, becomes an unseparated function of the $(A-1)$-coordinate 
in ${\cal H}$. 
This means that the system, when correctly expressed in terms of relative coordinates only, contains a correlation among the constituents, 
which goes beyond the dynamical one. One can call it a {\it CM correlation}. The latter  can be easily understood from the fact that
the movement of one particle will affect all the others, since the CM momentum and position remain fixed (conserved).

%A final remark on the notation:  in the rest of these notes the Hamiltonian denoted with H will always indicate the translational 
%and Galileian invariant  Hamiltonian 
%${\cal H}$ of Eq.~(\ref{Hlab}

\section{Classification of {\it Ab Initio} Approaches for Ground-state Calculations}\label{sec:CLASS}
As it was stated above the NRQM problem can be formulated in different ways. Therefore one can 
classify the {\it ab initio} methods in terms of just such different formulations, grouping them in different classes:
\begin{itemize} 
 \item  The Faddeev-Yakubowski (FY) method,
 \item  Methods based on the variational theorem,
 \item  Methods based on similarity transformations,
 \item  Quantum Monte Carlo methods.
\end{itemize}
In the following we will concentrate in particular on two of the methods rooted in the  variational theorem. However, in the following
a brief summary of the main peculiarities characterizing each group will be given. 
A more extensive description of the methods can be found in the quoted original references. A recent review can be found in~\cite{WlO12}. 
 
 \subsection{The Faddeev-Yakubowski (FY) Method}\label{sec:FY}
The very nice feature of the 
FY method is that it is formulated in a way that it is applicable to both bound and scattering states.
This method starts from the Lipmann-Schwinger (LS) reformulation 
of the Schr\"odinger equation and therefore deals with integral equations instead of differential equations. 
Today it is well known that a direct application of LS-type equations to the scattering problem for a system with more than two particles
does not lead to a unique solution. However, for quite some time  it was not clear 
how such a unique solution could be obtained. It was in 1961 that in his seminal work for the three-body system~\cite{FADDEEV:1961} Faddeev
showed how the problem can be solved. He derived the right set of coupled integral equations which
have taken his name: the Faddeev equations. 
Yakubowski~\cite{YAKUBOWSKY:1967} generalized the approach, in principle to any number of particles. However, 
the number of coupled equations to solve becomes prohibitive for more than four particles. 
 
A sort of variation of the FY equations are those introduced by Alt, Grassberger
and Sandhas (AGS equations)~\cite{AGS} who looked for possible further reductions of the FY problem. 
Assuming a separable form for the NN t-matrix leads to one-variable integral equations, which are much simpler 
to solve than the FY equations.

\subsection{Methods Based on the Variational Theorem (Diagonalization Methods)}\label{sec:VAR}
The diagonalization methods  are based on the Rayleigh-Ritz variational theorem~\cite{Ra870,Ri909}.
This theorem, which is very profitably applied every time  the solution of some useful equation renders stationary
some proper functional, finds a large application in quantum mechanics. In particular, one can show that the solution of the
Schr\"odinger equation (for a  state with finite norm) renders stationary  the energy
functional
\begin{equation}
 E[\Psi]=\frac{\langle\Psi|H|\Psi\rangle}{\langle\Psi|\Psi\rangle}\,.
\end{equation}

An important  lemma complements the fundamental variational theorem, stating that the value of the energy functional 
calculated with any trial function is always greater than the ground-state energy and equal to it, only when the trial function 
coincides with the exact ground-state wave function.
This means that one can find the ground state energy of a system by solving a minimization problem  
 \begin{equation}
 \delta E[\Psi]=0\,.\label{vareq}
\end{equation}

Numerous approaches use this variational principle to find the ground-state energy of a many-body system.
The approach is efficient if the trial function has a parametrized functional form that is both convenient 
and suitable to the problem to be solved. The various variational approaches differ by the choice of the trial function.
One very well known classical example is the Hartree-Fock method, where the trial function is a Slater determinant. 
In this case, however, it is not possible to give a theoretical estimate of how far the Hartree-Fock energy is from the correct result. 
Also the use of a parametrized functional form for the trial function and the minimization with respect to the parameters 
does not allow a theoretical  estimate of the error. More sophisticated approaches exist like the resonating group method~\cite{RGM1,RGM2}, 
where the trial function 
is chosen according to a cluster picture of the system  or the variational Monte Carlo (VMC) technique where the trial function reflects 
the form of the potential.

A more systematic approach consists in choosing the trial function $\Psi_T$ 
as an expansion on a complete (or over complete) set of square
integrable functions $\phi_n$ that respect the symmetries of the Hamiltonian:
\be
\label{Psi_trial}
|\Psi_T(N)\rangle = \sum_{n=1}^N c_n |\phi_n\rangle\,.
\ee
In this case the minimization procedure 
corresponds to finding the solution of a (generalized) eigenvalue problem 
\be
({\bf H}-E\,{\bf M})\,C=0 \,,\label{secular}
\ee
where $\bf H$ and $\bf M$ are $N \times N$ Hermitean matrices of the Hamiltonian ($H_{nm} = \langle \phi_n|  H |\phi_m\rangle$) 
and overlap integrals of the basis functions ($M_{nm} = \langle \phi_n|\phi_m\rangle$),
while $C$ represents the $N$-component vector formed by the linear parameters $c_n$.
With growing $N$ the size of the Hamiltonian matrix, 
represented on the chosen basis, increases and the true ground-state energy is approached from above. 
The basis can be complete as the hyperspherical harmonics (HH) or the harmonic oscillator (HO) basis,  or over-complete.
In principle the true result would be obtained only for an infinite number of basis functions, however,
the convergence of the smallest energy obtained after the diagonalization for large enough $N$ gives
the ground state energy. An estimate of the error can also be given, related to the convergence pattern.
One can  consider this as an  {\it ab initio} result. 

Here one should also mention another interesting variational method, namely the  stochastic variational method (SVM)~\cite{SVM1,SVM2}. 
Here again  the variational procedure does not proceed systematically 
by the diagonalization of a larger and larger Hamiltonian matrix,  
but in a stochastic way (trial and error), obtaining nevertheless rather good results when compared to other approaches~\cite{bench_2001}.

Variational approaches also allow to obtain the wave function corresponding to the minimal energy,
which can then be used to calculate other ground-state observables. However, one has to remember
that the difference between the exact value of the energy and that obtained with the trial function $\Psi_T$
which minimizes the energy functional, 
is an infinitesimal of higher order than the difference between the true wave function and $\Psi_T$. Therefore one should expect
a slower convergence and less accuracy for such observables.

\subsubsection{The Hyperspherical Harmonics (HH) Method}\label{sec:HH}

The HH method is a variational method where the trial function is written as
an expansion on the {\it hyper}-spherical harmonics (HH) basis. 
The HH are the generalization of the
spherical harmonics $Y_{lm}$. In fact as the latter represent a basis for the relative wave function of a two-body system,
the HH represent a general basis for the internal wave function of an $A$-body system.
Because of this,  they   are expressed in terms of the hyperspherical coordinates which are defined by a transformation 
of the Jacobi vectors.

Let us remember that the set of $A$ Jacobi vectors is composed 
by  $\R_{CM}$ and the $N=A-1$ relative vectors $\bm{\eta}_1,...,\bm{\eta}_N$ in Eq.(\ref{jacobi}), for a total of $3 A$ coordinates. 
The hyperspherical coordinates are defined by further transforming  
the $3N$ coordinates $\bm{\eta}_1,...,\bm{\eta}_N$ as follows:
the $2N$ polar angles $\theta_i$ and $\phi_i$ of the $\bm{\eta}_i\equiv(\eta_i,\theta_i,\phi_i)$ are left unaffected 
by the transformation. The remaining 
$N$  hyperspherical coordinates $\eta_i$ are expressed in terms of one {\it hyper}-radius $\rho_N$ and $(N-1)$ 
{\it hyper}-angles $\alpha_n$ defined by
\be
\sin\alpha_n= {\frac {|\eta_n|}{\rho_n}} \,; \,\,\,\,\,\,\,\,\,\,\,\,\,\,\rho^2_n  = \sum_{i=1}^n \eta_i^{\,2}
 \,, \,\,\, n=2,...,N \,.
\ee

A very interesting feature of the hyperspherical coordinates is that, when expressed in such coordinates,
the $A$-body kinetic energy operator of $A$ nucleons of equal masses is a sum of two terms ($\rho^2 \equiv \rho^2_N$)
~\cite{HiD56}
 \begin{equation}\label{Trho}
  T =  T_\rho + T_K(\rho)\,,\,\,\,\, {\rm with} \qquad T_\rho= - \frac{1}{2m}\Delta_{\rho} \,, \qquad 
T_K(\rho)=\frac{1}{2 m} \frac{{\bf K}_N^2}{\rho^2}\,,
\end{equation} 
namely it has a form which is  in perfect analogy to the three-dimensional case, 
with a {\it hyper}-radial dependent Laplacian $T_\rho$ and a {\it hyper}-centrifugal barrier $T_K(\rho)$.

The {\it hyper}-angular momentum operator ${\bf K}_N$ depends on all the $(3N-1)$ angles (denoted by 
$\hat\Omega_{[N]}$) and has a rather complicated form. But the main point here is that the HH 
are the orthonormal eigenfunctions ${\cal Y}_{[K_N]}(\hat\Omega_{[N]})$ of ${\bf K}_N^2$
\begin{equation}
{\bf K}_N^2\,{\cal Y}_{[K_N]}(\hat\Omega_{[N]})= K_N(K_N+3N-2){\cal Y}_{[K_N]}(\hat\Omega_{[N]})\,.
\label{eigeneqK2}
\end{equation} 
As one sees the eigenvalues are expressed in terms of the quantum number $K_N$. 
The subscript $[K_N]$ stands for the total set of $(3N-1)$ quantum numbers corresponding to commuting operators, namely
the hyperangular momenta ${\bf K}^2_N, {\bf K}^2_{N-1},... {\bf K}^2_2$ relative to the subsets of $N,N-1...2$ $\bm \eta$-coordinates, 
the angular momenta relative to each of the $N$ Jacobi coordinates $\vec l^2_N,\vec l^2_{N-1},...\vec l^2_1 $,
the total angular momenta $\vec L^2_N,\vec L^2_{N-1},...\vec L^2_2$ of the same subsets of $N,N-1...2$ $\bm \eta$-coordinates, 
and the third component of the total angular momentum $L^z$.
 
The ${\cal Y}_{[K_N]}(\hat\Omega_{[N]})$ are good  basis functions for the hyperangular part of the $A$-body internal 
wave function, however, one also needs good basis functions for the hyperradial part of the wave function.
A suitable choice are the orthogonal Laguerre polynomials, because of their exponential weight function, reproducing the 
correct asymptotic behavior of the wave function.

The basis obtained  by the product of Laguerre polinomials and HH is a translation invariant  CM ``correlated``  basis 
(all particles are connected to each other!) and
has good asymptotic conditions, therefore one can expect a faster convergence with respect to using a translation invariant HO basis. 
However, just because of the mentioned correlation, the basis presents big difficulties when coping with the Pauli
principle, a  problem also common to the translation invariant HO basis. 
Based only on intuition one can guess that permutations of particles will lead to different definitions 
of the Jacobi coordinates and consequently of the hyperspherical coordinates. It would be a miracle if the  HH would have 
definite permutational symmetries. And in fact they do not, but they possess different  components 
of the irreducible representations of the symmetry group $S_A$. Even when this problem is overcome 
(see~\cite{NOVOSELSKY:1994,BARNEA:1997+8, Gatto:2011,Deflorian:2013}), as the number of particles increases the convergence 
becomes rather slow.

In order to speed up the convergence two ways have been followed: the Correlated Hyperspherical Harmonics expansion (CHH)  and the 
Hyperspherical Harmonics expansion with Effective Interaction (EIHH). The latter will be explained in Section~\ref{sec:EIHH}. 
The main idea of the CHH approach consists in acting on the bare HH functions with a Jastrow
operator $\hat J$ embodying the short range correlation due to the repulsive part of the potential.
Such a repulsion leads to high momentum components in the wave function which is responsible for the 
slow convergence of the bare HH expansion. The correlation operator $\hat J$ takes the form
\begin{equation}
 \hat J={\cal S}\prod_{i<j}\sum_{s,t}f_{st}(r_{ij})P_{st}(i,j)
\end{equation}
where $P_{st}(i,j)$  are projection operators onto nucleon pairs $(ij)$ with spin $s$ and isospin $t$ and
${\cal S}$ is a particle symmetrization operator. This method has been applied only to A=3,4 systems, since the loss of orthonormality
of the CHH limits its efficiency. In fact calculating the matrix elements of the potential requires $3N$ dimensional integrals.
The reason why  this is not the case for the uncorrelated HH basis is explained in the following.

When expressed in HH coordinates the invariant Hamiltonian $\cal H$ is
\begin{equation}
 {\cal H}=   T_\rho + T_K(\rho) + V(\rho, \hat\Omega_{[N]})\,.
\end{equation}
However, supposing for simplicity that the potential has a two-body character $\sum_{i<j} V_{ij}$
(but the present argument can be easily extended to three-body potentials) its matrix element 
on antisymmetric functions will be the sum of $A(A-1)$ identical integrals 
with, say, $i=A$ and $j=A-1$. This means that $V$  will be a function only of the Jacobi vector $\bm{\eta}_N$, namely
$V(\sqrt{2}\rho \sin \alpha_N, \theta_N,\phi_N)$, 
the recursive  construction of the HH allows then to use the orthonormality condition 
of the ${\cal Y}_{[K_{N-1}]}(\hat\Omega_{[N-1]})$ and reduce the calculation of the matrix element of the potential 
to a (at most) four-dimensional integral, for any number of particles. When the orthonormality condition of the HH is lost, like in the 
CHH case, this is no longer true and one is left with 3N-dimensional integrals.


\subsection{Methods Based on Similarity Transformations~\label{sec:SIM}}
Another reformulation of the quantum mechanical many-body problem is based on the use of similarity
transformations~\cite{Ok54,CoK60,DaS64,SuL80}.
In this case one considers that the following mean value
\be
E_0 = \langle\Psi_0| H|\Psi_0\rangle \,,
\ee
where $|\Psi_0\rangle$ is the ground state  of the Hamiltonian $H$, 
is invariant under similarity transformations $e^S$, i.e.
\begin{equation} 
E_0= \langle\Psi_0|e^{-S} \,e^{S}\, H\,e^{-S} \, e^{S}|\Psi_0\rangle
\equiv \langle \bar\Phi|{\bar H} |\Phi\rangle\label{simileq} \, 
\end{equation}
with  
\be
|\Phi\rangle = e^{S}|\Psi_0\rangle \,, \qquad |\bar\Phi\rangle = e^{-S^\dagger}|\Psi_0\rangle \,,
 \qquad{\bar H} = e^{S}\,H\,e^{-S}\,.
\ee 

At this point one may consider a subspace P of the  Hilbert space  with  eigenprojector $\hat P$ given by 
\be
\hat P=\sum_{n=1}^N |\phi_n\rangle\langle\phi_n|\,, 
\ee 
where the $|\phi_n\rangle$ are eigenfunctions of some well known Hamiltonian (e.g. $H_{HO})$.
Indicating by $\hat Q=I-\hat P$ the corresponding eigenprojector on the  residual space, one can write Eq.~(\ref{simileq})
as 
\begin{equation}
E=\langle \bar\Phi|(\hat P + \hat Q){\bar H}(\hat P + \hat Q) |\Phi\rangle =
\langle \bar\Phi|\hat P{\bar H}\hat P+\hat P{\bar H}\hat Q+\hat Q{\bar H}\hat P+\hat Q{\bar H}\hat Q|\Phi\rangle\,.
\end{equation}
If  the following decoupling condition is satisfied
\begin{equation}
 \hat Q{\bar H}\hat P = \hat Q e^{S}\, H\,e^{-S}\hat P = 0\,,
\label{decoup}
\end{equation}
one has
\begin{equation}
E=\langle \bar\Phi|\hat P{\bar H}\hat P|\Phi\rangle\,.
\end{equation} 
This means that if one solves the decoupling equation (\ref{decoup}) it is possible, in principle, to determine $S$ and therefore
calculate $E_0$ as the mean value of the {\it effective} operator ${\bar H}$ on the P-space.  

Notice that, while in the {\it bare} Hamiltonian $H$ the operators may have  a two- or three-body nature, the  
{\it effective} operator ${\bar H}$ will be in principle an $A$-body operator. So, in general the operator $S$, 
which generates the similarity transformation, may be written as a combination of operators of any $n\leq A$-body nature. 
It is clear that in actual calculations one has to apply some restrictions on the number of these $n$-body
operators. In this respect the similarity transformation approach has been used in two ways: 
\begin{itemize}
 \item  in the so called Coupled Cluster (CC) method~\cite{Zab:1978,HJ:2004}  (discussed elsewhere in this book) 
 to calculate the ground state energies and radii of A-body nuclei ($n\leq 3$, known as CCSD and CCSDT);
 \item to construct effective two- or three- body potentials, as described in the next Section~\ref{sec:LS},
 in order to accelerate convergence in the variational diagonalization approaches using the  HO  
and the HH basis (see Sections~\ref{sec:NCSM} and~\ref{sec:EIHH}).
\end{itemize}


%In the first case one operates a restriction to a finite number of terms (2 or 3), but the sum can be systematically enlarged 
%in principle. In the second case   and ii) in the construction of effective interactions
%in order to accelerate convergence in the variational diagonalization approaches, in particular with the HO  
%and the HH basis (see Sections~\ref{sec:NCSM} and~\ref{sec:EIHH})  
%treated below).

\subsubsection{The Similarity Transformation Method for Effective Interactions}
\label{sec:LS}
%The approach that is chosen to construct the effective Hamiltonian, and consequently the effective interaction, 
%appropriate to the finite P${\!_A}$-space, 
%is that described in Section~\ref{sec:SIM}. However, 

As was stressed in the previous Section, the similarity transformation leads to  an  effective Hamiltonian 
which is an $A$-body operator. To avoid this complication, an approximation is made. It consists 
in first finding only a two-body effective interaction $\tilde V_{ij}^{\rm [2,eff]}$
which is then used to replace  the {\it bare} interaction term $ V_{ij}$. This approach is often referred to
as the Lee-Suzuki (LS) method~\cite{SuL80}. The effective interaction $V_{ij}^{\rm [2,eff]}$
is obtained  by applying the decoupling condition of Eq.~(\ref{decoup}) to a two-nucleon
Hamiltonian $H^{[2]}$  that arises from ${\cal H}$ by restricting  the kinetic and potential operators
to two nucleons only (e.g. $i=A$ and $j=A-1$). In this simple case the decoupling condition can be solved.
In fact the two-body problem can be fully solved and in this case one has the knowledge of both the $P$ and the $Q$ sapce.

Once the effective interaction $ V_{12}^{\rm [2,eff]}$ is obtained one replaces it in the $\sum_{i<j} V_{ij}$.
The replacement in the potential term of the effective interaction $ V_{ij}^{\rm [2,eff]}$ 
makes the approach no longer  variational. 
The $n$-body terms neglected in the full effective Hamiltonian, could either increase
or decrease the binding energy. On the other hand, one has an important result: as the P${\!_A}$-space  is increased
the result has to converge to the exact solution.
This may be illustrated in a pictorial way as in Fig.~\ref{fig:1}. At each P$_{\!A}$,  since
the similarity transformation transfers information from the Q$_{2}$ space
to the P$_2$ space, there is much less information left out. Consequently the convergence on $E_0$ is much faster. 
When  P$_{\!\!A}$ is  sufficiently large, so that it covers almost the 
whole Hilbert space, the effective interaction practically coincides  with the bare one, and one has an accurate result. 
From the figure one can infer that the exact result could be reached also applying the similarity transformation to the 
three-body and then four-body  Hamiltonian etc., namely sistematically applying the similarity transformation to
move the information from the larger $Q_3$ space into the
$P_3$ space, from the $Q_4$ space into the $P_4$ space, etc..

However, this is much more problematic and  impossible in practice, since one would need to
know the entire $n$-body spectrum to construct the $n$-body effective interaction. 
Of course if three-body forces are present in the original Hamiltonian one has 
to apply the  procedure at least up to $n=3$.  
\begin{figure}
\sidecaption
% Use the relevant command for your figure-insertion program
% to insert the figure file.{\pi}
% For example, with the option graphics use
\includegraphics[scale=.65]{Chapter7-figures/fig1.eps}
%
% If not, use
%\picplace{5cm}{2cm} % Give the correct figure height and width in cm
%
\caption{The various P and Q spaces relevant for the construction
of the two-body effective interaction (see text).}
\label{fig:1}       % Give a unique label
\end{figure}

\subsection{Monte Carlo Methods}\label{sec:MC}

The Monte Carlo (MC) methods are based on a formulation of the quantum mechanical many-body problem
which is suited to a stochastic approach.  These methods are
the Green's Function MC (GFMC)~\cite{Ka62},  Diffusion MC (DMC)~\cite{Ca87} and Auxiliary Field Diffusion MC (AFDMC)~\cite{ScF99}, 
the Chiral Effective Field Theory on a Lattice (LCEFT)~\cite{Le09},
the Monte Carlo Shell Model Diagonalization (MCSMD)~\cite{KoD97,HoM95,OtH01}, 
and the Variational Monte Carlo (VMC)~\cite{PiW01}. 
The GFMC, DMC and AFDMC methods are based on the path integral formulation of quantum mechanics
(there are small differences between the GFMC and DMC methods so that in the literature they are often interchanged).
The LCEFT is  a DMC approach, except that the dynamical degrees of freedom are nucleon and pion fields rather than particles.
Both GFMC and LCEFT methods are based on the Euclidean time (imaginary time) evolution of the system.  
The MCSMD, although inspired by the imaginary time formulation, is  effectively a variational method. 
Starting from an  imaginary time evolved trial function,  after some manipulation,  MCSMD leads to an expression that suggests 
a way of constructing a variational shell model basis. In this case the imaginary time is just one of the non-linear parameters.
The  VMC method is a fully variational method. Here the MC technique is used to evaluate 
the many-dimensional energy functional integrals. The variational wave function  obtained with this method  
usually serves as starting trial function for the GFMC imaginary time evolution.

An extensive treatment of   Monte Carlo approaches can be found elsewhere in this book.

\section{Two Diagonalization Methods with Effective Interactions}\label{sec:TWOEI}

In this section we will describe two diagonalization methods, which have much in common: the No Core Shell Model (NCSM) 
and the Effective Interaction Hyperspherical Harmonic (EIHH) methods. They  both make use of the LS similarity transformation method  
 (see Section \ref{sec:LS}), in order to speed up the convergence. 
They differ only by the choice of the basis on which the Hamiltonian is diagonalized and, 
for $A > 4$, also for the way they treat translation invariance. 

\subsection {The No Core Shell Model Method (NCSM)}\label{sec:NCSM}
The name NCSM means that all the nucleons  are taken into
account  explicitly as degrees of freedom, namely  it is not assumed that there is an inert core, like in the traditional shell model. 
The NCSM couples the  advantage of the shell model (i.e. working in a HO basis) with the accuracy of an {\it ab initio} approach.  

In the literature two versions of the NCSM exist , which differ in the treatment of translation invariance.
In one version, the Hamiltonian of Eq.~(\ref{Hint})
is modified~\cite{ShF74} by adding a harmonic oscillator CM Hamiltonian $H_{\rm CM}$ to the intrinsic Hamiltonian $\cal H$
 \begin{eqnarray}
H^{[A]}_\Omega &=& {\cal H} + H_{\rm CM}^{\rm HO} = {\cal H} +  \frac {{\P}_{\rm \rm CM}^2} {2Am} + 
{\frac{Am}{2}} \Omega^2 {\bf R}_{\rm CM}^2 \\
\label{prova}
 &=& \sum_{i=1}^A \left[ {\frac{{\bf p}_i^2}{2m}}
+\frac{1}{2}m\Omega^2 {\bf r}^2_i
\right] + \sum_{i<j=1}^A \left[ V_{ij}
-\frac{m\Omega^2}{2A}
({\bf r}_i-{\bf r}_j)^2 
\right]\\
&\equiv& \sum_{i=1}^A h_i^{\rm HO} + \sum_{i<j=1}^A\tilde V_{ij} \,,
\label{HOmega}
\end{eqnarray}
where $\tilde V_{jk}$ is a modified potential which depends  on  both the HO frequency $\Omega$ and the nuclear 
system via the mass number $A$: 
\be
\label{Vtilde}
\tilde V_{ij}=\left[ V_{ij}
-\frac{m\Omega^2}{2A}
({\bf r}_i-{\bf r}_j)^2\right]\,.
\ee
Of course the added center of mass HO term has no influence on the internal motion. 
Therefore the ground-state energy $E$ of ${\cal H}$ is obtained
by subtraction of the CM ground-state energy $3\hbar \Omega/2$ from the ground-state energy 
$E^{[A]}_\Omega$ of $H^{[A]}_\Omega$. 

In order to obtain an accurate result the calculation of $E^{[A]}_\Omega$  should be  performed with the $\tilde V$ of Eq.(\ref{Vtilde})
in a finite model space  P${\!_A}$ spanned
by all the $A$-body HO Slater determinants formed by filling the single-particle HO eigenstates
with $N\leq N_{\rm max}$ ($N$ is the total number of single-particle HO quanta) and increasing  
P${\!_A}$, namely $N_{\rm max}$, up to convergence. However, since the convergence is very slow one can speed it up
using a $\tilde V^{\rm [2,eff]}$. This is obtained as described in Section.~\ref{sec:LS}, namely applying the SL procedure
to a two-body internal Hamiltonian obtained from (\ref{prova}) 
by restricting the sums to two nucleons only (e.g. nucleons $A$ and $A-1)$, keeping however the original mass number $A$ 
in the interaction term $\tilde V$:

\begin{equation}\label{HO2mega}
{\cal H}^{[2]}_\Omega = \left[ \frac{\bm {\pi}^2}{2 m}
+\frac{1}{2}m\Omega^2 \bm {\eta}^2\right ]+
\tilde V_{A(A-1)}
\equiv {\cal H}^{[2]}_{\rm HO} +\tilde V_{A(A-1)}  \,.
\end{equation}

 
%\begin{eqnarray}
%{\cal H}^{[2]}_\Omega &=&  \frac{{\bf p}_1^2}{2m}+\frac{{\bf p}_2^2}{2m}
%+\left[\frac{1}{2}m\Omega^2 {\bf r}^2_1+\frac{1}{2}m\Omega^2 {\bf r}^2_2
%\right] 
%+  \left[ V_{12}
%-\frac{m\Omega^2}{2A}
%({\bf r}_1-{\bf r}_2)^2\right]
%\equiv H^{[2]}_{\rm HO} +\tilde V_{12}  \,.
%\label{HO2mega}
%\end{eqnarray}

The effective Hamiltonian ${\cal H}^{[2]}_{\rm eff}$ is determined in  
the P$_{2}$-space ($\hat P_2+ \hat Q_2 =I_2$), a subspace of $P$, via 
the two-body transformation operator $S^{[2]}=\hat Q_2 S^{[2]}  \hat P_2$. 
Then by subtracting ${\cal H}^{[2]}_{\rm HO}$ from ${\cal H}^{[2]}_{\rm eff}$ the two-body effective interaction is obtained  i.e. 
\begin{equation}
\tilde V_{12}^{[2,\rm eff]}  = {\cal H}^{[2]}_{\rm eff} - {\cal H}^{[2]}_{\rm HO}\,.
\end{equation}
The obtained  $\tilde V_{ij}^{[2,\rm eff]}$  is then used in Eq.~(\ref{HOmega}).
As is clear from Ref.~\cite{DaS64}, this procedure 
is equivalent to  {\bf (i)} limiting the similarity operator $S$ of Section~\ref{sec:SIM}   to
a  two-body operator $S^{[2]}$ and {\bf (ii)} truncating the effective Hamiltonian   at the two-body operator level. 
When the diagonalization of the Hamiltonian is performed with the two-body effective interaction the  NCSM is no longer  variational. 
In fact, as was already stated above the real effective interaction obtained by a full similarity transformation at a fixed P-space
is an $A$-body interaction. The neglected $n>2$-body terms  could either increase
or decrease the binding energy. However, as the P-space increases
the result converges to the exact solution, as was already discussed in Section~\ref{sec:LS}. 

Another version of the NCSM exists~\cite{NAV:1998}, where the problem is  formulated directly in terms of Jacobi coordinates 
and the $A$-body basis is 
the translationally invariant HO basis.  However, it is  restricted to $A=3,\,4$, 
because of the same complications generated by the Pauli principle, as those was mentioned in Section~\ref{sec:EIHH}.


\subsection{The Hyperspherical Harmonics Method with Effective Interaction (EIHH)}\label{sec:EIHH}

The idea is very similar to that of the NCSM approach, but with the following two differences:
one is that the P-space is defined by a maximal value $K_{\rm max}$ of the 
grand angular quantum number $K_N$ ($N=A-1$), and the second is that 
the EIHH two-body Hamiltonian  undergoing the similarity transformation, $H_A^{[2]}$, is a {\it quasi} 
two-body Hamiltonian, because it  contains information about the dynamics of the entire $A$-body system via an hyperradial dependence. 
\begin{figure}
\sidecaption
% Use the relevant command for your figure-insertion program
% to insert the figure file.{\pi}
% For example, with the option graphics use
\includegraphics[scale=.35]{Chapter7-figures/fig2.eps}
%
% If not, use
%\picplace{5cm}{2cm} % Give the correct figure height and width in cm
%
\caption{Comparison between EIHH (full squares), pure HH 
(open squares)  and NCSM results obtained with different values of the HO
parameter.}
\label{fig:2}       % Give a unique label
\end{figure}

To be more explicit, the HH {\it quasi} two-body Hamiltonian is chosen as  
\begin{equation}
\label{H_quasi}
 H^{[2]}_A(\rho)= T_K(\rho) + V_{A,A-1}(\sqrt{2}\rho \sin{\alpha_N} \, \theta_N,\phi_N)\,,
\end{equation} 
where $T_K(\rho)$ is the collective hyperspherical kinetic energy of the entire $A$-body system 
(see Eqs.~(\ref{Trho})) and we make explicit only the spatial dependence of $V$ on two particles 
(chosen to be $A$ and $A-1$, consistent with  Jacobi coordinates constructed  in normalized reversed order, like those
in Eq.(\ref{jacobi}. 

The  transformation is applied separately for each value of the hyperradius $\rho$,
therefore the effective interaction becomes a function of $\rho$.
In addition to its $\rho$-dependence, the HH effective interaction
also depends on some quantum numbers of the
residual system. The  additional many-body information contained in the HH effective interaction 
(obtained by subtracting the hypercentrifugal term) is responsible for 
leading to a fast convergence of the HH expansion as can be seen in an example reported in Fig.~\ref{fig:2} from~\cite{BaL00}.
In this figure the binding energy of $^4$He was obtained using a central NN potential. One can notice the enormous advantage 
in the convergence pattern of the EIHH results in comparison to the HH ones. It is also interesting to notice the comparison with the NCSM 
results, which are spread around the EIHH one, depending on the HO parameter, as well as the non-variational behavior 
of the convergence pattern.


\section{Excited States}\label{sec:EXCITED}
%While with the FY method   
%the calculation of a discrete excited state follows the same line as that of the ground state  the situation is different 
%for the diagonalization approaches.
By diagonalizing the Hamiltonian matrix on a given finite basis of square integrable functions one gets a spectrum of $N$
eigenstates $\Psi_i$ with eigenvalues $E_i$. If the trial function has the same quantum numbers of the ground state
of the nucleus,  the lowest $E_i$ corresponds to the ground-state energy. The other solutions correspond to discrete excited 
states of the system, 
with the same quantum numbers as the ground state. Getting convergence for the energy of such excited states requires 
in general larger matrices. 
To find discrete excited states with quantum numbers different from those of the ground state one has to implement 
the proper quantum numbers into the basis set.

Few-nucleon systems have very few, if any, discrete excited states, in fact most of them lie 
in the continuum. Continuum eigenvalues 
are larger than the threshold energy $E_{th}$, corresponding to their ground state energy plus
the nucleon separation energy (the proton separation energy 
is always smaller than that of a neutron if the Coulomb force is considered). 
The  energy states larger than $E_{th}$ obtained by diagonalization do not have the proper 
continuum boundary conditions. 
Proceeding in such a way one obtains a fake discretization of the continuum. 

In order to avoid finding the continuum solutions of the Schr\"odinger equation, corresponding to 
solve the many-body scattering problem,
one can try again to reformulate the quantum mechanical problem in an accessible way.
This will be done in the next sections,  referring in  particular to observables that require the knowledge  of such continuum states.

\subsection{Response Functions to Perturbative Probes}\label{sec:RPP}

Here we focus on a particular family of reactions involving states in the continuum: we deal with nuclear reactions on light systems 
induced by perturbations. Typical examples are electroweak (e.w.) reactions like electron or neutrino scattering on nuclei or 
nuclear photabsorption.

The strength of the interaction Hamiltonian $H_{int}$ between an electromagnetic or weak probe and the nucleus is very weak when
compared to the strong interaction among the nucleons. Therefore the cross section can be calculated in first order 
perturbation theory, using the 
Fermi Golden rule. This means that the cross section will contain the transition rate proportional to the square modulus of the 
matrix element $ |\langle f|H_{int}|i\rangle|^2$ with $|i\rangle$ and $|f\rangle$ the initial and final states of the nucleus, 
as well as an energy conserving $\delta$-function
\begin{equation}
 \sigma\propto|\langle f|H_{int}|i\rangle|^2 \delta(\omega-E_f+E_i)\,,
\end{equation}
where $\omega$ is the energy transferred by the probe to the nucleus $(\hbar=c=1)$ and $E_f,E_i$ are the energies of the nuclear 
final and initial states, generally the ground state $|0\rangle$ and one of its $|n\rangle$ eigenstates (we suppose 
that in $\omega$ the  energy 
that has served to recoil the nucleus has been subtracted). 
In general the interaction $H_{int}$ can be described as the product of the ''current`` densities inside the nucleus  
and the field generated  by the probe, e.g. the charge density $\rho(\vec r)$ and the electromagnetic scalar potential $\varphi(\vec r)$
generated by the electrons in an electron scattering experiment
\begin{equation}
 H_{int}=\int d\vec r\,\rho(\vec r)\,\varphi(\vec r)
\end{equation}
In this example the nuclear  charge density is due to the presence of the protons
\begin{equation}
\rho(r)=\sum_{j=1}^Z e\, \delta(\vec r-\vec r_j)=e \sum_{j=1}^A  \delta(\vec r-\vec r_j) \frac{1+\tau_j^3}{2}\,.
\end{equation}
The initial and final states will be the product of the CM wave functions 
and the internal ones, which are antisymmetric and translation invariant 
\begin{eqnarray}
|i\rangle &=& |\Phi_i(\vec R_{CM})\rangle|\psi_i(\bm{\eta}_1...\bm{\eta}_{A-1})\rangle\\
|f\rangle &=& |\Phi_f(\vec R_{CM})\rangle|\psi_f(\bm{\eta}_1...\bm{\eta}_{A-1})\rangle
\end{eqnarray}
Therefore 
\begin{eqnarray}\label{ME} 
 \langle f|H_{int}|i\rangle &=& e \int d\vec r \,\varphi(\vec r)\,\int d\vec R_{CM}  
 e^{-i\vec P_f\cdot\vec R_{CM}}\,e^{i \vec P_i\cdot\vec R_{CM}}
  \,\,\times \nonumber\\
 &\times& \int d\bm{\eta}_1...d\bm{\eta}_{A-1}\psi^*_f(\bm{\eta}_1...\bm{\eta}_{A-1}; 
 \sigma_1^z...\sigma_A^z;\tau_1^z...\tau_A^z)\,\,\times \nonumber\\
 &\times&\sum_{j=1}^A  \delta(\vec r-\vec r_j)\frac{1+\tau_j^3}{2} \psi_i(\bm{\eta}_1...\bm{\eta}_{A-1}; 
 \sigma_1^z...\sigma_A^z;\tau_1^z...\tau_A^z)\,.
\end{eqnarray}
Because the $\psi$ are antisymmetric the matrix element above is a sum of $A$ equal integrals, with the charge density operator 
limited to only one element of the sum (e.g. j=1). 

At this point it is necessary to express 
$\delta(\vec r-\vec r_1)$ in terms of the integration variables. Using the definition of the Jacobi coordinate $\bm{\eta}_1$  
as in (\ref{jacobi}
\begin{equation}
\bm{\eta}_1=\sqrt{\frac{A}{A-1}}(\vec r_1 - \vec R_{CM})
\end{equation}
the matrix element in \ref{ME} becomes
\begin{eqnarray}\label{MEE} 
 \langle f|H_{int}|i\rangle &=& A e \int d\vec r \,\varphi(\vec r)\,\int d\vec R_{CM}  e^{i\vec q\cdot\vec R_{CM}}
  \,\,\times \nonumber\\
 &\times& \int d\bm{\eta}_1...d\bm{\eta}_{A-1}\psi^*_f(\bm{\eta}_1...\bm{\eta}_{A-1}; 
 \sigma_1^z...\sigma_A^z;\tau_1^z...\tau_A^z)\,\,\times \nonumber\\
 &\times&\sum_{j=1}^A  \delta\left(\vec r-\vec R_{CM}-\sqrt{\frac{A-1}{A}} 
 \bm{\eta}_1\right)\frac{1+\tau_1^3}{2} \psi_i(\bm{\eta}_1...\bm{\eta}_{A-1}; 
 \sigma_1^z...\sigma_A^z;\tau_1^z...\tau_A^z)\,.
\end{eqnarray}
After performing the integral in $d\vec R_{CM}$ one has
\begin{eqnarray}   
 \langle f|H_{int}|i\rangle &=& A e \int d\vec r \,\varphi(\vec r)\,e^{i\vec q\cdot\vec r}\,\int d\bm{\eta}_1...d\bm{\eta}_{A-1} 
  \,\,\times \nonumber\\
&\times&\, \psi^*_f(\bm{\eta}_1...\bm{\eta}_{A-1}; \sigma_1^z...\sigma_A^z;\tau_1^z...\tau_A^z)\,
e^{i\vec q\cdot\sqrt{\frac{A-1}{A}}\bm{\eta}_1}\,\frac{1+\tau_1^3}{2} \,\psi_i(\bm{\eta}_1...\bm{\eta}_{A-1}; 
 \sigma_1^z...\sigma_A^z;\tau_1^z...\tau_A^z)\,.
\end{eqnarray}
One can notice that in the matrix element one has a factorization in two terms: the Fourier transform of the field 
\begin{equation}
 \varphi(\vec q)=\int d\vec r \,\varphi(\vec r)\,e^{i\vec q\cdot\vec r}
\end{equation}
and the Fourier transform of the so called {\it proton transition density}, defined as
\begin{equation}
 \rho_{i,f}^p(\bm{\eta}_1) =\int d\bm{\eta}_2...d\bm{\eta}_{A-1} 
 \psi^*_f(\bm{\eta}_1...\bm{\eta}_{A-1}; \sigma_1^z...\sigma_A^z;\tau_1^z...\tau_A^z)\,
\,\sum_j\frac{1+\tau_j^3}{2} \,\psi_i(\bm{\eta}_1...\bm{\eta}_{A-1}; 
 \sigma_1^z...\sigma_A^z;\tau_1^z...\tau_A^z)\,.
\end{equation}
Since $\sqrt{\frac{A-1}{A}}\bm{\eta}_1=\vec r_1-\vec R_{CM}\equiv \vec r'$ one can write  the final expression as
\begin{equation}
\langle f|H_{int}|i\rangle =  e \varphi(\vec q) \sqrt{\frac{A}{A-1}}\int d\vec r' \,e^{i\vec q\cdot\vec r'}\, \rho^p_{i,f}(\vec \r')\,.
\end{equation}

Some remarks are in order here: 

\begin{itemize}
 \item if $\psi_i=\psi_f=\psi_0$, namely the nucleus recoils, but does not 
excite, the cross section is called {\it elastic} and will be proportional to the square modulus of what is called
the {\it charge form factor}. This is the Fourier transform of the average charge distribution 
with respect to the center of mass $\rho_{0,0}^p(r')$; 
\item if $\psi_i\neq\psi_f$ the cross section is called {\it inelastic}
and one has the Fourier transform of the transition density $\rho^p_{i,f}(\vec \r')$ (called the {\it transition form factor});
\item the matrix element entering in the cross section is involving an integral 
of the wave functions. This means that in principle one does not need to know the whole detailed wave function, but only an 
integral of it (an infinite number of different wave functions will have the same integral!).
\end{itemize}

The previous derivation has been done for the case of electron scattering on a nucleus,
considering only the interaction between the charge density and the electromangnetic scalar field. 
Similar results are obtained in case of other kinds of interactions, e.g. 
between the nuclear current density and the vector  field or axial currents and axial  field for neutrino scattering.
What changes is the form of the transition form factors which will not result from the charge density operator, 
but from other kinds of operators. 

From what has been illustrated above  we can conclude what is the main ingredient of the cross section 
for  perturbative {\it inclusive} experiments. Such experiments are those where
the only experimentally controlled quantities are the energy $\omega$ and the momentum $\vec q$ transferred 
to the nucleus, while nothing is known on what has happened to it (in how many fragments it has possibly broken).
The cross section will be proportional to the so called {\it structure} or {\it response} function
$S(\vec q, \omega)$
\begin{equation}\label{S}
S(\vec q,\omega)= \sum_{n=0}^\infty |\langle n|G|0\rangle|^2 \delta(\omega-E_n+E_0)\,.
\end{equation}
The operator $G$ denotes the general operator that interacts with the field created by the probe.

Notice that in ~\ref{S}, because of the presence of the energy conserving $\delta$ function, we have been allowed to introduce 
the sum on all the eigenstates of the nuclear Hamiltonian. 
Of course, depending whether the energy transferred to the system corresponds to the discrete or the continuum 
part of the spectrum, that sum may be intended as an integral. Moreover it has to be noticed that for each energy $E_n$ in the continuum 
one has the degeneration given by the different possible ''asymptotic channels``, namely the possibility that such a many-body system 
breaks in many different sets of fragments at the same energy. 

In the following we will concentrate on the inelastic part of the response function, namely $S(\vec q,\omega)$ where the term
$n=0$ in the sum is excluded
 \begin{equation}
R(\vec q,\omega)= \sum_{n \neq 0}^\infty |\langle n|G|0\rangle|^2 \delta(\omega-E_n+E_0)\,.
\end{equation}
The reason is that the elastic part requires only the knowledge of the ground state of the system and its calculation 
can be done with one of the methods described above.
Here, instead we want to face the inelastic scattering problem, and in particular the situation when the energy is large 
enough to break the system and continuum states $|n\rangle$ are in the game.  
To this aim we  define a fluctuation operator $\Theta= G - \langle0|G|0\rangle$. Since $\langle0|\Theta|0\rangle=0$ one can write
 \begin{equation}
R(\vec q,\omega)= \sum_{n = 0}^\infty\!\!\!\!\!\!\!\!\!\!\!\int\, |\langle n|\Theta|0\rangle|^2 \delta(\omega-E_n+E_0)\,,
\end{equation}
where the complete sum (or the integral) on all states has been recovered. Having a complete sum on the Hamiltonian eigenstates 
is crucial as it will be clear below. 

\section{Integral Transform Approaches}\label{sec:ITA}

Integral transform  approaches allow to calculate inclusive response functions, 
avoiding to calculate the wave functions in the  continuum.
One starts from the consideration, already mentioned above, that 
the amount of information contained in the  wave function is  
redundant with respect to the transition matrix elements needed, since the latter involve their integrals.
Therefore, one tries to avoid 
the difficult task of solving the Schr\"odinger
equation for positive energies and one concentrates instead on $R(\vec q,\omega)$, directly. 


An integral transform of $R(\omega)$ (here and in the following we drop the dependence on $\vec q$) is defined as
\begin{equation}\label{phisigma1}
 \Phi(\sigma)=\int\, K(\sigma,\omega)\,R(\omega) \,d\omega\,,
\end{equation}
with a smooth kernel $K$. Performing the integral in $d\omega$ and applying 
the closure property of Hamiltonian eigenstates 
$\sum_{n=0}^\infty |n\rangle\langle n|=I$  (see the remark at the end of Section~\ref{sec:RPP}) one has
\begin{equation}\label{phisigma2}
\Phi(\sigma)=\langle 0|\Theta^\dagger\hat{K}\left(\sigma,(\hat{H}-E_0)\right)\,\Theta|0\rangle\,.
\end{equation}
From Eq. (\re{phisigma2}) one can see that the calculation of $\Phi(\sigma)$ seems to require in principle the knowledge of the ground state 
only. However, the possibility to actually calculate $\Phi(\sigma)$ depends on how complicate is the operator $\hat{K}(\sigma,\hat{H})$. 
Moreover, in order to access the quantity of interest, namely $R(\omega)$ an inversion of the transform is necessary.

\subsection{Sum Rules}\label{SR}

Before proceeding towards discussing useful kernels for our aim, let us recall that the so called '' method of moments`` 
has something to do with this approach. The moments of $R(\omega)$, seen as a probability distribution, would be the 
$\Phi(\sigma)$ obtained by the kernel $ K(\sigma,\omega)=\omega^\sigma$ with $\sigma$ integer (they are also known as {\it sum rules}). 
\begin{equation}\label{sumrule}
\Phi(\sigma)\equiv m_\sigma=\int \, d\omega\, \omega^\sigma R(\omega) = \langle 0|\Theta^\dagger (H-E_0)^\sigma\,\Theta|0\rangle\,.
\end{equation}
For integer positive $\sigma$ the moments $m(\sigma)$ can also be written as mean values on the ground state of multiple commutators or 
anticommutators of the Hamiltonian with the $\Theta $ operator (see e.g~\cite{OTrep}). 
As already stressed they can be evaluated using the knowledge of 
the ground state only. The knowledge of only few moments can give information about some gross features of $R(\omega)$, like
its normalization ($m_0$) the mean $\omega$ value $(m_1/m_0)$ etc. However, the real problem 
of such an approach is that the very detailed knowledge of $R(\omega)$ would require the calculation of a large number 
of $m_\sigma$, arising then the problem of their existence, since the high energy behavior of $R(\omega)$ is in general 
unknown. Cumulant expansion approaches~\cite{Rosenf:1980} have been suggested in order to get a lot of information on $R(\omega)$ 
by the knowledge of only the first few $m_\sigma$, however with limited success.

\subsection{Integral Transform with the Laplace Kernel}\label{sec:LAPLACE} 
\begin{figure}
\sidecaption
% Use the relevant command for your figure-insertion program
% to insert the figure file.{\pi}
% For example, with the option graphics use
\includegraphics[scale=.65]{Chapter7-figures/fig3.eps}
%
% If not, use
%\picplace{5cm}{2cm} % Give the correct figure height and width in cm
%
\caption{Left panel: two different $R(\omega)$. Right panel: their relative Laplace transforms $\Lambda(\sigma)$. 
See discussion in the text.}
\label{fig:3}       % Give a unique label
\end{figure}
There is a kernel that is used extensively in many different contexts. This is the Laplace kernel $e^{\,-\sigma\,\omega}$. 
Before coming to that let's  first  consider  the Fourier transform of  $R(\omega)$ 
\begin{equation}
F(t)=\int\, d\omega\, e^{\,i\, \omega\, t}\, R(\omega)\,,
\end{equation}
one can again make use of closure and write it as  
\begin{equation}
F(t)=\langle 0|\,\Theta^\dagger\, e^{\,i\, (H-E_0)\, t} \,\Theta\,|0\rangle\,,
\end{equation}
or, in Heisemberg representation  
\begin{equation}
F(t)=\langle 0|\,\Theta^\dagger(t) \,\Theta(0)\,|0\rangle\,.
\end{equation}
This is a complex quantity of the real time variable $t$. However, if one extends its domain in the complex plane, one realizes 
that $F^*(t)=F(-t^*)$ and for imaginary time, i.e. $t=i \tau$,  the function $F(\tau)$ is real. Since $R(\omega) = 0$ for $\omega\leq 0$,
$F(\tau)$  corresponds to its Laplace transform. It turns out that GFMC or DMC methods are suitable
to calculate $F(\tau)$. However, one is left with the thorny problem of the inversion of the Laplace transform.

It is well known that the inversion of a Laplace transform is a typical {\it ill posed problem}. In order to explain in simpler terms 
what an {\it ill posed problem}  means in practice, consider two examples for $R(\omega)$ as plotted in the left panel of~\ref{fig:3}. 
Suppose that the dashed curve
is the real one and the full line a wrong one. If you look at their relative Laplace transforms $\Lambda(\sigma)$ in the right panel, 
you will notice
that they can fall both within a possible numerical error of the size shown in the figure, so that no inversion algorithm 
can discriminate with absolute certitude between the right and wrong $R(\omega)$.
This is due to the fact that the exponential kernel tends to wash out rapidly any information at high $\omega$.
Moreover errors in the transform can generate oscillations (see discussion in Section~\ref{sec:INV}). 

\subsection{Integral Transform with the Lorentzian kernel}\label{sec:LIT}

One may ask what would be the ''perfect'' kernel, in the sense of a kernel that returns a transform which reproduces the features of
$R(\omega)$. It would trivially be $\delta(\sigma-\omega)$. This is of course of no use. However, probably a representation 
of the $\delta$-function could be an ``almost perfect'' kernel. This is the idea that was pursued in~ \cite{ELO94}, 
when the Lorentz Integral Transform (LIT) was proposed. It is clear that a good kernel has not only to be able to reproduce 
 the features of the original $R(\omega)$ in the transformed function, in order to make the inversion procedure easier and reliable, 
but it has also to allow the calculation
of $\Phi(\sigma)$ in practice. In the following it will be illustrated that if the kernel is a Lorentzian 
function (one of the representations of the $\delta$-function) this is just the case.

The (normalized) Lorentzian kernel can be defined for a complex $\sigma=\sigma_R+i\sigma_I$ as 
\begin{equation}
 K_L(\sigma,\omega)\,=\, \frac{\sigma_I}{\pi}\,\frac{1}{(\omega-\sigma_R)^2+\sigma_I^2}=
 \frac{\sigma_I}{\pi}\,\frac{1}{\omega-\sigma_R-i\sigma_I}\,\,\frac{1}{\omega-\sigma_R+i\sigma_I}\,.
\end{equation}
With such a kernel Eq.(\ref{phisigma1}) becomes
\begin{equation}\label{Lsr}
L(\sigma_R,\sigma_I)= \frac{\sigma_I}{\pi}\,\langle\,0\,|\,\Theta^\dagger \,
\frac{1}{\hat H -E_0 -\sigma_R-i\sigma_I}\,\,\frac{1}{\hat H -E_0 -\sigma_R+i\sigma_I}\,\Theta|\,0\,\rangle\,.
\end{equation}
If one defines
\begin{equation}\label{psitilde}
|\tilde\Psi\rangle\equiv\sqrt{\frac{\sigma_I}{\pi}}\,\frac{1}{\hat H -E_0-\sigma_R+i\sigma_I}\,\Theta|\,0\,\rangle\,,
\end{equation}
the LIT of $R(\omega)$ corresponds to the norm of this $|\tilde\Psi\rangle$, namely
\begin{equation}\label{norm}
L(\sigma_R,\sigma_I)=\langle\tilde\Psi |\tilde\Psi\rangle\,.
\end{equation}
The most important feature of $|\tilde\Psi\rangle$ is that its norm is finite since 
\begin{equation}
L(\sigma_R,\sigma_I)=\frac{\sigma_I}{\pi}\,\int\,\frac{1}{(\omega-\sigma_R)^2+\sigma_I^2}\,\,R(\omega) \,d\omega\,<\,\infty
\la{phiL}
\end{equation}
This means that $|\tilde\Psi\rangle$ has bound-state like asymptotic behavior and therefore it can be calculated using one or more 
of the methods described in Section~\ref{sec:CLASS}. 

In the literature there are examples of LIT calculations  with 
 the Faddeev method for inclusive electron scattering\cite{Sara95} and photoabsorption on $^3$He and $^3$H~\cite{Golak_Benchmark2002}, 
 the CHH method, again for electron scattering~\cite{long3H_2004} and photoabsorption~\cite{accuracy_2006}
 of three-nucleon systems,
 the NCSM method for  photoabsorption of $^4$He~\cite{Stetcu_2005}, 
 the EIHH method for  photoabsorption of $^4$He~\cite{Gazit_2006}, $^6$He and $^6$Li~\cite{He6_2004}, $^7$Li~\cite{Li7_2004}, 
for neutrino scattering  on $^4$He~\cite{neutrino_2007} and more recently with CC on photoabsorption of $^{16}$O and 
 $^{22}$O~\cite{CC_O16}. 

Now we will illustrate how the Lanczos method can be particularly useful  to calculate $L(\sigma_R,\sigma_I)$ in practice.
One easily realizes that, because of Eqs.~(\ref{Lsr})-(\ref{phiL}), one has 
\begin{equation}
\langle\tilde\Psi |\tilde\Psi\rangle= \frac{1}{\pi}\,Im\left[\, \langle \,0\,|\,\Theta^\dagger \,
\frac{1}{\hat H - E_0-\sigma_R-i\sigma_I}\,\Theta|\,0\,\rangle\,\right]\,.
\end{equation}
Due to the finiteness of the norm of $|\tilde\Psi\rangle$  one can represent the Hamiltonian on a 
basis of localized functions. After a Lanczos diagonalization, $L(\sigma_R,\sigma_I)$ will appear as 
\begin{equation}\label{convolution}
L(\sigma_R,\sigma_I)= \frac{1}{\pi}\,Im\left[\,\sum_\mu \langle \,0\,|\,\Theta^\dagger \,
\frac{1}{\epsilon_\mu - E_0-\sigma_R-i\sigma_I}\,\Theta|\,0\,\rangle\,\right]=
\frac{\sigma_I}{\pi}\sum_\mu \frac{|\langle\, \mu\,|\Theta|\,0\,\rangle|^2}{(\epsilon_\mu-E_0-\sigma_R)^2+\sigma_I^2}\,,
\end{equation}
namely a superposition
of Lorentzians functions of width $\sigma_I$, centered on $\epsilon_\mu-E_0$ ($\epsilon_\mu$ is the Hamiltonian matrix eigenvalue)
and weighted by
the strength of the transition between the ground state and the Hamiltonian matrix eigenvectors $|\mu\rangle$, induced by $\Theta$.
In principle the ``exact result'' is reached when the Hamiltonian matrix is large enough to let $L(\sigma_R,\sigma_I)$ become 
a converged curve. 

A few remarks are in order here: 

\begin{itemize}
 \item the range of $\sigma_R$, where it is convenient to calculate $L(\sigma_R,\sigma_I)$, is connected 
to the $\omega$-range of interest for $R(\omega)$. In fact one should remember that the kernel is a representation of the 
$\delta$-function and $L(\sigma_R,\sigma_I)$ resembles $R(\omega)$ more and more as $\sigma_I$ becomes small; 
 \item the choice of $\sigma_I$
is determined by the kind of resolution that one wants to have on $R(\omega)$. If $\sigma_I$ 
is of the same order as the experimental resolution one does not even need to invert the transform and one can compare 
$L(\sigma_R,\sigma_I)$ to data, directly; 
  \item even in case that an inversion is necessary, $\sigma_I$ is related to the  
the kind of resolution one wishes to have for $R(\omega)$. A crucial quantity to look 
at is the average distance between contiguous $\epsilon_\mu$.
In fact  for  $\sigma_I$ larger than this  average distance the convergence 
of  $L(\sigma_R,\sigma_I)$ is rather easy to reach, since in this case  the different Lorentzians that compose $L(\sigma_R,\sigma_I)$
(see Eq.~\ref{convolution}) overlap smoothly. 
The inversion of a 
well converged result is safe. However, if the inverted result shows oscillations with wavelength smaller that $\sigma_I$, one has to
check the result, calculating the LIT up to convergence with an as small $\sigma_I$ (see discussion in Section~\ref{sec:INV}). 
\item when $\sigma_I$ is smaller 
than the average distance between contiguous $\epsilon_\nu$, $L(\sigma_R,\sigma_I)$ will appear as a set of separated peaks,
which in most cases move around as the matrix increases. In fact enlarging the matrix does not necessarily mean that the density of 
$\epsilon_\mu$ increases in the $\sigma_R$  region of interest. Only if this is the case one could reach a convergence 
in $L(\sigma_R,\sigma_I)$. The choice of an appropriate basis in representing the Hamiltonian plays an essential role for 
this problem, as it was shown in~\cite{Leidemann_2015}.
\end{itemize}

An extensive review of the LIT method, together with more examples of applications can be found in~\cite{report_2007}.

\subsection{Integral Transform with the Sumudu Kernel}\label{sec:SUMUDU}

One of the advantages of the LIT method is that the kernel is a representation of the $\delta$-function. This  makes 
the transform resemble
$R(\omega)$, reducing in this way the difficulties in the inversion considerably. On the other hand, 
the necessity to diagonalize, up to convergence of the transform, 
an Hamiltonian matrix built on a many-body basis, seems to limit its applicability to rather small $A$. One could ask whether it is possible 
to find a kernel that is both a representation of the $\delta$-function and that allows to calculate the transform with one of the 
Monte Carlo methods, which are rather powerful also for a rather large number of particles. 
Such a  kernel has been proposed in Ref.~\cite{Roggero_2013}. 
It is a combination of Sumudu kernels, (or more simply of exponentials) and reads

\begin{equation}
%\label{eq:NEW4}
%K_P(\sigma,\omega)=({2}^{-\frac{\omega}{\sigma}}-{4}^{-\frac{\omega}{\sigma}})^P.
%\end{equation}
\label{eq:NEW4}
K_P(\sigma,\omega)=N\left[ \frac{e^{-\mu\frac{\omega}{\sigma}}}{\sigma}-\frac{e^{-\nu\frac{\omega}{\sigma}}}{\sigma} \right]^P\,,
\end{equation}
where 
\begin{equation}
\mu=\frac{ln[b]-ln[a]}{b-a}a\,;\,\,\,\,\,\,\,
\nu=\frac{ln[b]-ln[a]}{b-a}b\,,
\end{equation}
and the parameters $P,a,b$ are integer numbers  with $b>a$. The normalization constant $N$ is a function of $P,a,b$
such that $\int d\sigma K_P(\sigma,\omega)=1$.

Independent on the choice of $a$ and $b$ the kernel $K_P(\sigma,\omega)$ converges to 
$\delta(\omega-\sigma)$ in the $P \to \infty$ limit. 
For a finite $P$, at each value of $\omega$ the kernel has a 
finite width which depends on $P$ and  represents the {\it resolution} with which one can study
$R(\omega)$. The maximum of $\sigma K_P(\sigma,\omega)$ is at $\omega=\sigma$, 
therefore one can choose both the energy range 
of interest (the $\sigma$ values) and the resolution (the larger is $P$, the  higher is the resolution). 
This makes the transform with such a kernel  extremely flexible, similarly to the case of the LIT method.

Using the binomial expansion the kernel becomes 
a linear combination of the so-called Sumudu transform kernels~\cite{Roggero_2013}
\begin{equation}
\label{nsum_kernel}
\begin{split}
K_P(\sigma,\omega)& =\frac{N}{\sigma} \sum^{P}_{k=0} {
\begin{pmatrix}
P\\
k
\end{pmatrix}
} (-1)^k e^{-\ln(b/a)[\frac{a}{b-a}P+k]\frac{\omega}{\sigma}}. 
\end{split}
\end{equation}
This expression makes it clear how to calculate the transform by  quantum Monte Carlo.   
In fact by operating the usual substitution $\omega \to \hat{H}$ (see Eqs.~(\ref{phisigma1})-(\ref{phisigma2})), one has  a simple 
linear combination of imaginary-time propagators 
\begin{equation}\label{deltalimit}
K_P(\sigma,\hat{H}) =\frac{N}{\sigma} \sum^{P}_{k=0} {
\begin{pmatrix}
P\\
k
\end{pmatrix}
} (-1)^k e^{-\tau_{Pk}\hat{H}},
\end{equation}
taken at different imaginary-time points
\begin{equation} 
\tau_{Pk}=\ln(b/a)[\frac{a}{b-a}P+k]/\sigma.
\end{equation}

Until now this transform has only been applied to the case of bosons (liquid Helium)~\cite{Roggero_2013}. It would be desirable
to investigate its  application for nuclear systems.


 
\subsection{Integral Transform  with the Stieltjes Kernel}\label{sec:STIELTJES}

The integral transform with the Stieltjes kernel, $K(\sigma,\omega)=\frac{1}{\omega+\sigma}$ and $\sigma>0$, is given by
\begin{equation}\label{Stieltjes}
 {\cal S}(\sigma)= \int\,d\omega \,\frac{1}{(\omega+\sigma)} R(\omega)=\langle\,0\,|\,\Theta^\dagger\, 
 \frac{1}{\hat H -E_0+\sigma}\,\Theta|\,0\,\rangle\,.
\end{equation}
It was introduced long ago~\cite{Efros85} to study reactions in the continuum. It has  
had, however, a limited application, since it shares the same problems as the Laplace transform. In fact  
the kernel has a similar decreasing behavior in $\sigma$, even if much smoother than the exponential, washing out informations at higher
energies, useful for the success of the inversion algorithms.
In~\cite{Stieltjes_1993} the test on the electron scattering  longitudinal response function has shown that its inversion
can lead to rather large uncertainties, even in presence of rather small errors in the calculation of the transform.

There is, however, an interesting application of such a transform to calculate a dynamical quantity, which in general requires  the 
knowledge of the continuum part of the spectrum. This quantity is the {\it generalized} polarizability (e.g. dipole polarizability
or magnetic susceptibility)  of a nucleus in response to
a constant perturbative field.  

Applying time dependent perturbation theory one can show that these polarizabilities $\alpha_\Theta$
are related to the inverse moment $m_{-1}$ of $R(\omega)$ as
\begin{equation}\label{alpha}
\alpha_\Theta=2 m_{-1}= 2 \langle\,0\,|\,\Theta^\dagger\, \frac{1}{\hat H -E_0}\,\Theta|\,0\,\rangle\,,
\end{equation}
where $\Theta$ represents the operator relevant for the kind of probe used.
Comparing Eqs.~(\ref{Stieltjes}) and ~(\ref{alpha}) it is clear that the polarizability corresponds to the limit for vanishing $\sigma$
  of  ${\cal S}(\sigma)$. It turns out that this limit represents a viable way to calculate the polarizability. In fact, one notices that
the polarizability can be written as
\begin{equation}
\alpha_\Theta=  2 \langle\,0\,|\,\Theta^\dagger\, |\,\tilde\Phi\,\rangle 
\end{equation}
where $|\,\tilde\Phi\,\rangle$  is the solution of the Schr\"odinger-like  equation  with a source
\begin{equation}\label{ss}
({\hat H -E_0+\sigma})|\tilde\Phi\rangle\,=\,\Theta\,|0\,\rangle
\end{equation} 
One notices that $\tilde\Phi$ is  a function of finite norm, since
\begin{equation}
\langle\,\tilde\Phi|\,\tilde\Phi\,\rangle= \int\,d\omega \,\frac{1}{(\omega+\sigma)^2} R(\omega)\,<\,\infty.
\end{equation}
Therefore  representing the Hamiltonian and $\tilde\Phi$  on bound state-like functions,  equation~\ref{ss} becomes a linear 
matrix equation
to be solved up to convergence in the size of the matrix. Repeating the calculation  for smaller and smaller $\sigma$ 
one can plot $\cal S(\sigma)$ as a function of $\sigma$ and read $\alpha_\Theta$ as the extrapolated value at $\sigma=0$.
A first application of this procedure for the calculation of the dipole polarizability of light system can be found in~\cite{Miorelli_2016}.

\subsection{Methods of Inversion}\label{sec:INV}

A crucial part of the  integral transform method  is the inversion of the transforms. 
The inversion  has to be made with care, since errors in the transform can generate spurious oscillations in $R(\omega)$. 
To illustrate this let us consider a well defined $R(\omega)$ to which we add a
term $\Delta_\nu R(\omega)$ oscillating at a high frequency $\nu$. Such a term leads 
to an additional $\Delta_\nu \Phi (\sigma)$ in the transform of Eq.~(\ref{phisigma1}). One should realize that, for any
amplitude of the oscillation, $\Delta_\nu \Phi$ decreases with increasing $\nu$.
This means that for some value of $\nu$ the term $\Delta_\nu \Phi$ may be 
 smaller than the size of the errors in the
calculation. Therefore in this case  $\Delta_\nu R$ cannot be discriminated. In principle, by reducing
the error in the calculation one can push the frequency of the undiscriminated $\Delta_\nu R$ 
to higher and higher values, so to render their spuriosity manifest. 

At this point something should be said  regarding the  algorithms which are commonly used to invert  the transforms. 
The Laplace transforms are obtained by
Monte Carlo methods and are affected by statistical errors. In this case the inversion algorithms are necessarily based on 
Bayes theorem. Therefore,  one gets  
the probability of a solution  based on known input conditions. Because of the highly ill-posed nature of the Laplace transform 
inversion it may be necessary to use many input conditions to obtain a highly probable solution.
The best known algorithm  is the Maximum Enthropy method\cite{MEM_book} with its many  sophisticated versions. 

The calculation of the LIT is much less affected by numerical errors of statistical nature (they are generally tiny) 
and much more by the systematic errors related to  estimations of  the convergence quality. Therefore in this case it is more 
convenient to use another algorithm  called the {\it regularization method}~\cite{TIKHONOV:1977}.
This has led to very safe inversion results. Alternative inversion methods of the same nature are discussed in~\cite{ANDREASI:2005}. 

A  LIT inversion method that has been largely used with success  consists in making the following ansatz for
the response function
\begin{equation}
R(\omega') = \sum_{n=1}^{N_{max}} c_n \chi_n(\omega',\alpha_i) \,,
\end{equation}
with $\omega'=\omega-E_{th}$, where $E_{th}$ is the break--up threshold energy  
into the continuum (calculable with bound state methods). The $\chi_n$ are given functions with nonlinear parameters $\alpha_i$.
A basis set that has rather frequently been used to invert the   LIT inversions is
\begin{equation}
\label{bset}
\chi_n(\omega',\alpha_i) = \omega'^{\alpha_1} \exp(- {\frac {\alpha_2 \omega'} {n}}) \,.
\end{equation}
In addition also possible information on narrow levels
could be incorporated easily into the set $\chi_n$.
Substituting such an expansion into the right hand side of~(\ref{phiL})  one obtains (here too 
the $\sigma_I$ dependence is omitted)
\begin{equation}\label{sumphi}
L(\sigma_R) =
\sum_{n=1}^{N_{max}} c_n \tilde\chi_n(\sigma_R,\alpha_i) \,,
\end{equation}
where
\begin{equation}\label{chi}
\tilde\chi_n(\sigma_R,\alpha_i) =
\int_0^\infty {\frac {\chi_n(\omega',\alpha_i)} {(\omega'-\sigma_R)^2 + \sigma_I^2}}
\,\,.
\end{equation}
For given $\alpha_i$ the linear parameters $c_n$ are determined from a least--square best fit of
$L(\sigma_R)$ of equation~(\ref{sumphi}) to the calculated
$L(\sigma_R)$ for a number of $\sigma_R$ points
much larger than $N_{max}$.

For every value of $N_{max}$ the overall best fit is 
selected and then the procedure is repeated for $N'_{max}=N_{max}+1$ till
a stability of the inverted response is obtained and taken as inversion
result. A further increase of $N_{max}$ will eventually reach a point, where the
inversion becomes unstable leading typically to random oscillations. The
reason is that $L(\sigma_R)$ of equation~(\ref{convolution}) is not determined
precisely enough so that a randomly oscillating $R(\omega)$ leads to a better
fit than the true response. If the accuracy in the determination of $L(\sigma_R)$
is increased, one may include more basis functions in the expansion
(\ref{sumphi}).

The LIT method has to be understood as an approach with a {\it controlled resolution}. If
one expects that $R(\omega)$ has structures of  width $\Gamma$, then the LIT resolution
parameter $\sigma_I$ should be similar in size. Then it is sufficient to determine the corresponding LIT 
with a moderately high precision, and the inversion should 
lead to reliable results for $R(\omega)$, if in fact no structures with a width smaller than $\Gamma$
are present. If, however, there is a reason to believe that $R(\omega)$ exhibits
such smaller structures one should
reduce $\sigma_I$ accordingly and perform again a calculation of the Lorentz integral transform  with
the same relative precision as before. Such a calculation is of course more expensive than
the previous one with larger $\sigma_I$, but in principle one can reduce
the LIT resolution parameter $\sigma_I$ more and more. 

The advantage of the LIT approach as compared with a conventional approach is evident. In the
LIT case one makes the calculation with the proper resolution, while in a
conventional calculation an infinite resolution (corresponding to $\sigma_I=0$) is requested, which
often makes such a calculation not feasible.

There are several tests of the reliability of the inversion. First of all, performing the calculation 
at different $\sigma_I$ one has to obtain the same stable result for $R(\omega)$ from the inversion. If $\sigma_I$
is too small the solution reaches its asymptotic behavior to  zero very slowly, 
therefore for $\sigma_I < \sigma_I^{min}$  one may have convergence problems, 
turning into large errors for the LIT. As already said above, this will show up in unphysical oscillations in $R(\omega)$.
This means that the stable result obtained with $\sigma_I \geq \sigma_I^{min}$ is the correct one, at that resolution. 

Another test is the control of the moments, in fact the moments of $R(\omega)$ can be calculating both integrating $R(\omega)$ or 
by Eq.~(\ref{sumrule}),  which needs  only the knowledge of the ground state.

\section{Conclusion}
From  what has been written in these short lecture notes it should be clear that they are only a partial presentation of 
the amount of work that the theoretical nuclear physicists have done in the last two  decades in the attempt to account 
for nuclear observables from first principles, for a number of particles that exceeds the classical few-body definition, 
traditionally limited to A=2,3. 

In describing the methods, here only the main points have been exposed, leaving out the complicated formalism of the more 
detailed presentation needed by possible practitioners. The reference to the numerous original works should 
compensate this lack.

Very active theoretical research is still going on in this field, especially in the attempt to find unifying approaches for structure and reactions 
and to reach regions of the nuclear data chart still unexplored. Fortunately this research is accompanied by an as active experimental 
activity, which produces observables that at the same time need a theoretical explanation and constitute the reference
for testing the models and the methods. All that shows the relevance of the ab initio approaches in nuclear physics as the necessary  bridge   between QCD, the fundamental theory 
of strong interaction and nuclear phenomena, many of which are at the basis of the evolution of the Universe. 

\begin{thebibliography}{00}
%[1]
\bibitem{EpM11} see e.g E. Epelbaum and U.-G. Mei{\ss}ner, Ann. Rev. Nucl. Part. Sci. {\bf 62}, 159-185 (2012)

%[2]        
\bibitem{WlO12} W. Leidemann and G. Orlandini,  Progr. Part.  Nucl. Phys. {\bf 68}, 158-214  (2013)

%[3]
\bibitem{FADDEEV:1961} L.D. Faddeev,
        Sov. Phys. JETP, {\bf 12}, 1014 (1961)
%[4]       
\bibitem{YAKUBOWSKY:1967} O. Yakubowsky,   
        Yad. Fiz., {\bf 5}, 1312 (1967) [{\it Sov. J. Nucl. Phys.} {\bf 5} 937] 
%[5]
\bibitem{AGS}  E.O. Alt, P. Grassberger, W. Sandhas, Nucl. Phys. B {\bf 2},  167 (1967) ;
               P. Grassberger, W. Sandhas, Nucl. Phys. B {\bf 2} 181(1967) 
%[6]
\bibitem{Ra870} J.M. Rayleigh, Phil. Trans. {\bf 161}, 77 (1870)

%[7]
\bibitem{Ri909} W. Ritz, 
   Journal f\"ur die Reine und Angewandte Mathematik, {\bf 135}, 1 (1909)
%[8]  
\bibitem{RGM1} K. Wildermuth, Y.C. Tang, \textit{A Unified Theory of the Nucleus}, Vieweg, Braunschweig, 1977;
                Y.C. Tang, M. Lemere, D.R. Thompson, Phys. Rep. {\bf 47} 169 (1978) 
%[9]
\bibitem{RGM2} H.M. Hofmann, G.M. Hale, Nucl. Phys. A {\bf 613}, 69 (1997) ;
             Phys. Rev. C {\bf 77} 044002 (2008) 
%[10] 
\bibitem{SVM1} V.I. Kukulin, V.M. Krasnopolsk, J. Phys. {\bf 3}  795 (1977)

%[11] 
\bibitem{SVM2}  Y. Suzuki, K. Varga, \textit{Stochastic Variational Approach to Quantum Mechnical Few-Body Problems},
                  Springer-Verlag, Berlin, 1998
                  
%[12]                  
\bibitem{bench_2001}  H. Kamada {\it et al.} Phys. Rev. C {\bf 64}, 044001 (2001)       

%[13]                
\bibitem{HiD56} J.O. Hirschfelder and J. Dahler, Proc. Natl. Acad. Sci. USA {\bf 42}, 363 (1956)                                 

%[14]  
\bibitem{NOVOSELSKY:1994} A. Novoselsky and J.  Katriel,  
	 Phys. Rev A {\bf 49},  833  (1994)
%[15]
\bibitem{BARNEA:1997+8}  N. Barnea  and A. Novoselsky,
	 Ann. Phys. (N.Y.)  {\bf 256}, 192, (1997); Phys. Rev.  A {\bf 57},  48 (1998)
%[16]	 
\bibitem{Gatto:2011}  M. Gattobigio, A. Kievsky, and M. Viviani, Phys. Rev. C {\bf 83}, 024001 (2011)

%[17]
\bibitem{Deflorian:2013}	 
	 S. Deflorian, N. Barnea, W. Leidemann, G. Orlandini, Few-Body Syst. {\bf 54}, 1879 (2013) 
%[18]	 
\bibitem{Ok54} S. Okubo, \PRO, {\bf 12}, 603 (1954)

%[19]
\bibitem{CoK60} F. Coester and H. K{\" u}mmel, Nucl. Phys.  {\bf 17}, 477 (1960)

%[20]
\bibitem{DaS64} J. da Providencia and C.M. Shakin, \ANNP, {\bf 30}, 95 (1964) 

%[21]
\bibitem{SuL80} K. Suzuki and S.Y. Lee, \PRO, {\bf 64},  2091  (1980)

%[22]
\bibitem{Zab:1978} H. K\"ummel, K.H. L\"uhrmann, J.G. Zabolitzky, Phys. Rep. {\bf 36},1 (1978) 

%[23]
\bibitem{HJ:2004} D.J. Dean, M. Hjorth-Jensen, Phys. Rev. C  {\bf 69}, 054320 (2004) 

%[24]
\bibitem{Ka62} M.H. Kalos,  \PREV,  {\bf 128},  1791 (1962) 

%[25]
\bibitem{Ca87} J. Carlson,  \PRC,  {\bf 36},  2026   (1987) 

%[26]
\bibitem{ScF99} K. E. Schmidt and S. Fantoni,  \PLB,  {\bf 446},  99  (1999) 

%[27]
\bibitem{Le09} D. Lee,   Prog. Part. Nucl. Phys., {\bf 64}, 117 (2009)

%[28]
\bibitem{KoD97} S.E. Koonin, D.J. Dean, and K. Langanke,  \PREP, {\bf 278}, 1 (1997)

%[29]
\bibitem{HoM95} M. Honma, T. Mizusaki, and T. Otsuka,  \PRL,  {\bf 75}, 1284 (1995)

%[30]
\bibitem{OtH01} T. Otsuka et al.,  Prog. Part. Nucl. Phys., {\bf 47},  319 (2001)

%[31]
\bibitem{PiW01} S. C. Pieper and R. B. Wiringa, Ann. Rev. Nucl. Part. Sci., {\bf 51}, 53 (2001)

%[32]
\bibitem{ShF74} A. de Shalit and H. Feshbach, {\it Theoretical Nuclear Physics: Nuclear Structure}, 
               John Wiley \& Sons Inc, 1974                
%[33]               
\bibitem{NAV:1998} P. Navratil, B. R. Barrett, Phys.Rev.C {\bf 57}, 562 (1998) 

%[34]
\bibitem{BaL00}N. Barnea , W.Leidemann, G. Orlandini, Phys.Rev. C {\bf 61}, 054001 (2000)

%[35]
\bibitem{OTrep} G. Orlandini and M. Traini, Rep. Prog. Phys. {\bf 54} 257 (1991) 

%[36]
\bibitem{Rosenf:1980} R.Rosenfelder, Ann. Phys. {\bf 128}, 188 (1980)

%[37]
\bibitem{ELO94} V.D. Efros, W. Leidemann  and G. Orlandini, Phys. Lett. B, {\bf 338}, 130 (1994)

%[38]
\bibitem{Sara95} S. Martinelli et al., Phys. Rev. C {\bf 52} (1995) 1778.

%[39]
\bibitem{Golak_Benchmark2002} J. Golak, R. Skibinski, W. Gl\"ockle, H. Kamada, A. Nogga, 
H. Witala, V. D. Efros, W. Leidemann, G. Orlandini, and E.L. Tomusiak, 
Nucl. Phys. {\bf A707}, 365 (2002)
  
%[40]   
\bibitem{long3H_2004} V. D. Efros, W. Leidemann, G. Orlandini, E. L. Tomusiak,  Phys.Rev. C {\bf  69},  044001 (2004)

%[41]  
\bibitem{accuracy_2006} N. Barnea, W. Leidemann, G. Orlandini, V. D. Efros, E. L. Tomusiak, Few-Body Syst. C {\bf 39}, 1 (2006)

%[42]
\bibitem{Stetcu_2005} I. Stetcu, B. R. Barrett, P. Navr\'atil, and J. P. Vary, Phys. Rev. C \textbf{71}, 044325 (2005)

%[43]
\bibitem{Gazit_2006} D. Gazit, S. Bacca, N. Barnea, W. Leidemann, G. OD.J. Dean, M. Hjorth-Jensen, Phys. Rev. C 69 (2004) 054320.rlandini, Phys. Rev. Lett.{\bf  96}, 112301 (2006) 

%[44]
\bibitem{He6_2004} S. Bacca, N. Barnea, W. Leidemann, G. Orlandini, Phys. Rev. C {\bf 69}, 057001 (2004) 

%[45]
\bibitem{Li7_2004} S. Bacca, H. Arenhoevel, N. Barnea, W. Leidemann, G. Orlandini,  Phys.Lett. B {\bf 603}, 159 (2004)

%[46]
\bibitem{neutrino_2007}D. Gazit, N. Barnea, Nucl. Phys. A {\bf 790}, 3D.J. Dean, M. Hjorth-Jensen, Phys. Rev. C 69 (2004) 054320.56 (2007) 

%[47]
\bibitem{CC_O16} S. Bacca, N. Barnea, G. Hagen, M. Miorelli, G. Orlandini, T. Papenbrock, Phys. Rev. C {\bf 90}, 064619 (2014)

%[48]
\bibitem{Leidemann_2015} W. Leidemann, Phys. Rev. C {\bf 91}, 054001 (2015) 

%[49]
\bibitem{report_2007} V. D. Efros, W. Leidemann, G. Orlandini and N. Barnea,
J. Phys. G: Nucl. Part. Phys. {\bf 34}, R459 (2007)  

%[50]
\bibitem{Roggero_2013} A. Roggero, F. Pederiva, and G. Orlandini, Phys. Rev. B {\bf 88}, 094302 (2013)

%[51]
\bibitem{Efros85} V.D. Efros, Sov. J. Nucl. Phys. {\bf 41}, 949 (1985); 
V.D. Efros, W. Leidemann and {\bf G. Orlandini}

%[52]
\bibitem{Stieltjes_1993} V.D. Efros,  W. Leidemann, G. Orlandini, Few-Body Sys. {\bf 14}, 151 (1993)

%[53]
\bibitem{Miorelli_2016} M. Miorelli, S. Bacca, N. Barnea, G. Hagen, G. R. Jansen, G. Orlandini, T. Papenbrock, arXiv:1604.05381
 
%[54]
\bibitem{MEM_book}  E. T. Jaynes, {\it Information Theory and Statistical Mechanics}, in “Statistical Physics“, Ed. K.
Ford New York: Benjamin, p. 181, 1963.

%[55]
\bibitem{TIKHONOV:1977} A.N. Tikhonov, V.Y. Arsenin, {\it Solutions of ill--posed Problems}, Winston, 1977

%[56]
\bibitem{ANDREASI:2005}
D. Andreasi, W. Leidemann, C. Reiss, and M. Schwamb, Eur. Phys. J. A {\bf 24}, 361 (2005)

\end{thebibliography}



\label{chap:chapter7}
\newcommand{\element}[3]{\bra{#1}#2\ket{#3}}
\newcommand{\normord}[1]{\left\{#1\right\}}
\newcommand*{\kpr}[1]{\left\{#1\right\}}
\newcommand*{\fpr}[1]{\left[#1\right]}
\newcommand*{\for}[3]{\langle#1|#2|#3\rangle} 

\title{Computational Nuclear Physics and Post Hartree-Fock Methods}
\author{Justin G.~Lietz, Samuel Novario, Gustav R.~Jansen, Gaute
  Hagen, and Morten Hjorth-Jensen,} \institute{Justin G.~Lietz \at
  Department of Physics and Astronomy and National Superconducting
  Cyclotron Laboratory, Michigan State University, East Lansing,
  Michigan, USA, \email{lietz@nscl.msu.edu}, \and Samuel Novario \at
  Department of Physics and Astronomy and National Superconducting
  Cyclotron Laboratory, Michigan State University, East Lansing,
  Michigan, USA, \email{novarios@nscl.msu.edu}, \and Gustav R.~Jansen
  \at National Center for Computational Sciences and Physics Division, Oak Ridge National Laboratory, Oak Ridge,
  Tennessee, USA, \email{jansen@ornl.gov},
  \and Gaute Hagen \at Physics Division, Oak Ridge National Laboratory, Oak Ridge, Tennessee, USA and Department of Physics and
  Astronomy, University of Tennessee, Knoxville, Tennessee, USA,
  \email{hageng@ornl.gov}, \and Morten Hjorth-Jensen \at Department of
  Physics and Astronomy and National Superconducting Cyclotron
  Laboratory, Michigan State University, East Lansing, Michigan, USA
  and Department of Physics, University of Oslo, Oslo, Norway,
\email{hjensen@msu.edu}}

\maketitle 
\abstract{We present a computational approach to infinite
  nuclear matter employing Hartree-Fock theory, many-body perturbation
  theory and coupled cluster theory. These lectures are closely linked
  with those in chapters 9, 10 and 11 and serve as input for the
  correlation functions employed in Monte Carlo calculations in
  chapter 9, the in-medium similarity renormalization group theory of
  dense fermionic systems of chapter 10 and the Green's function
  approach in chapter 11.  We provide extensive code examples and
  benchmark calculations, allowing thereby an eventual reader to start
  writing her/his own codes. We start with an object-oriented serial
  code and end with in-depth discussions on strategies for porting the
  code to present and planned high-performance computing facilities. }


%\maketile

\section{Introduction}\label{sec:chap8intro}


Studies of dense baryonic matter are of central importance to our
basic understanding of the stability of nuclear matter, spanning from
matter at high densities and temperatures to matter as found within
dense astronomical objects like neutron stars.  An object like a
neutron star offers an intriguing interplay between nuclear processes
and astrophysical observables, spanning many orders of magnitude in
density and several possible compositions of matter, from the crust of
the star to a possible quark matter phase in its interior, see for
example
Refs.~\cite{lattimer2007,lattimer2012,weber1999,hh2000}
for discussions.  A central issue in studies of infinite nuclear
matter is the determination of the equation of state (EoS), which can
in turn be used to determine properties like the mass range, the
mass-radius relationship, the thickness of the crust and the rate by
which a neutron star cools down over time. The EoS is also an
important ingredient in studies of the energy release in supernova
explosions.

The determination and our understanding of the EoS for dense nuclear
matter is intimately linked with our capability to solve the nuclear
many-body problem. In particular, to be able to provide precise
constraints on the role of correlations beyond the mean field, is
crucial for improved and controlled calculations of the EoS of
nucleonic matter.  In recent years, there has been a considerable
algorithmic development of first principle (or {\em ab initio})
methods for solving the nuclear many-body problem. Linked with a
similar progress in the derivation of nuclear forces based on
effective field theory (EFT), see chapters four, five and six of the
present text and Refs.~\cite{vankolck1994,machleidt2011,epelbaum2009},
we are now in a situation where reliable results can be provided at
different levels of approximation.  The nuclear Hamiltonians which are
now used routinely in nuclear structure and nuclear matter
calculations, include both nucleon-nucleon ($NN$) and three-nucleon forces (3NFs) derived from EFT, see for
example
Refs.~\cite{binder2013,hergert2013,roth2012,hagen2012a,hagen2012b,cipollone2013,carbone2013}.
Parallel to the development of nuclear forces from EFT, which employs
symmetries of quantum chromodynamics, there are recent and promising
approaches to derive the EoS using forces constrained from lattice
quantum chromodynamics calculations \cite{tetsuo2013}, see chapters 2
and 3 of the present text.

Theoretical studies of nuclear matter and the pertinent EoS span back
to the very early days of nuclear many-body physics. These early
developments are nicely summarized in for example the review of Day
\cite{day1967} from 1967. These early state-of-the-art calculations
were performed using what is known as Brueckner-Bethe-Goldstone theory
\cite{brueckner1954,brueckner1955}, see for example
Refs.\cite{hh2000,baldo2012,baldo2012a} for recent reviews and
developments.  In these calculations, mainly particle-particle
correlations were summed to infinite order.  Other correlations were
often included in a perturbative way. A systematic inclusion of other
correlations in a non-perturbative way are nowadays accounted for in
many-body methods like coupled cluster theory
\cite{bartlett2007,shavittbartlett2009}, various Monte Carlo methods
\cite{gandolfi2009,lovato2012}, Green's function approaches
\cite{baldo2012a,carbone2013,dickhoff2004} and density functional
theories \cite{erler2013}, just to mention a few of the actual
many-body methods which are used for nuclear matter studies. Many of these methods
are discussed in detail in this and the following chapters.


The aim of this part of the lectures (comprising this chapter and the
three subsequent ones) is to provide the necessary ingredients for
perfoming studies of neutron star matter (or matter in
$\beta$-equilibrium) and symmetric nuclear matter.  We will mainly
focus on pure neutron matter, but the framework and formalism can
easily be extended to other dense and homogeneous fermionic systems
such as the electron gas in two and three dimensions. The introductory
material we present here forms also the basis for the next three
chapters, starting with the definition of the single-particle basis
and our Hamiltonians as well as Hartree-Fock theory. For infinite
matter, due to the translational invariance of the Hamiltonian, the
single-particle basis, in terms of plane waves, is unchanged under
Hartree-Fock calculations.

Neutron star matter at densities of 0.1 fm$^{-3}$ and greater, is
often assumed to be made of mainly neutrons, protons, electrons and
muons in beta equilibrium. However, other baryons like various
hyperons may exist, as well as possible mesonic condensates and
transitions to quark degrees of freedom at higher densities
\cite{hh2000,prakash1996,steiner2010}.  In these notes we limit ourselves to matter composed
of neutrons only.  Furthermore, we will also consider matter at
temperatures much lower than the typical Fermi energies.

In the next section we present some of the basic quantities that enter
the different many-body methods discussed in this and the three
subsequent chapters. All these methods start with some single-particle
basis states, normally obtained via the solution of mean-field
approaches like Hartree-Fock theory. Contributions from correlations
beyond such a mean-field basis to selected observables, are then
obtained via a plethora of many-body methods. These methods represent
different mathematical algorithms used to solve either
Schr\"{o}dinger's or Dirac's equations for many interacting
fermions. After the definitions of our basis states, we derive the
Hartree-Fock equations in the subsequent section and move on with
many-body perturbation theory, full configuration interaction theory
and coupled cluster theory.  Monte Carlo methods, Green's function
theory approaches and Similarity Renormlization group approaches are
discussed in the subsequent three chapters.

The strengths and weaknesses of these methods are discussed throughout
these chapters, with applications to either a simple pairing model
and/or pure neutron matter. Our focus will be on pure neutron matter,
starting with a simple model for the interaction between
nucleons. This allows us to focus on pedagogical and algorithmic
aspects of the various many-body methods, avoiding thereby the full
complexity of nuclear forces.  If properly written however, the codes
can easily be extended to include models of the nuclear interactions
based on effective field theory (see chapters four, five and six of
the present text) and other baryon species than just neutrons. In our
conclusions we point back to models for nuclear forces and their links
to the underlying theory of the strong interaction discussed in the
first chapters of this book, bridging thereby the gap between the
theory of nuclear Hamiltonians and many-body methods.


\section{Single-particle basis, Hamiltonians and models for the nuclear force}\label{sec:chap8forces}

\subsection{Introduction to nuclear matter and Hamiltonians}

Although our focus here and in the coming chapters is on neutron
matter only, our formalism lends itself easily to studies of nuclear
matter with a given proton fraction and electrons. In this section we
outline some of the background details, with a focus on the
calculational basis and the representation of a nuclear Hamiltonian.

Neutron star matter is not composed of only neutrons. Rather, matter
is composed of various baryons and leptons in chemical and charge
equilibrium.  The equilibrium conditions are governed by the weak
processes (normally referred to as the processes for
$\beta$-equilibrium)
\begin{equation} 
      b_1 \rightarrow b_2 + l +\bar{\nu}_l \hspace{1cm} b_2 +l
      \rightarrow b_1 +\nu_l,
      \label{eq:betadecay}
\end{equation}
where $b_1$ and $b_2$ refer to different types of baryons, for example
a neutron and a proton.  The symbol $l$ is either an electron or a
muon and $\bar{\nu}_l $ and $\nu_l$ their respective anti-neutrinos
and neutrinos. Leptons like muons appear at a density close to nuclear
matter saturation density, the latter being
\[
     n_0 \approx 0.16 \pm 0.02 \hspace{0.1cm} \mathrm{fm}^{-3},
\]
with a corresponding energy per baryon ${\cal E}_0$ for symmetric
nuclear matter at saturation density of
\[
     {\cal E}_0 = B/A=-15.6\pm 0.2 \hspace{0.1cm} \mathrm{MeV}.
\]
The energy per baryon is the central quantity in the present
studies. From the energy per baryon, we can define the pressure $P$
which counteracts the gravitional forces and hinders the gravitational
collapse of a neutron star. The pressure is defined through the
relation
\begin{equation}
    P=n^2\frac{\partial {\cal E}}{\partial n}= n\frac{\partial
      \varepsilon}{\partial n}-\varepsilon,
\end{equation}
where $\varepsilon$ is the energy density.  Similarly, the chemical
potential for particle species $i$ is given by
\begin{equation}
     \mu_i = \left(\frac{\partial \varepsilon}{\partial n_i}\right).
     \label{eq:chemicalpotdef}
\end{equation}
In calculations of properties of neutron star matter in
$\beta$-equilibrium, we need to calculate the energy per baryon ${\cal E}$ 
for several proton fractions $x_p$. The proton fraction
corresponds to the ratio of protons as compared to the total nucleon
number ($Z/A$). It is defined as
\begin{equation}
    x_p = \frac{n_p}{n},
\end{equation}
where $n=n_p+n_n$ is the total baryonic density if neutrons and
protons are the only baryons present. If this is the case, the total
Fermi momentum $k_F$ and the Fermi momenta $k_{Fp}$, $k_{Fn}$ for
protons and neutrons, respectively, are related to the total nucleon density $n$ by
\begin{align}
     n = & \frac{2}{3\pi^2} k_F^3 \nonumber \\ = & x_p n + (1-x_p) n
     \nonumber \\ = & \frac{1}{3\pi^2} k_{Fp}^3 + \frac{1}{3\pi^2}
     k_{Fn}^3.
    \label{eq:densi}
\end{align}
The energy per baryon will thus be labelled as ${\cal E}(n,x_p)$. The
quantity ${\cal E}(n,0)$ refers then to the energy per baryon for pure
neutron matter while ${\cal E}(n,\frac{1}{2})$ is the
corresponding value for symmetric nuclear matter. Furthermore, in this work, subscripts
$n,p,e,\mu$ will always refer to neutrons, protons, electrons and
muons, respectively.


Since the mean free path of a neutrino in a neutron star is bigger
than the typical radius of such a star ($\sim 10$ km), we will
throughout assume that neutrinos escape freely from the neutron star,
see for example Ref.~\cite{prakash2006} for a discussion on trapped
neutrinos. Eq.~(\ref{eq:betadecay}) yields then the following
conditions for matter in $\beta$-equilibrium with for example
nucleonic degrees freedom only
\begin{equation}
    \mu_n=\mu_p+\mu_e,
     \label{eq:npebetaequilibrium}
\end{equation}
and
\begin{equation}
     n_p = n_e,
     \label{eq:chargeconserv}
\end{equation}
where $\mu_i$ and $n_i$ refer to the chemical potential and number
density in fm$^{-3}$ of particle species $i$.  If muons are present as
well, we need to modify the equation for charge conservation,
Eq. (\ref{eq:chargeconserv}), to read
\[
     n_p = n_e+n_{\mu},
\]
and require that $\mu_e = \mu_{\mu}$.

An important ingredient in the discussion of the EoS and the criteria
for matter in $\beta$-equilibrium is the so-called symmetry energy
${\cal S} (n)$, defined as the difference in energy for symmetric
nuclear matter and pure neutron matter
\begin{equation}
      {\cal S} (n) = {\cal E} (n,x_p=0) - {\cal E} (n,x_p=1/2 ).
      \label{eq:symenergy}
\end{equation}
If we expand the energy per baryon in the case of nucleonic degrees of
freedom only in the proton concentration $x_p$ about the value of the
energy for SNM ($x_p=\frac{1}{2}$), we obtain,
\begin{equation}
     {\cal E} (n,x_p)={\cal E} (n,x_p=\frac{1}{2})+
     \frac{1}{2}\frac{d^2 {\cal E}}{dx_p^2}
     (n)\left(x_p-1/2\right)^2+\dots ,
     \label{eq:energyexpansion}
\end{equation}
where the term $d^2 {\cal E}/dx_p^2$ is to be associated with the
symmetry energy ${\cal S} (n)$ in the empirical mass formula. If we
assume that higher order derivatives in the above expansion are small,
then through the conditions for $\beta$-equilbrium of
Eqs.~(\ref{eq:npebetaequilibrium}) and (\ref{eq:chargeconserv}) and
Eq.~(\ref{eq:chemicalpotdef}) we can define the proton fraction by the
symmetry energy as
\begin{equation}  
    \hbar c\left(3\pi^2nx_p\right)^{1/3} = 4{\cal S}
    (n)\left(1-2x_p\right),
    \label{eq:crudeprotonfraction}
\end{equation}
where the electron chemical potential is given by $\mu_e = \hbar c
k_F$, i.e.\ ultrarelativistic electrons are assumed.  Thus, the
symmetry energy is of paramount importance for studies of neutron star
matter in $\beta$-equilibrium.  One can extract information about the
value of the symmetry energy at saturation density $n_0$ from
systematic studies of the masses of atomic nuclei. However, these
results are limited to densities around $n_0$ and for proton fractions
close to $\frac{1}{2}$, see for example the various contributions in
Ref.~\cite{symmetryenergy2013}.  Typical values for ${\cal S} (n)$ at
$n_0$ are in the range $27-38$ MeV.  For densities greater than $n_0$
it is more difficult to get a reliable information on the symmetry
energy, and thereby the related proton fraction.


With this background, we are now ready to define our basic inputs and
approximations to the various many-body theories discussed in this
chapter and the three subsequent ones.  We will assume that the
interacting part of the Hamiltonian can be approximated by a two-body
interaction.  This means that our Hamiltonian is written as the sum of
a one-body part and a two-body part
\begin{equation}
    \hat{H} = \hat{H}_0 + \hat{H}_I = \sum_{i=1}^A \hat{h}_0(x_i) +
    \sum_{i < j}^A \hat{v}(r_{ij}),
\label{Hnuclei}
\end{equation}
with
\begin{equation}
  H_0=\sum_{i=1}^A \hat{h}_0(x_i).
\label{hinuclei}
\end{equation}
The one-body operator is defined as
\[
\hat{h}_0(x_i)=\hat{t}(x_i) + \hat{u}_{\mathrm{ext}}(x_i),
\]
where $\hat{t}$ represents the kinetic energy and $x_i$ represents
both spatial and spin degrees of freedom.  For many-body calculations
of finite nuclei, the external potential $u_{\mathrm{ext}}(x_i)$ is
normally approximated by a harmonic oscillator or Woods-Saxon
potential. For atoms, the external potential is defined by the Coulomb
interaction an electron feels from the atomic nucleus. However, other
potentials are fully possible, such as one derived from the
self-consistent solution of the Hartree-Fock equations to be discussed
below. Since we will work with infinite matter and plane wave basis states, the one-body operator is
simply given by the kinetic energy operator.

Our Hamiltonian is invariant under the permutation (interchange) of
two particles.  Since we deal with fermions however, the total wave
function is anti-symmetric.  Let $\hat{P}$ be an operator which
interchanges two particles.  Due to the symmetries we have ascribed to
our Hamiltonian, this operator commutes with the total Hamiltonian,
\[
[\hat{H},\hat{P}] = 0,
 \]
meaning that a many-body eigenstate $\Psi_{\lambda}(x_1, x_2, \dots ,
x_A)$ of $\hat{H}$ is an eigenfunction of $\hat{P}$ as well.

In our case we assume that we can approximate the exact eigenfunction
for say the ground state with a Slater determinant
\begin{equation}
   \Phi_0(x_1, x_2,\dots ,x_A,\alpha,\beta,\dots,
   \sigma)=\frac{1}{\sqrt{A!}}  \left| \begin{array}{ccccc}
     \psi_{\alpha}(x_1)& \psi_{\alpha}(x_2)& \dots & \dots &
     \psi_{\alpha}(x_A)\\ \psi_{\beta}(x_1)&\psi_{\beta}(x_2)& \dots &
     \dots & \psi_{\beta}(x_A)\\ \dots & \dots & \dots & \dots & \dots
     \\ \dots & \dots & \dots & \dots & \dots
     \\ \psi_{\sigma}(x_1)&\psi_{\sigma}(x_2)& \dots & \dots &
     \psi_{\sigma}(x_A)\end{array} \right|, \label{eq:HartreeFockDet}
\end{equation}
where $x_i$ stand for the coordinates and spin values of particle
$i$ and $\alpha,\beta,\dots, \gamma$ are quantum numbers needed to
describe remaining quantum numbers.

The single-particle function $\psi_{\alpha}(x_i)$ are eigenfunctions
of the one-body Hamiltonian $h_0$, that is
\[
\hat{h}_0(x_i) \psi_{\alpha}(x_i)=\left(\hat{t}(x_i) +
\hat{u}_{\mathrm{ext}}(x_i)\right)\psi_{\alpha}(x_i)=\varepsilon_{\alpha}\psi_{\alpha}(x_i).
\]
The energies $\varepsilon_{\alpha}$ are the so-called non-interacting
single-particle energies, or unperturbed energies.  The total energy
is in this case the sum over all single-particle energies, if no
two-body or more complicated many-body interactions are present.

The properties of the determinant lead to a straightforward
implementation of the Pauli principle since no two particles can be at
the same place (two columns being the same in the above determinant)
and no two particles can be in the same state (two rows being the
same).  As a practical matter, however, Slater determinants beyond
$N=4$ quickly become unwieldy. Thus we turn to the occupation
  representation or second quantization to simplify
calculations. For a good introduction to second quantization see for
examples Ref.~\cite{shavittbartlett2009}.

We start with a set of orthonormal single-particle states $\{
\psi_{\alpha}(x) \}$.  To each single-particle state
$\psi_{\alpha}(x)$ we associate a creation operator
$a^\dagger_{\alpha}$ and an annihilation operator $a_{\alpha}$.  When
acting on the vacuum state $| 0 \rangle$, the creation operator
$a^\dagger_{\alpha}$ causes a particle to occupy the single-particle
state $\psi_{\alpha}(x)$
\[
\psi_{\alpha}(x) \rightarrow a^\dagger_{\alpha} |0 \rangle.
\]
But with multiple creation operators we can occupy multiple states
\[
\psi_{\alpha}(x) \psi_{\beta}(x^\prime) \psi_{\delta}(x^{\prime
  \prime}) \rightarrow a^\dagger_{\alpha} a^\dagger_{\beta}
a^\dagger_{\delta} |0 \rangle.
\]

Now we impose anti-symmetry by having the fermion operators satisfy
the anti-commutation relations
\[
a^\dagger_{\alpha} a^\dagger_{\beta} + a^\dagger_{\beta}
a^\dagger_{\alpha} = \left\{ a^\dagger_{\alpha}
,a^\dagger_{\beta}\right\}= 0,
\]
with the consequence that
\[
a^\dagger_{\alpha} a^\dagger_{\beta} = - a^\dagger_{\beta}
a^\dagger_{\alpha}.
\]
Because of this property, we obtain $a^\dagger_{\alpha}
a^\dagger_{\alpha} = 0$, enforcing the Pauli exclusion principle.
Thus we can represent a Slater determinant using creation operators as
\[
a^\dagger_{\alpha} a^\dagger_{\beta} a^\dagger_{\delta} \ldots |0\rangle,
\]
where each index $\alpha,\beta,\delta, \ldots$ has to be unique.

We will now find it convenient to define a Fermi level and introduce a
new reference vacuum. The Fermi level is normally  defined in terms of all
occupied single-particle states below a certain
single-particle energy.  With the definition of a Fermi level, we can
in turn define our ansatz for the ground state, represented by a
Slater determinant $\Phi_0$.  We will throughout the rest of this
text use creation and annihilation operators to represent quantum
mechanical operators and states.  It means that our compact
representation for a given Slater determinant in Fock space is
\[
  \Phi_{0}=|i_1 \dots i_A\rangle= a_{i_1}^{\dagger} \dots
  a_{i_A}^{\dagger} |0\rangle
\]
where $\vert 0\rangle$ is the true vacuum and we have defined the
creation and annihilation operators as
    \[
        a_p^\dagger|0\rangle = |p\rangle, \quad a_p |q\rangle =
        \delta_{pq}|0\rangle
    \]
with the anti-commutation relations
\[
  \delta_{pq} = \left\{a_p, a_q^\dagger \right\},
\]
and
\[
\left\{a_p^\dagger, a_q \right\} = \left\{a_p, a_q \right\} =
\left\{a_p^\dagger, a_q^\dagger \right\}=0.
\]

We can rewrite the ansatz for the ground state as
\[
\vert\Phi_0\rangle = \prod_{i\le F}a_{i}^{\dagger} |0\rangle,
\]
where we have introduced the shorthand labels for states below the
Fermi level $F$ as $i,j,\ldots \leq F$. For single-particle states
above the Fermi level we reserve the labels $a,b,\ldots > F$, while
the labels $p,q, \ldots$ represent any possible single particle state.


Since our focus is on infinite systems, the one-body part of the
Hamiltonian is given by the kinetic energy operator only.  In second
quantization it is defined as
\[
\hat{H}_0=\hat{T} = \sum_{pq} \langle p|\hat{t}|q\rangle a_p^\dagger
a_q,
\]
where the matrix elements $\langle p|\hat{t}|q\rangle$ are defined in
Eq.~(\ref{eq:kineticenergy}).  The two-body interaction reads
\[
\hat{H}_I=\hat{V} = \frac{1}{4} \sum_{pqrs} \langle
pq|\hat{v}|rs\rangle_{AS} a_p^\dagger a_q^\dagger a_s a_r,
\]
where we have defined the anti-symmetrized matrix elements
\[
\langle pq|\hat{v}|rs\rangle_{AS} = \langle pq|\hat{v}|rs\rangle -
\langle pq|\hat{v}|sr\rangle.
\]

We can also define a three-body operator
\[
\hat{V}_3 = \frac{1}{36} \sum_{pqrstu} \langle
pqr|\hat{v}_3|stu\rangle_{AS} a_p^\dagger a_q^\dagger a_r^\dagger a_u
a_t a_s,
\]
with the anti-symmetrized matrix element
\begin{align}
            \langle pqr|\hat{v}_3|stu\rangle_{AS} = \langle
            pqr|\hat{v}_3|stu\rangle + \langle
            pqr|\hat{v}_3|tus\rangle + \langle
            pqr|\hat{v}_3|ust\rangle- \langle pqr|\hat{v}_3|sut\rangle
            - \langle pqr|\hat{v}_3|tsu\rangle - \langle
            pqr|\hat{v}_3|uts\rangle.
\end{align}
In this and the forthcoming chapters we will limit ourselves to
two-body interactions at most.  Throughout this chapter and the subsequent three we will drop the subscript $AS$ and use only anti-symmetrized matrix elements.

Using the ansatz for the ground state $\vert \Phi_0\rangle$ as new reference
vacuum state, we need to redefine the anticommutation relations to
\[
\left\{a_p^\dagger, a_q \right\}= \delta_{pq}, p, q \leq F,
\]
and
\[
\left\{a_p, a_q^\dagger \right\} = \delta_{pq}, \hspace{0.1cm} p, q > F.
\]
It is easy to see that
\[
        a_i|\Phi_0\rangle = |\Phi_i\rangle\ne 0, \hspace{0.5cm}
        a_a^\dagger|\Phi_0\rangle = |\Phi^a\rangle\ne 0,
\]
and
\[
a_i^\dagger|\Phi_0\rangle = 0 \hspace{0.5cm} a_a|\Phi_0\rangle = 0.
\]
With the new reference vacuum state the Hamiltonian can be rewritten
as, see problem \ref{problem:prob8.1},
\[
\hat{H}=E_{\mathrm{Ref}}+\hat{H}_N,
\]
with the reference energy defined as the expectation value of the
Hamiltonian using the reference state $\Phi_0$
\[
E_{\mathrm{Ref}}=\langle \Phi_0 \vert \hat{H} \vert \Phi_0\rangle =
\sum_{i\le F} \langle i|\hat{h}_0|i\rangle + \frac{1}{2} \sum_{ij\le
  F}\langle ij|\hat{v}|ij\rangle,
\]
and the new normal-ordered Hamiltonian is defined as
\begin{equation}\label{eq:Hnormalorder}
\hat{H}_N = \sum_{pq} \langle p|\hat{h}_0|q\rangle \left\{a^\dagger_p
a_q\right\}+\frac{1}{4} \sum_{pqrs} \langle pq|\hat{v}|rs\rangle
\left\{a^\dagger_p a^\dagger_q a_s a_r\right\}+\sum_{pq,i\le F}
\langle pi|\hat{v}|qi\rangle \left\{a^\dagger_p a_q\right\},
\end{equation}
where the curly brackets represent normal-ordering with respect to the
new reference vacuum state.  The normal-ordered Hamiltonian can be
rewritten in terms of a new one-body operator and a two-body operator
as
\[
\hat{H}_N=\hat{F}_N+\hat{V}_N,
\]
with
\begin{equation}\label{eq:hfn}
\hat{F}_N=\sum_{pq} \langle p|\hat{f}|q\rangle \left\{a^\dagger_pa_q\right\},
\end{equation}
where
\[
\langle p|\hat{f}|q\rangle= \langle p|\hat{h}_0|q\rangle +\sum_{i\le F}
\langle pi|\hat{v}|qi\rangle.
\]
The last term on the right hand side represents a medium modification
to the single-particle Hamiltonian due to the two-body interaction.
Finally, the two-body interaction is given by
\begin{equation}\label{eq:hvn}	     
\hat{V}_N = \frac{1}{4} \sum_{pqrs} \langle pq|\hat{v}|rs\rangle
\left\{a^\dagger_p a^\dagger_q a_s a_r\right\}.
\end{equation}	     

\subsection{Single-particle basis for infinite matter}

Infinite nuclear or neutron matter is a homogeneous system and the
one-particle wave functions are given by plane wave functions
normalized to a volume $\Omega$ for a box with length $L$ (the limit
$L\rightarrow \infty$ is to be taken after we have computed various
expectation values)
\[
\psi_{\mathbf{k}\sigma}(\mathbf{r})=
\frac{1}{\sqrt{\Omega}}\exp{(i\mathbf{kr})}\xi_{\sigma}
\]
where $\mathbf{k}$ is the wave number and $\xi_{\sigma}$ is the spin
function for either spin up or down nucleons
\[ 
\xi_{\sigma=+1/2}=\left(\begin{array}{c} 1
  \\ 0 \end{array}\right) \hspace{0.5cm}
\xi_{\sigma=-1/2}=\left(\begin{array}{c} 0 \\ 1 \end{array}\right).
\]

We focus first on the kinetic energy operator.  We assume that we have
periodic boundary conditions which limit the allowed wave numbers to
\[
k_i=\frac{2\pi n_i}{L}\hspace{0.5cm} i=x,y,z \hspace{0.5cm} n_i=0,\pm
1,\pm 2, \dots
\]
The operator for the kinetic energy can be written as, see problem
\ref{problem:prob8kinetic},
\[
\hat{T}=\sum_{\mathbf{p}\sigma_p}\frac{\hbar^2k_P^2}{2m}a_{\mathbf{p}\sigma_p}^{\dagger}a_{\mathbf{p}\sigma_p}.
\]
When using periodic boundary conditions, the discrete-momentum
single-particle basis functions (excluding spin and/or isospin degrees
of freedom) result in the following single-particle energy
\begin{align}
  \varepsilon_{n_{x}, n_{y}, n_{z}} = \frac{\hbar^{2}}{2m} \left(
  \frac{2\pi }{L}\right)^{2} \left( n_{x}^{2} + n_{y}^{2} +
  n_{z}^{2}\right)=\frac{\hbar^2}{2m}\left(k_{n_x}^2+k_{n_y}^2+k_{n_z}^2\right),
\end{align} 
for a three-dimensional system with
\[
k_{n_i}=\frac{2\pi n_i}{L} \hspace{0.1cm} n_i = 0, \pm 1, \pm 2,
\dots,
\]
We will select the single-particle basis such that both the occupied
and unoccupied single-particle spaces have a closed-shell
structure. This means that all single-particle states corresponding to
energies below a chosen cutoff are included in the basis. We study
only the unpolarized spin phase, in which all orbitals are occupied
with one spin-up and one spin-down fermion (neutrons and protons in
our case).  With the kinetic energy rewritten in terms of the
discretized momenta we can set up a similar table and obtain (assuming
identical particles one and including spin up and spin down solutions)
for energies less than or equal to $n_{x}^{2}+n_{y}^{2}+n_{z}^{2}\le
3$, as shown in Table \ref{tab:table1}
\begin{table}
\begin{center}
\caption{Total number of particle filling $N_{\uparrow \downarrow }$
  for various $n_{x}^{2}+n_{y}^{2}+n_{z}^{2}$ values for one spin-1/2
  fermion species.  Borrowing from nuclear shell-model terminology,
  filled shells corresponds to all single-particle states for one
  $n_{x}^{2}+n_{y}^{2}+n_{z}^{2}$ value being occupied.  For matter
  with both protons and neutrons, the filling degree increased with a
  factor of $2$.} \label{tab:table1}
\begin{tabular}{ccccc}
\hline \multicolumn{1}{c}{ $n_{x}^{2}+n_{y}^{2}+n_{z}^{2}$ } &
\multicolumn{1}{c}{ $n_{x}$ } & \multicolumn{1}{c}{ $n_{y}$ } &
\multicolumn{1}{c}{ $n_{z}$ } & \multicolumn{1}{c}{ $N_{\uparrow
    \downarrow }$ } \\ \hline 0 & 0 & 0 & 0 & 2 \\ \hline 1 & -1 & 0 &
0 & \\ 1 & 1 & 0 & 0 & \\ 1 & 0 & -1 & 0 & \\ 1 & 0 & 1 & 0 & \\ 1 & 0
& 0 & -1 & \\ 1 & 0 & 0 & 1 & 14 \\ \hline 2 & -1 & -1 & 0 & \\ 2 & -1
& 1 & 0 & \\ 2 & 1 & -1 & 0 & \\ 2 & 1 & 1 & 0 & \\ 2 & -1 & 0 & -1 &
\\ 2 & -1 & 0 & 1 & \\ 2 & 1 & 0 & -1 & \\ 2 & 1 & 0 & 1 & \\ 2 & 0 &
-1 & -1 & \\ 2 & 0 & -1 & 1 & \\ 2 & 0 & 1 & -1 & \\ 2 & 0 & 1 & 1 &
38 \\ \hline 3 & -1 & -1 & -1 & \\ 3 & -1 & -1 & 1 & \\ 3 & -1 & 1 &
-1 & \\ 3 & -1 & 1 & 1 & \\ 3 & 1 & -1 & -1 & \\ 3 & 1 & -1 & 1 & \\ 3
& 1 & 1 & -1 & \\ 3 & 1 & 1 & 1 & 54 \\ \hline
\end{tabular}
\end{center}
\end{table}


Continuing in this way we get for $n_{x}^{2}+n_{y}^{2}+n_{z}^{2}=4$ a
total of 22 additional states, resulting in $76$ as a new magic
number. For the lowest six energy values the degeneracy in energy
gives us $2$, $14$, $38$, $54$, $76$ and $114$ as magic numbers. These
numbers will then define our Fermi level when we compute the energy in
a Cartesian basis. When performing calculations based on many-body
perturbation theory, Coupled cluster theory or other many-body
methods, we need then to add states above the Fermi level in order to
sum over single-particle states which are not occupied.

If we wish to study infinite nuclear matter with both protons and
neutrons, the above magic numbers become $4, 28, 76, 108, 132, 228,
\dots$.

Every number of particles for filled shells defines also the number of
particles to be used in a given calculation. The number of particles
can in turn be used to define the density $n$ (or the Fermi momentum)
of the system via
\[
n = g \frac{k_F^3}{6\pi^2},
\]
where $k_F$ is the Fermi momentum and the degeneracy $g$, which is two
for one type of spin-$1/2$ particles and four for symmetric nuclear
matter.  With the density defined and having fixed the number of
particles $A$ and the Fermi momentum $k_F$, we can define the length
$L$ of the box used with periodic boundary contributions via the
relation
\[
  V= L^3= \frac{A}{n}.
\]
With $L$ we can to define the spacing between various
$k$-values given by
\[
  \Delta k = \frac{2\pi}{L}.
\]
Here, $A$ is the number of nucleons. If we deal with the electron
gas only, this needs to be replaced by the number of electrons $N$.
Exercise \ref{problem:spbasissetup} deals with setting up a program
that establishes the single-particle basis for nuclear matter
calculations with input a given number of nucleons and a user
specificied density or Fermi momentum.


\subsection{Two-body interaction}

As mentioned above, we will employ a plane wave basis for our
calculations of infinite matter properties. With a cartesian basis it
means that we can calculate directly the various matrix elements, as
discussed in the previous subsection. However, a cartesian basis
represents an approximation to the thermodynamical limit. In order to
compare the stability of our basis with results from the
thermodynamical limit, it is convenient to rewrite the nucleon-nucleon
interaction in terms of a partial wave expansion. This will allow us
to compute the Hartree-Fock energy of the ground state in the
thermodynamical limit (with the caveat that we need to limit the
number of partial waves). In order to find the expressions for the
Hartree-Fock energy in a partial wave basis, we will find it
convenient to rewrite our two-body force in terms of the relative and
center-of-mass motion momenta.

The direct matrix element, with single-particle three-dimensional
momenta $\mathbf{k}_p$, spin $\sigma_p$ and isospin $\tau_p$, is
defined as
\[
\langle \mathbf{k}_p\sigma_p\tau_p \mathbf{k}_q\sigma_q\tau_q \vert
\hat{v}\vert \mathbf{k}_r\sigma_r\tau_r \mathbf{k}_s\sigma_s\tau_s
\rangle,
\]
or in a more compact form as $\langle \mathbf{p}\mathbf{q}\vert
\hat{v} \vert \mathbf{r}\mathbf{s} \rangle$ where the boldfaced
letters $\mathbf{p}$ etc represent the relevant quantum numbers, here
momentum, spin and isospin. Introducing the relative momentum
\[
\mathbf{k} = \frac{1}{2}\left(\mathbf{k}_p-\mathbf{k}_q\right),
\]
and the center-of-mass momentum
\[
\mathbf{K} = \mathbf{k}_p+\mathbf{k}_q,
\]
we have
\[
\langle \mathbf{k}_p\sigma_p\tau_p \mathbf{k}_q\sigma_q\tau_q \vert
\hat{v}\vert \mathbf{k}_r\sigma_r\tau_r \mathbf{k}_s\sigma_s\tau_s
\rangle=\langle \mathbf{k}\mathbf{K}\sigma_p\tau_p \sigma_q\tau_q
\vert \hat{v}\vert \mathbf{k}'\mathbf{K}'\sigma_r\tau_r \sigma_s\tau_s
\rangle.
\]
The nucleon-nucleon interaction conserves the total momentum and
charge, implying that the above uncoupled matrix element reads
\[
\langle \mathbf{k}\mathbf{K}\sigma_p\tau_p \sigma_q\tau_q \vert
\hat{v}\vert \mathbf{k}'\mathbf{K}'\sigma_r\tau_r \sigma_s\tau_s
\rangle=\delta_{T_z,T_z'}\delta(\mathbf{K}-\mathbf{K}')\langle
\mathbf{k}T_zS_z=(\sigma_a+\sigma_b) \vert \hat{v}\vert
\mathbf{k}'T_zS_z'=(\sigma_c+\sigma_d) \rangle,
\]
where we have defined the isospin projections $T_z=\tau_p+\tau_q$ and
$T_z'=\tau_r+\tau_s$.  Defining
$\hat{v}=\hat{v}(\mathbf{k},\mathbf{k}' )$, we can rewrite the
previous equation in a more compact form as
\[
\delta_{T_z,T_z'}\delta(\mathbf{K}-\mathbf{K}')\langle
\mathbf{k}T_zS_z=(\sigma_p+\sigma_q) \vert \hat{v}\vert
\mathbf{k}'T_zS_z'=(\sigma_r+\sigma_s)
\rangle=\delta_{T_z,T_z'}\delta(\mathbf{K}-\mathbf{K}')\langle
T_zS_z\vert\hat{v}(\mathbf{k},\mathbf{k}' ) \vert T_zS_z' \rangle.
\]
These matrix elements can in turn be rewritten in terms of the total
two-body quantum numbers for the spin $S$ of two spin-1/2 fermions as
\[
\langle \mathbf{k}T_zS_z \vert \hat{v}(\mathbf{k},\mathbf{k}' )\vert
\mathbf{k}'T_zS_z' \rangle=\sum_{SS'}\langle
\frac{1}{2}\sigma_p\frac{1}{2}\sigma_q\vert SS_z\rangle \langle
\frac{1}{2}\sigma_r\frac{1}{2}\sigma_s\vert S'S_z'\rangle \langle
\mathbf{k}T_zSS_z\vert \hat{v}(\mathbf{k},\mathbf{k}' )\vert
\mathbf{k}T_zS'S_z' \rangle
\]
The coefficients $\langle \frac{1}{2}\sigma_p\frac{1}{2}\sigma_q\vert
SS_z\rangle$ are so-called Clebsch-Gordan recoupling coefficients.  We
will assume that our interactions break charge and isospin
symmetry. We will refer to $T_z=0$ as the $pn$ (proton-neutron)
channel, $T_z=-1$ as the $pp$ (proton-proton) channel and $T_z=1$ as
the $nn$ (neutron-neutron) channel.

The nucleon-nucleon force is often derived and analyzed theoretically
in terms of a partial wave expansion. A state with linear momentum
$\mathbf{k}$ can be written as
\[
\vert \mathbf{k} \rangle =
\sum_{l=0}^{\infty}\sum_{l_l=-l}^{L}\imath^lY_{l}^{m_l}(\hat{k}\vert
klm_l\rangle.
\]

In terms of the relative and center-of-mass momenta $\mathbf{k}$ and
$\mathbf{K}$, the potential in momentum space is related to the
nonlocal operator $V(\mathbf{r},\mathbf{r}')$ by
\begin{equation}
      \langle \mathbf{k'K'}\vert \hat{v} \vert \mathbf{k'K} \rangle=
      \int d\mathbf{r}d \mathbf{r'} e^{-\imath
        \mathbf{k'r'}}V(\mathbf{r'},\mathbf{r}) e^{\imath \mathbf{kr}}
      \delta(\mathbf{K},\mathbf{K'}).
\end{equation}
We will assume that the interaction is spherically symmetric and use
the partial wave expansion of the plane waves in terms of spherical
harmonics.  This means that we can separate the radial part of the
wave function from its angular dependence. The wave function of the
relative motion is described in terms of plane waves as
\begin{equation}
       e^{\imath \mathbf{kr}} = \langle\mathbf{r}\vert
       \mathbf{k}\rangle = 4\pi \sum_{lm} \imath ^{l} j_{l} (kr)
       Y_{lm}^{*}(\mathbf{\hat{k}}) Y_{lm}(\mathbf{\hat{r}}),
\end{equation}
where $j_l$ is a spherical Bessel function and $Y_{lm}$ the spherical
harmonic.  This partial wave basis is useful for defining the operator
for the nucleon-nucleon interaction, which is symmetric with respect
to rotations, parity and isospin transformations. These symmetries
imply that the interaction is diagonal with respect to the quantum
numbers of total angular momentum $J$, spin $S$ and isospin $T$. Using
the above plane wave expansion, and coupling to final $J$, $S$ and $T$
we get
\begin{equation}
      \langle \mathbf{k'}\vert V \vert \mathbf{k}\rangle = (4\pi)^2
      \sum_{JM}\sum_{lm}\sum_{l'm'} \imath ^{l+l'}
      Y_{lm}^{*}(\mathbf{\hat{k}}) Y_{l'm'}(\mathbf{\hat{k}'}) {\cal
        C}_{m'M_SM}^{l'SJ}{\cal C}_{mM_SM}^{lSJ} \langle k'l'STJM
      \vert V \vert klSTJM \rangle,
\label{eq:vpartial}
\end{equation}
where we have defined
\begin{equation}
    \langle k'l'STJM\vert V \vert klSTJM\rangle = \int
    j_{l'}(k'r')\langle l'STJM\vert V(r',r)\vert lSTJM \rangle j_l(kr)
    {r'}^2 dr' r^2 dr.
\end{equation}
We have omitted the momentum of the center-of-mass motion $\mathbf{K}$
and the corresponding orbital momentum $L$, since the interaction is
diagonal in these variables. The potentials we will employ in this
work, like those of the Bonn group, are all non-local potentials
defined in momentum space, and we will therefore not need the last
equation.


The interaction we will use for these calculations is a semirealistic
nucleon-nucleon potential known as the Minnesota potential
\cite{minnesota} which has the form, $V_{\alpha}\left(
r\right)=V_{\alpha}\exp{(-\alpha r^{2})}$. The spin and isospin
dependence of the Minnesota potential is given by,
\begin{equation}
V\left( r\right)=\frac{1}{2}\left( V_{R}+\frac{1}{2}\left(
1+P_{12}^{\sigma}\right) V_{T}+\frac{1}{2}\left(
1-P_{12}^{\sigma}\right) V_{S}\right)\left(
1-P_{12}^{\sigma}P_{12}^{\tau}\right),
\end{equation}
where $P_{12}^{\sigma}=\frac{1}{2}\left(
1+\sigma_{1}\cdot\sigma_{2}\right)$ and
$P_{12}^{\tau}=\frac{1}{2}\left( 1+\tau_{1}\cdot\tau_{2}\right)$ are
the spin and isospin exchange operators, respectively. A Fourier
transform to momentum space of the radial part $V_{\alpha}\left(
r\right)$ is rather simple, see problem \ref{problem:fourier}, since
the radial depends only on the magnitude of the relative distance and
thereby the relative momentum
$\vec{q}=\frac{1}{2}\left(\vec{k}_{p}-\vec{k}_{q}-\vec{k}_{r}+\vec{k}_{s}\right)$. Omitting
spin and isospin dependencies, the momentum space version of the
interaction reads
\begin{equation}
\langle \mathbf{k}_p \mathbf{k}_q \vert V_{\alpha}\vert
\mathbf{k}_r\mathbf{k}_s\rangle=\frac{V_{\alpha}}{L^{3}}\left(\frac{\pi}{\alpha}\right)^{3/2}\exp{(\frac{-q^{2}}{4\alpha})}\delta_{\vec{k}_{p}+\vec{k}_{q},\vec{k}_{r}+\vec{k}_{s}}
\end{equation}
The various parameters defining the interaction model used in this
work are listed in Table \ref{tab:minnesotatab}.
\begin{table}
\caption{Parameters used to define the Minnesota interaction model
  \cite{minnesota}.}\label{tab:minnesotatab}
\begin{center}
  \begin{tabular}{| l | l | l |}
    \hline $\alpha$ & $V_{\alpha}$ & $\kappa_{\alpha}$ \\ \hline $R$ &
    200 $\mathrm{MeV}$ & 1.487 $\mathrm{fm}^{-2}$ \\ \hline $T$ & 178
    $\mathrm{MeV}$ & 0.639 $\mathrm{fm}^{-2}$ \\ \hline $S$ & 91.85
    $\mathrm{MeV}$ & 0.465 $\mathrm{fm}^{-2}$ \\ \hline
  \end{tabular}
\end{center}
\end{table}


\subsection{Models from Effective field theory for the two- and three-nucleon interactions}\label{subsec:forcemodels}

During the past two decades it has been demonstrated that chiral
effective field theory represents a powerful tool to deal with
hadronic interactions at low energy in a systematic and
model-independent way (see
Refs.~\cite{weinberg1990,weinberg1991,ordonez1992,ordonez1994,ordonez1996,vankolck1999,machleidt2011,epelbaum2009,ekstrom2013,ekstromPRX}).
Effective field theories (EFTs) are defined in terms of effective Lagrangians which are given by
an infinite series of terms with increasing number of derivatives
and/or nucleon fields, with the dependence of each term on the pion
field prescribed by the rules of broken chiral symmetry.  Applying
this Lagrangian to a particular process, an unlimited number of
Feynman graphs can be drawn. Therefore, a scheme is needed that makes
the theory manageable and calculable.  This scheme which tells us how
to distinguish between large (important) and small (unimportant)
contributions is chiral perturbation theory (ChPT).  Chiral perturbation theory  allows for
an expansion in terms of $(Q/\Lambda_\chi)^\nu$, where $Q$ is generic
for an external momentum (nucleon three-momentum or pion
four-momentum) or a pion mass, and $\Lambda_\chi \sim 1$ GeV is the
chiral symmetry breaking scale.  Determining the power $\nu$ has
become known as power counting.

Nuclear potentials are defined as sets of irreducible graphs up to a
given order.  The power $\nu$ of a few-nucleon diagram involving $A$
nucleons is given in terms of naive dimensional analysis by:
\begin{equation} 
\nu = -2 +2A - 2C + 2L + \sum_i \Delta_i \, ,
\label{eq_nu} 
\end{equation}
with
\begin{equation}
\Delta_i \equiv d_i + \frac{n_i}{2} - 2 \, ,
\label{eq_Deltai}
\end{equation}
where $A$ labels the number of nucleons, $C$ denotes the number of separately connected pieces and $L$
the number of loops in the diagram; $d_i$ is the number of derivatives
or pion-mass insertions and $n_i$ the number of nucleon fields
(nucleon legs) involved in vertex $i$; the sum runs over all vertices
contained in the diagram under consideration.  Note that $\Delta_i
\geq 0$ for all interactions allowed by chiral symmetry.  In this work
we will focus on optimized two- and three-nucleon forces at order
NNLO, as indicated in Fig.~\ref{fig_diagNNLO}.  

Below we revisit briefly the formalism and results presented in
Refs.~\cite{ekstromPRX}. For further details on chiral effective
field theory and nuclear interactions, see for example
Refs.~\cite{machleidt2011,epelbaum2009,ekstrom2013}
\begin{figure}[t]\centering
%\vspace*{-0.25cm}
\scalebox{0.14}{\includegraphics{Chapter8-figures/diagNNLO.pdf}}
%\vspace*{-1.0cm}
\caption{Nuclear forces in ChPT up to NNLO. Solid lines represent
  nucleons and dashed lines pions.  Small dots, large solid dots, and
  solid squares denote vertices of index $\Delta_i= \, $ 0, 1, and 2,
  respectively.}
\label{fig_diagNNLO}
\end{figure}
For an irreducible $NN$ diagram (``two-nucleon potential'', $A=2$,
$C=1$), Eq.~(\ref{eq_nu}) collapses to
\begin{equation} 
\nu = 2L + \sum_i \Delta_i \, .
\label{eq_nunn} 
\end{equation}
Thus, in terms of naive dimensional analysis or ``Weinberg counting''
\cite{weinberg1990,weinberg1991}, the various orders of the
irreducible graphs which define the chiral $NN$ potential are given by
(cf.\ Fig.~\ref{fig_diagNNLO})
\begin{eqnarray}
V_{\rm LO} & = & V_{\rm ct}^{(0)} + V_{1\pi}^{(0)}
\label{eq_VLO}
\\ V_{\rm NLO} & = & V_{\rm LO} + V_{\rm ct}^{(2)} + V_{1\pi}^{(2)} +
V_{2\pi}^{(2)}
\label{eq_VNLO}
\\ V_{\rm NNLO} & = & V_{\rm NLO} + V_{1\pi}^{(3)} + V_{2\pi}^{(3)}
\label{eq_VNNLO}
\end{eqnarray}
where the superscript denotes the order $\nu$ of the low-momentum
expansion.  LO stands for leading order, NLO for next-to-leading order
and NNLO stands for next-to-next-to leading order.  Contact potentials
carry the subscript ``ct'' and pion-exchange potentials can be
identified by an obvious subscript.

The charge-independent one-pion-exchange (1PE) potential reads
\begin{equation}
V_{1\pi} ({\vec k}~', \vec k) = -
%\frac{1}{(2\pi)^3} \,
\frac{g_A^2}{4f_\pi^2} \: {\vec \tau}_1 \cdot {\vec \tau}_2 \: \frac{
  \vec \sigma_1 \cdot \vec q \,\, \vec \sigma_2 \cdot \vec q} {q^2 +
  m_\pi^2} \,,
\label{eq:eq_1PEci}
\end{equation}
where ${\vec k}~'$ and $\vec k$ represent the final and initial
nucleon momenta in the center-of-mass system (CMS) and $\vec q \equiv
{\vec k}~' - \vec k$ is the momentum transfer; $\vec \sigma_{1,2}$ and
$\vec \tau_{1,2}$ are the spin and isospin operators of nucleon 1 and
2; $g_A$, $f_\pi$, and $m_\pi$ denote axial-vector coupling constant,
the pion decay constant, and the pion mass, respectively.  Since
higher order corrections contribute only to mass and coupling constant
renormalizations and since, on shell, there are no relativistic
corrections, the on-shell 1PE has the form of Eq.~(\ref{eq:eq_1PEci})
to all orders.

It is well known that for high-precision $NN$ potentials, charge
dependence is important.  Therefore, we will take the charge
dependence of the 1PE into account.  Defining a pion-mass dependent
1PE by
\[
V_{1\pi} (m_\pi) \equiv - \,
%\frac{1}{(2\pi)^3} \,
\frac{g_A^2}{4f_\pi^2} \, \frac{ \vec \sigma_1 \cdot \vec q \,\, \vec
  \sigma_2 \cdot \vec q} {q^2 + m_\pi^2} \,,
\]
the 1PE for proton-proton ($pp$) and neutron-neutron ($nn$) scattering
is
\[
V_{1\pi}^{(pp)} ({\vec k}~', \vec k) = V_{1\pi}^{(nn)} ({\vec k}~',
\vec k) = V_{1\pi} (m_{\pi^0}) \,,
\]
while for neutron-proton ($np$) scattering we have
\[
V_{1\pi}^{(np)} ({\vec k}~', \vec k) = -V_{1\pi} (m_{\pi^0}) +
(-1)^{T+1}\, 2\, V_{1\pi} (m_{\pi^\pm}) \,,
\]
where $T$ denotes the isospin of the two-nucleon system.  We use
$m_{\pi^0}=134.9766$ MeV and $m_{\pi^\pm}=139.5702$ MeV.  For the
leading-order, next-to-leading order and NNLO, we refer the reader to
Refs.~\cite{machleidt2011,ekstromPRX}.  The final interaction at
order NNLO is multiplied with the following factors
\cite{machleidt2011},
\begin{equation}
\widehat{V}({\vec k}~',{\vec k}) \equiv \frac{1}{(2\pi)^3}
\sqrt{\frac{M_N}{E_{p'}}}\: {V}({\vec p}~',{\vec p})\:
\sqrt{\frac{M_N}{E_{p}}}
\label{eq_minrel1}
\end{equation}
with $E_p=\sqrt{M_N^2+p^2}$ and where the factor $1/(2\pi)^3$ is just
added for convenience.  The potential $\widehat{V}$ satisfies the
nonrelativistic Lippmann-Schwinger (LS) equation,
\begin{equation}
 \widehat{T}({\vec k}~',{\vec k})= \widehat{V}({\vec k}~',{\vec k})+
 \int d^3p''\: \widehat{V}({\vec k}~',{\vec k}~'')\: \frac{M_N} {{
     k}^{2}-{k''}^{2}+i\epsilon}\: \widehat{T}({\vec k}~'',{\vec k})
 \, .
\label{eq_LS}
\end{equation}
In $pp$ scattering, we use $M_N=M_p=938.2720$ MeV, and in $nn$
scattering, $M_N=M_n=939.5653$ MeV.  Moreover, the on-shell momentum
is simply
\begin{equation}
p^2 = \frac12 M_N T_{\rm lab} \,,
\end{equation}
where $T_{\rm lab}$ denotes the kinetic energy of the incident nucleon
in the laboratory system (``Lab.\ Energy'').  For $np$ scattering, we
have
\begin{eqnarray}
M_N &=& \frac{2M_pM_n}{M_p+M_n} = 938.9182 \mbox{ MeV, and} \\ p^2 & =
& \frac{M_p^2 T_{\rm lab} (T_{\rm lab} + 2M_n)} {(M_p + M_n)^2 +
  2T_{\rm lab} M_p} \,,
\end{eqnarray}
which is based upon relativistic kinematics.

Iteration of $\widehat V$ in the Lippman-Schwinger equation, Eq.~(\ref{eq_LS}),
requires cutting $\widehat V$ off for high momenta to avoid
infinities.  This is consistent with the fact that ChPT is a
low-momentum expansion which is valid only for momenta $Q \ll
\Lambda_\chi \approx 1$ GeV.  Therefore, the potential $\widehat V$ is
multiplied with the regulator function $f(k',k)$,
\begin{equation}
{\widehat V}(\vec{ k}~',{\vec k}) \longmapsto {\widehat V}(\vec{
  k}~',{\vec k}) \, f(k',k)
\end{equation}
with
\begin{equation}
f(p',p) = \exp[-(p'/\Lambda)^{2n}-(p/\Lambda)^{2n}] \,.
\label{eq:eq_f}
\end{equation}


Up to NNLO in chiral perturbation theory there are, in addition to the
two-body interaction diagrams discussed above, also a few three-body
interaction diagrams, see Fig.~\ref{fig_diagNNLO}. In chiral
perturbation theory, the orders are generated systematically, and at a
given chiral order the number of Feynman diagrams is
finite. Consistency requires that a calculation includes all diagrams
which are present at the chosen order. In this work we employ chirally
consistent many-body calculations including all NNLO nuclear
interactions.

There are in total five contact terms that determine the strength of
the NNLO three-nucleon force (3NF); $c_1,c_3,$ and $c_4$ are
associated with the three-body two-pion-exchange (2PE) diagram, $c_D$
and $c_E$ determine the strength of the one-pion-exchange plus contact
(1PE) diagram and the pure contact (CNT) diagram,
respectively. Following the notation of Ref.~\cite{epelbaum2002} the
three-body diagrams are given by
\begin{equation}
V^{\textnormal{2PE}}_{ijk} = \sum_{i \neq j \neq k} \frac{1}{2} \left(
\frac{g_A}{f_{\pi}} \right)^2 \frac{(\vec{\sigma_i} \cdot
  \vec{q_i})(\vec{\sigma_j}\cdot \vec{q_j})}{(\vec{q_i}^2 +
  m_{\pi)}^2)(\vec{q_j}^2 + m_{\pi)}^2)} F_{ijk}^{\alpha \beta}
\tau_i^{\alpha}\tau_{j}^{\beta} \,,
\end{equation}
where $q_{i}$ denotes the momentum transfer associated with nucleon
$i$, and
\begin{align}
F_{ijk}^{\alpha \beta} ={}& \delta^{\alpha \beta} \left[
  -\frac{4c_1m_{\pi}^2}{f_{\pi}^2} +
  \frac{2c_3}{f_{\pi}^2}\vec{q_i}\cdot
  \vec{q_j}\right]\\ {}&+\sum_{\gamma}\frac{c_4}{f_{\pi}^2}\epsilon^{\alpha
  \beta \gamma} \tau_{k}^{\gamma} \vec{\sigma_k} \cdot [\vec{q_i}
  \times \vec{q_j}] \,.
\end{align}
For this diagram, no new parameters are introduced since the
$c_1,c_3,c_4$ appear already in the 2PE two-nucleon interaction. The
remaining two three-body terms are given by
\begin{equation}
V^{\textnormal{1PE}}_{ijk} = -\sum_{i \neq j \neq k}
\frac{g_A}{8f_{\pi}^2} \frac{c_D}{f_{\pi}^2\Lambda_{\chi}} \frac{
  (\vec{\sigma_j} \cdot \vec{q_j})}{(\vec{q_j}^2 + m_{\pi}^2)} (\tau_i
\cdot \tau_j)(\vec{\sigma_i}\cdot \vec{\sigma_j})
\end{equation}
and
\begin{equation}
V^{\textnormal{CNT}}_{ijk} = \frac{1}{2}\sum_{i \neq j \neq k}
\frac{c_E}{f_{\pi}^4 \Lambda_{\chi}} (\tau_i \cdot \tau_j)
\end{equation}
with $\Lambda_{\chi}=700$ MeV. Following~\cite{navratil2007}, we use a
regulator depending on the momentum-transfer $q$,
\begin{equation}
f(q) = \exp[-q^4/\Lambda]
\end{equation}
and thus obtain a local three-body force.

\section{Hartree-Fock theory}\label{sec:chap8hf}

Hartree-Fock (HF) theory is an algorithm for finding an approximative
expression for the ground state of a given Hamiltonian. The basic
ingredients contain a single-particle basis $\{\psi_{\alpha}\}$
defined by the solution of the following eigenvalue problem
\[ 
\hat{h}^{\mathrm{HF}}\psi_{\alpha} =\varepsilon_{\alpha}\psi_{\alpha},
\]
with the Hartree-Fock Hamiltonian defined as
\[
\hat{h}^{\mathrm{HF}}=\hat{t}+\hat{u}_{\mathrm{ext}}+\hat{u}^{\mathrm{HF}}.
\]

The term $\hat{u}^{\mathrm{HF}}$ is a single-particle potential to be
determined by the HF algorithm. The HF algorithm means to choose
$\hat{u}^{\mathrm{HF}}$ in order to have
\[ \langle \hat{H} \rangle = E^{\mathrm{HF}}= \langle \Phi_0^{HF} | \hat{H}|\Phi_0^{HF} \rangle,
\]
as a local minimum with a Slater determinant
$\Phi_0^{HF}$ being the ansatz for the ground state.  The variational
principle ensures that $E^{\mathrm{HF}} \ge E_0$, with $E_0$ the representing the exact
ground state energy.

We will show that the Hartree-Fock Hamiltonian $\hat{h}^{\mathrm{HF}}$
equals our definition of the operator $\hat{f}$ discussed in
connection with the new definition of the normal-ordered Hamiltonian,
that is we have, for a specific matrix element
\[
\langle p |\hat{h}^{\mathrm{HF}}| q \rangle =\langle p |\hat{f}| q
\rangle=\langle p|\hat{t}+\hat{u}_{\mathrm{ext}}|q \rangle +\sum_{i\le
  F} \langle pi | \hat{V} | qi\rangle,
\]
meaning that
\[
\langle p|\hat{u}^{\mathrm{HF}}|q\rangle = \sum_{i\le F} \langle pi |
\hat{V} | qi\rangle.
\]
The so-called Hartree-Fock potential $\hat{u}^{\mathrm{HF}}$ brings an
explicit medium dependence due to the summation over all
single-particle states below the Fermi level $F$. It brings also in an
explicit dependence on the two-body interaction (in nuclear physics we
can also have complicated three- or higher-body forces). The two-body
interaction, with its contribution from the other bystanding fermions,
creates an effective mean field in which a given fermion moves, in
addition to the external potential $\hat{u}_{\mathrm{ext}}$ which
confines the motion of the fermion. For systems like nuclei or
infinite nuclear matter, there is no external confining
potential. Nuclei and nuclear matter are examples of self-bound
systems, where the binding arises due to the intrinsic nature of the
strong force. For nuclear systems thus, there would be no external
one-body potential in the Hartree-Fock Hamiltonian.


Another possibility is to expand the single-particle functions in a
known basis and vary the coefficients, that is, the new
single-particle wave function is written as a linear expansion in
terms of a fixed chosen orthogonal basis (for example the well-known
harmonic oscillator functions or the hydrogen-like functions etc).  We
define our new Hartree-Fock single-particle basis by performing a
unitary transformation on our previous basis (labelled with greek
indices) as
\begin{equation}
\psi_p^{HF} = \sum_{\lambda}
C_{p\lambda}\phi_{\lambda}. \label{eq:newbasis}
\end{equation}
In this case we vary the coefficients $C_{p\lambda}$. If the basis has
infinitely many solutions, we need to truncate the above sum.  We
assume that the basis $\phi_{\lambda}$ is orthogonal. A unitary
transformation keeps the orthogonality, as discussed in problem
\ref{problem:unitarity} below.


It is normal to choose a single-particle basis defined as the
eigenfunctions of parts of the full Hamiltonian. The typical situation
consists of the solutions of the one-body part of the Hamiltonian,
that is we have
\[
\hat{h}_0\phi_{\lambda}=\epsilon_{\lambda}\phi_{\lambda}.
\]
For infinite nuclear matter $\hat{h}_0$ is given by the kinetic energy
operator and the states are given by plane wave functions. Due to the
translational invariance of the two-body interaction, the Hartree-Fock
single-particle eigenstates are also given by the same functions. For
infinite matter thus, it is only the single-particle energies that
change when we solve the Hartree-Fock equations.

The single-particle wave functions $\phi_{\lambda}({\bf r})$, defined
by the quantum numbers $\lambda$ and ${\bf r}$ are defined as the
overlap
\[
   \phi_{\lambda}({\bf r}) = \langle {\bf r} | \lambda \rangle .
\]
In our discussions we will use our definitions of single-particle
states above and below the Fermi ($F$).  


We use greek letters to refer to our original single-particle
basis. The expectation value for the energy with the ansatz $\Phi_0$
for the ground state reads (see problem \ref{problem:referenceE}, with
application to infinite nuclear matter)
\[
  E[\Phi_0] = \sum_{\mu=1}^A \langle \mu | h | \mu \rangle +
  \frac{1}{2}\sum_{{\mu}=1}^A\sum_{{\nu}=1}^A \langle
  \mu\nu|\hat{v}|\mu\nu\rangle.
\]
Now we are interested in defining a new basis defined in terms of a
chosen basis as defined in Eq.~(\ref{eq:newbasis}). We can then
rewrite the energy functional as
\begin{equation}
  E[\Phi^{HF}] = \sum_{i\le F} \langle i | h | i \rangle +
  \frac{1}{2}\sum_{ij\le F}\langle
  ij|\hat{v}|ij\rangle, \label{FunctionalEPhi2}
\end{equation}
where $\Phi^{HF}$ is the new Slater determinant defined by the new
basis of Eq.~(\ref{eq:newbasis}).





Using Eq.~(\ref{eq:newbasis}) we can rewrite
Eq.~(\ref{FunctionalEPhi2}) as
\begin{equation}
  E[\Psi] = \sum_{i\le F} \sum_{\alpha\beta}
  C^*_{i\alpha}C_{i\beta}\langle \alpha | h | \beta \rangle +
  \frac{1}{2}\sum_{ij\le F}\sum_{{\alpha\beta\gamma\delta}}
  C^*_{i\alpha}C^*_{j\beta}C_{i\gamma}C_{j\delta}\langle
  \alpha\beta|\hat{v}|\gamma\delta\rangle. \label{FunctionalEPhi3}
\end{equation}


We wish now to minimize the above functional. We introduce a set of
Lagrange multipliers, noting that since $\langle i | j \rangle =
\delta_{i,j}$ and $\langle \alpha | \beta \rangle =
\delta_{\alpha,\beta}$, the coefficients $C_{i\gamma}$ obey the
relation
\[
 \langle i | j \rangle=\delta_{i,j}=\sum_{\alpha\beta}
 C^*_{i\alpha}C_{i\beta}\langle \alpha | \beta \rangle= \sum_{\alpha}
 C^*_{i\alpha}C_{i\alpha},
\]
which allows us to define a functional to be minimized that reads
\begin{equation}
  F[\Phi^{HF}]=E[\Phi^{HF}] - \sum_{i\le F}\epsilon_i\sum_{\alpha}
  C^*_{i\alpha}C_{i\alpha}.
\end{equation}


Minimizing with respect to $C^*_{i\alpha}$ (the equations for
$C^*_{i\alpha}$ and $C_{i\alpha}$ can be written as two independent
equations) we obtain
\[
\frac{d}{dC^*_{i\alpha}}\left[ E[\Phi^{HF}] -
  \sum_{j}\epsilon_j\sum_{\alpha} C^*_{j\alpha}C_{j\alpha}\right]=0,
\]
which yields for every single-particle state $i$ and index $\alpha$
(recalling that the coefficients $C_{i\alpha}$ are matrix elements of
a unitary matrix, or orthogonal for a real symmetric matrix) the
following Hartree-Fock equations
\[
\sum_{\beta} C_{i\beta}\langle \alpha | h | \beta \rangle+
\sum_{j\le F}\sum_{\beta\gamma\delta}
C^*_{j\beta}C_{j\delta}C_{i\gamma}\langle
\alpha\beta|\hat{v}|\gamma\delta\rangle=\epsilon_i^{HF}C_{i\alpha}.
\]


We can rewrite this equation as (changing dummy variables)
\[
\sum_{\beta} \left\{\langle \alpha | h | \beta \rangle+
\sum_{j\le F}\sum_{\gamma\delta} C^*_{j\gamma}C_{j\delta}\langle
\alpha\gamma|\hat{v}|\beta\delta\rangle\right\}C_{i\beta}=\epsilon_i^{HF}C_{i\alpha}.
\]
Note that the sums over greek indices run over the number of basis set
functions (in principle an infinite number).

Defining
\[
h_{\alpha\beta}^{HF}=\langle \alpha | h | \beta \rangle+
\sum_{j\le F}\sum_{\gamma\delta} C^*_{j\gamma}C_{j\delta}\langle
\alpha\gamma|\hat{v}|\beta\delta\rangle,
\]
we can rewrite the new equations as
\begin{equation}
\sum_{\gamma}h_{\alpha\beta}^{HF}C_{i\beta}=\epsilon_i^{HF}C_{i\alpha}. \label{eq:newhf}
\end{equation}
The latter is nothing but a standard eigenvalue problem.  Our
Hartree-Fock matrix is thus
\[
\hat{h}_{\alpha\beta}^{HF}=\langle \alpha | \hat{h}_0 | \beta \rangle+
\sum_{j\le F}\sum_{\gamma\delta} C^*_{j\gamma}C_{j\delta}\langle
\alpha\gamma|\hat{v}|\beta\delta\rangle.
\]
The Hartree-Fock equations are solved in an iterative way starting
with a guess for the coefficients $C_{j\gamma}=\delta_{j,\gamma}$ and
solving the equations by diagonalization till the new single-particle
energies $\epsilon_i^{\mathrm{HF}}$ do not change anymore by a
prefixed quantity.

Normally we assume that the single-particle basis $|\beta\rangle$
forms an eigenbasis for the operator $\hat{h}_0$, meaning that the
Hartree-Fock matrix becomes
\[
\hat{h}_{\alpha\beta}^{HF}=\epsilon_{\alpha}\delta_{\alpha,\beta}+
\sum_{j\le F}\sum_{\gamma\delta} C^*_{j\gamma}C_{j\delta}\langle
\alpha\gamma|\hat{v}|\beta\delta\rangle.
\]

\subsection{Hartree-Fock algorithm with simple Python code}

The equations are often rewritten in terms of a so-called density matrix,
which is defined as 
\begin{equation}
\rho_{\gamma\delta}=\sum_{i\le F}\langle\gamma|i\rangle\langle i|\delta\rangle = \sum_{i=1}^{N}C_{i\gamma}C^*_{i\delta}.
\end{equation}
It means that we can rewrite the Hartree-Fock Hamiltonian as
\[
\hat{h}_{\alpha\beta}^{HF}=\epsilon_{\alpha}\delta_{\alpha,\beta}+
\sum_{\gamma\delta} \rho_{\gamma\delta}\langle \alpha\gamma|V|\beta\delta\rangle.
\]
It is convenient to use the density matrix since we can precalculate in every iteration the product of the eigenvector components $C$. 

The Hartree-Fock equations are, in their simplest form, solved in an
iterative way, starting with a guess for the coefficients
$C_{i\alpha}$. We label the coefficients as $C_{i\alpha}^{(n)}$, where
the superscript $n$ stands for iteration $n$.  To set up the algorithm
we can proceed as follows.

\begin{svgraybox}
\begin{enumerate}
\item We start with a guess
  $C_{i\alpha}^{(0)}=\delta_{i,\alpha}$. Alternatively, we could have
  used random starting values as long as the vectors are
  normalized. Another possibility is to give states below the Fermi
  level a larger weight. We construct then the density matrix and the 
Hartree-Fock Hamiltonian. 
\item The Hartree-Fock matrix simplifies then to
\[
\hat{h}_{\alpha\beta}^{HF}(0)=\epsilon_{\alpha}\delta_{\alpha,\beta}+
\sum_{\gamma\delta} \rho_{\gamma\delta}^{(0)}\langle \alpha\gamma|V|\beta\delta\rangle.
\]
Solving the Hartree-Fock eigenvalue problem yields then new eigenvectors $C_{i\alpha}^{(1)}$ and eigenvalues
$\epsilon_i^{\mathrm{HF}}(1)$. 
\item With the new eigenvectors we can set up a new Hartree-Fock potential 
\[
\sum_{\gamma\delta} \rho_{\gamma\delta}^{(1)}\langle \alpha\gamma|V|\beta\delta\rangle.
\]
The diagonalization with the new Hartree-Fock potential yields new eigenvectors and eigenvalues.
\item This process is continued till a user defined test is
  satisfied. As an example, we can require that
\[
\frac{\sum_{p} |\epsilon_i^{(n)}-\epsilon_i^{(n-1)}|}{m} \le \lambda,
\]
where $\lambda$ is a small number defined by the user ($\lambda \sim
10^{-8}$ or smaller) and $p$ runs over all calculated single-particle
energies and $m$ is the number of single-particle states.

\end{enumerate}
\end{svgraybox}
The following simple Python program implements the above algorithm using the density matrix formalism outlined above.
We have omitted the functions that set up the single-particle basis and the anti-symmetrized two-body interaction matrix elements.
These have to be provided, see \url{https://github.com/ManyBodyPhysics/LectureNotesPhysics/blob/master/doc/src/Chapter8-programs/python/hfnuclei.py}
for full code and matrix elements.
\begin{lstlisting}
# We skip here functions that set up the one- and two-body parts of the Hamiltonian 
# These functions need to be defined by the user. The two-body interaction below is
# calculated by calling the function TwoBodyInteraction(alpha,gamma,beta,delta)
# Similalry, the one-body part is computed by the function singleparticleH(alpha)
# We have omitted specific quantum number tests as well (isopsin conservation, 
# momentum conservation etc)
import numpy as np 
from decimal import Decimal

if __name__ == '__main__':
	
	""" Star HF-iterations, preparing variables and density matrix """

        """ Coefficients for setting up density matrix, assuming only one along the diagonals """
	C = np.eye(spOrbitals) # HF coefficients
        DensityMatrix = np.zeros([spOrbitals,spOrbitals])
        for gamma in range(spOrbitals):
            for delta in range(spOrbitals):
                sum = 0.0
                for i in range(Nparticles):
                    sum += C[gamma][i]*C[delta][i]
                DensityMatrix[gamma][delta] = Decimal(sum)
        maxHFiter = 100
        epsilon =  1.0e-5 
        difference = 1.0
	hf_count = 0
	oldenergies = np.zeros(spOrbitals)
	newenergies = np.zeros(spOrbitals)
	while hf_count < maxHFiter and difference > epsilon:
   	        HFmatrix = np.zeros([spOrbitals,spOrbitals])		
		for alpha in range(spOrbitals):
			for beta in range(spOrbitals):
                            """  Setting up the Fock matrix using the density matrix and antisymmetrized two-body interaction  """
     		            sumFockTerm = 0.0
                            for gamma in range(spOrbitals):
                                for delta in range(spOrbitals):
                                    sumFockTerm += DensityMatrix[gamma][delta]*
                                                   TwoBodyInteraction(alpha,gamma,beta,delta)
                            HFmatrix[alpha][beta] = Decimal(sumFockTerm)
                            """  Adding the one-body term """
                            if beta == alpha:   HFmatrix[alpha][alpha] += singleparticleH(alpha)
		spenergies, C = np.linalg.eigh(HFmatrix)
                """ Setting up new density matrix """
                DensityMatrix = np.zeros([spOrbitals,spOrbitals])
                for gamma in range(spOrbitals):
                    for delta in range(spOrbitals):
                        sum = 0.0
                        for i in range(Nparticles):
                            sum += C[gamma][i]*C[delta][i]
                        DensityMatrix[gamma][delta] = Decimal(sum)
		newenergies = spenergies
                """ Brute force computation of difference between previous and new sp HF energies """
                sum =0.0
                for i in range(spOrbitals):
                    sum += (abs(newenergies[i]-oldenergies[i]))/spOrbitals
                difference = sum
                oldenergies = newenergies
                print "Single-particle energies, ordering may have changed "
                for i in range(spOrbitals):
                    print('{0:4d}  {1:.4f}'.format(i, Decimal(oldenergies[i])))
		hf_count += 1
\end{lstlisting}





  We end this section by rewriting the ground state energy by adding
  and subtracting $\hat{u}^{HF}$
  \[
    E_0^{HF} =\langle \Phi_0 | \hat{H} | \Phi_0\rangle = \sum_{i\le
      F}^A \langle i | \hat{h}_0 +\hat{u}^{HF}| j\rangle+
    \frac{1}{2}\sum_{i\le F}^A\sum_{j \le F}^A\left[\langle ij
      |\hat{v}|ij \rangle-\langle
      ij|\hat{v}|ji\rangle\right]-\sum_{i\le F}^A \langle i
    |\hat{u}^{HF}| i\rangle,
  \]
  which results in
  \[
    E_0^{HF} = \sum_{i\le F}^A \varepsilon_i^{HF} +
    \frac{1}{2}\sum_{i\le F}^A\sum_{j \le F}^A\langle ij\vert\hat{v}\vert ij \rangle-\sum_{i\le F}^A \langle i\vert\hat{u}^{HF}\vert i\rangle.
  \]
  Our single-particle states $ijk\dots$ are now single-particle states
  obtained from the solution of the Hartree-Fock equations.



  Using our definition of the Hartree-Fock single-particle energies we
  obtain then the following expression for the total ground-state
  energy
  \[
    E_0^{HF} = \sum_{i\le F}^A \varepsilon_i - \frac{1}{2}\sum_{i\le
      F}^A\sum_{j \le F}^A\left[\langle ij |\hat{v}|ij \rangle-\langle
      ij|\hat{v}|ji\rangle\right].
  \]
  This equation demonstrates that the total energy is not given as the
  sum of the individual single-particle energies.

  \section{Full Configuration Interaction Theory}\label{sec:chap8fci}

  Full configuration theory (FCI), which represents a discretized
  variant of the continuous eigenvalue problem, allows for, in
  principle, an exact (to numerical precision) solution of Schr\"odinger's equation
  for many interacting fermions or bosons with a given basis set. This
  basis set defines an effective Hilbert space.  For fermionic
  problems, the standard approach is to define an upper limit for the
  set of singel-particle states. As an example, if we use the harmonic
  oscillator one-body Hamiltonian to generate an orthogonal
  single-particle basis, truncating the basis at some oscillator
  excitation energy provides thereby an upper limit.  Similarly,
  truncating the maximum values of $n_{x,y,z}$ for plane wave states
  with periodic boundary conditions, yields a similar upper limit.
  Table \ref{tab:table1} lists several possible truncations to the
  basis set in terms of the single-particle energies as functions of
  $n_{x,y,z}$.  This single-particle basis is then used to define all
  possible Slater determinants which can be constructed with a given
  number of fermions $A$.  The total number of Slater determinants
  determines thereafter the dimensionality of the Hamiltonian matrix
  and thereby an effective Hilbert space. 
  If we are able to set up the Hamiltonian matrix
  and solve the pertinent eigenvalue problem within this basis set,
  FCI provides numerically exact solutions to all states of interest
  for a given many-body problem. The dimensionality of the problem
  explodes however quickly. To see this it suffices to consider
  the total number of Slater determinants which can be built with say
  $N$ neutrons distributed among $n$ single-particle states. The total number is
  \[
  \left (\begin{array}{c} n \\ N\end{array} \right)
    =\frac{n!}{(n-N)!N!}.
  \]
  As an example, for a model space which comprises the first four
  major harmonic oscillator shells only, that is the $0s$, $0p$,
  $1s0d$ and $1p0f$ shells we have $40$ single particle states for
  neutrons and protons.  For the eight neutrons of oxygen-16 we would
  then have
  \[
  \left (\begin{array}{c} 40 \\ 8\end{array} \right)
    =\frac{40!}{(32)!8!}\sim 8\times 10^{7},
  \]
  possible Slater determinants. Multiplying this with the number of
  proton Slater determinants we end up with approximately $d\sim 10^{15}$ 
  possible Slater determinants and a Hamiltonian matrix of
  dimension $10^{15}\times 10^{15}$, an intractable problem if we wish to diagonalize the Hamiltonian matrix. The
  dimensionality can be reduced if we look at specific symmetries,
  however these symmetries will never reduce the problem to
  dimensionalities which can be handled by standard eigenvalue
  solvers. These are normally lumped into two main categories, direct
  solvers for matrices of dimensionalities which are smaller than
  $d\sim 10^5$, and iterative eigenvalue solvers (when only selected
  states are being sought after) for dimensionalities up to
  $10^{10}\times 10^{10}$.

  Due to its discreteness thus, the effective Hilbert space will
  always represent an approximation to the full continuous problem.
  However, with a given Hamiltonian matrix and effective Hilbert
  space, FCI provides us with true benchmarks that can convey
  important information on correlations beyond Hartree-Fock theory and
  various approximative many-body methods like many-body perturbation
  theory, coupled cluster theory, Green's function theory and the
  Similarity Renormalization Group approach. These methods are all
  discussed in this text.  Assuming that we can diagonalize the
  Hamiltonian matrix, and thereby obtain the exact solutions, this
  section serves the aim to link the exact solution obtained from FCI with various
  approximative methods, hoping thereby that eventual differences can
  shed light on which correlations  play a major role and should be
  included in the above approximative methods. The simple pairing
  model discussed in problem \ref{problem:pairingmodel} is an example
  of a system that allows us to compare exact solutions with
  those defined by many-body perturbation theory to a given order in
  the interaction, coupled cluster theory, Green's function theory and
  the Similarity Renormalization Group (SRG). Many-body perturbation
  theory and coupled cluster theory are discussed in this chapter,
  while various SRG approaches are discussed in Chapter 10. Green's
  function theory is discussed in Chapter 11.

  In order to familiarize the reader with these approximative
  many-body methods, we start with the general definition of the full
  configuration interaction problem.

  We have defined the ansatz for the ground state as
  \[
  |\Phi_0\rangle = \left(\prod_{i\le
    F}\hat{a}_{i}^{\dagger}\right)|0\rangle,
  \]
  where the variable  $i$ defines different single-particle states up to
  the Fermi level. We have assumed that we have $A$ nucleons and that the chosen single-particle basis are eigenstates of 
the one-body Hamiltonian $\hat{h}_0$ (defining thereby an orthogonal basis set).  
A given
  one-particle-one-hole ($1p1h$) state can be written as
  \[
  |\Phi_i^a\rangle = \hat{a}_{a}^{\dagger}\hat{a}_i|\Phi_0\rangle,
  \]
  while a $2p2h$ state can be written as
  \[
  |\Phi_{ij}^{ab}\rangle =
  \hat{a}_{a}^{\dagger}\hat{a}_{b}^{\dagger}\hat{a}_j\hat{a}_i|\Phi_0\rangle,
  \]
  and a general $ApAh$ state as
  \[
  |\Phi_{ijk\dots}^{abc\dots}\rangle =
  \hat{a}_{a}^{\dagger}\hat{a}_{b}^{\dagger}\hat{a}_{c}^{\dagger}\dots\hat{a}_k\hat{a}_j\hat{a}_i|\Phi_0\rangle.
  \]

  As before, we use letters $ijkl\dots$ for states below the Fermi level and
  $abcd\dots$ for states above the Fermi level. A general
  single-particle state is given by letters $pqrs\dots$.

  We can then expand our exact state function for the ground state as
  \[
  |\Psi_0\rangle=C_0|\Phi_0\rangle+\sum_{ai}C_i^a|\Phi_i^a\rangle+\sum_{abij}C_{ij}^{ab}|\Phi_{ij}^{ab}\rangle+\dots
  =(C_0+\hat{C})|\Phi_0\rangle,
  \]
  where we have introduced the so-called correlation operator
  \[
  \hat{C}=\sum_{ai}C_i^a\hat{a}_{a}^{\dagger}\hat{a}_i
  +\sum_{abij}C_{ij}^{ab}\hat{a}_{a}^{\dagger}\hat{a}_{b}^{\dagger}\hat{a}_j\hat{a}_i+\dots
  \]
  Since the normalization of $\Psi_0$ is at our disposal and since
  $C_0$ is by assumption not zero, we may arbitrarily set $C_0=1$ with
  corresponding proportional changes in all other coefficients. Using
  this so-called intermediate normalization we have
  \[
  \langle \Psi_0 | \Phi_0 \rangle = \langle \Phi_0 | \Phi_0 \rangle =
  1,
  \]
  resulting in
  \[
  |\Psi_0\rangle=(1+\hat{C})|\Phi_0\rangle.
  \]


  We rewrite
  \[
  |\Psi_0\rangle=C_0|\Phi_0\rangle+\sum_{ai}C_i^a|\Phi_i^a\rangle+\sum_{abij}C_{ij}^{ab}|\Phi_{ij}^{ab}\rangle+\dots,
  \]
  in a more compact form as
  \[
  |\Psi_0\rangle=\sum_{PH}C_H^P\Phi_H^P=\left(\sum_{PH}C_H^P\hat{A}_H^P\right)|\Phi_0\rangle,
  \]
  where $H$ stands for $0,1,\dots,n$ hole states and $P$ for
  $0,1,\dots,n$ particle states. The operator $\hat{A}_H^P$ represents
  a given set of particle-hole excitations. For a two-particle-to-hole
  excitation this operator is given by
  $\hat{A}_{2h}^{2p}=\hat{a}_{a}^{\dagger}\hat{a}_{b}^{\dagger}\hat{a}_j\hat{a}_i$.
  Our requirement of unit normalization gives
  \[
  \langle \Psi_0 | \Psi_0 \rangle = \sum_{PH}|C_H^P|^2= 1,
  \]
  and the energy can be written as
  \[
  E= \langle \Psi_0 | \hat{H} |\Psi_0 \rangle=
  \sum_{PP'HH'}C_H^{*P}\langle \Phi_H^P | \hat{H} |\Phi_{H'}^{P'}
  \rangle C_{H'}^{P'}.
  \]
  The last equation is normally solved by diagonalization, with the
  Hamiltonian matrix defined by the basis of all possible Slater
  determinants. A diagonalization is equivalent to finding the
  variational minimum of
  \[
   \langle \Psi_0 | \hat{H} |\Psi_0 \rangle-\lambda \langle \Psi_0
   |\Psi_0 \rangle,
  \]
  where $\lambda$ is a variational multiplier to be identified with
  the energy of the system.

  The minimization process results in
  \begin{align}
  0=&\delta\left[ \langle \Psi_0 | \hat{H} |\Psi_0 \rangle-\lambda
    \langle \Psi_0 |\Psi_0 \rangle\right]\\
  =& \sum_{P'H'}\left\{\delta[C_H^{*P}]\langle \Phi_H^P | \hat{H}
  |\Phi_{H'}^{P'} \rangle C_{H'}^{P'}+ C_H^{*P}\langle \Phi_H^P |
  \hat{H} |\Phi_{H'}^{P'} \rangle \delta[C_{H'}^{P'}]- \lambda(
  \delta[C_H^{*P}]C_{H'}^{P'}+C_H^{*P}\delta[C_{H'}^{P'}]\right\}.
  \end{align}
  Since the coefficients $\delta[C_H^{*P}]$ and $\delta[C_{H'}^{P'}]$
  are complex conjugates it is necessary and sufficient to require the
  quantities that multiply with $\delta[C_H^{*P}]$ to vanish.

  This leads to
  \[
  \sum_{P'H'}\langle \Phi_H^P | \hat{H} |\Phi_{H'}^{P'} \rangle
  C_{H'}^{P'}-\lambda C_H^{P}=0,
  \]
  for all sets of $P$ and $H$.

  If we then multiply by the corresponding $C_H^{*P}$ and sum over
  $PH$ we obtain
  \[ 
  \sum_{PP'HH'}C_H^{*P}\langle \Phi_H^P | \hat{H} |\Phi_{H'}^{P'}
  \rangle C_{H'}^{P'}-\lambda\sum_{PH}|C_H^P|^2=0,
  \]
  leading to the identification $\lambda = E$. This means that we have
  for all $PH$ sets
  \begin{equation}
  \sum_{P'H'}\langle \Phi_H^P | \hat{H} -E|\Phi_{H'}^{P'} \rangle =
  0. \label{eq:fullci}
  \end{equation}



  An alternative way to derive the last equation is to start from
  \[
  (\hat{H} -E)|\Psi_0\rangle = (\hat{H}
  -E)\sum_{P'H'}C_{H'}^{P'}|\Phi_{H'}^{P'} \rangle=0,
  \]
  and if this equation is successively projected against all
  $\Phi_H^P$ in the expansion of $\Psi$, we end up with
  Eq.~(\ref{eq:fullci}).

  If we are able to solve this equation by numerical diagonalization in
  a large Hilbert space (it will be truncated in terms of the number
  of single-particle states included in the definition of Slater
  determinants), it can then serve as a benchmark for other many-body
  methods which approximate the correlation operator $\hat{C}$.  Our
  pairing model discussed in problem \ref{problem:pairingmodel} is an
  example of a system which can be diagonalized exactly, providing
  thereby benchmarks for different approximative methods.


  To better understand the meaning of possible configurations and the
  derivation of a Hamiltonian matrix, we consider here a simple
  example of six fermions. We assume we can make an ansatz for the
  ground state with all six fermions below the Fermi level. We 
  label this state as a zero-particle-zero-hole state $0p-0h$. With
  six nucleons we can make at most $6p-6h$ excitations. If we have an
  infinity of single particle states above the Fermi level, we will
  obviously have an infinity of say $2p-2h$ excitations. Each specific way
  to distribute the particles represents a configuration. We will
  always have to truncate  the basis of single-particle states.
  This gives us a finite number of possible Slater determinants. Our
  Hamiltonian matrix would then look like (where each block which is
  marked with an $x$ can contain a large quantity of non-zero matrix
  elements) as shown here
  \begin{table}[h]
  \begin{center}
  \begin{tabular}{cccccccc}
  \hline \multicolumn{1}{c}{ } & \multicolumn{1}{c}{ $0p-0h$ } &
  \multicolumn{1}{c}{ $1p-1h$ } & \multicolumn{1}{c}{ $2p-2h$ } &
  \multicolumn{1}{c}{ $3p-3h$ } & \multicolumn{1}{c}{ $4p-4h$ } &
  \multicolumn{1}{c}{ $5p-5h$ } & \multicolumn{1}{c}{ $6p-6h$ }
  \\ \hline $0p-0h$ & x & x & x & 0 & 0 & 0 & 0 \\ $1p-1h$ & x & x & x
  & x & 0 & 0 & 0 \\ $2p-2h$ & x & x & x & x & x & 0 & 0 \\ $3p-3h$ &
  0 & x & x & x & x & x & 0 \\ $4p-4h$ & 0 & 0 & x & x & x & x & x
  \\ $5p-5h$ & 0 & 0 & 0 & x & x & x & x \\ $6p-6h$ & 0 & 0 & 0 & 0 &
  x & x & x \\ \hline
  \end{tabular}
  \end{center}
  \end{table}
  if the Hamiltonian contains at most a two-body interaction, as
  demonstrated in problem \ref{problem:hamiltoniansetup}.  If we use a
  so-called canonical Hartree-Fock basis \cite{shavittbartlett2009}, this corresponds to a particular unitary
  transformation where matrix elements of the type 
$\langle 0p-0h\vert \hat{H} \vert 1p-1h\rangle =\langle \Phi_0 |\hat{H}|\Phi_{i}^{a}\rangle=0$.
With a canonical Hartree-Fock basis our Hamiltonian matrix reads
  \begin{table}[h]
  \begin{center}
  \begin{tabular}{cccccccc}
  \hline \multicolumn{1}{c}{ } & \multicolumn{1}{c}{ $0p-0h$ } &
  \multicolumn{1}{c}{ $1p-1h$ } & \multicolumn{1}{c}{ $2p-2h$ } &
  \multicolumn{1}{c}{ $3p-3h$ } & \multicolumn{1}{c}{ $4p-4h$ } &
  \multicolumn{1}{c}{ $5p-5h$ } & \multicolumn{1}{c}{ $6p-6h$ }
  \\ \hline $0p-0h$ & $\tilde{x}$ & 0 & $\tilde{x}$ & 0 & 0 & 0 & 0
  \\ $1p-1h$ & 0 & $\tilde{x}$ & $\tilde{x}$ & $\tilde{x}$ & 0 & 0 & 0
  \\ $2p-2h$ & $\tilde{x}$ & $\tilde{x}$ & $\tilde{x}$ & $\tilde{x}$ &
  $\tilde{x}$ & 0 & 0 \\ $3p-3h$ & 0 & $\tilde{x}$ & $\tilde{x}$ &
  $\tilde{x}$ & $\tilde{x}$ & $\tilde{x}$ & 0 \\ $4p-4h$ & 0 & 0 &
  $\tilde{x}$ & $\tilde{x}$ & $\tilde{x}$ & $\tilde{x}$ & $\tilde{x}$
  \\ $5p-5h$ & 0 & 0 & 0 & $\tilde{x}$ & $\tilde{x}$ & $\tilde{x}$ &
  $\tilde{x}$ \\ $6p-6h$ & 0 & 0 & 0 & 0 & $\tilde{x}$ & $\tilde{x}$ &
  $\tilde{x}$ \\ \hline
  \end{tabular}
  \end{center}
  \end{table}
  If we do not make any truncations in the possible sets of Slater
  determinants (many-body states) we can make by distributing $A$
  nucleons among $n$ single-particle states, we call such a
  calculation for a full configuration interaction (FCI) approach.  If
  we make truncations, we have several different possibilities to
  reduce the dimensionality of the problem.  A well-known example is
  the standard nuclear shell-model. For the shell model, we define an
  effective Hilbert space with respect to a given core. The
  calculations are normally then performed for all many-body states
  that can be constructed from the effective Hilbert spaces. This
  approach requires a properly defined effective Hamiltonian.  Another
  possibility to constrain the dimensionality of the problem is to
  truncate in the number of excitations. As an example, we can limit
  the possible Slater determinants to only $1p-1h$ and $2p-2h$
  excitations. This is called a configuration interaction calculation
  at the level of singles and doubles excitations. If we truncate at
  the level of three-particle-three-hole excitations we end up with
  singles, doubles and triples excitations.  Such truncations reduce
  considerably the size of the Hamiltonian matrices to be
  diagonalized, but can lead to so-called unlinked contributions, and
  thereby wrong results, for a given expectation value
  \cite{barrettmusial2007}.  A third possibility is to constrain the
  number of excitations by an energy cutoff. This cutoff defines a
  maximum excitation energy. The maximum excitation energy is normally given by the 
sum of single-particle energies defined by the unperturbed one-body part of the Hamiltonian.
A commonly used basis in nuclear physics is the harmonic oscillator. The 
cutoff in energy is then defined by the maximum number of harmonic oscillator excitations.
If we do not define a core, this defines normally what is
  called the no-core shell-model approach \cite{navratil2009}.


  \subsection{A non-practical way of solving the eigenvalue problem}

  For reasons to come (links with coupled cluster theory and many-body
  perturbation theory), we will rewrite Eq.~(\ref{eq:fullci}) as a set
  of coupled non-linear equations in terms of the unknown coefficients
  $C_H^P$.  To obtain the eigenstates and eigenvalues in terms of
  non-linear equations is, from a computational perspective as 
  inefficient as finding the roots (and thereby eigenvalues) of the
  characteristic polynomial. However, this digression serves the scope
  of linking full configuration interaction theory with approximative
  solutions to the many-body problem.

  To see this, we look at the contributions arising from
  \[
  \langle \Phi_H^P | = \langle \Phi_0|
  \]
  in Eq.~(\ref{eq:fullci}), that is we multiply with $\langle \Phi_0|$ from the left in
  \[
  (\hat{H} -E)\sum_{P'H'}C_{H'}^{P'}|\Phi_{H'}^{P'} \rangle=0.
  \]
  If we assume that we have a two-body operator at most, the
  Slater-Condon rule for a two-body interaction, see problem
  \ref{problem:hamiltoniansetup}, results in an expression for the
  correlation energy in terms of $C_i^a$ and $C_{ij}^{ab}$ only, namely
  \[
  \langle \Phi_0 | \hat{H} -E| \Phi_0\rangle + \sum_{ai}\langle \Phi_0
  | \hat{H} -E|\Phi_{i}^{a} \rangle C_{i}^{a}+ \sum_{abij}\langle
  \Phi_0 | \hat{H} -E|\Phi_{ij}^{ab} \rangle C_{ij}^{ab}=0,
  \]
  or
  \[
  E-E_{\mathrm{Ref}} =\Delta E=\sum_{ai}\langle \Phi_0 |
  \hat{H}|\Phi_{i}^{a} \rangle C_{i}^{a}+ \sum_{abij}\langle \Phi_0 |
  \hat{H}|\Phi_{ij}^{ab} \rangle C_{ij}^{ab},
  \]
  where the energy $E_{\mathrm{Ref}}$ is the reference energy and
  $\Delta E$ defines the so-called correlation energy.  The
  single-particle basis functions could  result from  a
  Hartree-Fock calculation or they could be the eigenstates of the one-body operator that defined the
  non-interacting part of the Hamiltonian.

  In our Hartree-Fock discussions, we have already computed
  the matrix $\langle \Phi_0 | \hat{H}|\Phi_{i}^{a}\rangle $ and
  $\langle \Phi_0 | \hat{H}|\Phi_{ij}^{ab}\rangle$.  If we are using a
  Hartree-Fock basis we have $\langle \Phi_0 | \hat{H}|\Phi_{i}^{a}\rangle=0$
  and we are left with a \emph{correlation energy} given by
  \[
  E-E_{\mathrm{Ref}} =\Delta E^{HF}=\sum_{abij}\langle \Phi_0 |
  \hat{H}|\Phi_{ij}^{ab} \rangle C_{ij}^{ab}.
  \]


  Inserting the various matrix elements we can rewrite the previous
  equation as
  \begin{equation}\label{eq:correlationenergy}
  \Delta E=\sum_{ai}\langle i| \hat{f}|a \rangle C_{i}^{a}+
  \sum_{abij}\langle ij | \hat{v}| ab \rangle C_{ij}^{ab}.
  \end{equation}
  This equation determines the correlation energy but not the
  coefficients $C$.  We need more equations. Our next step is to set
  up
  \[ 
  \langle \Phi_i^a | \hat{H} -E| \Phi_0\rangle + \sum_{bj}\langle
  \Phi_i^a | \hat{H} -E|\Phi_{j}^{b} \rangle C_{j}^{b}+
  \sum_{bcjk}\langle \Phi_i^a | \hat{H} -E|\Phi_{jk}^{bc} \rangle
  C_{jk}^{bc}+ \sum_{bcdjkl}\langle \Phi_i^a | \hat{H}
  -E|\Phi_{jkl}^{bcd} \rangle C_{jkl}^{bcd}=0,
  \]
  as this equation will allow us to find an expression for the
  coefficents $C_i^a$  through 
  \begin{equation}\label{eq:c1p1h}
  \langle i | \hat{f}| a\rangle +\langle \Phi_i^a |
  \hat{H}|\Phi_{i}^{a} \rangle C_{i}^{a}+ \sum_{bj\ne ai}\langle
  \Phi_i^a | \hat{H}|\Phi_{j}^{b} \rangle C_{j}^{b}+
  \sum_{bcjk}\langle \Phi_i^a | \hat{H}|\Phi_{jk}^{bc} \rangle
  C_{jk}^{bc}+ \sum_{bcdjkl}\langle \Phi_i^a |
  \hat{H}|\Phi_{jkl}^{bcd} \rangle C_{jkl}^{bcd}=EC_i^a.
  \end{equation}

  We see that on the right-hand side we have the energy $E$. This
  leads to a non-linear equation in the unknown coefficients since the coefficients appear also in the definition of the correlation
energy of Eq.~(\ref{eq:correlationenergy}).  These
  equations are normally solved iteratively, that is we  start
  with a guess for the coefficients $C_i^a$. A common choice is to
  use perturbation theory as a starting point for the unknown coefficients. For the one-particle-one-hole coefficients, the wave operator
(see section \ref{sec:chap8mbpt}) to first order in the interaction is given by
  \[
   C_{i}^{a}=\frac{\langle i | \hat{f}|
     a\rangle}{\epsilon_i-\epsilon_a}.
  \]

  The observant reader will however see that we need an equation for
  $C_{jk}^{bc}$ and $C_{jkl}^{bcd}$ and more complicated particle-hole excitations as well.  To find the equations for
  these coefficients we need then to continue our multiplications from
  the left with the various $\Phi_{H}^P$ terms.


  For $C_{jk}^{bc}$ we have
  \begin{align}\label{eq:c2p2h}
  0=&\langle \Phi_{ij}^{ab} | \hat{H} -E| \Phi_0\rangle +
  \sum_{kc}\langle \Phi_{ij}^{ab} | \hat{H} -E|\Phi_{k}^{c} \rangle
  C_{k}^{c}+ \\
  &\sum_{cdkl}\langle \Phi_{ij}^{ab} | \hat{H} -E|\Phi_{kl}^{cd}
  \rangle C_{kl}^{cd}+\sum_{cdeklm}\langle \Phi_{ij}^{ab} | \hat{H}
  -E|\Phi_{klm}^{cde} \rangle C_{klm}^{cde}+\sum_{cdefklmn}\langle
  \Phi_{ij}^{ab} | \hat{H} -E|\Phi_{klmn}^{cdef} \rangle
  C_{klmn}^{cdef}.
  \end{align}
  We can isolate the coefficients $C_{kl}^{cd}$ in a similar way
  as we did for the coefficients $C_{i}^{a}$.  A standard choice for
  the first iteration is to use again perturbation theory to first order in the interaction and set
  \[
  C_{ij}^{ab} =\frac{\langle ij \vert \hat{v} \vert ab
    \rangle}{\epsilon_i+\epsilon_j-\epsilon_a-\epsilon_b}.
  \]
  At the end we can rewrite our solution of the Schr\"odinger equation
  in terms of a series coupled equations for the coefficients $C_H^P$.
  This is a very cumbersome way of solving a many-body
  problem. However, by using this iterative scheme we can illustrate
  how we can compute the various terms in the wave operator or
  correlation operator $\hat{C}$. We will later identify the
  calculation of the various terms $C_H^P$ as parts of different
  many-body approximations to full configuration interaction theory.

  \subsection{Short summary}


  If we can directly diagonalize large matrices, full configuration interaction
  theory is the method of choice since  we obtain all  eigenvectors and eigenvalues. 
  The eigenvectors are obtained directly from the coefficients
    $C_H^P$ which result from the diagonalization.  We can then
compute expectation values of other operators,
    as well as transition probabilities. Moreover, correlations are easy to understand in terms of contributions
    to a given operator beyond the Hartree-Fock contribution. 
For larger dimensionalities $d$, with $d > 10^5$, iterative Krylov methods \cite{krylov} like Lanczos' \cite{lanczos} or Davidson's \cite{davidson} 
algorithms are frequently used. These methods yield, with a finite number of iteration, only a subset of all eigenvalues of interest. Lanczos' algorithm converges to the extreme values, yielding the lowest-lying and highest-lying eigenstates, see for example Ref.~\cite{golubvanloan} for a proof. 

With the eigenvectors we can compute
  the correlation energy, which is defined as (with a two-body Hamiltonian)
  \[
  \Delta E=\sum_{ai}\langle i| \hat{f}|a \rangle C_{i}^{a}+
  \sum_{abij}\langle ij | \hat{v}| ab \rangle C_{ij}^{ab}.
  \]
The energy of  the ground state is then
  \[
  E=E_{\mathrm{Ref}}+\Delta E,
  \]
where $E_{\mathrm{Ref}}$ is the reference energy.
  However, as we have seen, even for a
  small case like the four first major shells and 
  oxygen-16 with 16 active nucleons, the dimensionality becomes quickly intractable. If we
  wish to include single-particle states that reflect weakly bound
  systems, we need a much larger single-particle basis. We need thus
  approximative methods that sum specific correlations to infinite
  order.  All these methods start normally with a Hartree-Fock basis
  as the calculational basis. In the next section we discuss one of
  these possible approximative methods, namely many-body perturbation
  theory.

  \section{Many-body perturbation theory}\label{sec:chap8mbpt}

  We assume here that we are only interested in the non-degenerate
  ground state of a given system and expand the exact wave function in
  terms of a series of Slater determinants
  \[
  \vert \Psi_0\rangle = \vert \Phi_0\rangle +
  \sum_{m=1}^{\infty}C_m\vert \Phi_m\rangle,
  \]
  where we have assumed that the true ground state is dominated by the
  solution of the unperturbed problem, that is
  \[
  \hat{H}_0\vert \Phi_0\rangle= W_0\vert \Phi_0\rangle.
  \]
  The state $\vert \Psi_0\rangle$ is not normalized and we employ again
  intermediate normalization via $\langle \Phi_0 \vert
  \Psi_0\rangle=1$.

  The Schr\"odinger equation is given by
  \[
  \hat{H}\vert \Psi_0\rangle = E\vert \Psi_0\rangle,
  \]
  and multiplying the latter from the left with $\langle \Phi_0\vert $
  gives
  \[
  \langle \Phi_0\vert \hat{H}\vert \Psi_0\rangle = E\langle
  \Phi_0\vert \Psi_0\rangle=E,
  \]
  and subtracting from this equation
  \[
  \langle \Psi_0\vert \hat{H}_0\vert \Phi_0\rangle= W_0\langle
  \Psi_0\vert \Phi_0\rangle=W_0,
  \]
  and using the fact that the  operators $\hat{H}$ and $\hat{H}_0$
  are hermitian results in
  \begin{equation}\label{eq:mbptcorrel}
  \Delta E=E-W_0=\langle \Phi_0\vert \hat{H}_I\vert \Psi_0\rangle,
  \end{equation}
  which is an exact result. This resembles our previous definition of the correlation energy except that the reference energy is now defined
by the unperturbed energy $W_0$. The reader should contrast this equation to our previous definition of the correlation energy
  \[
  \Delta E=\sum_{ai}\langle i| \hat{f}|a \rangle C_{i}^{a}+
  \sum_{abij}\langle ij | \hat{v}| ab \rangle C_{ij}^{ab},
  \]
and the total energy
  \[
  E=E_{\mathrm{Ref}}+\Delta E,
  \]
where the reference energy is given by
\[   
   E_{\mathrm{Ref}}= \langle \Phi_0 \vert \hat{H} \vert \Phi_0\rangle.
\]

  Equation (\ref{eq:mbptcorrel}) forms the starting point for all perturbative
  derivations. However, as it stands it represents nothing but a mere
  formal rewriting of Schr\"odinger's equation and is not of much
  practical use. The exact wave function $\vert \Psi_0\rangle$ is
  unknown. In order to obtain a perturbative expansion, we need to
  expand the exact wave function in terms of the interaction
  $\hat{H}_I$.

  Here we have assumed that our model space defined by the operator
  $\hat{P}$ is one-dimensional, meaning that
  \[
  \hat{P}= \vert \Phi_0\rangle \langle \Phi_0\vert ,
  \]
  and
  \[
  \hat{Q}=\sum_{m=1}^{\infty}\vert \Phi_m\rangle \langle \Phi_m\vert .
  \]


  We can thus rewrite the exact wave function as
  \[
  \vert \Psi_0\rangle= (\hat{P}+\hat{Q})\vert \Psi_0\rangle=\vert
  \Phi_0\rangle+\hat{Q}\vert \Psi_0\rangle.
  \]
  Going back to the Schr\"odinger equation, we can rewrite it as,
  adding and a subtracting a term $\omega \vert \Psi_0\rangle$ as
  \[
  \left(\omega-\hat{H}_0\right)\vert
  \Psi_0\rangle=\left(\omega-E+\hat{H}_I\right)\vert \Psi_0\rangle,
  \]
  where $\omega$ is an energy variable to be specified later.


  We assume also that the resolvent of $\left(\omega-\hat{H}_0\right)$
  exits, that is it has an inverse which defines the unperturbed
  Green's function as
  \[
  \left(\omega-\hat{H}_0\right)^{-1}=\frac{1}{\left(\omega-\hat{H}_0\right)}.
  \]

  We can rewrite Schr\"odinger's equation as
  \[
  \vert
  \Psi_0\rangle=\frac{1}{\omega-\hat{H}_0}\left(\omega-E+\hat{H}_I\right)\vert
  \Psi_0\rangle,
  \]
  and multiplying from the left with $\hat{Q}$ results in
  \[
  \hat{Q}\vert
  \Psi_0\rangle=\frac{\hat{Q}}{\omega-\hat{H}_0}\left(\omega-E+\hat{H}_I\right)\vert
  \Psi_0\rangle,
  \]
  which is possible since we have defined the operator $\hat{Q}$ in
  terms of the eigenfunctions of $\hat{H}_0$.

 Since these operators commute we have
  \[
  \hat{Q}\frac{1}{\left(\omega-\hat{H}_0\right)}\hat{Q}=\hat{Q}\frac{1}{\left(\omega-\hat{H}_0\right)}=\frac{\hat{Q}}{\left(\omega-\hat{H}_0\right)}.
  \]
  With these definitions we can in turn define the wave function as
  \[
  \vert \Psi_0\rangle=\vert
  \Phi_0\rangle+\frac{\hat{Q}}{\omega-\hat{H}_0}\left(\omega-E+\hat{H}_I\right)\vert
  \Psi_0\rangle.
  \]
  This equation is again nothing but a formal rewrite of
  Schr\"odinger's equation and does not represent a practical
  calculational scheme.  It is a non-linear equation in two unknown
  quantities, the energy $E$ and the exact wave function $\vert
  \Psi_0\rangle$. We can however start with a guess for $\vert
  \Psi_0\rangle$ on the right hand side of the last equation.



   The most common choice is to start with the function which is
   expected to exhibit the largest overlap with the wave function we
   are searching after, namely $\vert \Phi_0\rangle$. This can again
   be inserted in the solution for $\vert \Psi_0\rangle$ in an
   iterative fashion and if we continue along these lines we end up
   with
  \[
  \vert
  \Psi_0\rangle=\sum_{i=0}^{\infty}\left\{\frac{\hat{Q}}{\omega-\hat{H}_0}\left(\omega-E+\hat{H}_I\right)\right\}^i\vert
  \Phi_0\rangle,
  \]
  for the wave function and
  \[
  \Delta E=\sum_{i=0}^{\infty}\langle \Phi_0\vert
  \hat{H}_I\left\{\frac{\hat{Q}}{\omega-\hat{H}_0}\left(\omega-E+\hat{H}_I\right)\right\}^i\vert
  \Phi_0\rangle,
  \]
  which is now a perturbative expansion of the exact energy in terms
  of the interaction $\hat{H}_I$ and the unperturbed wave function
  $\vert \Psi_0\rangle$.



  In our equations for $\vert \Psi_0\rangle$ and $\Delta E$ in terms
  of the unperturbed solutions $\vert \Phi_i\rangle$ we have still an
  undetermined parameter $\omega$ and a dependecy on the exact energy
  $E$. Not much has been gained thus from a practical computational
  point of view.

  In Brilluoin-Wigner perturbation theory \cite{haxton,shavittbartlett2009} it is customary to set
  $\omega=E$. This results in the following perturbative expansion for
  the energy $\Delta E$
  \begin{align}
  \Delta E=&\sum_{i=0}^{\infty}\langle \Phi_0\vert
  \hat{H}_I\left\{\frac{\hat{Q}}{\omega-\hat{H}_0}\left(\omega-E+\hat{H}_I\right)\right\}^i\vert
  \Phi_0\rangle=\\ &\langle \Phi_0\vert
  \left(\hat{H}_I+\hat{H}_I\frac{\hat{Q}}{E-\hat{H}_0}\hat{H}_I+
  \hat{H}_I\frac{\hat{Q}}{E-\hat{H}_0}\hat{H}_I\frac{\hat{Q}}{E-\hat{H}_0}\hat{H}_I+\dots\right)\vert
  \Phi_0\rangle.
  \end{align}
  This expression depends however on the exact energy $E$ and is again
  not very convenient from a practical point of view. It can obviously
  be solved iteratively, by starting with a guess for $E$ and then
  solve till some kind of self-consistency criterion has been reached.

  Defining $e=E-\hat{H}_0$ and recalling that $\hat{H}_0$ commutes
  with $\hat{Q}$ by construction and that $\hat{Q}$ is an idempotent
  operator $\hat{Q}^2=\hat{Q}$, we can rewrite the denominator in the above
  expansion for $\Delta E$ as
  \[
  \hat{Q}\frac{1}{\hat{e}-\hat{Q}\hat{H}_I\hat{Q}}=\hat{Q}\left[\frac{1}{\hat{e}}+\frac{1}{\hat{e}}\hat{Q}\hat{H}_I\hat{Q}
    \frac{1}{\hat{e}}+\frac{1}{\hat{e}}\hat{Q}\hat{H}_I\hat{Q}
    \frac{1}{\hat{e}}\hat{Q}\hat{H}_I\hat{Q}\frac{1}{\hat{e}}+\dots\right]\hat{Q}.
  \]

  Inserted in the expression for $\Delta E$ we obtain
  \[
  \Delta E= \langle \Phi_0\vert
  \hat{H}_I+\hat{H}_I\hat{Q}\frac{1}{E-\hat{H}_0-\hat{Q}\hat{H}_I\hat{Q}}\hat{Q}\hat{H}_I\vert
  \Phi_0\rangle.
  \]
  In Rayleigh-Schr\"odinger (RS) perturbation theory \cite{shavittbartlett2009} we set $\omega = W_0$ and obtain the
  following expression for the energy difference
  \begin{align}
  \Delta E =& \sum_{i=0}^{\infty}\langle \Phi_0\vert
  \hat{H}_I\left\{\frac{\hat{Q}}{W_0-\hat{H}_0}\left(\hat{H}_I-\Delta
  E\right)\right\}^i\vert \Phi_0\rangle\\ & \langle \Phi_0\vert
  \left(\hat{H}_I+\hat{H}_I\frac{\hat{Q}}{W_0-\hat{H}_0}(\hat{H}_I-\Delta
  E)+ \hat{H}_I\frac{\hat{Q}}{W_0-\hat{H}_0}(\hat{H}_I-\Delta
  E)\frac{\hat{Q}}{W_0-\hat{H}_0}(\hat{H}_I-\Delta
  E)+\dots\right)\vert \Phi_0\rangle.
  \end{align}



  Recalling that $\hat{Q}$ commutes with $\hat{H_0}$ and since $\Delta
  E$ is a constant we obtain that
  \[
  \hat{Q}\Delta E\vert \Phi_0\rangle = \hat{Q}\Delta E\vert
  \hat{Q}\Phi_0\rangle = 0.
  \]
  Inserting this results in the expression for the energy results in
  \[
  \Delta E=\langle \Phi_0\vert
  \left(\hat{H}_I+\hat{H}_I\frac{\hat{Q}}{W_0-\hat{H}_0}\hat{H}_I+
  \hat{H}_I\frac{\hat{Q}}{W_0-\hat{H}_0}(\hat{H}_I-\Delta
  E)\frac{\hat{Q}}{W_0-\hat{H}_0}\hat{H}_I+\dots\right)\vert
  \Phi_0\rangle.
  \]



  We can now expand this expression in terms of a perturbative expression in
  terms of $\hat{H}_I$ where we iterate the last expression in terms
  of $\Delta E$, that is 
  \[
  \Delta E=\sum_{i=1}^{\infty}\Delta E^{(i)}.
  \]
  We get the following expression for $\Delta E^{(i)}$
  \[
  \Delta E^{(1)}=\langle \Phi_0\vert \hat{H}_I\vert \Phi_0\rangle,
  \] 
  which is just the contribution to first order in perturbation
  theory,
  \[
  \Delta E^{(2)}=\langle\Phi_0\vert
  \hat{H}_I\frac{\hat{Q}}{W_0-\hat{H}_0}\hat{H}_I\vert \Phi_0\rangle,
  \]
  which is the contribution to second order and
  \[
  \Delta E^{(3)}=\langle \Phi_0\vert
  \hat{H}_I\frac{\hat{Q}}{W_0-\hat{H}_0}\hat{H}_I\frac{\hat{Q}}{W_0-\hat{H}_0}\hat{H}_I\Phi_0\rangle-
  \langle\Phi_0\vert \hat{H}_I\frac{\hat{Q}}{W_0-\hat{H}_0}\langle
  \Phi_0\vert \hat{H}_I\vert
  \Phi_0\rangle\frac{\hat{Q}}{W_0-\hat{H}_0}\hat{H}_I\vert
  \Phi_0\rangle,
  \]
  being the third-order contribution.
  There exists a formal theory for the calculation
  of $\Delta E_0$, see for example Ref.~\cite{shavittbartlett2009}.  According to the well-known Goldstone
  linked-diagram theory, the energy shift $\Delta E_0$ is given
  exactly by the diagrammatic expansion shown in
  Fig.~\ref{fig:goldstone}, where ground state diagrams to third order are listed. This theory is a linked-cluster
  perturbation expansion for the ground state energy of a many-body
  system, and applies equally well to both nuclear matter and
  closed-shell nuclei such as the doubly magic nucleus $^{40}$Ca.  
We assume the reader is familiar 
with the standard rules for deriving and setting up the analytical expressions for various Feymann-Goldstone diagrams \cite{svavittbartlett2009}.
\begin{figure}[hbtp]
    \includegraphics[width=\linewidth]{Chapter8-figures/mbpt.pdf}
      \caption{Diagrams which enter the definition of the ground-state
      correlation energy $\Delta E_0$ to third order in the interaction. We have not included the first-order contribution.}
      \label{fig:goldstone}
\end{figure}
In an infinite system like nuclear matter or the homogenous electron gas, all diagrams with so-called Hartree-Fock insertions like diagrams (2), (6), (7), (10-16) are zero 
due to lack of momentum conservation. They would also be zero in case a canonical \cite{shavittbartlett2009} Hartree-Fock basis is employed. 

  Using the above standard diagram rules, the various
  diagrams contained in Fig.~\ref{fig:goldstone} can be readily calculated (in
  an uncoupled scheme). Diagram  (1) results in
  \begin{equation}
     (1)=\frac{1}{2^2}\sum_{ij\leq F}\sum_{ab>F}
    \frac{\langle ij\vert\hat{v}\vert ab\rangle \langle
      ab\vert\hat{v}\vert ij\rangle}
         {\varepsilon_i+\varepsilon_j-\varepsilon_a-\varepsilon_b},
  \end{equation}
 while diagram (2) is zero due to lack of momentum conservation. We have two factors of $1/2$ since there are two equivalent pairs of fermions (two  particle states and two hole states) starting at the same vertex and ending at the same vertex. The expression for diagram (3)  is 
  \begin{equation}\label{eq:diag3}
     (3)=\sum_{ijk\leq k_F}\sum_{abc > F}
    \frac{\langle ij\vert\hat{v}\vert ab\rangle \langle
      bk\vert\hat{v}\vert ic\rangle \langle ac\vert\hat{v}\vert
      ik\rangle}
         {(\varepsilon_i+\varepsilon_j-\varepsilon_a-\varepsilon_b)
           (\varepsilon_i+\varepsilon_k-\varepsilon_a-\varepsilon_c)}.
  \end{equation}
  Diagrams (4) and (5) read
  \begin{equation}\label{eq:diag4}
     (4)=\frac{1}{2^3}\sum_{ij\le F}\sum_{abcd > F}
    \frac{\langle ij\vert\hat{v}\vert cd\rangle \langle
      cd\vert\hat{v}\vert ab\rangle \langle ab\vert\hat{v}\vert
      ij\rangle}
         {(\varepsilon_i+\varepsilon_j-\varepsilon_c-\varepsilon_d)
           (\varepsilon_i+\varepsilon_j-\varepsilon_a-\varepsilon_b)},
  \end{equation}
  \begin{equation}\label{eq:diag5}
     (5)=\frac{1}{2^3}\sum_{ijkl\le F}\sum_{ab > F}
    \frac{\langle ab\vert\hat{v}\vert kl\rangle \langle
      kl\vert\hat{v}\vert ij\rangle \langle ij\vert\hat{v}\vert
      ab\rangle}
         {(\varepsilon_i+\varepsilon_j-\varepsilon_a-\varepsilon_b)
           (\varepsilon_k+\varepsilon_l-\varepsilon_a-\varepsilon_b)},
  \end{equation}
where the factor $(1/2)^3$ arises due to three equivalent pairs of lines starting and ending at the same vertex. The last two contributions 
have an even number of hole lines and closed loops, resulting thus in a positive sign. In problem \ref{problem:diagrams}, you are asked to calculate the expressions for diagrams like (8) and (9) in the above figure. 

  In the expressions for the various diagrams the quantity $\varepsilon$ denotes the single-particle energies
  defined by $H_0$.  The steps leading to the above expressions for
  the various diagrams are rather straightforward. Though, if we wish
  to compute the matrix elements for the interaction $\hat{v}$, a serious
  problem arises. Typically, the matrix elements will contain a term
  $V(|{\mathbf r}|)$
  which represents the interaction potential $V$ between two nucleons,
  where ${\mathbf r}$ is the internucleon distance.  All modern models
  for $V$ have a strong short-range repulsive core. Hence, matrix
  elements involving $V(|{\mathbf r}|)$, will result in large (or
  infinitely large for a potential with a hard core) and repulsive
  contributions to the ground-state energy. Thus, the diagrammatic
  expansion for the ground-state energy in terms of the potential
  $V(|{\mathbf r}|)$ becomes meaningless. A perturbative expansion in terms of
such interaction matrix elements may thus lead to a slowly converging expansion.
A standard recipe to circumvent such problems has been to sum up 
a selected class of correlations. We discuss such possibilities in section \ref{sec:cc}.  



  \subsection{Interpreting the correlation energy and the wave operator}

  In section \ref{sec:chap8fci} we showed that we could rewrite the exact state
  function for the ground state as a linear expansion in terms of all
  possible Slater determinants.  We expanded then (as an example) our
  exact state function for the ground state as
  \[
  |\Psi_0\rangle=C_0|\Phi_0\rangle+\sum_{ai}C_i^a|\Phi_i^a\rangle+\sum_{abij}C_{ij}^{ab}|\Phi_{ij}^{ab}\rangle+\dots
  =(C_0+\hat{C})|\Phi_0\rangle,
  \]
  where we introduced the so-called correlation operator
  \[
  \hat{C}=\sum_{ai}C_i^a\hat{a}_{a}^{\dagger}\hat{a}_i
  +\sum_{abij}C_{ij}^{ab}\hat{a}_{a}^{\dagger}\hat{a}_{b}^{\dagger}\hat{a}_j\hat{a}_i+\dots
  \]
  In a shell-model calculation, the unknown coefficients in $\hat{C}$
  are the eigenvectors that result from the diagonalization of the
  Hamiltonian matrix.

  How can we use perturbation theory to determine the same
  coefficients? Let us study the contributions to second order in the
  interaction, namely
  \[
  \Delta E^{(2)}=\langle\Phi_0\vert
  \hat{H}_I\frac{\hat{Q}}{W_0-\hat{H}_0}\hat{H}_I\vert \Phi_0\rangle.
  \]
  This contribution will also be discussed in connection with the
  development of a many-body program for nuclear matter, as well as
  the simple pairing model of problem \ref{problem:pairingmodel}.  The
  intermediate states given by $\hat{Q}$ can at most be of a $2p-2h$
  nature if we have a two-body Hamiltonian. This means that to second
  order in perturbation theory we can at most have $1p-1h$ and $2p-2h$
  excitations as intermediate states. When we diagonalize, these
  contributions are included to infinite order. This means that in
  order to include such correlations to higher order in the
  interaction, we need to go to higher-orders in perturbation theory.

  If we limit the attention to a Hartree-Fock basis, we have that
  $\langle\Phi_0\vert \hat{H}_I \vert 2p-2h\rangle$ is the only
  contribution since matrix elements involving $\langle\Phi_0\vert
  \hat{H}_I \vert 1p-1h\rangle$ are zero and the contribution to the
  energy from second order in Rayleigh-Schr\"odinger perturbation
  theory reduces to
  \[
  \Delta E^{(2)}=\frac{1}{4}\sum_{abij}\langle ij\vert \hat{v}\vert
  ab\rangle \frac{\langle ab\vert \hat{v}\vert
    ij\rangle}{\epsilon_i+\epsilon_j-\epsilon_a-\epsilon_b}.
  \]
Here we have used the results from problem \ref{problem:hamiltoniansetup}. 
  If we compare this to the correlation energy obtained from full
  configuration interaction theory with a Hartree-Fock basis, we found
  that
  \[
  E-E_{\mathrm{Ref}} =\Delta E=\sum_{abij}\langle ij | \hat{v}| ab
  \rangle C_{ij}^{ab},
  \]
  where the energy $E_{\mathrm{Ref}}$ is the reference energy and
  $\Delta E$ defines the so-called correlation energy.

  We see that if we set
  \[
  C_{ij}^{ab} =\frac{1}{4}\frac{\langle ab \vert \hat{v} \vert ij
    \rangle}{\epsilon_i+\epsilon_j-\epsilon_a-\epsilon_b},
  \]
  we have a perfect agreement between configuration interaction  theory  and many-body perturbation
  theory. However, configuration interaction  theory includes $2p-2h$ (and more complicated ones as
  well) correlations to infinite order. In order to make a meaningful
  comparison we would at least need to sum such correlations to
  infinite order in perturbation theory. The last equation serves
  however as a very useful comparison between configuration interaction  theory and many-body
  perturbation theory. Furthermore, for our nuclear matter studies,
  one-particle-one-hole intermediate excitations are zero due to the
  requirement of momentum conservation in infinite systems. These
  two-particle-two-hole correlations can also be summed to infinite
  order and a particular class of such excitations are given by
  two-particle excitations only. These represent in case of nuclear interactions, which
  are strongly repulsive at short interparticle distances, a
  physically intuitive way to understand the renomalization of nuclear
  forces. Such correlations are easily computed by simple matrix
  inversion techniques and have been widely employed in nuclear
  many-body theory.  Summing up two-particle excitations to infinite
  order leads to an effective two-body interaction which renomalizes
  the short-range part of the nuclear interactions. 

In summary, many-body perturbation theory introduces order-by-order
specific correlations and we can make comparisons with exact
calculations like those provided by configuration interaction theory.
The advantage of for example Rayleigh-Schr\"odinger perturbation
theory is that at every order in the interaction, we know how to
calculate all contributions since they can be either tabulated or
computed on the fly.  However, many-body perturbation theory suffers
from not being variational theory and there is no guarantee that
higher-order terms will improve the order-by-order convergence.  It is
also extremely tedious to compute terms beyond third order, in
particular if one is interested in effective valence space
interactions.  There are however classes of correlations which can be
summed up to infinite order in the interaction.  The hope is that such
correlations can mitigate specific convergence issues, although there
is no a priori guarantee thereof.  Examples are the so-called TDA and
RPA class of diagrams \cite{ripkablaizot}, as well as the resummation
of two-particle-two-hole correlations discussed in chapter 11. If we limit ourselves to the
resummation of two-particle correlations only, these lead us to the
so-called $G$-matrix resummation of diagrams, see for example Ref.~\cite{day1967}.
There are however computationally inexpensive methods which
sum larger classes of correlations to infinite order in the
interaction. This leads us to section \ref{sec:chap8cctheory} and the final
many-body method of this chapter, coupled cluster theory. 


\section{Coupled cluster theory}\label{sec:chap8cctheory}
  
Coester and Kummel developed the ideas that led to coupled cluster
theory in the late 1950s. The basic idea is that the correlated wave function
of a many-body system $\mid\Psi\rangle$
can be formulated as an exponential of correlation
operators $T$ acting on a reference state $\mid\Phi\rangle$.
\begin{equation}
\mid\Psi\rangle = \exp\left(-\hat{T}\right)\mid\Phi\rangle,
\end{equation}
We will discuss how to define the operators later in this work. This simple
ansatz carries enormous power. It leads to a non-perturbative many-body
theory that includes summation of ladder diagrams \cite{brueckner1955}, ring
diagrams \cite{GB1957}, and an infinite-order
generalization of many-body perturbation theory
\cite{Bart_silver74, Bart_silver76}.
Developments and applications of coupled cluster theory took
different routes in chemistry and nuclear physics. In quantum
chemistry, coupled cluster developments and applications have proven
to be extremely useful, see for example the review by Barrett and
Musial as well as the recent textbook by Shavitt and Bartlett \cite{shavittbartlett2009}.
Many previous applications to nuclear physics struggled with the
repulsive character of the nuclear forces and limited basis sets
used in the computations. Most of these problems have been overcome
during the last decade and coupled cluster theory is one of the
computational methods of preference for doing nuclear physics, with
  applications ranging from light nuclei to medium-heavy nuclei, see
  for example the recent reviews by Hagen {\em et al} \cite{papenbrock2014,hagen2016}.


\subsection{A quick tour of Coupled Cluster theory}

  The ansatz for the wavefunction (ground state) is given by
  \begin{equation}
     \vert \Psi_0\rangle = \vert \Psi_{CC}\rangle = e^{\hat{T}} \vert
     \Phi_0\rangle = \left( \sum_{n=1}^{A} \frac{1}{n!} \hat{T}^n
     \right) \vert \Phi_0\rangle,
  \end{equation}
  where $A$ represents the maximum number of particle-hole excitations
  and $\hat{T}$ is the cluster operator defined as
  \begin{align}
              \hat{T} &= \hat{T}_1 + \hat{T}_2 + \ldots + \hat{T}_A
              \\ \hat{T}_n &= \left(\frac{1}{n!}\right)^2
              \sum_{\substack{ i_1,i_2,\ldots i_n \\ a_1,a_2,\ldots
                  a_n}} t_{i_1i_2\ldots i_n}^{a_1a_2\ldots a_n}
              a_{a_1}^\dagger a_{a_2}^\dagger \ldots a_{a_n}^\dagger
              a_{i_n} \ldots a_{i_2} a_{i_1}.
          \end{align}
      The energy is given by
      \begin{equation}
          E_{\mathrm{CC}} = \langle\Phi_0\vert \overline{H}\vert
          \Phi_0\rangle,
      \end{equation}
      where $\overline{H}$ is a similarity transformed Hamiltonian
      \begin{align}
          \overline{H}&= e^{-\hat{T}} \hat{H}_N e^{\hat{T}}
          \\ \hat{H}_N &= \hat{H} - \langle\Phi_0\vert \hat{H} \vert
          \Phi_0\rangle.
      \end{align}

      The coupled cluster energy is a function of the unknown cluster
      amplitudes $t_{i_1i_2\ldots i_n}^{a_1a_2\ldots a_n}$, given by
      the solutions to the amplitude equations
      \begin{equation}\label{eq:amplitudeeq}
          0 = \langle\Phi_{i_1 \ldots i_n}^{a_1 \ldots a_n}\vert
          \overline{H}\vert \Phi_0\rangle.
      \end{equation}
In order to set up the above equations, 
the similarity transformed Hamiltonian $\overline{H}$ is expanded
  using the Baker-Campbell-Hausdorff expression,
      \begin{equation}\label{eq:bch}
          \overline{H}= \hat{H}_N + \left[ \hat{H}_N, \hat{T} \right]
          + \frac{1}{2} \left[\left[ \hat{H}_N, \hat{T} \right],
            \hat{T}\right] + \ldots + \frac{1}{n!} \left[
            \ldots \left[ \hat{H}_N, \hat{T} \right], \ldots \hat{T}
            \right] +\dots
      \end{equation}
  and simplified using the connected cluster theorem
      \begin{equation}
          \overline{H}= \hat{H}_N + \left( \hat{H}_N \hat{T}\right)_c
          + \frac{1}{2} \left( \hat{H}_N \hat{T}^2\right)_c + \dots +
          \frac{1}{n!} \left( \hat{H}_N \hat{T}^n\right)_c +\dots
      \end{equation}
We will discuss parts of the the derivation below.
For the full derivation of these expressions, see for example Ref.~\cite{shavittbartlett2009}. 

A much used approximation is to truncate the cluster operator
$\hat{T}$ at the $n=2$ level. This defines the so-called singes and
doubles approximation to the coupled cluster wavefunction, normally
shortened to CCSD.
The coupled cluster wavefunction is now given by
  \begin{equation}
              \vert \Psi_{CC}\rangle = e^{\hat{T}_1 + \hat{T}_2} \vert
              \Phi_0\rangle
  \end{equation}
  where
          \begin{align*}
              \hat{T}_1 &= \sum_{ia} t_{i}^{a} a_{a}^\dagger a_i
              \\ \hat{T}_2 &= \frac{1}{4} \sum_{ijab} t_{ij}^{ab}
              a_{a}^\dagger a_{b}^\dagger a_{j} a_{i}.
          \end{align*}

  The amplutudes $t$ play a role similar to the coefficients $C$ in
  the shell-model calculations. They are obtained by solving a set of
  non-linear equations similar to those discussed above in connection
  with the configuration interaction theory  discussion, see Eqs.~(\ref{eq:c1p1h}) and (\ref{eq:c2p2h}). 

  If we truncate our equations at the CCSD level, it corresponds to
  performing a transformation of the Hamiltonian matrix of the
  following type for the six particle problem (with a two-body
  Hamiltonian) discussed in section \ref{sec:chap8fci}
  \begin{table}
\caption{Schematic representation of blocks of matrix elements which are set to zero using the CCSD approximation.}
  \begin{center}
  \begin{tabular}{cccccccc}
  \hline \multicolumn{1}{c}{ } & \multicolumn{1}{c}{ $0p-0h$ } &
  \multicolumn{1}{c}{ $1p-1h$ } & \multicolumn{1}{c}{ $2p-2h$ } &
  \multicolumn{1}{c}{ $3p-3h$ } & \multicolumn{1}{c}{ $4p-4h$ } &
  \multicolumn{1}{c}{ $5p-5h$ } & \multicolumn{1}{c}{ $6p-6h$ }
  \\ \hline $0p-0h$ & $\tilde{x}$ & $\tilde{x}$ & $\tilde{x}$ & 0 & 0
  & 0 & 0 \\ $1p-1h$ & 0 & $\tilde{x}$ & $\tilde{x}$ & $\tilde{x}$ & 0
  & 0 & 0 \\ $2p-2h$ & 0 & $\tilde{x}$ & $\tilde{x}$ & $\tilde{x}$ &
  $\tilde{x}$ & 0 & 0 \\ $3p-3h$ & 0 & $\tilde{x}$ & $\tilde{x}$ &
  $\tilde{x}$ & $\tilde{x}$ & $\tilde{x}$ & 0 \\ $4p-4h$ & 0 & 0 &
  $\tilde{x}$ & $\tilde{x}$ & $\tilde{x}$ & $\tilde{x}$ & $\tilde{x}$
  \\ $5p-5h$ & 0 & 0 & 0 & $\tilde{x}$ & $\tilde{x}$ & $\tilde{x}$ &
  $\tilde{x}$ \\ $6p-6h$ & 0 & 0 & 0 & 0 & $\tilde{x}$ & $\tilde{x}$ &
  $\tilde{x}$ \\ \hline
  \end{tabular}
  \end{center}
  \end{table}

  In our configuration interaction theory discussion the correlation energy is defined as (with a two-body Hamiltonian) 
  \[
  \Delta E=\sum_{ai}\langle i| \hat{f}|a \rangle C_{i}^{a}+
  \sum_{abij}\langle ij | \hat{v}| ab \rangle C_{ij}^{ab}.
  \]
We can obtain a similar expression for the correlation energy using coupled cluster theory. 
Using Eq.~(\ref{eq:bch}) we can write the expression for the coupled cluster  ground state energy as an infinite sum over nested commutators
        \begin{align*}
            E_{CC} &= \bra{\Psi_0}\Bigl( \hat{H}_N + \left[ \hat{H}_N, \hat{T} \right] +
                \frac{1}{2} \left[\left[ \hat{H}_N, \hat{T} \right], \hat{T}\right] \\
                & \qquad + \frac{1}{3!} \left[ \left[\left[ \hat{H}_N, \hat{T} \right], \hat{T}\right], \hat{T} \right] \\
                & \qquad + \frac{1}{4!} \left[ \left[ \left[\left[ \hat{H}_N, \hat{T} \right], \hat{T}\right], \hat{T} \right], \hat{T} \right] ++ \Bigr) \ket{\Psi_0}. \\
        \end{align*}
One can show that this infinite series truncates naturally at a given order of nested commutators \cite{shavittbartlett2009}. 
Let us demonstrate briefly how we can construct the expressions for the correlation 
energy by approximating $\hat{T}$ at the CCSD level, that is $\hat{T}\approx \hat{T}_1+\hat{T}_2$.
        The first term is zero by construction
        \begin{equation*}
            \bra{\Psi_0} \hat{H}_N \ket{\Psi_0} = 0.
        \end{equation*}       
     The second term can be split into the following contributions
        \begin{align*}
        \bra{\Psi_0}\left[ \hat{H}_N, \hat{T} \right]\ket{\Psi_0} & = 
            \bra{\Psi_0} \Bigl(\left[ \hat{F}_N, \hat{T}_1 \right] + \left[ \hat{F}_N, \hat{T}_2 \right]
            + \left[ \hat{V}_N, \hat{T}_1 \right] + \left[ \hat{V}_N, \hat{T}_2 \right] \Bigr) \ket{\Psi_0}.
    \end{align*}
Let us start with $\left[ \hat{F}_N, \hat{T}_1 \right]$, where the one-body operator $\hat{F}_N$ is defined in Eq.~(\ref{eq:hfn}). 
In the equations below we employ the shorthand $f^p_q = \langle p\vert \hat{f} \vert q\rangle$.  
We write out the commutator  as
    \begin{align*}
        \left[ \hat{F}_N, \hat{T}_1 \right] &= \sum_{pqia} \left(f_q^p \normord{a_p^\dagger a_q} 
            t_i^a \normord{a_a^\dagger a_i} - t_i^a \normord{a_a^\dagger a_i} f_q^p \normord{a_p^\dagger a_q} \right) \\
        &= \sum_{pqia} f_q^p t_i^a \left( \normord{a_p^\dagger a_q} \normord{a_a^\dagger a_i} -
                \normord{a_a^\dagger a_i} \normord{a_p^\dagger a_q} \right).
    \end{align*}
We have skipped here the curly brackets that indicate that the chains of operators are normal ordered with respect to the new reference state. 
If we consider the second set of operators and rewrite them with curly brackets (bringing back the normal ordering) we have
        \begin{align*}
            \left\{a_a^\dagger a_i\right\} \left\{a_p^\dagger a_q \right\} &= \normord{a_a^\dagger a_i a_p^\dagger a_q}
                = \normord{a_p^\dagger a_q a_a^\dagger a_i} \\ 
        \normord{a_p^\dagger a_q}\normord{a_a^\dagger a_i} &= \normord{a_p^\dagger a_q a_a^\dagger a_i}  \\
            & \quad + \normord{
               \contraction{}{a}{{}^\dagger_pa_q a^\dagger_a}{a}
                a^\dagger_p a_q a^\dagger_a a_i} +
            \normord{
                \contraction{a^\dagger_p}{a}{{}_q}{a}
                a^\dagger_p a_q a^\dagger_a a_i} \\ 
            & \quad + \normord{
                \contraction[1.5ex]{}{a}{{}^\dagger_pa_q a^\dagger_a}{a}
                \contraction{a^\dagger_p}{a}{{}_q}{a}
                a^\dagger_p a_q a^\dagger_a a_i} \\ 
            &=  \normord{a_p^\dagger a_q a_a^\dagger a_i} + \delta_{qa} \normord{a_p^\dagger a_i} + \delta_{pi} \normord{a_q a_a^\dagger}
            + \delta_{qa}\delta_{pi}.
        \end{align*}
We can then rewrite the two sets of operators as
        \begin{align*}
            \left\{a_p^\dagger a_q \right\}\left\{a_a^\dagger a_i\right\} - \left\{a_a^\dagger a_i\right\} \left\{a_p^\dagger a_q \right\} &= \delta_{qa} \left\{ a_p^\dagger a_i\right\} + \delta_{pi} \left\{ a_q a_a^\dagger \right\} + \delta_{qa}\delta_{pi}.
    \end{align*}

        Inserted into the original expression, we arrive at the explicit value of the commutator
        \[
        \left[ \hat{F}_N, \hat{T}_1 \right] = \sum_{pai} f_a^p t_i^a \normord{a_p^\dagger a_i} + 
                \sum_{qai} f_q^i t_i^a \normord{a_q a_a^\dagger} + \sum_{ai} f_a^i t_i^a.
        \]
We are now ready to compute the expectation value with respect to our reference state. Since the two first terms require the ground state linking to 
a one-particle-one-hole state, the first two terms are zero and we are left with 
\begin{equation}\label{eq:firsttermE}
\langle \Phi_0 \vert \left[ \hat{F}_N, \hat{T}_1 \right] \vert \Phi_0\rangle = \sum_{ai} f_a^i t_i^a.
\end{equation}
The two first terms will however contribute to the calculation of the Hamiltonian  matrix element which connects the ground state and a one-particle-one-hole excitation. 
  
Let us next look at the term $\left[ \hat{F}_N, \hat{T}_2 \right]$. We have
    \begin{align*}
        \left[ \hat{F}_N, \hat{T}_2 \right] 
        &= \left[\sum_{pq} f_q^p \normord{a^\dagger_p a_q},
            \frac{1}{4}\sum_{ijab} t_{ij}^{ab} \normord{a^\dagger_a a^\dagger_b a_j a_i} \right] \\
        &= \frac{1}{4}\sum_{\substack{pq\\ijab}}
            f_q^p t_{ij}^{ab}
        \left( \normord{a^\dagger_p a_q} \normord{a^\dagger_a a^\dagger_b a_j a_i}
            - \normord{a^\dagger_a a^\dagger_b a_j a_i} \normord{a^\dagger_p a_q} \right).
    \end{align*}
The last set of operators can be rewritten as 
    \begin{align*}
        \normord{a^\dagger_a a^\dagger_b a_j a_i} \normord{a^\dagger_p a_q} &= 
            \normord{a^\dagger_a a^\dagger_b a_j a_i a^\dagger_p a_q} \\ 
        &= \normord{a^\dagger_p a_q a^\dagger_a a^\dagger_b a_j a_i} 
        \\
        \normord{a^\dagger_p a_q} \normord{a^\dagger_a a^\dagger_b a_j a_i} &= 
            \normord{a^\dagger_p a_q a^\dagger_a a^\dagger_b a_j a_i}
        + \left\{ 
        \contraction{}{a}{{}^\dagger_pa_q a^\dagger_a a^\dagger_b}{a}
        a^\dagger_p a_q a^\dagger_a a^\dagger_b a_j a_i\right\}
        + \left\{ 
        \contraction{}{a}{{}^\dagger_pa_q a^\dagger_a a^\dagger_b a_j}{a}
        a^\dagger_p a_q a^\dagger_a a^\dagger_b a_j a_i\right\} \\
        & \quad 
        + \left\{ 
        \contraction{a^\dagger_p}{a}{{}_q}{a}
        a^\dagger_p a_q a^\dagger_a a^\dagger_b a_j a_i\right\}
        + \left\{ 
        \contraction{a^\dagger_p}{a}{{}_q a^\dagger_a}{a}
        a^\dagger_p a_q a^\dagger_a a^\dagger_b a_j a_i\right\}
        + \left\{ 
        \contraction{a^\dagger_p}{a}{{}_q}{a}
        \contraction[1.5ex]{}{a}{{}^\dagger_pa_q a^\dagger_a a^\dagger_b}{a}
        a^\dagger_p a_q a^\dagger_a a^\dagger_b a_j a_i\right\} \\
        & \quad 
        + \left\{ 
        \contraction{a^\dagger_p}{a}{{}_q}{a}
        \contraction[1.5ex]{}{a}{{}^\dagger_pa_q a^\dagger_a a^\dagger_b a_j}{a}
        a^\dagger_p a_q a^\dagger_a a^\dagger_b a_j a_i\right\}
        + \left\{ 
        \contraction{a^\dagger_p}{a}{{}_q a^\dagger_a}{a}
        \contraction[1.5ex]{}{a}{{}^\dagger_pa_q a^\dagger_a a^\dagger_b}{a}
        a^\dagger_p a_q a^\dagger_a a^\dagger_b a_j a_i\right\}
        + \left\{ 
        \contraction{a^\dagger_p}{a}{{}_q a^\dagger_a}{a}
        \contraction[1.5ex]{}{a}{{}^\dagger_pa_q a^\dagger_a a^\dagger_b a_j}{a}
        a^\dagger_p a_q a^\dagger_a a^\dagger_b a_j a_i\right\} \\ 
        &= \normord{a^\dagger_p a_q a^\dagger_a a^\dagger_b a_j a_i}
        - \delta_{pj} \normord{a_q a^\dagger_a a^\dagger_b a_i}
        + \delta_{pi} \normord{a_q a^\dagger_a a^\dagger_b a_j} \\
        & \quad + \delta_{qa} \normord{a^\dagger_p a^\dagger_b a_j a_i}
        - \delta_{qb}\normord{a^\dagger_p a^\dagger_a a_j a_i} 
        - \delta_{pj} \delta_{qa} \normord{a^\dagger_b a_i} \\
        & \quad + \delta_{pi} \delta_{qa} \normord{a^\dagger_b a_j}
        + \delta_{pj} \delta_{qb} \normord{a^\dagger_a a_i}
        - \delta_{pi} \delta_{qb} \normord{a^\dagger_a a_j}.
    \end{align*}
 We can then rewrite the two sets of operators as
    \begin{align*}
        & \Bigl( \normord{a^\dagger_p a_q} \normord{a^\dagger_a a^\dagger_b a_j a_i}
            - \normord{a^\dagger_a a^\dagger_b a_j a_i} \normord{a^\dagger_p a_q} \Bigr) = \\
        & \qquad - \delta_{pj} \normord{a_q a^\dagger_a a^\dagger_b a_i}
        + \delta_{pi} \normord{a_q a^\dagger_a a^\dagger_b a_j}
        + \delta_{qa} \normord{a^\dagger_p a^\dagger_b a_j a_i} \\
        & \qquad - \delta_{qb}\normord{a^\dagger_p a^\dagger_a a_j a_i} 
        - \delta_{pj} \delta_{qa} \normord{a^\dagger_b a_i}
        + \delta_{pi} \delta_{qa} \normord{a^\dagger_b a_j}
        + \delta_{pj} \delta_{qb} \normord{a^\dagger_a a_i} \\
        & \qquad - \delta_{pi} \delta_{qb} \normord{a^\dagger_a a_j},
    \end{align*}
which, when    inserted into the original expression gives us
    \begin{align*}
        \left[ \hat{F}_N, \hat{T}_2 \right]
        &= \frac{1}{4} \sum_{\substack{pq\\abij}} f_q^p t_{ij}^{ab} \Bigl(
        - \delta_{pj} \normord{a_q a^\dagger_a a^\dagger_b a_i}
        + \delta_{pi} \normord{a_q a^\dagger_a a^\dagger_b a_j} \\
        & \quad + \delta_{qa} \normord{a^\dagger_p a^\dagger_b a_j a_i}
        - \delta_{qb}\normord{a^\dagger_p a^\dagger_a a_j a_i} 
        - \delta_{pj} \delta_{qa} \normord{a^\dagger_b a_i} \\
        & \quad + \delta_{pi} \delta_{qa} \normord{a^\dagger_b a_j}
        + \delta_{pj} \delta_{qb} \normord{a^\dagger_a a_i}
        - \delta_{pi} \delta_{qb} \normord{a^\dagger_a a_j} \Bigr).
    \end{align*}
    After renaming indices and changing the order of operators, we arrive at the explicit expression
    \[
        \left[ \hat{F}_N, \hat{T}_2 \right]
        = \frac{1}{2} \sum_{qijab} f_q^i t_{ij}^{ab} \normord{a_q a^\dagger_a a^\dagger_b a_j}
        + \frac{1}{2} \sum_{pijab} f_a^p t_{ij}^{ab} \normord{a^\dagger_p a^\dagger_b a_j a_i}+ \sum_{ijab} f_a^i t_{ij}^{ab} \normord{a^\dagger_b a_j}. 
    \]
In this case we have two sets of two-particle-two-hole operators and one-particle-one-hole operators and all these terms result in zero expectation values. 
However, these terms are important for the amplitude equations.  
In a similar was we can compute the terms involving the interaction $\hat{V}_N$.  
We obtain then
    \begin{align*}
        \bra{\Phi_0} \left[ \hat{V}_N, \hat{T}_1 \right] \ket{\Phi_0} &= 
            \bra{\Phi_0}
                \left[ \frac{1}{4} \sum_{pqrs} \bra{pq}\hat{v}\ket{rs} \normord{a^\dagger_p a^\dagger_q a_s  a_r},
                    \sum_{ia} t_i^a \normord{a^\dagger_a a_i} \right] \ket{\Phi_0} \\ 
            &= \frac{1}{4}\sum_{\substack{
                pqr \\
                sia}} \bra{pq}\ket{rs} t_i^a \bra{\Phi_0} 
                \left[ \normord{a^\dagger_p a^\dagger_q a_s  a_r}, \normord{a^\dagger_a a_i} \right]
                \ket{\Phi_0} \\ 
        &= 0,
    \end{align*}
and
    \begin{align*}
        & \bra{\Phi_0} \left[ \hat{V}_N, \hat{T}_2 \right] \ket{\Phi_0}= \\
            & \quad \bra{\Phi_0}
                \left[ \frac{1}{4} \sum_{pqrs} \bra{pq}\hat{v}\ket{rs} \normord{a^\dagger_p a^\dagger_q a_s  a_r},
                    \frac{1}{4}\sum_{ijab} t_{ij}^{ab} \normord{a^\dagger_a a^\dagger_b a_j a_i} \right] \ket{\Phi_0} \\ 
            &= \frac{1}{16}\sum_{\substack{
                    pqr \\
                    sijab}} \bra{pq}\hat{v}\ket{rs}t_{ij}^{ab} \bra{\Phi_0} 
                \left[ \normord{a^\dagger_p a^\dagger_q a_s  a_r}, \normord{a^\dagger_a a^\dagger_b a_j a_i} \right]
                \ket{\Phi_0} \\ 
            &= \frac{1}{16}\sum_{\substack{
                    pqr \\
                    sijab}} \bra{pq}\hat{v}\ket{rs}t_{ij}^{ab} \bra{\Phi_0}
            \Bigl(
            \left\{
            \contraction{a^\dagger_p a^\dagger_q a_s}{a}{{}_r}{a}
            \contraction[1.25ex]{a^\dagger_p a^\dagger_q}{a}{{}_s a_r a^\dagger_a}{a}
            \contraction[1.50ex]{a^\dagger_p}{a}{{}^\dagger_q a_s a_r a^\dagger_a a^\dagger_b}{a}
            \contraction[1.75ex]{}{a}{{}^\dagger_p a^\dagger_q a_s a_r a^\dagger_a a^\dagger_b a_j}{a}
            a^\dagger_p a^\dagger_q a_s  a_r a^\dagger_a a^\dagger_b a_j a_i \right\}
            + \left\{
            \contraction{a^\dagger_p a^\dagger_q}{a}{{}_s a_r}{a}
            \contraction[1.25ex]{a^\dagger_p a^\dagger_q a_s}{a}{{}_r a^\dagger_p}{a}
            \contraction[1.50ex]{a^\dagger_p}{a}{{}^\dagger_q a_s a_r a^\dagger_a a^\dagger_b}{a}
            \contraction[1.75ex]{}{a}{{}^\dagger_p a^\dagger_q a_s a_r a^\dagger_a a^\dagger_b a_j}{a}
            a^\dagger_p a^\dagger_q a_s  a_r a^\dagger_a a^\dagger_b a_j a_i \right\} \\
            & \quad \left\{
            \contraction{a^\dagger_p a^\dagger_q a_s}{a}{{}_r}{a}
            \contraction[1.25ex]{a^\dagger_p a^\dagger_q}{a}{{}_s a_r a^\dagger_a}{a}
            \contraction[1.5ex]{}{a}{{}^\dagger_p a^\dagger_q a_s a_r a^\dagger_a a^\dagger_b}{a}
            \contraction[1.75ex]{a^\dagger_p}{a}{{}^\dagger_q a_s a_r a^\dagger_a a^\dagger_b a_j}{a}
            a^\dagger_p a^\dagger_q a_s  a_r a^\dagger_a a^\dagger_b a_j a_i \right\}
            + \left\{
            \contraction{a^\dagger_p a^\dagger_q}{a}{{}_s a_r}{a}
            \contraction[1.25ex]{a^\dagger_p a^\dagger_q a_s}{a}{{}_r a^\dagger_p}{a}
            \contraction[1.5ex]{}{a}{{}^\dagger_p a^\dagger_q a_s a_r a^\dagger_a a^\dagger_b}{a}
            \contraction[1.75ex]{a^\dagger_p}{a}{{}^\dagger_q a_s a_r a^\dagger_a a^\dagger_b a_j}{a}
            a^\dagger_p a^\dagger_q a_s  a_r a^\dagger_a a^\dagger_b a_j a_i \right\}
            \Bigr) \ket{\Phi_0} \\ 
            &= \frac{1}{4} \sum_{ijab} \bra{ij}\hat{v}\ket{ab} t_{ij}^{ab}.
    \end{align*}
The final contribution to the correlation energy comes from the non-linear terms with the amplitudes squared. 
The contribution from the $\hat{T}^2$ is given by
    \begin{align*}
        & \bra{\Phi_0} \frac{1}{2} \left( \hat{V}_N \hat{T}_1^2 \right) \ket{\Phi_0} = \\
            & \quad \frac{1}{8} \sum_{pqrs} \sum_{ijab} \bra{pq}\hat{v}\ket{rs} t_i^a t_j^b 
            \bra{\Phi_0} \left(\normord{a^\dagger_p a^\dagger_q a_s  a_r} 
            \normord{a^\dagger_a a_i} \normord{a^\dagger_b a_j} \right)_c\ket{\Phi_0} \\ 
        &= \frac{1}{8} \sum_{pqrs} \sum_{ijab} \bra{pq}\hat{v}\ket{rs} t_i^a t_j^b \bra{\Phi_0} \\
        & \quad \Bigl( 
        \left\{
        \contraction{a^\dagger_p a^\dagger_q a_s}{a}{{}_r}{a}
        \contraction[1.25ex]{a^\dagger_p}{a}{{}^\dagger_q a_s a_r a^\dagger_a}{a}
        \contraction[1.5ex]{a^\dagger_p a^\dagger_q }{a}{{}_s a_r a^\dagger_a a_i}{a}
        \contraction[1.75ex]{}{a}{{}^\dagger_p a^\dagger_q a_s a_r a^\dagger_a a_i a^\dagger_b}{a}
        a^\dagger_p a^\dagger_q a_s  a_r a^\dagger_a a_i a^\dagger_b a_j \right\}
        +\left\{
        \contraction{a^\dagger_p a^\dagger_q}{a}{{}_s a_r}{a}
        \contraction[1.25ex]{a^\dagger_p}{a}{{}^\dagger_q a_s a_r a^\dagger_a}{a}
        \contraction[1.5ex]{a^\dagger_p a^\dagger_q a_s}{a}{{}_r a^\dagger_a a_i}{a}
        \contraction[1.75ex]{}{a}{{}^\dagger_p a^\dagger_q a_s a_r a^\dagger_a a_i a^\dagger_b}{a}
        a^\dagger_p a^\dagger_q a_s  a_r a^\dagger_a a_i a^\dagger_b a_j \right\}
        + \left\{
        \contraction{a^\dagger_p a^\dagger_q a_s}{a}{{}_r}{a}
        \contraction[1.25ex]{}{a}{{}^\dagger_p q^\dagger_q a_s a_r a^\dagger_a}{a}
        \contraction[1.5ex]{a^\dagger_p a^\dagger_q }{a}{{}_s a_r a^\dagger_a a_i}{a}
        \contraction[1.75ex]{a^\dagger_p}{a}{{}^\dagger_q a_s a_r a^\dagger_a a_i a^\dagger_b}{a}
        a^\dagger_p a^\dagger_q a_s  a_r a^\dagger_a a_i a^\dagger_b a_j \right\} \\
        & \quad +\left\{
        \contraction{a^\dagger_p a^\dagger_q}{a}{{}_s a_r}{a}
        \contraction[1.25ex]{}{a}{{}^\dagger_p q^\dagger_q a_s a_r a^\dagger_a}{a}
        \contraction[1.5ex]{a^\dagger_p a^\dagger_q a_s}{a}{{}_r a^\dagger_a a_i}{a}
        \contraction[1.75ex]{a^\dagger_p}{a}{{}^\dagger_q a_s a_r a^\dagger_a a_i a^\dagger_b}{a}
        a^\dagger_p a^\dagger_q a_s  a_r a^\dagger_a a_i a^\dagger_b a_j \right\}
        \Bigr) \ket{\Phi_0} \\ 
        &= \frac{1}{2} \sum_{ijab} \bra{ij}\hat{v}\ket{ab} t_i^a t_j^b.
    \end{align*}
 Collecting all terms we have   the final expression for the correlation energy with a two-body interaction given by
  \begin{equation}\label{eq:energyccsd}
  \Delta E=\sum_{ai}\langle i| \hat{f}|a \rangle t_{i}^{a}+\frac{1}{2} \sum_{ijab} \bra{ij}\hat{v}\ket{ab} t_i^a t_j^b+
  \frac{1}{4}\sum_{ijab}\langle ij | \hat{v}| ab \rangle t_{ij}^{ab}.
  \end{equation}
We leave it as a challenge to the reader to derive the corresponding equations for the Hamiltonian matrix elements of Eq.~\ref{eq:amplitudeeq}.


  There are several interesting features with the coupled cluster
  equations. With a truncation like CCSD or even with the inclusion of
  triples (CCSDT), we can include to infinite order correlations based
  on one-particle-one-hole and two-particle-two-hole contributions.
  We can include a large basis of single-particle states, normally not
  possible in standard FCI calculations. Typical FCI calculations for
  light nuclei $A\le 16$ can be performed in at most some few harmonic
  oscillator shells. For heavier nuclei, at most two major shells can
  be included due to too large dimensionalities.  However, Coupled
  Cluster theory is non-variational and if we want to find properties
  of excited states, additional calculations via for example equation
  of motion methods are needed \cite{shavittbartlett2009,hagen2014}.
  If correlations are strong, a single-reference ansatz may not be the
  best starting point and a multi-reference approximation is needed
  \cite{jansen2015}. Furthermore, we cannot quantify properly the
  error we make when truncations are made in the cluster operator.

  \subsection{The CCD approximation}

  We will now approximate the cluster operator $\hat{T}$ to include
  only $2p-2h$ correlations. This leads to the so-called CCD
  approximation, that is
  \[
  \hat{T}\approx
  \hat{T}_2=\frac{1}{4}\sum_{abij}t_{ij}^{ab}a^{\dagger}_aa^{\dagger}_ba_ja_i,
  \]
  meaning that we have
  \[
  \vert \Psi_0 \rangle \approx \vert \Psi_{CCD} \rangle =
  \exp{\left(\hat{T}_2\right)}\vert \Phi_0\rangle.
  \]

  Inserting these equations in the expression for the computation of
  the energy we have, with a Hamiltonian defined with respect to a
  general reference vacuum
  \[
  \hat{H}=\hat{H}_N+E_{\mathrm{ref}},
  \]
  with
  \[
  \hat{H}_N=\sum_{pq}\langle p \vert \hat{f} \vert q \rangle
  a^{\dagger}_pa_q + \frac{1}{4}\sum_{pqrs}\langle pq \vert \hat{v}
  \vert rs \rangle a^{\dagger}_pa^{\dagger}_qa_sa_r,
  \]
  we obtain that the energy can be written as
  \[
  \langle \Phi_0 \vert
  \exp{-\left(\hat{T}_2\right)}\hat{H}_N\exp{\left(\hat{T}_2\right)}\vert
  \Phi_0\rangle = \langle \Phi_0 \vert \hat{H}_N(1+\hat{T}_2)\vert
  \Phi_0\rangle = E_{CCD}.
  \]
  This quantity becomes
  \[
  E_{CCD}=E_{\mathrm{ref}}+\frac{1}{4}\sum_{abij}\langle ij \vert
  \hat{v} \vert ab \rangle t_{ij}^{ab},
  \]
  where the latter is the correlation energy from this level of
  approximation of coupled cluster  theory.  Similarly, the expression for the
  amplitudes reads (see problem \ref{problem:amplitudes})
  \[
  \langle \Phi_{ij}^{ab} \vert
  \exp{\left(-\hat{T}_2\right)}\hat{H}_N\exp{\left(\hat{T}_2\right)}\vert
  \Phi_0\rangle = 0.
  \]
  These equations can be reduced to (after several applications of
  Wick's theorem), for all $i > j$ and all $a > b$,
  \begin{align}
  0 = \langle ab \vert \hat{v} \vert ij \rangle +
  \left(\epsilon_a+\epsilon_b-\epsilon_i-\epsilon_j\right)t_{ij}^{ab}
  & \nonumber \\ +\frac{1}{2}\sum_{cd} \langle ab \vert \hat{v} \vert
  cd \rangle t_{ij}^{cd}+\frac{1}{2}\sum_{kl} \langle kl \vert \hat{v}
  \vert ij \rangle t_{kl}^{ab}+\hat{P}(ij\vert ab)\sum_{kc} \langle kb
  \vert \hat{v} \vert cj \rangle t_{ik}^{ac} & \nonumber
  \\ +\frac{1}{4}\sum_{klcd} \langle kl \vert \hat{v} \vert cd \rangle
  t_{ij}^{cd}t_{kl}^{ab}+\hat{P}(ij)\sum_{klcd} \langle kl \vert
  \hat{v} \vert cd \rangle t_{ik}^{ac}t_{jl}^{bd}& \nonumber
  \\ -\frac{1}{2}\hat{P}(ij)\sum_{klcd} \langle kl \vert \hat{v} \vert
  cd \rangle t_{ik}^{dc}t_{lj}^{ab}-\frac{1}{2}\hat{P}(ab)\sum_{klcd}
  \langle kl \vert \hat{v} \vert cd \rangle t_{lk}^{ac}t_{ij}^{db},&
  \label{eq:ccd}
  \end{align}
  where we have defined
  \[
  \hat{P}\left(ab\right)= 1-\hat{P}_{ab},
  \]
  where $\hat{P}_{ab}$ interchanges two particles occupying the
  quantum numbers $a$ and $b$.  The operator $\hat{P}(ij\vert ab)$ is
  defined as
  \[
  \hat{P}(ij\vert ab) = (1-\hat{P}_{ij})(1-\hat{P}_{ab}).
  \]
  Recall also that the unknown amplitudes $t_{ij}^{ab}$ represent
  anti-symmetrized matrix elements, meaning that they obey the same
  symmetry relations as the two-body interaction, that is
  \[
  t_{ij}^{ab}=-t_{ji}^{ab}=-t_{ij}^{ba}=t_{ji}^{ba}.
  \]
  The two-body matrix elements are also anti-symmetrized, meaning that
  \[
  \langle ab \vert \hat{v} \vert ij \rangle = -\langle ab \vert
  \hat{v} \vert ji \rangle= -\langle ba \vert \hat{v} \vert ij
  \rangle=\langle ba \vert \hat{v} \vert ji \rangle.
  \]
  The non-linear equations for the unknown amplitudes $t_{ij}^{ab}$
  are solved iteratively. We discuss the implementation of these
  equations below.

  \subsection{Approximations to the full CCD equations.}
  It is useful to make approximations to the equations for the
  amplitudes. These serve as important benchmarks when we are to develop a many-body code.
The standard method for solving these equations is to
  set up an iterative scheme where method's like Newton's method or
  similar root searching methods are used to find the amplitudes see for example Ref.~\cite{hagen2007}.

  Iterative solvers need a guess for the amplitudes. A good starting
  point is to use the correlated wave operator from perturbation
  theory to first order in the interaction.  This means that we define
  the zeroth approximation to the amplitudes as
  \[
  t^{(0)}=\frac{\langle ab \vert \hat{v} \vert ij
    \rangle}{\left(\epsilon_i+\epsilon_j-\epsilon_a-\epsilon_b\right)},
  \]
  leading to our first approximation for the correlation energy at the
  CCD level to be equal to second-order perturbation theory without
  $1p-1h$ excitations, namely
  \[
  \Delta E_{\mathrm{CCD}}^{(0)}=\frac{1}{4}\sum_{abij} \frac{\langle
    ij \vert \hat{v} \vert ab \rangle \langle ab \vert \hat{v} \vert
    ij
    \rangle}{\left(\epsilon_i+\epsilon_j-\epsilon_a-\epsilon_b\right)}.
  \]

  With this starting point, we are now ready to solve
  Eq. (\ref{eq:ccd}) iteratively. Before we attack the full equations,
  it is however instructive to study a truncated version of the
  equations. We will first study the following approximation where we
  take away all terms except the linear terms that involve the
  single-particle energies and the two-particle intermediate
  excitations, that is
  \begin{equation}
  0 = \langle ab \vert \hat{v} \vert ij \rangle +
  \left(\epsilon_a+\epsilon_b-\epsilon_i-\epsilon_j\right)t_{ij}^{ab}+\frac{1}{2}\sum_{cd}
  \langle ab \vert \hat{v} \vert cd \rangle t_{ij}^{cd}.
  \label{eq:ccd1}
  \end{equation}

  Setting the single-particle energies for the hole states equal to an
  energy variable $\omega = \epsilon_i+\epsilon_j$,
  Eq.~(\ref{eq:ccd1}) reduces to the well-known equations for the
  so-called $G$-matrix, widely used in infinite matter and finite
  nuclei studies, see for example Refs.~\cite{day1967,hh2000}.  The equation can then be reordered
  and solved by matrix inversion.  To see this let us define the
  following quantity
  \[
  \tau_{ij}^{ab}=
  \left(\omega-\epsilon_a-\epsilon_b\right)t_{ij}^{ab},
  \]
  and inserting
  \[
  1=\frac{\left(\omega-\epsilon_c-\epsilon_d\right)}{\left(\omega-\epsilon_c-\epsilon_d\right)},
  \]
  in the intermediate sums over $cd$ in Eq.~(\ref{eq:ccd1}), we can
  rewrite the latter equation as
  \[
  \tau_{ij}^{ab}(\omega)= \langle ab \vert \hat{v} \vert ij \rangle +
  \frac{1}{2}\sum_{cd} \langle ab \vert \hat{v} \vert cd \rangle
  \frac{1}{\omega-\epsilon_c-\epsilon_d}\tau_{ij}^{cd}(\omega),
  \]
  where we have indicated an explicit energy dependence. This
  equation, transforming a two-particle configuration into a single
  index, can be transformed into a matrix inversion problem.  Solving
  the equations for a fixed energy $\omega$ allows us to compare
  directly with results from Green's function theory when only
  two-particle intermediate states are included.

  To solve Eq. (\ref{eq:ccd1}), we would thus start with a guess for
  the unknown amplitudes, normally using the wave operator defined by
  first order in perturbation theory, leading to a zeroth
  approximation to the energy given by second-order perturbation
  theory for the correlation energy.  A simple approach to the
  solution of Eq.~(\ref{eq:ccd1}), is to thus to
  \begin{enumerate}
  \item Start with a guess for the amplitudes and compute the zeroth
    approximation to the correlation energy

  \item Use the ansatz for the amplitudes to solve Eq. (\ref{eq:ccd1})
    via for example your root-finding method of choice (Newton's
    method or modifications thereof can be used) and continue these
    iterations till the correlation energy does not change more than a
    prefixed quantity $\lambda$; $\Delta E_{\mathrm{CCD}}^{(i)}-\Delta
    E_{\mathrm{CCD}}^{(i-1)} \le \lambda$.

  \item It is common during the iterations to scale the amplitudes
    with a parameter $\alpha$, with $\alpha \in (0,1]$ as
      $t^{(i)}=\alpha t^{(i)}+(1-\alpha)t^{(i-1)}$.
  \end{enumerate}

  \noindent
  The next approximation is to include the two-hole term in
  Eq.~(\ref{eq:ccd}), a term which allow us to make a link with
  Green's function theory with two-particle and two-hole
  correlations discussed in chapter 11. This means that we solve
  \begin{equation}
  0 = \langle ab \vert \hat{v} \vert ij \rangle +
  \left(\epsilon_a+\epsilon_b-\epsilon_i-\epsilon_j\right)t_{ij}^{ab}+\frac{1}{2}\sum_{cd}
  \langle ab \vert \hat{v} \vert cd \rangle
  t_{ij}^{cd}+\frac{1}{2}\sum_{kl} \langle kl \vert \hat{v} \vert ij
  \rangle t_{kl}^{ab}.
  \label{eq:ccd2}
  \end{equation}
  This equation is solved the same way as we would do for
  Eq. (\ref{eq:ccd1}). The final step is then to include all terms in
  Eq. (\ref{eq:ccd}).

\section{Developing a numerical project}\label{sec:chap8numproject}

When developing numerical projects we would like to emphasize some
topics we feel can help in structuring a large computational project.
Amongst these, the validation and verification of the correctness of
the employed algorithms and programs are central issues which can
save you a lot of time when developing a numerical project. In the
discussions below we will use repeatedly the simple pairing model of
problem \ref{problem:pairingmodel}.  This model allows for benchmarks
against exact results. In addition, it provides analytical answers to
several approximations, from perturbation theory to specific terms in the solution of the coupled
cluster equations, the in-medium similarity renormalization group
approach of chapter 10 and the Green's function approach of chapter
11. 

It is also important to develop an understanding of how our
algorithms can go wrong and how they can be implemented in order to
run as efficiently as possible. When working on a numerical project it
is important to keep in mind that computing covers numerical as well
as symbolic computing and paper and pencil solutions. Furthermore,
version control is something we strongly recommend. It does not only
save you time in case you struggle with odd errors in a new version of
your code. It allows you also to make science reproducible.  Making
your codes available to a larger audience and provinding proper
benchmarks allows fellow scientists to test what you have developed,
and perhaps come with considerable improvements and/or find flaws
or errors you were not aware of. 
Sharing your codes using for example modern version control
software makes thus your science reproducible and adds in a natural
way a sound ethical scientific element to what you develop.
In the discussions below, you will thus find several links to our 
common repository for the codes that we have developed.


When building up a numerical project there are several elements you should think of, amongst these we take the liberty of mentioning the following:
\begin{itemize}
\item Structure a code in terms of functions.
\item Modularize your codes.
\item Be able to  read input data flexibly.
\item Write unit tests (test functions) and let your code undergo heavy testing.
\item Refactor code in terms of classes (instead of functions only).
\item Conduct and automate large-scale numerical experiments.
\item New code is added in a modular fashion to a library (modules).
\item Programs are run through convenient user interfaces.
\item Use scripts in order to automatize  tedious manual work.
\item Make sure your scientific investigations are reproducible and document properly your results.
\end{itemize}


\subsection{Developing a CCD code for infinite matter}
  This section will focus on writing a working CCD code from
  scratch. If you are familiar with writing quantum many-body physics
  codes, feel free to skip ahead as we are going to go into some
  detail about implementing CCD as a computer code now. As we saw
  earlier, the CCD equations can be written as
  \begin{align}
  \left(\epsilon_i+\epsilon_j-\epsilon_a-\epsilon_b\right)t_{ij}^{ab}
  = \langle ab \vert \hat{v} \vert ij \rangle & \nonumber
  \\ +\frac{1}{2}\sum_{cd} \langle ab \vert \hat{v} \vert cd \rangle
  t_{ij}^{cd}+\frac{1}{2}\sum_{kl} \langle kl \vert \hat{v} \vert ij
  \rangle t_{kl}^{ab}+\hat{P}(ij\vert ab)\sum_{kc} \langle kb \vert
  \hat{v} \vert cj \rangle t_{ik}^{ac} & \nonumber
  \\ +\frac{1}{4}\sum_{klcd} \langle kl \vert \hat{v} \vert cd \rangle
  t_{ij}^{cd}t_{kl}^{ab}+\hat{P}(ij)\sum_{klcd} \langle kl \vert
  \hat{v} \vert cd \rangle t_{ik}^{ac}t_{jl}^{bd}& \nonumber
  \\ -\frac{1}{2}\hat{P}(ij)\sum_{klcd} \langle kl \vert \hat{v} \vert
  cd \rangle t_{ik}^{dc}t_{lj}^{ab}-\frac{1}{2}\hat{P}(ab)\sum_{klcd}
  \langle kl \vert \hat{v} \vert cd \rangle t_{lk}^{ac}t_{ij}^{db},&
  \label{eq:ccd2}
  \end{align}
  for all $i < j$ and all $a < b$, using the standard notation that
  $a,b,...$ are particle states and $i,j,...$ are hole states. With
  the CCD correlation energy given by
  \begin{equation}
  \Delta E_{CCD} = \frac{1}{4} \sum_{ijab}
  \braket{ij|\hat{v}|ab}t^{ab}_{ij}.
  \label{eq:ccdcorr}
  \end{equation}
  One way to solve these equations, is to write equation
  (\ref{eq:ccd2}) as a series of iterative nonlinear algebraic
  equations
  \begin{align}
  t_{ij}^{ab}{}^{(n+1)} = \frac{1}{\epsilon^{ab}_{ij}} \bigg(\langle
  ab \vert \hat{v} \vert ij \rangle & \nonumber
  \\ +\frac{1}{2}\sum_{cd} \langle ab \vert \hat{v} \vert cd \rangle
  t_{ij}^{cd}{}^{(n)}+\frac{1}{2}\sum_{kl} \langle kl \vert \hat{v}
  \vert ij \rangle t_{kl}^{ab}{}^{(n)}+\hat{P}(ij\vert ab)\sum_{kc}
  \langle kb \vert \hat{v} \vert cj \rangle t_{ik}^{ac}{}^{(n)} &
  \nonumber \\ +\frac{1}{4}\sum_{klcd} \langle kl \vert \hat{v} \vert
  cd \rangle
  t_{ij}^{cd}{}^{(n)}t_{kl}^{ab}{}^{(n)}+\hat{P}(ij)\sum_{klcd}
  \langle kl \vert \hat{v} \vert cd \rangle
  t_{ik}^{ac}{}^{(n)}t_{jl}^{bd}{}^{(n)}& \nonumber
  \\ -\frac{1}{2}\hat{P}(ij)\sum_{klcd} \langle kl \vert \hat{v} \vert
  cd \rangle
  t_{ik}^{dc}{}^{(n)}t_{lj}^{ab}{}^{(n)}-\frac{1}{2}\hat{P}(ab)\sum_{klcd}
  \langle kl \vert \hat{v} \vert cd \rangle
  t_{lk}^{ac}{}^{(n)}t_{ij}^{db}{}^{(n)} \bigg),&
  \label{eq:ccd3}
  \end{align}
  for all $i < j$ and all $a < b$, where $\epsilon^{ab}_{ij} =
  \left(\epsilon_i+\epsilon_j-\epsilon_a-\epsilon_b\right)$, and
  $t_{ij}^{ab}{}^{(n)}$ is the $t$ amplitude for the nth iteration of
  the series. This way, given some starting guess
  $t_{ij}^{ab}{}^{(0)}$, we can generate subsequent $t$ amplitudes
  that converges to some value. Discussion of the mathematical details
  regarding convergence will be presented below; for now we will talk
  about implementing these equations into a computer program and
  assume convergence. In pseudocode, the function that updates the $t$
  amplitudes looks like

\begin{svgraybox}
  \begin{algorithmic} 
  \State CCD\_Update() \For{$i \in \{0,N_{fermi}-1\}$ } \For{$j \in
    \{0,N_{fermi}-1\}$ } \For{$a \in \{N_{fermi},N_{sp}-1\}$ } \For{$b
    \in \{N_{fermi},N_{sp}-1\}$ } \State $\text{sum} \gets
  \text{TBME}[\text{index}(a,b,i,j)$] \For{$c \in
    \{N_{fermi},N_{sp}-1\}$ } \For{$d \in \{N_{fermi},N_{sp}-1\}$ }
  \State $\text{sum} \gets \text{sum} +
  0.5\times\text{ME}[\text{index}(a,b,c,d)] \times
  t\_\text{amplitudes}\_\text{old}[\text{index}(c,d,i,j)]$ \EndFor
  \EndFor \State ...  \State sum $\gets$ sum + (all other terms)
  \State ...  \State energy\_denom =
  SP\_energy[$i$]+SP\_energy[$j$]-SP\_energy[$a$]-SP\_energy[$b$]
  \State t\_amplitudes[index($a,b,i,j$)] = sum/energy\_denom \EndFor
  \EndFor \EndFor \EndFor
  \end{algorithmic}
\end{svgraybox} 
 Where $N_{fermi}$ is the fermi level and $N_{sp}$ is the total
  number of single particle (s.p.) states, indexed from 0 to
  $N_{sp}-1$. At the most basic level, the CCD equations are just the
  addition of many products containing $t_{ij}^{ab}$ amplitudes and
  two-body matrix elements (TBMEs) $\braket{ij|\hat{v}|ab}$.
  Care should thus be placed into how we store these objects. These are
  objects with four indices and a  sensible first implementation
  of the CCD equations would be to create two four-dimensional arrays to store the
  objects. However, it is often more convenient to work with simple
  one-dimensional arrays instead. The function $index()$ maps the four
  indices onto one index so that a one-dimensional array can be used. An example
  of such a function is
\begin{svgraybox} 
 \begin{algorithmic}
  \Function{index}{$p,q,r,s$} \State \textbf{return} $p\times N_{sp}^3 +
  q\times N_{sp}^2 + r\times N_{sp} + s$ \EndFunction
  \end{algorithmic}
\end{svgraybox}
  Because elements with repeated indices vanish,
  $t_{ii}^{ab}=t_{ij}^{aa}=0$ and
  $\braket{pp|\hat{v}|rs}=\braket{pq|\hat{v}|rr}=0$, data structures
  using this index function will contain many elements that are
  automatically zero. This means that we need to discuss more efficient storage
  strategies later. Notice also that we are looping over all
  indices $i,j,a,b$, rather than the restricted indices. This means that we
  are doing redundant work. The reason for presenting the equations this way is merely pedagogical. When developing a program, we would recommend to write a code which is as close as possible to the mathematical expressions. The first version of our code will then often be slow. However It is  often less error-prone and simpler to code in this way. 
Below we will however unrestrict these indices in order to achieve a better speed up of our code. 

   The goal of our code is to calculate the correlation energy,
   $\Delta E_{CCD}$, meaning that after each iteration of our equations, we use
   our newest $t$ amplitudes to update the correlation energy
  \begin{equation}
  \Delta E_{CCD}^{(n)} = \frac{1}{4} \sum_{ijab}
  \braket{ij|\hat{v}|ab}t^{ab}_{ij}{}^{(n)}.
  \end{equation}
  We check that our result is converged by testing whether the
  most recent iteration has changed the correlation energy by less
  than some tolerance threshold $\eta$,
  \begin{equation}
  \eta > | \Delta E_{CCD}^{(n+1)} - \Delta E_{CCD}^{(n)} |.
  \end{equation}
  The basic structure of the iterative process will look like
\begin{svgraybox}
  \begin{algorithmic}
    \While {(abs(energy\_Diff) $>$ tolerance)} \State CCD\_Update()
    \State correlation\_Energy $\gets$ CCD\_Corr\_Energy() \State
    energy\_Diff $\gets$ correlation\_Energy -
    correlation\_Energy\_old \State correlation\_Energy\_old $\gets$
    correlation\_Energy \State t\_amplitudes\_old $\gets$
    t\_amplitudes \EndWhile
  \end{algorithmic}
\end{svgraybox}
  Prior to this algorithm, the $t$ amplitudes should be initalized,
  $t_{ij}^{ab}{}^{(0)}$. A particularly convenient choice, as discussed above, is to set
  $t_{ij}^{ab}{}^{(0)} = 0$. Notice that if this starting point is
  used, then
  \begin{equation}
  t_{ij}^{ab}{}^{(1)} = \frac{\langle ab \vert \hat{v} \vert ij\rangle}{\epsilon^{ab}_{ij}},
  \label{eq:ccdGuess}
  \end{equation}
resulting in the correlation energy
  \begin{equation}
  \Delta E_{CCD}^{(1)} = \frac{1}{4} \sum_{ijab}\frac{\langle ij \vert \hat{v} \vert ab \rangle\langle ab \vert \hat{v} \vert ij \rangle}{\epsilon^{ab}_{ij}},
  \end{equation}
  which is the result from many-body perturbation theory to second order (MBPT2).  This is a very useful result, as one iteration
  of the CCD equations can be ran, and checked against MBPT2 to give
  some confidence that everything is working correctly. To check that
  everything is working, it is useful to run the code using a minimal
  example. We turn then our attention to the simple pairing model Hamiltonian of problem \ref{problem:pairingmodel},
  \begin{equation}
  \hat{H}_0 = \xi \sum_{p \sigma} (p-1) a^{\dagger}_{p \sigma} a_{p
    \sigma}\label{eq:sppairing}
  \end{equation}
  \begin{equation}
  \hat{V} = -\frac{1}{2}g \sum_{pq} a^{\dagger}_{p+}a^{\dagger}_{p-}
  a_{q-}a_{q+}\label{eq:intpairing}
  \end{equation}
  which represents a basic pairing model with $p$ levels, each having a
  spin degeneracy of 2. The form of the coupled cluster equations 
  uses single-particle states that are eigenstates of the
  Hartree-Fock operator, $\left(\hat{u}+\hat{u}_{\text{HF}}\right)\vert
  p\rangle=\epsilon_{p}\vert p\rangle$. In the pairing model, this
  condition is already fulfilled. All we have to do is define the
  lowest $N_{\mathrm{Fermi}}$ states as holes and  redefine the single-particle
  energies,
  \begin{equation}\label{eq:pairingsp}
  \epsilon_q = h_{qq} + \sum_{i} \braket{qi|\hat{v}|qi}.
  \end{equation}
  To be more specific, let us look at the pairing model with four
  particles and eight single-particle states. These states (with $\xi =1.0$) could be labeled as shown in 
Table \ref{tab:pairingmodelsp}.
\begin{table}
\caption{Single-particle states and their quantum numbers and their energies from Eq.~(\ref{eq:pairingsp}). The degeneracy for every quantum number $p$ is equal to 2 due to the two possible spin values.} \label{tab:pairingmodelsp}
  \begin{center}
      \begin{tabular}{| l | l | l | l | l |}
      \hline State Label & p & 2s$_z$ & E & type\\ \hline 0 & 1 & 1 &
      -g/2 & hole \\ \hline 1 & 1 & -1 & -g/2 & hole \\ \hline 2 & 2 &
      1 & 1-g/2 & hole \\ \hline 3 & 2 & -1 & 1-g/2 & hole \\ \hline 4
      & 3 & 1 & 2 & particle \\ \hline 5 & 3 & -1 & 2 & particle
      \\ \hline 6 & 4 & 1 & 3 & particle \\ \hline 7 & 4 & -1 & 3 &
      particle \\ \hline
      \end{tabular}
  \end{center}
\end{table}
The Hamiltonian matrix for this   four-particle problem with no broken pairs is defined by six possible Slater determinants,
one representing the ground state and zero-particle-zero-hole excitations $0p-0h$, four representing various $2p-2h$ excitations and finally one representing a $4p-4h$ excitation. Problem \ref{problem:pairingmodel} gives us for this specific problem
  \[
  H = \begin{bmatrix}
  2\delta -g & -g/2 & -g/2 & -g/2 & -g/2 & 0 \\ -g/2 & 4\delta -g &
  -g/2 & -g/2 & -0 & -g/2 \\ -g/2 & -g/2 & 6\delta -g & 0 & -g/2 &
  -g/2 \\ -g/2 & -g/2 & 0 & 6\delta-g & -g/2 & -g/2 \\ -g/2 & 0 & -g/2
  & -g/2 & 8\delta-g & -g/2 \\ 0 & -g/2 & -g/2 & -g/2 & -g/2 &
  10\delta -g
  \end{bmatrix}
  \]
  The following python program diagonalizes the above Hamiltonian
  matrix for a given span of interaction strength values, performing
  a full configuration interaction calculation. It plots the correlation energy, that is the difference between the ground state energy and the reference energy. Furthermore, for the pairing model we have added results from perturbation theory to second order (MBPT2)
and third order in the interaction MBPT3. Second order perturbation theory includes diagram (2) of Fig.~\ref{fig:goldstone}
while MBPT3 includes diagrams (3), (4), (5), (8) and (9) as well. Note that diagram (3) is zero for the pairing model and that diagrams (8) and (9) contribute as well with a canonical Hartree-Fock basis. 
  In the case of the simple pairing model it is easy to calculate
  $\Delta E_{MBPT2}$ anyltically. This is a very useful  check of our codes since this analytical expression  can  also be used to check our first CCD iteration.
We restate this expression here but restrict the sums over single-particle states
  \[
  \Delta E_{MBPT2} = \frac{1}{4} \sum_{abij} \frac{\braket{ij|\hat{v}|ab}
    \braket{ab|\hat{v}|ij}}{ \epsilon_{ij}^{ab}} = \sum_{a<b,i<j}
  \frac{\braket{ij|\hat{v}|ab} \braket{ab|\hat{v}|ij}}{ \epsilon_{ij}^{ab}}
  \]
  For our pairing example we obtain the following result
  \[
  \Delta E_{MBPT2} = \frac{\braket{01|\hat{v}|45}^2}{\epsilon_{01}^{45}} +
  \frac{\braket{01|\hat{v}|67}^2}{\epsilon_{01}^{67}} +
  \frac{\braket{23|\hat{v}|45}^2}{\epsilon_{23}^{45}} +
  \frac{\braket{23|\hat{v}|67}^2}{\epsilon_{23}^{67}},
  \]
which translates into
  \[
  \Delta E_{MBPT2} = -\frac{g^2}{4} \bigg( \frac{1}{ 4 + g} +
  \frac{1}{ 6 + g} + \frac{1}{ 2 + g} + \frac{1}{ 4 + g} \bigg).
  \]
 This expression can be used to check the results
  for any value of $g$ and provides thereby an important test of  our codes.
See \url{https://github.com/ManyBodyPhysics/LectureNotesPhysics/blob/master/doc/src/Chapter8-programs/python/mbpt.py} for the Python code shown here.
\begin{lstlisting}
#!/usr/bin/python
from sympy import *
from pylab import *
import matplotlib.pyplot as plt

below_fermi = (0,1,2,3)
above_fermi = (4,5,6,7)
states = [(1,1),(1,-1),(2,1),(2,-1),(3,1),(3,-1),(4,1),(4,-1)]
N = 8
g = Symbol('g')

def h0(p,q):
    if p == q:
        p1, s1 = states[p]
        return (p1 - 1)
    else:
        return 0

def f(p,q):
    if p == q:
        return 0
    s = h0(p,q)
    for i in below_fermi:
        s += assym(p,i,q,i)
        return s


def assym(p,q,r,s):
    p1, s1 = states[p]
    p2, s2 = states[q]
    p3, s3 = states[r]
    p4, s4 = states[s]

    if p1 != p2 or p3 != p4:
        return 0
    if s1 == s2 or s3 == s4:
        return 0
    if s1 == s3 and s2 == s4:
        return -g/2.
    if s1 == s4 and s2 == s3:
        return g/2.

def eps(holes, particles):
    E = 0
    for h in holes:
        p, s = states[h]
        E += (p-1)
    for p in particles:
        p, s = states[p]
        E -= (p-1)
    return E


# Diagram 1
s1 = 0
for a in above_fermi:
    for b in above_fermi:
        for i in below_fermi:
            for j in below_fermi:
                s1 += 0.25*assym(a,b,i,j)*assym(i,j,a,b)/eps((i,j),(a,b))


# Diagram 3
s3 = 0
for a in above_fermi:
   for b in above_fermi:
       for c in above_fermi:
           for i in below_fermi:
               for j in below_fermi:
                   for k in below_fermi:
                       s3 += assym(i,j,a,b)*assym(a,c,j,k)*assym(b,k,c,i)
                             /eps((i,j),(a,b))/eps((k,j),(a,c))

# Diagram 4
s4 = 0
for a in above_fermi:
    for b in above_fermi:
        for c in above_fermi:
            for d in above_fermi:
                for i in below_fermi:
                    for j in below_fermi:
                        s4 += 0.125*assym(i,j,a,b)*assym(a,b,c,d)*assym(c,d,i,j)
                               /eps((i,j),(a,b))/eps((i,j),(c,d))

# Diagram 5
s5 = 0
for a in above_fermi:
    for b in above_fermi:
        for i in below_fermi:
            for j in below_fermi:
                for k in below_fermi:
                    for l in below_fermi:
                        s5 += 0.125*assym(i,j,a,b)*assym(k,l,i,j)*assym(a,b,k,l)
                              /eps((i,j),(a,b))/eps((k,l),(a,b))

# Diagram 8 
s8 = 0
for a in above_fermi:
    for b in above_fermi:
        for i in below_fermi:
            for j in below_fermi:
                for k in below_fermi:
                    s8 -= 0.5*assym(i,j,a,b)*assym(a,b,i,k)*f(k,j)
                            /eps((i,j),(a,b))/eps((i,k),(a,b))

# Diagram 9 
s9 = 0
for a in above_fermi:
    for b in above_fermi:
        for c in above_fermi:
            for i in below_fermi:
                for j in below_fermi:
                    s9 += 0.5*assym(i,j,a,b)*assym(a,c,i,j)*f(b,c)
                        /eps((i,j),(a,b))/eps((i,j),(a,c))

ga = linspace(-1,1,20)
e1 = []
corr2 = []
corr3 = []

for g_val in ga:
    H1 = matrix([[2-g_val , -g_val/2.,  -g_val/2., -g_val/2., -g_val/2.,     0],
                         [-g_val/2.,   4-g_val,  -g_val/2., -g_val/2.,    0., -g_val/2.],
                         [-g_val/2., -g_val/2.,    6-g_val,     0, -g_val/2., -g_val/2.],
                 [-g_val/2., -g_val/2.,      0,   6-g_val, -g_val/2., -g_val/2.],
                 [-g_val/2.,     0,  -g_val/2., -g_val/2.,   8-g_val, -g_val/2.],
                 [0    , -g_val/2.,  -g_val/2., -g_val/2., -g_val/2.,  10-g_val]])

    u1, v1 = linalg.eig(H1)
    e1.append(min(u1))
    corr2.append((s1).subs(g,g_val))
    corr3.append((s1+s3+s4+s5).subs(g,g_val))

exact = e1 - (2-ga)

plt.axis([-1,1,-0.5,0.05])
plt.xlabel(r'Interaction strength, $g$', fontsize=16)
plt.ylabel(r'Correlation energy', fontsize=16)
exact = plt.plot(ga, exact,'b-*',linewidth = 2.0, label = 'Exact')
mbpt2 = plt.plot(ga, corr2,'r:.', linewidth = 2.0, label = 'MBPT2')
mbpt3 = plt.plot(ga, corr3, 'm:v',linewidth = 2.0, label = 'MBPT3')
plt.legend()
plt.savefig('perturbationtheory.pdf', format='pdf')
plt.show()
\end{lstlisting}
Figure \ref{fig:diagpairing} shows the resulting correlation energies for the exact case, MBPT2 and MBPT3.
  \begin{figure}
    \includegraphics[width=\linewidth]{Chapter8-figures/perturbationtheory.pdf}
    \caption{Correlation energy for the pairing model with exact diagonalization, MBPT2 and perturbation theory to third order MBPT3 for a range of interaction values.}
    \label{fig:diagpairing}
  \end{figure}
We note from the above program that we have coded the expressions for the various diagrams
following strictly the mathematical expressions of for example Eqs.~(\ref{eq:diag3})-(\ref{eq:diag5}).
This means that for every diagram we loop explicitely  over every single-particle state. As we will
  see later, this is extremely inefficient from a computational point
  of view. In our discussions below, we will rewrite
  the computations of most diagrams in terms of efficient
  matrix-matrix multiplications or matrix-vector multiplications.  
  Figure \ref{fig:diagpairing} shows us that the approximation to both
  second and third order are very good when the interaction strength
  is small and contained in the interval $g\in[-0.5,0.5]$, but as the
  interaction gets stronger in absolute value the agreement  with the exact reference energy for  MBPT2 and MBPT3 worsens. We also note
  that the third-order result is actually worse than the second order
  result for larger values of the interaction strength, indicating
  that there is no guarantee that higher orders in many-body
  perturbation theory may reduce the relative error in a systematic
  way.  Adding fourth order contributions as shown in Fig.~\ref{fig:pairingccmbpt4}
  for negative interaction strengths gives a
  better result than second and third order, while for $g>0$ the
  relative error is worse.  The fourth order contributions (not shown in the above figure) include also four-particle-four-hole correaltions.
However, the disagreement for stronger interaction values hints at  the possibility that many-body perturbation theory may not converge order by order.  Perturbative studies of nuclear systems may thus be questionable, unless selected contributions that soften the interactions are properly softened.   
 We note also the non-variational character
  of many-body perturbation theory, with results at different levels of many-body perturbation theory either
  overshooting or undershooting the true ground state correlation energy. 
The coupled cluster results are included in Fig.~\ref{fig:pairingccmbpt4} where we display the difference between the exact correlation energy and the correlation energy obtained with many-body perturbation theory to fourth order.  The MBPT4 results show a better agreement with experiment, but here as well for larger positive values of the interaction, we see clear signs of deviations. 
  \begin{figure}
    \includegraphics[width=\linewidth]{Chapter8-figures/CCDMBPT4theory.pdf}
    \caption{Correlation energy for the pairing model with exact diagonalization, CCD and perturbation theory to fourth order (MBPT4) for a range of interaction values.}
    \label{fig:pairingccmbpt4}
  \end{figure}
In Table \ref{tab:selectedbenchmarks} we list for the sake of completeness also our coupled cluster results at the CCD level for the same system.
\begin{table}
\caption{Coupled cluster and MBPT2 results for the simple pairing model with eight single-particle levels and four spin $1/2$ fermions
for different values of the interaction strength $g$.} \label{tab:selectedbenchmarks}
  \begin{center}
      \begin{tabular}{| l | l | l | l |}
      \hline g & E$_{ref}$ & $\Delta E_{MBPT2}$ & $\Delta E_{CCD}$
      \\ \hline -1.0 & 3 & -0.46667 & -0.21895* \\ \hline -0.5 & 2.5 &
      -0.08874 & -0.06306 \\ \hline 0.0 & 2 & 0 & 0 \\ \hline 0.5 &
      1.5 & -0.06239 & -0.08336 \\ \hline 1.0 & 1 & -0.21905 &
      -0.36956 \\ \hline
      \end{tabular}
  \end{center}
\end{table}
  The $g=-1.0$ case diverges without implementing iterative
  mixing. Sometimes iterative solvers run into oscillating solutions,
  and mixing can help the iterations break this cycle.
  \begin{equation}
  t^{(i)} = \alpha t^{(i)}_{no\_mixing} + (1 - \alpha) t^{(i-1)}
  \end{equation}




  Once a working pairing model has been implemented, improvements can
  start to be made, all the while using the pairing model to make sure
  that the code is still working and giving correct answers. Realistic
  systems will be much larger than this small pairing example.

  One limitation that will be ran into while trying to do realistic
  CCD calculations is that of memory. The four-indexed two-body matrix elements (TBMEs) and
  $t$-amplitudes have to store a lot of elements, and the size of these
  arrays can quickly exceed the available memory on
  a machine. If a calculation wants to use 500 single-particle basis states, then
  a structure like $\braket{pq|v|rs}$ will need a  length of 500 for each of
  its four indices, which means it will have $500^4 = 625\times 10^8$
  elements. To get a handle on how much memory is used, consider the
  elements as double-precision floating point type. One double takes
  up 8 bytes of memory. This specific array would take up $8\times 625\times 10^8$ bytes
  = $5000 \times 10^8$ bytes = $500$ Gbytes of memory. Most personal
  computers in 2016 have 4-8 Gbytes of RAM, meaning that this calculation would
  be way out of reach. There are supercomputers that can handle
  applications using 500 Gbytes of memory, but we can quickly reduce
  the total memory required by applying some physical arguments. In
  addition to vanishing elements with repeated indices, mentioned
  above, elements that do not obey certain symmetries are also
  zero. Almost all realistic two-body forces preserve some quantities
  due to symmetries in the interaction. For example, an interaction
  with rotational symmetry will conserve angular momentum. This means
  that a two-body ket state $\ket{rs}$, which has some set of quantum
  numbers, will retain quantum numbers corresponding to the
  interaction symmetries after being acted on by $\hat{v}$. This state
  is then projected onto $\ket{pq}$ with its own set of quantum
  numbers. Thus $\braket{pq|v|rs}$ is only non-zero if $\ket{pq}$ and
  $\ket{rs}$ share the same quantum numbers that are preserved by
  $\hat{v}$. In addition, because the cluster operators represent
  excitations due to the interaction, $t_{ij}^{ab}$ is only non-zero
  if $\ket{ij}$ has the same relevant quantum numbers as $\ket{ab}$.

  To take advantage of this, these two-body ket states can be
  organized into ``channels'' of shared quantum numbers. In the case
  of the pairing model, the interaction preserves the total spin
  projection of a two-body state, $S_{z}=s_{z1}+s_{z2}$. The single
  particle states can have spin of +1/2 or -1/2, so there can be three
  two-body channels with $S_{z}=-1,0,+1$. These channels can then be
  indexed with a unique label in a similar way to the single particle
  index scheme. In more complicated systems, there will be many more
  channels involving multiple symmetries, so it is useful to create a
  data structure that stores the relevant two-body quantum numbers to
  keep track of the labeling scheme.

  It is more efficient to use two-dimensional array data
  structures, where the first index refers to the channel number and
  the second refers to the element within that channel. So to access
  matrix elements or $t$ amplitudes, you can loop over the channels
  first, then the indices within that channel. To get an idea of the
  savings using this block diagonal structure, let's look at a case
  with a plane wave basis, with three momentum and one spin quantum
  numbers, with an interaction that conserves linear momentum in all
  three dimensions, as well as the total spin projection. Using 502
  basis states, the TBME's require about 0.23 Gb of memory in block
  diagonal form, which is an enormous saving from the 500 Gb mentioned
  earlier in the na\"ive storage scheme.


  Since the calculation of all  zeros can now be avoided,
  improvements in speed and memory will now follow. To get a handle on
  how these CCD calculations are implemented we need only to look at the
  most expensive sum in equation \ref{eq:ccd2}. This corresponds to
  the sum over $klcd$. Since this sum is repeated for all $i < j$ and
  $a < b$, it means that these equations will scale as
  $\mathcal{O}(n_{p}^{4} n_{h}^{4})$. However,
  they can be rewritten using intermediates as

  \begin{align}
  0 = \braket{ab|\hat{v}|ij} + \hat{P}(ab) \sum_{c} \braket{b| \chi
    |c} \braket{ac| t |ij} - \hat{P}(ij) \sum_{k} \braket{k| \chi |j}
  \braket{ab| t |ik} & \nonumber \\ + \frac{1}{2}\sum_{cd} \braket{ab|
    \chi |cd} \braket{cd| t |ij} + \frac{1}{2} \sum_{kl} \braket{ab| t
    |kl} \braket{kl| \chi |ij} \\ + \hat{P}(ij)\hat{P}(ab) \sum_{kc}
  \braket{ac| t |ik}\braket{kb| \chi |cj} & \nonumber
  \end{align}
  for all $i,j,a,b$, the reason why these indices are now unrestricted
  will be explained later. The intermediates $\chi$ are defined as
  \begin{equation}
  \braket{b| \chi |c} = \braket{b|f|c} - \frac{1}{2} \sum_{kld}
  \braket{bd|t|kl} \braket{kl|v|cd}
  \end{equation}
  \begin{equation}
  \braket{k| \chi |j} = \braket{k|f|j} + \frac{1}{2} \sum_{cdl}
  \braket{kl|v|cd} \braket{cd|t|jl}
  \end{equation}
  \begin{equation}
  \braket{kl| \chi |ij} = \braket{kl|v|ij} + \frac{1}{2} \sum_{cd}
  \braket{kl|v|cd} \braket{cd|t|ij}
  \label{eq:mtxEx}
  \end{equation}
  \begin{equation}
  \braket{kb| \chi |cj} = \braket{kb|v|cj} + \frac{1}{2} \sum_{dl}
  \braket{kl|v|cd} \braket{db|t|lj}
  \end{equation}
  \begin{equation}
  \braket{ab| \chi |cd} = \braket{ab|v|cd}
  \end{equation}


  With the introduction of the above intermediates, the CCD equations scale now as $\mathcal{O}(n_{h}^{2}
  n_{p}^{4})$, which is an important improvement. This is of
  course at the cost of computing the intermediates at the beginning
  of each iteration, where the most expensive one, $\braket{kb| \chi |cj}$ scales as $\mathcal{O}(n_{h}^{3} n_{p}^{3})$. To
  further speed up these computations, we see that these sums can be
  written in terms of  matrix-matrix multiplications. It is not obvious how to
  write all of these sums in such a way, but it is useful to first
  recall that the expression for the multiplication of two matrices $\hat{C} =
  \hat{A}\times \hat{B}$ can be written as
  \begin{equation}
  C_{ij} = \sum_{k} A_{ik} \times B_{kj}.
  \end{equation}
  We obeserve then  that equation (\ref{eq:mtxEx}) can be written as
  \[
  \braket{K| \chi |I} = \braket{K|v|I} + \frac{1}{2} \sum_{C}
  \braket{K|v|C} \braket{C|t|I}
  \]
  by mapping the two index pairs $kl \to K, ij \to I, cd \to C$. The sum looks now 
like a matrix-matrix multiplication. This is
  useful because there are packages like BLAS (Basic Linear Algebra
  Subprograms) \cite{blas} which have extremely fast implementations of
  matrix-matrix multiplication.
  The simplest example to consider is the expression for the correlation energy from MBPT2. We rewrite 
  \begin{equation}\label{eq:bruteforceMBPT}
  \Delta E_{MBPT2} = \frac{1}{4}\sum_{abij} \frac{\braket{ij|\hat{v}|ab}\braket{ab|\hat{v}|ij}}{ \epsilon_{ij}^{ab}},
\end{equation}
by defining the matrices $\hat{A}$ and $\hat{B}$ with new indices $I=(ij)$ and $A=(ab)$. The individual matrix elements of these matrices are 
\[
A_{IA} = \langle I \vert \hat{v} \vert A \rangle,
\]
and 
\[
B_{AI} = \frac{\langle A \vert \hat{v} \vert I \rangle}{\epsilon^A_I}.
\]
We can then rewrite the correlation energy from MBPT2 as
\begin{equation}\label{eq:smartMBPT}
  \Delta E_{MBPT2} = \frac{1}{4}\hat{A}\times \hat{B}.
\end{equation}
Figure \ref{fig:speedup1} shows the difference between the brute force summation over single-particle states
of Eq.~(\ref{eq:bruteforceMBPT}) and the smarter matrix-matrix multiplication of Eq.~(\ref{eq:smartMBPT}) for a given number  of basis states.
In these calculations we have only considered pure neutron matter with $N=14$ neutrons and a density $n=0.08$ fm$^{-3}$ and plane wave single-particle states with periodic boundary conditions, allowing for up to 
$1600$ single-particle basis states. The Minnesota interaction model has been used in these calculations.  
\begin{figure}
    \includegraphics[width=\linewidth]{Chapter8-figures/blockvsfull_log.png}
    \caption{MBPT2 contribution to the correlation for pure neutron matter with $N=14$ neutrons and periodic boundary conditions. Up to approximately 1600 single-particle states have been included in the sums over intermediate states in Eqs.~(\ref{eq:bruteforceMBPT}) and (\ref{eq:smartMBPT})}. 
    \label{fig:speedup1}
  \end{figure}
With $40$ single-particle shells, see Table \ref{tab:table1} for example, we have in total  $2713$ single-particle states. 
Using the matrix-matrix algorithm the final calculation time is $2.4$ s (this is the average time from ten numerical experiments).
The total time using the brute force is $100.6$ s (again the average of ten numerical experiments), resulting in a speed up of 42. It is useful to dissect the final time in terms of different operations.
For the matrix-matrix multiplication most of the time is spent setting up the matrix elements for the two-body channels and to load the matrix elements. The final matrix-matrix multiplication takes only $1\%$ of the total time. For the brute force algorithm, the multiplication and summation over the various single-particle states takes almost half of the total time. In our discussion on high-performance computing below, we will discuss in more detail the program used to obtain these results.



With the definition of the intermediates and appropriate matrix-matrix multiplications and  a working CCD program, we can move on to more
  realistic cases. One such case is infinite nuclear matter using a
  plane-wave basis. These states are solutions to the free-particle
  Hamiltonian,
  \begin{equation}
  \frac{-\hbar^2}{2m}\nabla^2\mathop{\phi(\vec{x})}=\epsilon\mathop{\phi(\vec{x})}.
  \end{equation}
  For a finite basis, we approximate the problem by constructing a box
  with sides of length $L$, which quantizes the momentum, and impose
  periodic boundary conditions in each direction.
  \begin{align}
  \mathop{\phi(x_{i})}=\mathop{\phi(x_{i}+L)}
  \\ \mathop{\phi_{\vec{k}}(\vec{x})}=\frac{1}{\sqrt{L^{3}}}e^{i\vec{k}\cdot\vec{x}},\hspace{0.5cm}\vec{k}=\frac{2\pi\vec{n}}{L},\hspace{0.5cm}n_{i}
  \end{align}

  The first step in calculating infinite matter is to construct a
  model space by finding every single-particle state relevant to a
  given problem. In our case, this amounts to looping over the quantum
  numbers for spin, isospin, and the three momentum directions. To
  control the model space size, the momentum can be truncated to give
  a cubic space, where $n_{i}\leq n_{\text{max}}$, or a spherical
  space, where $n_{x}^{2}+n_{y}^{2}+n_{z}^{2}\leq N_{\text{max}}$. The
  number of single-particle states in a cubic space increases rapidly
  with $n_{\text{max}}$ compared to the spherical case with
  $N_{\text{max}}$. For example, in pure neutron matter a cubic space
  with $n_{\text{max}}=3$ has $668$ states while the spherical space
  with $N_{\text{max}}=17$ has $610$ states. Therefore, the spherical
  case will be used for the rest of the calculations here. The loop
  increases in energy by counting the number of shells, so states can
  be 'filled' by labeling the first $P$ proton and $N$ neutron states
  as holes. The following loop is for pure neutron matter and requires
  the number of neutrons, $N$ and density, $\rho=N/L^{3}$, as
  input. Symmetric nuclear matter requires an extra loop over isospin.
\begin{svgraybox}
  \begin{algorithmic}
    \State $n=0$ \For{$\text{shell}\in\{ 0,...,N_{\text{max}}\}$}
    \For{$-\sqrt{N_{\text{max}}}\leq n_{x}\leq\sqrt{N_{\text{max}}}$}
    \For{$-\sqrt{N_{\text{max}}}\leq n_{y}\leq\sqrt{N_{\text{max}}}$}
    \For{$-\sqrt{N_{\text{max}}}\leq n_{z}\leq\sqrt{N_{\text{max}}}$}
    \For{$s_{z}\in\{-\frac{1}{2},\frac{1}{2}\}$}
    \If{$n_{x}^{2}+n_{y}^{2}+n_{z}^{2}=\text{shell}$} \State
    $\text{Energy}=\frac{4\pi^{2}\hbar^{2}}{2m}\times\text{shell}$
    \If{$n<N$} \State $\text{type}=\text{``hole''}$ \Else \State
    $\text{type}=\text{``particle''}$ \EndIf \State STATES $\gets$
    ($n$, $n_{x}$, $n_{y}$, $n_{z}$, $s_{z}$, Energy, type) \State
    $n\gets n+1$ \EndIf \EndFor \EndFor \EndFor \EndFor \EndFor
  \end{algorithmic}
\end{svgraybox}
  The next step is to build every two-body state in the model space
  and separate them by their particle/hole character and combined
  quantum numbers. While each single-particle state is unique,
  two-body states can share quantum numbers with other members of a
  particular two-body channel. These channels allow us to remove
  matrix elements and cluster amplitudes that violate the symmetries
  of the interaction. This reduces greatly the size and speed of the
  calculation. Our structures will depend on direct two-body channels,
  $T$, where the quantum numbers are added, and cross two-body
  channels, $X$, where the quantum numbers are subtracted. Before
  filling the channels, it is helpful to order them with an index
  function which returns a unique index for a given set of two-body
  quantum numbers. Without an index function, one has to loop over all
  the channels for each two-body state, adding a substantial amount
  of time to this algorithm. An example of an index function for the
  direct channels in symmetric nuclear matter is, for
  $N_{x}=n_{x,1}+n_{x,2}$, $N_{y}=n_{y,1}+n_{y,2}$,
  $N_{z}=n_{z,1}+n_{z,2}$, $S_{z}=s_{z,1}+s_{z,2}$,
  $T_{z}=t_{z,1}+t_{z,2}$, $m=2\lfloor\sqrt{N_{\text{max}}}\rfloor$,
  and $M=2m+1$,
  \begin{equation}
  \text{Ind}\left( N_{x},N_{y},N_{z},S_{z},T_{z}\right)=2\left(
  N_{x}+m\right)M^{3}+2\left( N_{y}+m\right)M^{2}+2\left(
  N_{z}+m\right)M+2\left( S_{z}+1\right)+\left(T_{z}+1\right).
  \end{equation}
  This function, which can also be used for the cross-channel index
  function, is well suited for a cubic model space but can be applied
  in either case. An additional restriction for two-body states is
  that they must be composed of two different states to satisfy the
  Pauli-exclusion principle.
\begin{svgraybox}
  \begin{algorithmic}
    \For{$\text{sp1}\in STATES$} \For{$\text{sp2}\in STATES$}
    \If{$sp1\neq sp2$} \State $N_{i}\gets n_{i,1}+n_{i,2}$ \State
    $S_{z}\gets s_{z,1}+s_{z,2}$ \State $T_{z}\gets t_{z,1}+t_{z,2}$
    \State
    $\text{i\_dir}\gets\text{Ind}\left(N_{x},N_{y},N_{z},S_{z},T_{z}\right)$
    \State $T\gets$ (sp1, sp2, i\_dir) \State $N'_{i}\gets
    n_{i,1}-n_{i,2}$ \State $S'_{z}\gets s_{z,1}-s_{z,2}$ \State
    $T'_{z}\gets t_{z,1}-t_{z,2}$ \State
    $\text{i\_cross}\gets\text{Ind}\left(N'_{x},N'_{y},N'_{z},S'_{z},T'_{z}\right)$
    \State $X\gets$ (sp1, sp2, i\_cross) \EndIf \EndFor \EndFor
  \end{algorithmic}
\end{svgraybox}
  From the cross channels, one can construct the cross channel
  complements, $X'$, where $X\left( pq\right)\equiv X'\left(
  qp\right)$. Also from the direct channels, one can construct
  one-body, and corresponding three-body, channels for each
  single-particle state, $K$ by finding every combination of two
  two-body states within a direct channel that contains that single
  particle state, $T\left( pq\right)=T\left( rs\right)\Rightarrow
  K_{p}\gets\left( qrs\right)$.
\begin{svgraybox}
  \begin{algorithmic}
    \For{$\text{Chan}\in T$} \For{$\text{tb1}\in\text{Chan}$}
    \For{$\text{tb2}\in\text{Chan}$} \State $K\gets\text{tb1}_{1}$
    \State
    $K_{\text{tb1}_{1}}\gets\mathop{\text{tb1}_{2},\text{tb2}_{1},\text{tb2}_{2}}$
    \EndFor \EndFor \EndFor
  \end{algorithmic}
\end{svgraybox}
  These different structures can be further categorized by a two-body
  state's particle-hole character, $\braket{pp| t |hh}, \braket{hh| v
    |hh}, \braket{pp| v |pp}, \braket{hh| v |pp}$, and $\braket{hp| v
    |hp}$, which greatly simplifies the matrix-matrix multiplications
  of the CCD iterations by indexing the summed variables in a
  systematic way. Summations are constructed by placing two
  structures next to each other in such a way that the inner summed
  indices are of the same channel. The resulting structure is indexed
  by the outer channels as shown here
  \begin{align}
  \braket{b| \chi |c} = \braket{b|f|c} - \frac{1}{2} \sum_{kld}
  \braket{bd|t|kl} \braket{kl|v|cd} &\rightarrow f_{c}^{b}\left(
  K\left( b\right),K\left( c\right)\right)\\ \nonumber & -
  \frac{1}{2}t_{kl}^{bd}\left( K\left( b\right),K_{b}\left(
  kld\right)\right) v_{cd}^{kl}\left( K_{c}\left(
  kld\right),K\left( c\right)\right) \\ \braket{k| \chi |j} =
  \braket{k|f|j} + \frac{1}{2} \sum_{cdl} \braket{kl|v|cd}
  \braket{cd|t|jl} &\rightarrow f_{j}^{k}\left( K\left(
  k\right),K\left( j\right)\right) \\ \nonumber &+ \frac{1}{2}v_{cd}^{kl}\left(
  K\left( k\right),K_{k}\left( cdl\right)\right)
  t_{jl}^{cd}\left( K_{j}\left( cdl\right),K\left( j\right)\right)
  \\ \braket{kl| \chi |ij} = \braket{kl|v|ij} + \frac{1}{2} \sum_{cd}
  \braket{kl|v|cd} \braket{cd|t|ij} &\rightarrow v_{ij}^{kl}\left(
  T\left( kl\right),T\left( ij\right)\right)\\ \nonumber  &+ 
  \frac{1}{2}v_{cd}^{kl}\left( T\left( kl\right),T\left(
  cd\right)\right) t_{ij}^{cd}\left( T\left( cd\right),T\left(
  ij\right)\right) \\ \braket{kb| \chi |cj} = \braket{kb|v|cj} +
  \frac{1}{2} \sum_{dl} \braket{kl|v|cd} \braket{db|t|lj} &\rightarrow
  v_{cj}^{kb}\left( X\left( kc\right),X\left( jb\right)\right)\\ \nonumber  &+
  \frac{1}{2}v_{cd}^{kl}\left( X\left( kc\right),X\left(
  dl\right)\right) t_{lj}^{db}\left( X\left( dl\right),X\left(
  jb\right)\right) \\ \braket{ab| \chi |cd} = \braket{ab|v|cd}
  &\rightarrow v_{cd}^{ab}\left( T\left( ab\right),T\left(
  cd\right)\right) \\ \sum_{c} \braket{b| \chi |c} \braket{ac| t |ij}
  &\rightarrow \chi_{c}^{b}\left( K\left( b\right),K\left(
  c\right)\right)\cdot t_{ij}^{ac}\left( K\left( c\right), K_{c}\left(
  ija\right)\right) \\ \sum_{k} \braket{k| \chi |j} \braket{ab| t |ik}
  &\rightarrow \chi_{j}^{k}\left( K\left( j\right),K\left(
  k\right)\right)\cdot t_{ik}^{ab}\left( K\left( c\right), K_{c}\left(
  ija\right)\right) \\ \sum_{cd} \braket{ab| \chi |cd} \braket{cd| t
    |ij} &\rightarrow \chi_{cd}^{ab}\left( T\left( ab\right),T\left(
  cd\right)\right)\cdot t_{ij}^{cd}\left( T\left( cd\right),T\left(
  ij\right)\right) \\ \sum_{kl} \braket{ab| t |kl} \braket{kl| \chi
    |ij} &\rightarrow t_{kl}^{ab}\left( T\left( ab\right),T\left(
  kl\right)\right)\cdot \chi_{ij}^{kl}\left( T\left( kl\right),T\left(
  ij\right)\right) \\ \sum_{kc} \braket{ac| t |ik}\braket{kb| \chi
    |cj} = \sum_{kc} \braket{ai^{-1}| t |kc^{-1}}\braket{kc^{-1}| \chi
    |jb^{-1}} &\rightarrow t_{ik}^{ac}\left( X\left( ia\right),X\left(
  kc\right)\right)\cdot \chi_{cj}^{kb}\left( X\left( kc\right),X\left(
  jb\right)\right)
  \end{align}
Figure \ref{fig:fig1} displays the convergence of the energy per particle for pure neutron matter as function of number particles 
for  the CCD approximation with the Minnesota potential for different with $\mathrm{N_{max}=20}$.
  \begin{figure}
    \includegraphics[width=\linewidth]{Chapter8-figures/fig1.pdf}
    \caption{Energy per particle of pure neutron matter computed in
      the CCD approximation with the Minnesota potential for different
      numbers of particles with $\mathrm{N_{max}=20}$.}
    \label{fig:fig1}
  \end{figure}
Similarly, Fig.~{fig:fig2} shows the convergence in terms of different model space sizes $\mathrm{N_{max}=20}$ with 
a fixed number of neutrons $N=114$. 
  \begin{figure}
    \includegraphics[width=\linewidth]{Chapter8-figures/fig2.pdf}
    \caption{Energy per particle of pure neutron matter computed in
      the CCD approximation with the Minnesota potential for different
      model space sizes with $\mathrm{A=114}$.}
    \label{fig:fig2}
  \end{figure}
We see from the last figure that at the CCD level and neutron matter only there is a good convergence with  $\mathrm{N_{max}=25}$.
This results depends however on the type of interaction and many-body approximation. 

In these calculations we
approximated our problem with periodic boundary conditions,
  $\mathop{\phi(x_{i})}=\mathop{\phi(x_{i}+L)}$, but we could have
  chosen anti-periodic boundary conditions,
  $\mathop{\phi(x_{i})}=-\mathop{\phi(x_{i}+L)}$. The difference
  between these two shows how the correlation energy contains
  finite-size effects \cite{finitesize}. One solution to this problem is by integrating
  over solutions between periodic and anti-periodic conditions, known
  as twist-averaging \cite{twistaverage}. First, we multiply the single-particle states by
  a phase for each direction, characterized by a twist-angle,
  $\theta_{i}$.
  \begin{equation}
    \mathop{\phi_{\vec{k}}(\vec{x}+\vec{L})}\rightarrow\mathop{e^{i\vec{\theta}}\phi_{\vec{k}}(\vec{x})}
  \end{equation}
  $\theta_{i}=0$ for PBC and $\theta_{i}=\pi$ for APBC
  \begin{align}
  \vec{k}\rightarrow\vec{k}+\frac{\vec{\theta}}{L}
  \\ \epsilon_{\vec{k}}\rightarrow\epsilon_{\vec{k}}+\frac{\pi}{L}\vec{k}\cdot\vec{\theta}+\frac{\pi^{2}}{L^{2}}
  \end{align}
  Adding these phases changes the single-particle energies, the
  correction of which disappear as $L\rightarrow\infty$, depending on
  $\vec{\theta}$ and thus changes the shell structure so that hole
  states can jump up to particle states and vis a versa. So it's
  necessary to fill hole states separately for each
  $\vec{\theta}$. Integration over some quantitiy is approximated by a
  weighted sum, such as Gauss-Legendre quadrature, over the quantity
  for each set of twist angles. The algorithm becomes then
\begin{svgraybox}
  \begin{algorithmic}
    \State Build mesh points and weights for each direction $i$:
    $\{\theta_{i},w_{i}\}$ \State $E_{\text{twist}}=0$
    \For{$\mathop{(\theta_{x},w_{x})}\in\mathop{\{\theta_{x},w_{x}\}}$}
    \For{$\mathop{(\theta_{y},w_{y})}\in\mathop{\{\theta_{y},w_{y}\}}$}
    \For{$\mathop{(\theta_{z},w_{z})}\in\mathop{\{\theta_{z},w_{z}\}}$}
    \State Build Basis States with $k_{i}\rightarrow
    k_{i}+\frac{\theta_{i}}{L}$ \State Order States by Energy and Fill
    Holes \State Get Result $E$ (T,HF,CCD) \State
    $E_{\text{twist}}=E_{\text{twist}}+\frac{1}{\pi^{3}}w_{x}w_{y}w_{z}E$
    \EndFor \EndFor \EndFor
  \end{algorithmic}
\end{svgraybox}
  This technique gives results which depend much less on the particle
  number, but requires a full calculation for each set of twist
  angles, which can grow very quickly. For example, using 10 twist
  angles in each direction requires 1000 calculations. To see the
  effects of twist averaging, it's easy to calculate the kinetic
  energy per particle and the Hartree-Fock energy per particle, which
  avoids the full CCD calculation. These calculations can be compared
  to the exact values for infinite matter, which are calculated by
  integrating the the relevent values up to the fermi surface
  \begin{align}
    \text{T}_{\text{inf}}=\frac{3\hbar^{2}k_{f}^{2}}{10m}
    \\ \text{HF}_{\text{inf}}=\frac{1}{\mathop{(2\pi)^{6}}}\frac{L^{3}}{2\rho}\int_{0}^{k_{f}}d\vec{k}_{1}\int_{0}^{k_{f}}d\vec{k}_{2}\braket{\vec{k}_{1}\vec{k}_{2}|\hat{v}|\vec{k}_{1}\vec{k}_{2}}
  \end{align}
Figure \ref{fig:fig3} shows the CCD kinetic energy of pure neutron
      matter computed with the Minnesota potential as a function of
      the number of particles for both periodic boundary conditions (PBC)
      and twist-averaged boundary conditions (TABC5). We see clearly that the 
twist-averaged boundary conditions soften the dependence on finite size effects. 
  \begin{figure}
    \includegraphics[width=\linewidth]{Chapter8-figures/fig3.pdf}
    \caption{Finite-size effects in the kinetic energy of pure neutron
      matter computed with the Minnesota potential as a function of
      the number of particles for both periodic boundary conditions (PBC)
      and twist-averaged boundary conditions (TABC5).}
    \label{fig:fig3}
  \end{figure}
Similarly, Fig.~\ref{fig:fig4} displays the corresponding Hartree-Fock energy (the reference energy) 
obtained with Minnesota interaction using both periodic and twist-average boundary conditions.  
  \begin{figure}
    \includegraphics[width=\linewidth]{Chapter8-figures/fig4.pdf}
    \caption{Finite-size effects in the Hartree-Fock energy of pure
      neutron matter computed with the Minnesota potential as a
      function of the number of particles for both periodic boundary (PBC)
      conditions and twist-averaged boundary conditions (TABC5).}
    \label{fig:fig4}
  \end{figure}
The results show again a weaker dependence on finite size effects.

The codes we have developed for obtaining these results, with
benchmarks and selected computations, are discussed in more detail in
the appendix.  The discussions hitherto have focused on the
development of an efficient and flexible many-body code. As discussed
in the appendix, the codes have been structured in a very flexible
way, allowing the user to study and add different physical systems,
spanning from the simple pairing model to quantum dots and infinite
nuclear matter. Structuring the codes in for a example an abstract
system class and a solver class allows an eventual user to study
different physical systems and add new many-body solvers without
having to write a totally new program. The structure of the program
discussed in the appendix of this chapter adhere to this philosophy, and as demonstrated in chapter 11, 
with few additions one can add 
the widely popular similarity renormalization group method. We discuss this in more detail in chapter 11. 

Till now we have limited our discussion to the construction of a
many-body code, including elements like how to structure a code in
terms of functions, how to modularize the code, how to develop classes
(see the appendix) and how to verify and validate our computations
(our checks provided by the simple pairing model and many-body
perturbation theory to second order) against known results.  With the
rewriting of our equations in terms of efficient matrix and vector
operations we have also shown how to make our code more efficient.
Matrix and vector operations can easily be parallelized. Such
algorithmic improvements are necessary in order to be able to study
complicated physical systems.  Our codes can also be easily parallelized in order to run on shared memory architectures using either OpenMP \cite{openmp} and/or MPI/OpenMPI \cite{mpi,openmpi}.
However, we have till now not touched
upon topics like parallelization and an efficient utilization of exisiting and future
high-performance facility and possible improvements and bottlenecks which our codes can face. We end thus this section with a brief discussion
of high-performance computing topics of relevance for wave function based methods. We will focus on many-body perturbation theory to second order and the two algorithms discussed in Eqs.~(\ref{eq:bruteforceMBPT}) and (\ref{eq:smartMBPT}). 


\subsection{High-performance computing and analysis of the MBPT program}
The aim of this subsection is to discuss in more detail how we can make the computations discussed in connection with equations 
Eqs.~(\ref{eq:bruteforceMBPT}) and (\ref{eq:smartMBPT}) more efficient using physical constraints, algorithm improvements, and parallel processing. For pedagocical reasons, we will use the MBPT parts of the program due to their simplicity while still containing the important elements of a larger, more complicated calculation. 
The codes can be found at the github link 
\url{https://github.com/ManyBodyPhysics/LectureNotesPhysics/tree/master/doc/src/Chapter8-programs/cpp/MBPT/src}.
We start by discussing the basic structure of these codes by looking at the MBPT2 functions in MBPTfunctions.cpp.
All of the functions with MBPT2 in their name calculate Eq.~(\ref{eq:bruteforceMBPT}), but each iteration utilizes more techniques to minimize its runtime.

The first function is the most simple way to calculate the sum over $ijab$ with four nested loops. When all hole states are indexed below all particle states, the loops over $ab$ can be restricted above the number of holes while the loops over $ij$ can be restricted under the number of hole states. This function scales as $h^2p^2$, where $h$ is the number of hole states and $p$ is the number of particle states.

\lstset{language=c++}
\begin{lstlisting}
  void MBPT2_0(const Input_Parameters &Parameters, const Model_Space &Space)
  {
    double energy0;
    double energy = 0.0;
    for(int i = 0; i < Space.indhol; ++i){
      for(int j = 0; j < Space.indhol; ++j){
        if(i == j){ continue; }
        for(int a = Space.indhol; a < Space.indtot; ++a){
	  for(int b = Space.indhol; b < Space.indtot; ++b){
	    if(a == b){ continue; }
	    energy0 = V_Minnesota(Space, i, j, a, b);
	    energy0 *= energy0;
	    energy0 /= (Space.qnums[i].energy + Space.qnums[j].energy - Space.qnums[a].energy - Space.qnums[b].energy);
	    energy += energy0;
	  }
        }
      }
    }
    energy *= 0.25;
  }
\end{lstlisting}

The second function improves upon the first by splitting the top loop over $a$ into different processing elements so the inner loops can be computed simultaneously. At this point it's important to think about which variables will only be used on a certain thread and which will be shared. In this case, each thread has a private variable named ``energy0'' and share the ``energy'' variable, which is subject to the specified reduction operation at the end of each parallel section. Naively, this should improve the calculation time by a factor equal to the number of processing threads, which in the case of these calculations is four. However, there is some overhead processing that occurs to split and recombine the threads which slows the overall time. In addition, because each team of threads can only proceed when the slowest member finishes, there is overhead in unbalanced threads and in optimizing their order. Fortunately, the nested loops over $bij$ are all equivalent, so each thread is balanced and declaring ``schedule(static)'' tells the thread scheduler not to spend time optimizing them.

\begin{lstlisting}
  void MBPT2_1(const Input_Parameters &Parameters, const Model_Space &Space)
  {
    double energy = 0.0;
    double energy0;
    #pragma omp parallel private(energy0)
    {
      #pragma omp for schedule(static) reduction(+:energy)
      for(int i = 0; i < Space.indhol; ++i){
        for(int j = 0; j < Space.indhol; ++j){
	  if(i == j){ continue; }
	  for(int a = Space.indhol; a < Space.indtot; ++a){
	    for(int b = Space.indhol; b < Space.indtot; ++b){
	      if(a == b){ continue; }
	      energy0 = V_Minnesota(Space, i, j, a, b);
	      energy0 *= energy0;
	      energy0 /= (Space.qnums[i].energy + Space.qnums[j].energy - Space.qnums[a].energy - Space.qnums[b].energy);
	      energy += energy0;
	    }
	  }
        }
      }
    }
    energy *= 0.25;
  }
\end{lstlisting}

The rest of the functions begin by exploiting the symmetries of the interaction to split the single-particle states into channels. This essentially removes all of the zeros that break the symmetry. The function that sets up these channels uses an index function that maps the summed quantum numbers of two states onto a unique channel index. Then, because the MBPT term uses only pairs of holes and pairs of particles, the function constructs lists of these indicies, ``hhvec'' and ``ppvec''. Now, instead of four nested loops over all holes and particles, the sum is obtained by looping over hhvec and ppvec inside each channel.

\begin{lstlisting}
  void MBPT2_2(const Input_Parameters &Parameters, const Model_Space &Space, const Channels &Chan)
  {
    double energy = 0.0;
    double energy0;
    int nhh, npp, i, j, a, b;
    for(int chan = 0; chan < Chan.size; ++chan){
      nhh = Chan.nhh[chan];
      npp = Chan.npp[chan];
      if(nhh*npp == 0){ continue; }
      
      for(int hh = 0; hh < nhh; ++hh){
        i = Chan.hhvec[chan][2*hh];
        j = Chan.hhvec[chan][2*hh + 1];
        for(int pp = 0; pp < npp; ++pp){
	  a = Chan.ppvec[chan][2*pp];
	  b = Chan.ppvec[chan][2*pp + 1];
	  energy0 = V_Minnesota(Space, i, j, a, b);
	  energy0 *= energy0;
	  energy0 /= (Space.qnums[i].energy + Space.qnums[j].energy - Space.qnums[a].energy - Space.qnums[b].energy);
	  energy += energy0;
        }
      }
    }
    energy *= 0.25;
  }
\end{lstlisting}

The next function is exactly like the last except that it is parallelized like the improvement from the first to the second function. However, there are some obvious differences. Because the sizes of hhvec and ppvec for each channel can be wildly different, splitting the loop over channels into a team of threads could be massively unbalanced. An easy way to balance the threads to use a static scheduler is to parallelize the loops over ppvec inside each channel because each nested loop over ''hhvec'' is equivalent.

\begin{lstlisting}
  void MBPT2_3(const Input_Parameters &Parameters, const Model_Space &Space, const Channels &Chan)
  {
    double energy = 0.0;
    double energy0;
    int nhh, npp, i, j, a, b;
    for(int chan = 0; chan < Chan.size; ++chan){
      nhh = Chan.nhh[chan];
      npp = Chan.npp[chan];
      if(nhh*npp == 0){ continue; }
      
      #pragma omp parallel private(i, j, a, b, energy0)
      {
        #pragma omp for schedule(static) reduction(+:energy)
        for(int hh = 0; hh < nhh; ++hh){
	  i = Chan.hhvec[chan][2*hh];
	  j = Chan.hhvec[chan][2*hh + 1];
	  for(int pp = 0; pp < npp; ++pp){
	    a = Chan.ppvec[chan][2*pp];
	    b = Chan.ppvec[chan][2*pp + 1];
	    energy0 = V_Minnesota(Space, i, j, a, b);
	    energy0 *= energy0;
	    energy0 /= (Space.qnums[i].energy + Space.qnums[j].energy - Space.qnums[a].energy - Space.qnums[b].energy);
	    energy += energy0;
	  }
        }
      }
    }
    energy *= 0.25;
  }
\end{lstlisting}

The last MBPT2 function uses the ``hhvec'' and ``ppvec'' lists from the channel function to make matrices of the relevant interaction matrix elements and energy denominators. Then it uses a matrix-matrix multiplication function from \cite{blas} to sum the elements together like equation Eq.~(\ref{eq:mtxmtx}). The benefit to this strategy is that this function is optimized to compute the matrix-matrix multiplication faster than an equivalent loop structure. The detriment is the memory required to hold the different matrix elements.

\begin{lstlisting}
  void MBPT2_4(const Input_Parameters &Parameters, const Model_Space &Space, const Channels &Chan)
  {
    double energy = 0.0;
    double *V1, *V2, *S1;
    char N = 'N';
    double fac0 = 0.0;
    double fac1 = 1.0;
    int nhh, npp, i, j, a, b, idx;
    double energy0;
    for(int chan = 0; chan < Chan.size; ++chan){
      nhh = Chan.nhh[chan];
      npp = Chan.npp[chan];
      if(nhh*npp == 0){ continue; }
      
      V1 = new double[nhh * npp];
      V2 = new double[npp * nhh];
      S1 = new double[nhh * nhh];
      #pragma omp parallel shared(V1, V2) private(i, j, a, b, idx, energy0)
      {
        #pragma omp for schedule(static)
        for(int hh = 0; hh < nhh; ++hh){
	  i = Chan.hhvec[chan][2*hh];
	  j = Chan.hhvec[chan][2*hh + 1];
	  for(int pp = 0; pp < npp; ++pp){
	    a = Chan.ppvec[chan][2*pp];
	    b = Chan.ppvec[chan][2*pp + 1];
	    idx = hh * npp + pp;
	    energy0 = V_Minnesota(Space, i, j, a, b);
	    V1[idx] = energy0 / (Space.qnums[i].energy + Space.qnums[j].energy - Space.qnums[a].energy - Space.qnums[b].energy);
	    idx = pp * nhh + hh;
	    V2[idx] = energy0;
	  }
        }
      } 
      RM_dgemm(V1, V2, S1, &nhh, &nhh, &npp, &fac1, &fac0, &N, &N);
      delete V1; delete V2;
      
      for(int hh = 0; hh < nhh; ++hh){ energy += S1[nhh*hh + hh]; }
      delete S1;
    }
  }
\end{lstlisting}

\begin{figure}
  \includegraphics[width=\linewidth]{Chapter8-figures/MBPT2fig.pdf}
  \caption{Computation times of the MBPT2 energy correction for different algorithms and the time to construct symmetry channels. These calculations use the Minnesota potential with a density of $\mathrm{\rho=0.08\ fm^{-3}}$ and $\mathrm{14}$ neutrons.}
  \label{fig:fig5}
\end{figure}

Figure~(\ref{fig:fig5}) shows the computation times from these MBPT2 functions and the channel setup function for up to 28 major shells with 14 neutrons at a density 0f $0.08 fm^{-3}$. There is a clear improvement from the first to the second function and the third to the fourth, except in very small cases where the overhead from parallelization is overwhelming. It's also clear that there is an order of magnitude by first computing the symmetry channels. However, there are two aspects of these improvements that aren't clear. First, the time it takes to setup the channels begins to overcome the MBPT2 sums for large cases due to its higher scaling. Second, the matrix-matrix multiplication in the fifth function doesn't improve on the previous function because the same loops are needed to fill the matrices in the first place.

To more clearly see the power of matrix-matrix multiplication and why the scaling of the channel function becomes insignificant, we move to the particle-particle term in the MBPT3 energy term which is shown in diagram 4 of fig. 8.2 and in eqn. 8.65. This sum should naively scale as $h^2p^4$ which for larger model spaces dramatically increase the computation time. The first four functions for MBPT3 mirror very closely those for MBPT2 with additional loop(s) for the additional two particle indices.

\begin{lstlisting}
  void MBPT3_0(const Input_Parameters &Parameters, const Model_Space &Space)
  {
    double energy0, energy1;
    double energy = 0.0;
    for(int i = 0; i < Space.indhol; ++i){
      for(int j = 0; j < Space.indhol; ++j){
        if(i == j){ continue; }
        for(int a = Space.indhol; a < Space.indtot; ++a){
	  for(int b = Space.indhol; b < Space.indtot; ++b){
	    if(a == b){ continue; }
	    energy0 = V_Minnesota(Space, i, j, a, b);
	    energy0 /= (Space.qnums[i].energy + Space.qnums[j].energy - Space.qnums[a].energy - Space.qnums[b].energy);
	    for(int c = Space.indhol; c < Space.indtot; ++c){
	      for(int d = Space.indhol; d < Space.indtot; ++d){
	        if(c == d){ continue; }
	        energy1 = V_Minnesota(Space, a, b, c, d);
	        energy1 *= V_Minnesota(Space, c, d, i, j);
	        energy1 /= (Space.qnums[i].energy + Space.qnums[j].energy - Space.qnums[c].energy - Space.qnums[d].energy);
	        energy += energy0 * energy1;
	      }
	    }
	  }
        }
      }
    }
    energy *= 0.125;
  }
\end{lstlisting}

\begin{lstlisting}
  void MBPT3_1(const Input_Parameters &Parameters, const Model_Space &Space)
  {
    double energy = 0.0;
    double energy0, energy1;
    #pragma omp parallel private(energy0, energy1)
    {
      #pragma omp for schedule(static) reduction(+:energy)
      for(int i = 0; i < Space.indhol; ++i){
        for(int j = 0; j < Space.indhol; ++j){
	  if(i == j){ continue; }
	  for(int a = Space.indhol; a < Space.indtot; ++a){
	    for(int b = Space.indhol; b < Space.indtot; ++b){
	      if(a == b){ continue; }
	      energy0 = V_Minnesota(Space, i, j, a, b);
	      energy0 /= (Space.qnums[i].energy + Space.qnums[j].energy - Space.qnums[a].energy - Space.qnums[b].energy);
	      for(int c = Space.indhol; c < Space.indtot; ++c){
	        for(int d = Space.indhol; d < Space.indtot; ++d){
		  if(c == d){ continue; }
		  energy1 = V_Minnesota(Space, a, b, c, d);
		  energy1 *= V_Minnesota(Space, c, d, i, j);
		  energy1 /= (Space.qnums[i].energy + Space.qnums[j].energy - Space.qnums[c].energy - Space.qnums[d].energy);
		  energy += energy0 * energy1;
	        }
	      }
	    }
	  }
        }
      }
    }
    energy *= 0.125;
  }
\end{lstlisting}

\begin{lstlisting}
  void MBPT3_2(const Input_Parameters &Parameters, const Model_Space &Space, const Channels &Chan)
  {
    double energy = 0.0;
    double energy0, energy1;
    int nhh, npp, i, j, a, b, c, d;
    for(int chan = 0; chan < Chan.size; ++chan){
      nhh = Chan.nhh[chan];
      npp = Chan.npp[chan];
      if(nhh*npp == 0){ continue; }
      
      for(int hh = 0; hh < nhh; ++hh){
        i = Chan.hhvec[chan][2*hh];
        j = Chan.hhvec[chan][2*hh + 1];
        for(int pp0 = 0; pp0 < npp; ++pp0){
	  a = Chan.ppvec[chan][2*pp0];
	  b = Chan.ppvec[chan][2*pp0 + 1];
	  energy0 = V_Minnesota(Space, i, j, a, b);
	  energy0 /= (Space.qnums[i].energy + Space.qnums[j].energy - Space.qnums[a].energy - Space.qnums[b].energy);
	  for(int pp1 = 0; pp1 < npp; ++pp1){
	    c = Chan.ppvec[chan][2*pp1];
	    d = Chan.ppvec[chan][2*pp1 + 1];
	    energy1 = V_Minnesota(Space, a, b, c, d);
	    energy1 *= V_Minnesota(Space, c, d, i, j);
	    energy1 /= (Space.qnums[i].energy + Space.qnums[j].energy - Space.qnums[c].energy - Space.qnums[d].energy);
	    energy += energy0 * energy1;
	  }
        }
      }
    }
    energy *= 0.125;
  }
\end{lstlisting}

\begin{lstlisting}
  void MBPT3_3(const Input_Parameters &Parameters, const Model_Space &Space, const Channels &Chan)
  {
    double energy = 0.0;
    double energy0, energy1;
    int nhh, npp, i, j, a, b, c, d;
    for(int chan = 0; chan < Chan.size; ++chan){
      nhh = Chan.nhh[chan];
      npp = Chan.npp[chan];
      if(nhh*npp == 0){ continue; }
      
      #pragma omp parallel private(i, j, a, b, c, d, energy0, energy1)
      {
        #pragma omp for schedule(static) reduction(+:energy)
        for(int hh = 0; hh < nhh; ++hh){
	  i = Chan.hhvec[chan][2*hh];
	  j = Chan.hhvec[chan][2*hh + 1];
	  for(int pp0 = 0; pp0 < npp; ++pp0){
	    a = Chan.ppvec[chan][2*pp0];
	    b = Chan.ppvec[chan][2*pp0 + 1];
	    energy0 = V_Minnesota(Space, i, j, a, b);
	    energy0 /= (Space.qnums[i].energy + Space.qnums[j].energy - Space.qnums[a].energy - Space.qnums[b].energy);
	    for(int pp1 = 0; pp1 < npp; ++pp1){
	      c = Chan.ppvec[chan][2*pp1];
	      d = Chan.ppvec[chan][2*pp1 + 1];
	      energy1 = V_Minnesota(Space, a, b, c, d);
	      energy1 *= V_Minnesota(Space, c, d, i, j);
	      energy1 /= (Space.qnums[i].energy + Space.qnums[j].energy - Space.qnums[c].energy - Space.qnums[d].energy);
	      energy += energy0 * energy1;
	    }
	  }
        }
      }
    }
    energy *= 0.125;
  }
\end{lstlisting}

The last function is slightly different from its MBPT2 counterpart where loops over $ijab$ were needed to fill the matrices. In this function, the matrices are filled by looping over $ijab$ and $abcd$, respectively, instead of $ijabcd$. Then, the matrix-matrix multiplications take care of the sum over $ab$ between the two structures into an intermediate matrix. This essentially allows the function to scale as the dominant loop, $p^4$, naively improving the computation time by a factor of $h^2$.

\begin{lstlisting}
  void MBPT3_4(const Input_Parameters &Parameters, const Model_Space &Space, const Channels &Chan)
  {
    double energy = 0.0;
    double *V1, *S1, *V2, *S2;
    char N = 'N';
    char T = 'T';
    double fac0 = 0.0;
    double fac1 = 1.0;
    int nhh, npp, i, j, a, b, c, d, idx;
    double energy0;
    for(int chan = 0; chan < Chan.size; ++chan){
      nhh = Chan.nhh[chan];
      npp = Chan.npp[chan];
      if(nhh*npp == 0){ continue; }
      
      V1 = new double[nhh * npp];
      S1 = new double[nhh * npp];
      V2 = new double[npp * npp];
      S2 = new double[nhh * nhh];
      #pragma omp parallel shared(V1, V2) private(i, j, a, b, c, d, idx, energy0)
      {
        #pragma omp for schedule(static)
        for(int pp0 = 0; pp0 < npp; ++pp0){
	  a = Chan.ppvec[chan][2*pp0];
	  b = Chan.ppvec[chan][2*pp0 + 1];
	  for(int hh = 0; hh < nhh; ++hh){
	    i = Chan.hhvec[chan][2*hh];
	    j = Chan.hhvec[chan][2*hh + 1];
	    idx = hh * npp + pp0;
	    energy0 = V_Minnesota(Space, i, j, a, b);
	    energy0 /= (Space.qnums[i].energy + Space.qnums[j].energy - Space.qnums[a].energy - Space.qnums[b].energy);
	    V1[idx] = energy0;
	  }
	  for(int pp1 = 0; pp1 < npp; ++pp1){
	    c = Chan.ppvec[chan][2*pp1];
	    d = Chan.ppvec[chan][2*pp1 + 1];
	    idx = pp0 * npp + pp1;
	    V2[idx] = V_Minnesota(Space, a, b, c, d);
	  }
        }
      }
      
      RM_dgemm(V1, V2, S1, &nhh, &npp, &npp, &fac1, &fac0, &N, &N);
      RMT_dgemm(S1, V1, S2, &nhh, &nhh, &npp, &fac1, &fac0, &N, &T);
      delete V1; delete V2; delete S1;
      for(int hh = 0; hh < nhh; ++hh){
        energy += S2[nhh*hh + hh];
      }
      delete S2;
    }
    energy *= 0.125;
  }
\end{lstlisting}

\begin{figure}
  \includegraphics[width=\linewidth]{Chapter8-figures/MBPT3fig.pdf}
  \caption{Computation times of the MBPT3 particle-particle energy correction for different algorithms and the time to construct symmetry channels. These calculations use the Minnesota potential with a density of $\mathrm{\rho=0.08\ fm^{-3}}$ and $\mathrm{14}$ neutrons. The dotted extensions for functions MBPT3_0 and MBPT3_1 are extrapolations.}
  \label{fig:fig6}
\end{figure}

Figure\ref{fig:fig6} shows the run times from the MBPT3 functions and the channel setup function up to 28 major shells with 14 neutrons at a density 0f $0.08 fm^{-3}$. The obvious difference between MBPT3 and MBPT2 is the massive increase in computation time. For example, the largest case with the first function would take approximately eight months to compute. Because of this, the first two functions had to be extrapolated for the larger cases. Again, there is a clear improvement from the first to the second function, from the third to the fourth, and by first computing the symmetry channels. In this case, however, the scaling of the channel function is dwarfed by the $h^2p^4$ scaling of the sum. In addition, the optimized matrix-matrix multiplication improves the computation time by another order of magnitude by using an intermediate matrix. Because all many-body methods include similar sums to MBPT and also utilize interactions that benefit from exploiting its symmetries, all of these strategies can be used to make computation times managable.

\section{Conclusions}
In this chapter we have presented many of the basic ingredients that
enter theoretical studies of infinite nuclear matter, with possible
extensions to the homogeneous electron gas in two and three dimensions
or other quantum mechanical systems.  We have focused on the
construction of a single-particle and many-body basis appropriate for
such systems, as well as introducing post Hartree-Fock many-body
methods like full configuration interaction theory, many-body
perturbation theory and coupled cluster theory. The results here,
albeit being obtained with a simpler model for the nuclear forces, can
easily be extended to more complicated and realistic models for
nuclear interactions and to include other many-body methods. We have
however, for pedagogical reasons, tried to keep the theoretical inputs
to the various many-body methods as simple as possible. The reader
should however, with the inputs from chapters 2-6, be able to have a
better of understanding of nuclear forces and how these can be derived
from the underlying theory for the strong force and effective field
theory.  The last exercise in this chapter replaces the simple Minnesota potential with
realistic interaction from effective field theory. 



The subsequent chapters 9, 10 and 11 show how many of the theoretical
concepts and code elements discussed in this chapter can be used to
add other many-body methods, without having to develop a new numerical
project.  With a proper modularization and flexible classes, we can
add new physical systems as well as new many-body methods.  The codes
which have been developed in this chapter can be reused in the
development and analysis of the in-medium similarity renormalizaton
group approach of chapter 10 or the Green's function approach in chapter 11. Similarly, the theoretical concepts we
have developed in this chapter, such as the definition of a
single-particle basis using plane wave functions and correlations from
many-body perturbation theory or coupled cluster theory, can be used
in chapter 9, 10 and 11 as well. Chapter 9 for example, uses results
from coupled cluster theory in order to provide better ways to constrain
the Jastrow factor, which accounts for correlations beyond a mean-field picture, 
in Monte Carlo calculations.



  \section{Exercises}
  \begin{prob} \label{problem:prob8.1}
  Show that the one-body part of the Hamiltonian
      \begin{equation*}
          \hat{H}_0 = \sum_{pq} \element{p}{\hat{h}_0}{q} a^\dagger_p
          a_q
      \end{equation*}
  can be written, using standard annihilation and creation operators,
  in normal-ordered form as
      \[
          \hat{H}_0 = \sum_{pq} \element{p}{\hat{h}_0}{q}
          \left\{a^\dagger_p a_q\right\} + \sum_i
          \element{i}{\hat{h}_0}{i}
      \]
  Explain the meaning of the various symbols. Which reference vacuum
  has been used?
  \end{prob}


  \begin{prob} \label{problem:prob8.2}
  Show that the two-body part of the Hamiltonian
      \begin{equation*}
          \hat{H}_I = \frac{1}{4} \sum_{pqrs}
          \element{pq}{\hat{v}}{rs} a^\dagger_p a^\dagger_q a_s a_r
      \end{equation*}
  can be written, using standard annihilation and creation operators,
  in normal-ordered form as
      \begin{align*}
      \hat{H}_I &= \frac{1}{4} \sum_{pqrs} \element{pq}{\hat{v}}{rs}
      a^\dagger_p a^\dagger_q a_s a_r \nonumber \\ &= \frac{1}{4}
      \sum_{pqrs} \element{pq}{\hat{v}}{rs} \normord{a^\dagger_p
        a^\dagger_q a_s a_r} + \sum_{pqi} \element{pi}{\hat{v}}{qi}
      \normord{a^\dagger_p a_q} + \frac{1}{2} \sum_{ij}
      \element{ij}{\hat{v}}{ij}
      \end{align*}
  Explain again the meaning of the various symbols.
  \end{prob}



  \begin{prob}\label{problem:prob8.3}
  Derive the normal-ordered form of the threebody part of the
  Hamiltonian.
  \[
      \hat{H}_3 = \frac{1}{36} \sum_{\substack{ pqr \\ stu}}
      \element{pqr}{\hat{v}_3}{stu} a^\dagger_p a^\dagger_q
      a^\dagger_r a_u a_t a_s,
  \]
  and specify the contributions to the two-body, one-body and the
  constant part.
  \end{prob}



  \begin{prob}\label{problem:spbasissetup}
  Develop a program which sets up a single-particle basis for nuclear
  matter calculations with input a given number of nucleons and a user
  specificied density or Fermi momentum. Follow the setup discussed in
  Table \ref{tab:table1}.  You need to define the number of particles
  $A$ and the density of the system using
  \[
  n = g \frac{k_F^3}{6\pi^2}.
  \]
  Here you can either define the density itself or the Fermi momentum
  $k_F$.  With the density/Fermi momentum and a fixed number of
  nucleons $A$, we can define the length $L$ of the box used with our
  periodic boundary contributions via the relation
  \[
    V= L^3= \frac{A}{n}.
  \]
  We can then can use $L$ to define the spacing between various
  $k$-values, that is
  \[
    \Delta k = \frac{2\pi}{L}.
  \]
  \end{prob}


  \begin{prob}\label{problem:fourier}
  The interaction we will use for these calculations is a
  semirealistic nucleon-nucleon potential known as the Minnesota
  potential \cite{minnesota} which has the form, $V_{\alpha}\left(
  r\right)=V_{\alpha}\exp{-(\alpha r^{2})}$. The spin and isospin
  dependence of the Minnesota potential is given by,
  \[
  V\left( r\right)=\frac{1}{2}\left( V_{R}+\frac{1}{2}\left(
  1+P_{12}^{\sigma}\right) V_{T}+\frac{1}{2}\left(
  1-P_{12}^{\sigma}\right) V_{S}\right)\left(
  1-P_{12}^{\sigma}P_{12}^{\tau}\right),
  \]
  where $P_{12}^{\sigma}=\frac{1}{2}\left(
  1+\sigma_{1}\cdot\sigma_{2}\right)$ and
  $P_{12}^{\tau}=\frac{1}{2}\left( 1+\tau_{1}\cdot\tau_{2}\right)$ are
  the spin and isospin exchange operators, respectively.  Show that a
  Fourier transform to momentum space results in
  \[
  \langle \mathbf{k}_p \mathbf{k}_q \vert V_{\alpha}\vert
  \mathbf{k}_r\mathbf{k}_s\rangle=\frac{V_{\alpha}}{L^{3}}\left(\frac{\pi}{\alpha}\right)^{3/2}\exp{\frac{-q^{2}}{4\alpha}}\delta_{\vec{k}_{p}+\vec{k}_{q},\vec{k}_{r}+\vec{k}_{s}}.
  \]
  Write thereafter a function which sets up the full anty-symmetrized
  matrix elements for the Minnesota potential using the parameters
  listed in Table \ref{tab:minnesotatab}.
  \end{prob}



  \begin{prob}\label{problem:unitarity}
  Consider a Slater determinant built up of orthogonal single-particle
  orbitals $\psi_{\lambda}$, with $\lambda = 1,2,\dots,A$.

  The unitary transformation
  \[
  \psi_a = \sum_{\lambda} C_{a\lambda}\phi_{\lambda},
  \]
  brings us into the new basis.  The new basis has quantum numbers
  $a=1,2,\dots,A$.  Show that the new basis is orthogonal.
  \begin{enumerate}
  \item[a)] Show that the new Slater determinant constructed from the
    new single-particle wave functions can be written as the
    determinant based on the previous basis and the determinant of the
    matrix $C$.
  \item[b)] Show that the old and the new Slater determinants are
    equal up to a complex constant with absolute value unity.  Hint:
    $C$ is a unitary matrix.
  \end{enumerate}
  \end{prob}

  \begin{prob}\label{problem:referenceE}
  Use the ansatz for the ground state in second quantization
  \[
  |\Phi_0\rangle = \left(\prod_{i\le
    F}\hat{a}_{i}^{\dagger}\right)|0\rangle,
  \]
  where the index $i$ defines different single-particle states up to
  the Fermi level, to calculate using Wick's theorem (see the
  appendix) the expectation value
  \[
    E[\Phi_0]= E_{\mathrm{Ref}}= \sum_{i\le F}^A \langle i | \hat{h}_0
    | i \rangle + \frac{1}{2}\sum_{ij\le F}^A\langle
    ij|\hat{v}|ij\rangle.
  \]
  Insert thereafter the plane wave function basis for the various
  single-particle states and show that the above energy can be written
  as
  \[
    E[\Phi_0] = \sum_{i\le F}^A \langle k_i | \hat{t} | k_i \rangle +
    \frac{1}{2}\sum_{ij\le F}^A\langle
    k_ik_j|\hat{v}|k_ik_j\rangle,
  \]
  where we use the shorthand $\vert k_i\rangle = \vert
  \mathbf{k}_i,\sigma_i,\tau_{z_i}\rangle$ for the single-particle
  states in three dimensions.

  Replace then the discrete sums with integrals, that is
  \[
  \sum_i \rightarrow
  \sum_{\sigma_i}\sum_{\tau_{z_i}}\frac{L^3}{(2\pi)^3}\int_0^{\mathbf{k}_F}d\mathbf{k},
  \]
  and show that the energy per particle $A$ can be written as (for
  symmetric nuclear matter)
  \[
    \frac{E_{\mathrm{Ref}}}{A}=\frac{3\hbar^2k_F^2}{10M_N}+\frac{1}{2n}\frac{L^3}{(2\pi)^6}\sum_{\sigma_i\sigma_j}\sum_{\tau_{z_i}\tau_{z_j}}\int_0^{\mathbf{k}_F}d\mathbf{k}_i\int_0^{\mathbf{k}_F}d\mathbf{k}_j\langle
    k_ik_j|\hat{v}|k_ik_j\rangle,
  \]
  with the density $n=V/A=L^3/A$.

  Find the following expression for pure neutron matter. Use the
  Minnesota interaction and try to simplify the above six-dimensional
  integral for pure neutron matter (Hint: the interaction depends only
  the momentum transfer squared and fix one of the momentum
  integrations along the $z$-axis. Integrate out the dependence on the
  various angles).

  Finally, write a program which computes the above energy for pure
  nuetron matter using the Minnesota potential.
  \end{prob}

  \begin{prob}\label{problem:hamiltoniansetup}
  We will assume that we can build various Slater determinants using
  an orthogonal single-particle basis $\psi_{\lambda}$, with $\lambda= 1,2,\dots,A$.


  The aim of this exercise is to set up specific matrix elements that
  will turn useful when we start our discussions of different
  many-body methods. In particular you will notice, depending on the
  character of the operator, that many matrix elements will actually
  be zero.

  Consider three $A$-particle Slater determinants $|\Phi_0$,
  $|\Phi_i^a\rangle$ and $|\Phi_{ij}^{ab}\rangle$, where the notation
  means that Slater determinant $|\Phi_i^a\rangle$ differs from
  $|\Phi_0\rangle$ by one single-particle state, that is a
  single-particle state $\psi_i$ is replaced by a single-particle
  state $\psi_a$.  It will later be interpreted as a so-called
  one-particle-one-hole excitation.  Similarly, the Slater determinant
  $|\Phi_{ij}^{ab}\rangle$ differs by two single-particle states from
  $|\Phi_0\rangle$ and is normally thought of as a
  two-particle-two-hole excitation.

  Define a general one-body operator $\hat{F} =
  \sum_{i}^A\hat{f}(x_{i})$ and a general two-body operator
  $\hat{G}=\sum_{i>j}^A\hat{g}(x_{i},x_{j})$ with $g$ being invariant
  under the interchange of the coordinates of particles $i$ and $j$.
  \begin{enumerate}
  \item[a)]
  \[
  \langle \Phi_0 \vert\hat{F}\vert\Phi_0\rangle,
  \]
  and
  \[
  \langle \Phi_0\vert\hat{G}|\Phi_0\rangle.
  \]
  \item[b)] Find thereafter
  \[
  \langle \Phi_0 |\hat{F}|\Phi_i^a\rangle,
  \]
  and
  \[
  \langle \Phi_0|\hat{G}|\Phi_i^a\rangle,
  \]
  \item[c)] Finally, find
  \[
  \langle \Phi_0 |\hat{F}|\Phi_{ij}^{ab}\rangle,
  \]
  and
  \[
  \langle \Phi_0|\hat{G}|\Phi_{ij}^{ab}\rangle.
  \]
  \item[d)] What happens with the two-body operator if we have a
    transition probability of the type
  \[
  \langle \Phi_0|\hat{G}|\Phi_{ijk}^{abc}\rangle,
  \]
  where the Slater determinant to the right of the operator differs by
  more than two single-particle states?
  \item[e)] With an orthogonal basis of Slater determinants
    $\Phi_{\lambda}$, we can now construct an exact many-body state as
    a linear expansion of Slater determinants, that is, a given exact
    state
  \[
  \Psi_i = \sum_{\lambda =0}^{\infty}C_{i\lambda}\Phi_{\lambda}.
  \]
  In all practical calculations the infinity is replaced by a given
  truncation in the sum.

  If you are to compute the expectation value of (at most) a two-body
  Hamiltonian for the above exact state
  \[
  \langle \Psi_i \vert \hat{H} \vert \Psi_i\rangle,
  \]
  based on the calculations above, which are the only elements which
  will contribute?  (there is no need to perform any calculation here,
  use your results from exercises a), b), and c)).

  These results simplify to a large extent shell-model calculations.
  \end{enumerate}
  \end{prob}

  \begin{prob}\label{problem:diagrams}
Write down the analytical expressions for diagrams (8) and (9) in Fig.~\ref{fig:goldstone} and discuss whether these
diagrams should be accounted for or not in the calculation of the energy per particle of infinite matter. 
If a Hartree-Fock basis is used, should these diagrams be included?  
Show also that diagrams (2), (6)-(7) and (10)-(16) are zero in infinite matter due to the lack of momentum conservation. 
\end{prob}

  \begin{prob}\label{problem:pairingmodel}


  We present a simplified Hamiltonian consisting of an unperturbed
  Hamiltonian and a so-called pairing interaction term. It is a model
  which to a large extent mimicks some central features of atomic
  nuclei, certain atoms and systems which exhibit superfluiditity or
  superconductivity.  To study this system, we will use a mix of
  many-body perturbation theory (MBPT), Hartree-Fock (HF) theory and
  full configuration interaction (FCI) theory. The latter will also
  provide us with the exact answer.  When setting up the Hamiltonian
  matrix you will need to solve an eigenvalue problem.

  We define first the Hamiltonian, with a definition of the model
  space and the single-particle basis. Thereafter, we present the
  various exercises (some of them are solved).


  The Hamiltonian acting in the complete Hilbert space (usually
  infinite dimensional) consists of an unperturbed one-body part,
  $\hat{H}_0$, and a perturbation $\hat{V}$.

  We limit ourselves to at most two-body interactions and our
  Hamiltonian is represented by the following operators
  \[
  \hat{H} = \sum_{\alpha\beta}\langle \alpha |h_0|\beta\rangle
  a_{\alpha}^{\dagger}a_{\beta}+\frac{1}{4}\sum_{\alpha\beta\gamma\delta}\langle
  \alpha\beta| V|\gamma\delta\rangle
  a_{\alpha}^{\dagger}a_{\beta}^{\dagger}a_{\delta}a_{\gamma},
  \]
  where $a_{\alpha}^{\dagger}$ and $a_{\alpha}$ etc.~are standard
  fermion creation and annihilation operators, respectively, and
  $\alpha\beta\gamma\delta$ represent all possible single-particle
  quantum numbers.  The full single-particle space is defined by the
  completeness relation
  \[
  \hat{{\bf 1}} = \sum_{\alpha=1}^{\infty}|\alpha \rangle \langle
  \alpha|.
  \]
  In our calculations we will let the single-particle states
  $|\alpha\rangle$ be eigenfunctions of the one-particle operator
  $\hat{h}_0$. Note that the two-body part of the Hamiltonian contains
  anti-symmetrized matrix elements.


  The above Hamiltonian acts in turn on various many-body Slater
  determinants constructed from the single-basis defined by the
  one-body operator $\hat{h}_0$.  As an example, the two-particle
  model space $\mathcal{P}$ is defined by an operator
  \[
  \hat{P} = \sum_{\alpha\beta =1}^{m}|\alpha\beta \rangle \langle
  \alpha\beta|,
  \]
  where we assume that $m=\dim(\mathcal{P})$ and the full space is
  defined by
  \[
  \hat{P}+\hat{Q}=\hat{{\bf 1}},
  \]
  with the projection operator
  \[
  \hat{Q} = \sum_{\alpha\beta =m+1}^{\infty}|\alpha\beta \rangle
  \langle \alpha\beta|,
  \]
  being the complement of $\hat{P}$.


  Our specific model consists of $N$ doubly-degenerate and equally
  spaced single-particle levels labelled by $p=1,2,\dots$ and spin
  $\sigma=\pm 1$.  These states are schematically portrayed in
  Fig.~\ref{fig:schematic}.  The first two single-particle levels
  define a possible model space, indicated by the label $\mathcal{P}$.
  The remaining states span the excluded space $\mathcal{Q}$.

  We write the Hamiltonian as
  \[ \hat{H} = \hat{H}_0 + \hat{V} , \]
  where
  \[
  \hat{H}_0=\xi\sum_{p\sigma}(p-1)a_{p\sigma}^{\dagger}a_{p\sigma}
  \]
  and
  \[
  \hat{V}=-\frac{1}{2}g\sum_{pq}a^{\dagger}_{p+}
  a^{\dagger}_{p-}a_{q-}a_{q+}.
  \]
  Here, $H_0$ is the unperturbed Hamiltonian with a spacing between
  successive single-particle states given by $\xi$, which we will set
  to a constant value $\xi=1$ without loss of generality. The two-body
  operator $\hat{V}$ has one term only. It represents the pairing
  contribution and carries a constant strength $g$.

  The indices $\sigma=\pm$ represent the two possible spin values. The
  interaction can only couple pairs and excites therefore only two
  particles at the time.


  \begin{enumerate}
  \item[a)] Show that the unperturbed Hamiltonian $\hat{H}_0$ and
    $\hat{V}$ commute with both the spin projection $\hat{S}_z$ and
    the total spin $\hat{S}^2$, given by
  \[
    \hat{S}_z := \frac{1}{2}\sum_{p\sigma} \sigma
    a^{\dagger}_{p\sigma}a_{p\sigma}
  \]
  and
  \[
    \hat{S}^2 := \hat{S}_z^2 + \frac{1}{2}(\hat{S}_+\hat{S}_- +
    \hat{S}_-\hat{S}_+),
  \]
  where
  \[
    \hat{S}_\pm := \sum_{p} a^{\dagger}_{p\pm} a_{p\mp}.
  \]
  This is an important feature of our system that allows us to
  block-diagonalize the full Hamiltonian. We will focus on total spin
  $S=0$.  In this case, it is convenient to define the so-called pair
  creation and pair annihilation operators
  \[
  \hat{P}^{+}_p = a^{\dagger}_{p+}a^{\dagger}_{p-},
  \]
  and
  \[
  \hat{P}^{-}_p = a_{p-}a_{p+},
  \]
  respectively.
  \item[b)] Show that you can rewrite the Hamiltonian (with $\xi=1$)
    as
  \[
  \hat{H}=\sum_{p\sigma}(p-1)a_{p\sigma}^{\dagger}a_{p\sigma}
  -\frac{1}{2}g\sum_{pq}\hat{P}^{+}_p\hat{P}^{-}_q.
  \]
  \item[c)] Show also that Hamiltonian commutes with the product of
    the pair creation and annihilation operators.  This model
    corresponds to a system with no broken pairs. This means that the
    Hamiltonian can only link two-particle states in so-called
    spin-reversed states.

  \item[d)] Construct thereafter the Hamiltonian matrix for a system
    with no broken pairs and total spin $S=0$ for the case of the four
    lowest single-particle levels indicated in the
    Fig.~\ref{fig:schematic}. Our system consists of four particles
    only.  Our single-particle space consists of only the four lowest
    levels $p=1,2,3,4$.  You need to set up all possible Slater
    determinants.  Find all eigenvalues by diagonalizing the
    Hamiltonian matrix.  Vary your results for values of $g\in
    [-1,1]$.  We refer to this as the exact calculation. Comment the
    behavior of the ground state as function of $g$.
  \end{enumerate}
  \end{prob}



  \begin{prob}\label{problem:prob8.5}
  \begin{enumerate}
  \item[a)] We will now set up the Hartree-Fock equations by varying
    the coefficients of the single-particle functions. The
    single-particle basis functions are defined as
  \[
  \psi_p = \sum_{\lambda} C_{p\lambda}\psi_{\lambda}.
  \]
  where in our case $p=1,2,3,4$ and $\lambda=1,2,3,4$, that is the
  first four lowest single-particle orbits of
  Fig.~\ref{fig:schematic}.  Set up the Hartree-Fock equations for
  this system by varying the coefficients $C_{p\lambda}$ and solve
  them for values of $g\in [-1,1]$.  Comment your results and compare
  with the exact solution. Discuss also which diagrams in
  Fig.~\ref{fig:diagrams} that can be affected by a Hartree-Fock
  basis. Compute the total binding energy using a Hartree-Fock basis
  and comment your results.

  \item[b)] We will now study the system using non-degenerate
    Rayleigh-Schr\"odinger perturbation theory to third order in the
    interaction.  If we exclude the first order contribution, all
    possible diagrams (so-called anti-symmetric Goldstone diagrams)
    are shown in Fig.~\ref{fig:diagrams}.


  Based on the form of the interaction, which diagrams contribute to
  the binding energy of the ground state?  Write down the expressions
  for the diagrams that contribute and find the contribution to the
  ground state energy as function $g\in [-1,1]$. Comment your results.
  Compare these results with those you obtained from the exact
  diagonalization with and without the $4p-4h$ state.  Discuss your
  results for a canonical Hartree-Fock basis and a non-canonical
  Hartree-Fock basis.


  Diagram 1 in Fig.~\ref{fig:diagrams} represents a second-order
  contribution to the energy and a so-called $2p-2h$ contribution to
  the intermediate states. Write down the expression for the wave
  operator in this case and compare the possible contributions with
  the configuration interaction calculations without the $4p-4h$
  Slater determinant. Comment your results for various values of $g\in
  [-1,1]$.

  We limit now the discussion to the canonical Hartree-Fock case
  only. To third order in perturbation theory we can produce diagrams
  with $1p-1h$, $2p-2h$ and $3p-3h$ intermediate excitations as shown in


  Define first linked and unlinked diagrams and explain briefly
  Goldstone's linked diagram theorem.  Based on the linked diagram
  theorem and the form of the pairing Hamiltonian, which diagrams will
  contribute to third order?

  Calculate the energy to third order with a canonical Hartree-Fock
  basis for $g\in [-1,1]$ and compare with the full diagonalization
  case in exercise b). Discuss the results.
  \end{enumerate}



  \end{prob}


  \begin{prob}\label{problem:prob8.6}
  This project serves as a continuation of the pairing model with the
  aim being to solve the same problem but now by developing a program
  that implements the coupled cluster method with double excitations
  only. In doing so you will find it convenient to write classes which
  define the single-particle basis and the Hamiltonian. Your functions
  that solve the coupled cluster equations will then just need to set
  up variables which point to interaction elements and single-particle
  states with their pertinent quantum numbers. Use for example the
  setup discussed in the FCI lectures for the pairing model.

  \begin{enumerate}

  \item[a)] Explain why no single excitations are involved in this
    model.


  \item[b)] Set up the coupled cluster equations for doubles
    excitations and convince yourself about their meaning and
    correctness.

  \item[c)] Write a class which holds single-particle data like
    specific quantum numbers, single-particle Hamiltonian etc. Write
    also a class which sets up and stores two-body matrix elements
    defined by the single-particle states.  Write thereafter
    functions/classes which implement the coupled cluster method with
    doubles only.


  \item[d)] Compare your results with those from the exact
    diagonalization with and without the $4p4h$ excitation. Compare
    also your results to perturbation theory at different orders, in
    particular to second order. Discuss your results.  If other
    students are solving the same problem using Green's function
    theory, you can also compare your results with those obtained from
    Green's function theory. The aim is to finalize this part during
    the first week. The codes you will develop can be used as a
    starting point for the second part of the project.
  \end{enumerate}
  \end{prob}

\begin{prob} \label{problem:amplitudes}
Derive the amplitude equations of Eq.~(\ref{eq:ccd}) starting with 
 \[
          0 = \langle\Phi_{i_1 \ldots i_n}^{a_1 \ldots a_n}\vert
          \overline{H}\vert \Phi_0\rangle.
 \]
\end{prob}
  \begin{prob}\label{problem:realisticforces}
Replace the Minnesota interaction model with realistic models for nuclear forces based on effective field theory. In particular 
replace the Minnesota interaction with the low-order (LO) contribution which includes a contact term and a one-pion exchange
term only. The expressions are discussed in section  \ref{subsec:forcemodels} and Eq.~(\ref{eq:eq_1PEci}). Reference \cite{ekstromPRX} contains a
detailed compilation of all terms up to order NNLO, with tabulated values for all constants. 
When adding realistic interaction models we recommend that you use the many-body perturbation theory codes to second order in the interaction, see the code link at \url{https://github.com/ManyBodyPhysics/LectureNotesPhysics/blob/master/doc/src/Chapter8-programs/cpp/MBPT2/src/}. 
\end{prob}

  \section{Solutions to selected exercises}
  \begin{sol}{problem:prob8.1}
  To solve this problem, we start by introducing the shorthand label
  for single-particle states below the Fermi level $F$ as $i,j,\ldots
  \leq F$. For single-particle states above the Fermi level we reserve
  the labels $a,b,\ldots > F$, while the labels $p,q, \ldots$
  represent any possible single particle state.  Using the ansatz for
  the ground state $\vert \Phi_0$ as new reference vacuum state, the
  anticommutation relations are
  \[
  \left\{a_p^\dagger, a_q \right\}= \delta_{pq}, p, q \leq F,
  \]
  and
  \[
  \left\{a_p, a_q^\dagger \right\} = \delta_{pq}, \hspace{0.1cm} p, q
  > F.
  \]
  It is easy to see then that
  \[
          a_i|\Phi_0\rangle = |\Phi_i\rangle\ne 0, \hspace{0.5cm}
          a_a^\dagger|\Phi_0\rangle = |\Phi^a\rangle\ne 0,
  \]
  and
  \[
  a_i^\dagger|\Phi_0\rangle = 0 \hspace{0.5cm} a_a|\Phi_0\rangle = 0.
  \]
  We can then rewrite the one-body Hamiltonian as
   \begin{align*}
          \hat{H}_0 &= \sum_{pq} \element{p}{\hat{h}_0}{q} a^\dagger_p
          a_q \\ &= \sum_{pq} \element{p}{\hat{h}_0}{q}
          \left\{a^\dagger_p a_q\right\} + \delta_{pq\in i} \sum_{pq}
          \element{p}{\hat{h}_0}{q} \\ &= \sum_{pq}
          \element{p}{\hat{h}_0}{q} \left\{a^\dagger_p a_q\right\} +
          \sum_i \element{i}{\hat{h}_0}{i},
   \end{align*}
  where the curly brackets represent normal-ordering with respect to
  the new vacuum state. Withe respect to the new vacuum reference
  state, the
  \end{sol}

  \begin{sol}{problem:prob8.2}
  Using our anti-commutation rules, Wick's theorem discussed in the
  appendix and the definition of the creation and annihilation
  operators from the previous problem, we can rewrite the set of
  creation and annihilation operators of
      \begin{equation*}
          \hat{H}_I = \frac{1}{4} \sum_{pqrs}
          \element{pq}{\hat{v}}{rs} a^\dagger_p a^\dagger_q a_s a_r
      \end{equation*}
  as
  \begin{align*}
      a^\dagger_p a^\dagger_q a_s a_r &=\normord{a^\dagger_p
        a^\dagger_q a_s a_r} \\ & + \normord{
        \contraction{a^\dagger_p}{a}{{}^\dagger_q}{a}a^\dagger_p
        a^\dagger_q a_s a_r}+
      \normord{\contraction{a^\dagger_p}{a}{{}^\dagger_q a_s}{a}
        a^\dagger_p a^\dagger_q a_s a_r}+
      \normord{\contraction{}{a}{{}^\dagger_p
          a^\dagger_q}{a}a^\dagger_p a^\dagger_q a_s a_r} \\ & +
      \normord{\contraction{}{a}{{}^\dagger_p a^\dagger_q
          a_s}{a}a^\dagger_p a^\dagger_q a_s a_r}+
      \normord{\contraction[1.5ex]{}{a}{{}^\dagger_p a_q^\dagger
          a_s}{a}
        \contraction{a^\dagger_p}{a}{{}^\dagger_q}{a}a^\dagger_p
        a^\dagger_q a_s a_r}+ \normord{\contraction{}{a}{{}^\dagger_p
          a_q^\dagger}{a}\contraction[1.5ex]{a^\dagger_p}{a}{{}^\dagger_q
          a_s}{a}a^\dagger_p a^\dagger_q a_s a_r} \\ &=
      \normord{a^\dagger_p a^\dagger_q a_s a_r}+ \delta_{qs\in i}
      \normord{ a^\dagger_p a_r}- \delta_{qr \in i}
      \normord{a^\dagger_p a_s} - \delta_{ps \in i}
      \normord{a^\dagger_q a_r}\\ & + \delta_{pr \in i}
      \normord{a^\dagger_q a_s} + \delta_{pr \in i} \delta_{qs \in i}
      - \delta_{ps \in i} \delta_{qr \in i}.
  \end{align*}
   Inserting the redefinition of the creation and annihilation
   operators with respect to the new vacuum state, we have
      \begin{align*}
      \hat{H}_I &= \frac{1}{4} \sum_{pqrs} \element{pq}{\hat{v}}{rs}
      a^\dagger_p a^\dagger_q a_s a_r \\ &= \frac{1}{4} \sum_{pqrs}
      \element{pq}{\hat{v}}{rs} \normord{a^\dagger_p a^\dagger_q a_s
        a_r} + \frac{1}{4} \sum_{pqrs} \Bigl( \delta_{qs\in i}
      \element{pq}{\hat{v}}{rs} \normord{ a^\dagger_p a_r}\\ & -
      \delta_{qr \in i} \element{pq}{\hat{v}}{rs} \normord{a^\dagger_p
        a_s} - \delta_{ps \in i}\element{pq}{\hat{v}}{rs}
      \normord{a^\dagger_q a_r}\\ & + \delta_{pr \in i}
      \element{pq}{\hat{v}}{rs} \normord{a^\dagger_q a_s}+ \delta_{pr
        \in i} \delta_{qs \in i}- \delta_{ps \in i} \delta_{qr \in i}
      \Bigr) \\ &= \frac{1}{4} \sum_{pqrs} \element{pq}{\hat{v}}{rs}
      \normord{a^\dagger_p a^\dagger_q a_s a_r}\\ & + \frac{1}{4}
      \sum_{pqi} \Bigl(\element{pi}{\hat{v}}{qi} -
      \element{pi}{\hat{v}}{iq} - \element{ip}{\hat{v}}{qi} +
      \element{ip}{\hat{v}}{iq}\Bigr) \normord{a^\dagger_p a_q}\\ & +
      \frac{1}{4} \sum_{ij} \Bigl(\element{ij}{\hat{v}}{ij}-
      \element{ij}{\hat{v}}{ji}\Bigr)\\ &= \frac{1}{4} \sum_{pqrs}
      \element{pq}{\hat{v}}{rs} \normord{a^\dagger_p a^\dagger_q a_s
        a_r} + \sum_{pqi} \element{pi}{\hat{v}}{qi}
      \normord{a^\dagger_p a_q} + \frac{1}{2} \sum_{ij}
      \element{ij}{\hat{v}}{ij}.
      \end{align*}
  Summing up, we obtain a two-body part defined as
      \begin{equation*}
              \hat{V}_N = \frac{1}{4} \sum_{pqrs}
              \element{pq}{\hat{v}}{rs} \normord{a^\dagger_p
                a^\dagger_q a_s a_r},
      \end{equation*}
  a one-body part given by
      \begin{equation*}
              \hat{F}_N = \sum_{pqi} \element{pi}{\hat{v}}{qi}
              \normord{a^\dagger_p a_q},
      \end{equation*}
      and finally the so-called reference energy
      \begin{equation*}
                  E_{\mathrm{ref}}= \frac{1}{2} \sum_{ij}
                  \element{ij}{\hat{v}}{ij}.
      \end{equation*}
  which is the energy expectation value for the reference state.
  Thus, our normal-ordered Hamiltonian with at most a two-body
  nucleon-nucleon interaction is defined as
  \[
  \hat{H}_N =\frac{1}{4} \sum_{pqrs} \bra{pq}\hat{v}\ket{rs}
  \normord{a^\dagger_p a^\dagger_q a_s a_r} + \sum_{pq} f_q^p
  \normord{a^\dagger_p a_q}= \hat{V}_N + \hat{F}_N,
  \]
  with
  \[
  \hat{F}_N = \sum_{pq} f_q^p \normord{a^\dagger_p a_q},
  \]
  and
  \[
  \hat{V}_N = \frac{1}{4} \sum_{pqrs} \bra{pq}\hat{v}\ket{rs}
  \normord{a^\dagger_p a^\dagger_q a_s a_r},
  \]
  where
  \[
    f_q^p = \element{p}{\hat{h}_0}{q} + \sum_i
    \element{pi}{\hat{v}}{qi}
  \]
  \end{sol}

  \begin{sol}{problem:spbasissetup}
  The following python code sets up the quantum numbers for both
  infinite nuclear matter and neutron matter employing a cutoff in
  the value of $n$. The full code can be found at  \url{https://github.com/ManyBodyPhysics/LectureNotesPhysics/blob/master/doc/src/Chapter8-programs/python/spstatescc.py}.
  \begin{lstlisting}
from numpy import *

nmax =2
nshell = 3*nmax*nmax
count = 1
tzmin = 1

print "Symmetric nuclear matter:"  
print "a, nx,   ny,   nz,   sz,   tz,   nx^2 + ny^2 + nz^2"
for n in range(nshell): 
    for nx in range(-nmax,nmax+1):
         for ny in range(-nmax,nmax+1):
            for nz in range(-nmax, nmax+1):  
                for sz in range(-1,1+1):
                    tz = 1
                    for tz in range(-tzmin,tzmin+1):
                        e = nx*nx + ny*ny + nz*nz
                        if e == n:
                            if sz != 0: 
                                if tz != 0: 
                                    print count, "  ",nx,"  ",ny, "  ",nz,"  ",sz,"  ",tz,"         ",e
                                    count += 1
                                    
                                    
nmax =1
nshell = 3*nmax*nmax
count = 1
tzmin = 1
print "------------------------------------"
print "Neutron matter:"                                    
print "a, nx,   ny,   nz,   sz,    nx^2 + ny^2 + nz^2"
for n in range(nshell): 
    for nx in range(-nmax,nmax+1):
         for ny in range(-nmax,nmax+1):
            for nz in range(-nmax, nmax+1):  
                for sz in range(-1,1+1):
                    e = nx*nx + ny*ny + nz*nz
                    if e == n:
                        if sz != 0: 
                            print count, "  ",nx,"  ",ny, "  ",sz,"  ",tz,"         ",e
                            count += 1     
  \end{lstlisting}                               
  \end{sol}






  \section*{Appendix, Wick's theorem}
  \addcontentsline{toc}{section}{Appendix} Wick's theorem is based on
  two fundamental concepts, namely $\textit{normal ordering}$ and
  $\textit{contraction}$. The normal-ordered form of
  $\hat{A}\hat{B}..\hat{X}\hat{Y}$, where the individual terms are
  either a creation or annihilation operator, is defined as
  \begin{align}
  \label{def: Normal ordering}
  \kpr{\hat{A}\hat{B}..\hat{X}\hat{Y}} \equiv
  (-1)^p\fpr{\text{creation operators}}\cdot\fpr{\text{annihilation
      operators}}.
  \end{align}
  The $p$ subscript denotes the number of permutations that is needed
  to transform the original string into the normal-ordered form. A
  contraction between to arbitrary operators $\hat{X}$ and $\hat{Y}$
  is defined as
  \begin{align}
  \contraction[0.5ex]{}{\hat{X}}{}{\hat{Y}}{} \hat{X}\hat{Y} \equiv
  \for{0}{\hat{X}\hat{Y}}{0}.
  \end{align}
  It is also possible to contract operators inside a normal ordered
  products. We define the original relative position between two
  operators in a normal ordered product as $p$, the so-called
  permutation number. This is the number of permutations needed to
  bring one of the two operators next to the other one. A contraction
  between two operators with $p \neq 0$ inside a normal ordered is
  defined as
  \begin{align}
  \kpr{\contraction[0.5ex]{}{\hat{A}}{\hat{B}..}{\hat{X}}\hat{A}\hat{B}..\hat{X}\hat{Y}}
  = (-1)^p
  \kpr{\contraction[0.5ex]{}{\hat{A}}{}{\hat{B}}\hat{A}\hat{B}..\hat{X}\hat{Y}}.
  \end{align}
  In the general case with $m$ contractions, the procedure is similar,
  and the prefactor changes to
  \begin{align}
  (-1)^{p_1 + p_2 + .. + p_m}.
  \end{align} 

  Wick's theorem states that every string of creation and annihilation
  operators can be written as a sum of normalordered products with all
  possible ways of contractions,
  \begin{align}
  \label{def: Wick's theorem}
  \hat{A}\hat{B}\hat{C}\hat{D}..\hat{R}\hat{X}\hat{Y}\hat{Z} &=
  \kpr{\hat{A}\hat{B}\hat{C}\hat{D}..\hat{R}\hat{X}\hat{Y}\hat{Z}}\\ &+
  \sum_{[1]} \kpr{ \contraction[0.5ex]{}{\hat{A}}{}{\hat{B}}
    \hat{A}\hat{B}\hat{C}\hat{D}..\hat{R}\hat{X}\hat{Y}\hat{Z}}\\ &+
  \sum_{[2]}
  \kpr{\contraction[0.5ex]{}{\hat{A}}{\hat{B}}{\hat{C}}\contraction[1.0ex]{\hat{A}}{\hat{B}}{\hat{C}}{\hat{D}}\hat{A}\hat{B}\hat{C}\hat{D}..\hat{R}\hat{X}\hat{Y}\hat{Z}}\\ &+
  ...\\ &+
  \sum_{[\frac{N}{2}]}\kpr{\contraction[0.5ex]{}{\hat{A}}{\hat{B}}{\hat{C}}\contraction[1.0ex]{\hat{A}}{\hat{B}}{\hat{C}}{\hat{D}}
    \hat{A}\hat{B}\hat{C}\hat{D}..\contraction[0.5ex]{}{\hat{R}}{\hat{X}}{\hat{Y}}\contraction[1.0ex]{\hat{R}}{\hat{X}}{\hat{Y}}{\hat{Z}}\ \hat{R}\hat{X}\hat{Y}\hat{Z}}.
  \end{align}

  The $\sum_{[m]}$ means the sum over all terms with $m$ contractions,
  while $\fpr{\frac{N}{2}}$ means the largest integer that not do not
  exceeds $\frac{N}{2}$ where $N$ is the number of creation and
  annihilation operators. When $N$ is even,
  \begin{align}
  \label{exp: Wick condition}
  \fpr{\frac{N}{2}} = \frac{N}{2},
  \end{align}
  and the last sum in Eq. (\ref{def: Wick's theorem}) is over fully
  contracted terms. When $N$ is odd,
  \begin{align}
  \fpr{\frac{N}{2}} \neq \frac{N}{2},
  \end{align}
  and none of the terms in Eq. (\ref{def: Wick's theorem}) are fully
  contracted.

  An important extension of Wick's theorem allow us to define
  contractions between normal-ordered strings of operators. This is
  the so-called generalized Wick's theorem,
  \begin{align}
  \label{def: Generalized Wick's theorem}
  \kpr{\hat{A}\hat{B}\hat{C}\hat{D}..}\kpr{\hat{R}\hat{X}\hat{Y}\hat{Z}..}
  &=
  \kpr{\hat{A}\hat{B}\hat{C}\hat{D}..\hat{R}\hat{X}\hat{Y}\hat{Z}}\\ &+
  \sum_{[1]} \kpr{
    \contraction[0.5ex]{}{\hat{A}}{\hat{B}\hat{C}\hat{D}..}{\hat{R}}
    \hat{A}\hat{B}\hat{C}\hat{D}..\hat{R}\hat{X}\hat{Y}\hat{Z}}\\ &+
  \sum_{[2]}
  \kpr{\contraction[0.5ex]{}{\hat{A}}{\hat{B}\hat{C}\hat{D}..}{\hat{R}}\contraction[1.0ex]{\hat{A}}{\hat{B}}{\hat{C}\hat{D}..\hat{R}}{\hat{X}}\hat{A}\hat{B}\hat{C}\hat{D}..\hat{R}\hat{X}\hat{Y}\hat{Z}}\\ &+
  ...
  \end{align}

  Turning back to the many-body problem, the vacuum expectation value
  of products of creation and annihilation operators can be written,
  according to Wick's theoren in Eq. (\ref{def: Wick's theorem}), as a
  sum over normal ordered products with all possible numbers and
  combinations of contractions,
  \begin{align}
  \for{0}{\hat{A}\hat{B}\hat{C}\hat{D}..\hat{R}\hat{X}\hat{Y}\hat{Z}}{0}
  &=
  \for{0}{\kpr{\hat{A}\hat{B}\hat{C}\hat{D}..\hat{R}\hat{X}\hat{Y}\hat{Z}}}{0}\\ &+
  \sum_{[1]} \for{0}{\kpr{\contraction[0.5ex]{}{\hat{A}}{}{\hat{B}}
      \hat{A}\hat{B}\hat{C}\hat{D}..\hat{R}\hat{X}\hat{Y}\hat{Z}}}{0}\\ &+
  \sum_{[2]}\for{0}{\kpr{\contraction[0.5ex]{}{\hat{A}}{\hat{B}}{\hat{C}}\contraction[1.0ex]{\hat{A}}{\hat{B}}{\hat{C}}{\hat{D}}\hat{A}\hat{B}\hat{C}\hat{D}..\hat{R}\hat{X}\hat{Y}\hat{Z}}}{0}\\ &+
  ... \\ &+ \sum_{[\frac{N}{2}]}
  \for{0}{\kpr{\contraction[0.5ex]{}{\hat{A}}{\hat{B}}{\hat{C}}\contraction[1.0ex]{\hat{A}}{\hat{B}}{\hat{C}}{\hat{D}}
      \hat{A}\hat{B}\hat{C}\hat{D}..\contraction[0.5ex]{}{\hat{R}}{\hat{X}}{\hat{Y}}\contraction[1.0ex]{\hat{R}}{\hat{X}}{\hat{Y}}{\hat{Z}}\ \hat{R}\hat{X}\hat{Y}\hat{Z}}}{0}.
  \end{align}

  All vacuum expectation values of normal ordered products without
  fully contracted terms are zero. Hence, the only contributions to
  the expectation value are those terms that $\textit{is}$ fully
  contracted,
  \begin{align}
  \for{0}{\hat{A}\hat{B}\hat{C}\hat{D}..\hat{R}\hat{X}\hat{Y}\hat{Z}}{0}
  &= \sum_{[all]}
  \for{0}{\kpr{\contraction[0.5ex]{}{\hat{A}}{\hat{B}}{\hat{C}}\contraction[1.0ex]{\hat{A}}{\hat{B}}{\hat{C}}{\hat{D}}
      \hat{A}\hat{B}\hat{C}\hat{D}..\contraction[0.5ex]{}{\hat{R}}{\hat{X}}{\hat{Y}}\contraction[1.0ex]{\hat{R}}{\hat{X}}{\hat{Y}}{\hat{Z}}\ \hat{R}\hat{X}\hat{Y}\hat{Z}}}{0}\\ &=
  \sum_{[all]}
  \contraction[0.5ex]{}{\hat{A}}{\hat{B}}{\hat{C}}\contraction[1.0ex]{\hat{A}}{\hat{B}}{\hat{C}}{\hat{D}}
  \hat{A}\hat{B}\hat{C}\hat{D}..\contraction[0.5ex]{}{\hat{R}}{\hat{X}}{\hat{Y}}\contraction[1.0ex]{\hat{R}}{\hat{X}}{\hat{Y}}{\hat{Z}}\ \hat{R}\hat{X}\hat{Y}\hat{Z}.
  \end{align}

  To obtain fully contracted terms, Eq. (\ref{exp: Wick condition})
  must hold. When the number of creation and annihilation operators is
  odd, the vacuum expectation value can be set to zero at once. When
  the number is even, the expectation value is simply the sum of terms
  with all possible combinations of fully contracted terms. Observing
  that the only contractions that give nonzero contributions are
  \begin{align}
  \contraction{}{a_{\alpha}}{}{a^{\dagger}_{\beta}}
  a_{\alpha}a^{\dagger}_{\beta} = \delta_{\alpha \beta},
  \end{align}
  the terms that contribute are reduced even more.

  Wick's theorem provides us with an algebraic method for easy
  determination of the terms that contribute to the matrix element.

  \section*{Code structure with benchmark calculations}
In section \ref{sec:chap8numproject} we emphasized several elements we
feel are important during the various development stages of a
numerical project. Here we would like to reiterate some of these. In
particular we would like to stress modularization of code and
refactoring into general classes which can easily be extended upon in
order to accomodate new physical systems and/or many-methods.

The codes of this chapter, listed at the link
\url{https://github.com/ManyBodyPhysics/LectureNotesPhysics/blob/master/doc/src/Chapter8-programs/cpp/CCD/src},
abide to this general philosophy via two major general classes, one
for the physical systems under study and one for the specific
many-body methods. The latter allows us, in a fairly straightforward
way to add new many-body methods like the in-medium similarity
renormalization group approach of chapter 10 or to extend our nuclear
matter calculations to systems like the homogenous electron gas in two
or three dimensions.

To be more explicit, the general class  {\em abstractSPbasis} allows easily for extensions the homogenous electron gas or systems like quantum dots.  The abstract class is defined as (for the complete code see the above link)
\begin{lstlisting}
#ifndef ABSTRACTSPBASIS_HPP
#define ABSTRACTSPBASIS_HPP

class abstractSPbasis{
public:
  // double g;
  // double density;
  // double r_s;
  // int tzMax;
  // int nMax;
  // int shellMax;
  int Nspstates;
  int Nparticles;
  int Nchannels;
  int ** indexMap;
  double * spEnergy;
  channelBundle * chanValue;
  channelBundle * chanModValue;

  abstractSPbasis() {}
  // abstractSPbasis(double densityIn, int tzMaxIn, int shellMaxIn, int NparticlesIn);
  virtual void generateIndexMap() = 0;
  virtual void generateBasis() = 0;
  virtual int checkSympqrs(int p, int q, int r, int s) = 0;
  virtual int checkModSympqrs(int p, int q, int r, int s) = 0;
  virtual int checkChanSym(int p, int q, int ichan) = 0;  
  virtual int checkChanModSym(int p, int q, int ichan) = 0;  
  virtual void setUpTwoStateChannels() = 0;
  virtual void printBasis() = 0;
  virtual void deallocate() = 0;  

  virtual double calc_TBME(int p, int q, int r, int s) = 0;
};
#endif /*ABSTRACTSPBASIS_HPP*/
\end{lstlisting} 
This class defines general quantum numbers and functions that can in turn be inhereted by other systems.
The inheritance diagram for this specific class is shown in Fig.~\ref{fig:absinherit}.
\begin{figure}
\caption{Inheritance diagram for the abstract class.} \label{fig:absinherit}
\begin{center}
\includegraphics{Chapter8-figures/classabstractspbasis.pdf}
\end{center}
\end{figure}
As an example, the class that is used in the infinite matter calculations is shown here.
\begin{lstlisting}
#ifndef INFMATTER_HPP
#define INFMATTER_HPP

#include "abstractSPbasis.hpp"

// struct channelBundle{
//   int chanNx;
//   int chanNy;
//   int chanNz;
//   int chanSz;
//   int chanTz; 
// };

class infMatterSPBasis: public abstractSPbasis{
public:
   double density;
   int tzMax;
   int nMax;
   int shellMax;
  // int Nspstates;
  // int Nparticles;
  // int Nchannels;
  // int ** indexMap;
  // double * spEnergy;
  // channelBundle * chanValue;
  // channelBundle * chanModValue;

  infMatterSPBasis(double densityIn, int tzMaxIn, int shellMaxIn, int NparticlesIn);// : abstractSPbasis(densityIn, tzMaxIn, shellMaxIn, NparticlesIn) {}
  void generateIndexMap();
  void generateBasis();
  int checkSympqrs(int p, int q, int r, int s);
  int checkModSympqrs(int p, int q, int r, int s);
  int checkChanSym(int p, int q, int ichan);
  int checkChanSym2(int p, int q, int chanNx, int chanNy, int chanNz, int chanSz, int chanTz);
int checkChanModSym2(int p, int q, int chanNx, int chanNy, int chanNz, int chanSz, int chanTz);
  int checkChanModSym(int p, int q, int ichan);  
  void setUpTwoStateChannels();
  void printBasis();
  void deallocate();

  double calc_TBME(int p, int q, int r, int s);
  int spinExchangeMtxEle(int i, int j, int k, int l);
  int kron_del(int i, int j);

};

#endif /* INFMATTER_HPP */
\end{lstlisting}

  \begin{acknowledgement}
  We are much indebted to Thomas Papenbrock for many discussions on
  many-body theory.  Computational resources were provided by
  Michigan State University and the Research Council of Norway via the
  Notur project (Supercomputing grant NN2977K).  This work was
  supported by NSF Grant No.~PHY-1404159 (Michigan State University).
  \end{acknowledgement}



\bibliographystyle{spphys} 
\bibliography{lnplib}



\label{chap:chapter8}
\title{Variational and Diffusion Monte Carlo approaches to the nuclear few- and many-body problem}
\author{Francesco Pederiva, Alessandro Roggero, Kevin E. Schmidt}
\institute{Francesco Pederiva \at Physics Department, University of Trento, and INFN-TIFPA, Trento, Italy , \email{francesco.pederiva@unitn.it} \and Kevin E. Schmidt \at Department of Physics, Arizona State University, Tempe AZ 85283-1506 (USA) \email{kevin.schmidt@asu.edu} \and
Alessandro Roggero \at Institute for Nuclear Theory, University of Washington, Seattle WA (USA) , \email{roggero@uw.edu}}
\newcommand{\redd}[1]{\textcolor{red}{#1}}


\maketitle
\abstract{We review Qauntum Monte Carlo methods, a class of stochastic methods allowing for solving the many-body Schroedinger equation for an arbitrary Hamiltonian. The basic elements of
the stochastic integration theory are first presented, followed by the implementation to the
variational solution of the quantum many-body problem. Projection algorithms are then introduced, beginning with a formulation in coordinate space for central potentials, in order to illustrate
the fundamental ideas. The extension to Hamiltonians with an explicit dependence on the spin-isospin degrees of freedom is then presented by making use of auxiliary fields (Auxiliary Field Diffusion Monte Carlo, AFDMC). Finally, we present the Configuration Interaction Monte Carlo algorithm (CIMC) a method to compute the ground state of general, local or non-local, Hamiltonians based on the configuration space sampling.}

%\maketile


\section{Monte Carlo methods in quantum many-body physics}
\subsection{Expectations in Quantum Mechanics}
In the previous chapters the authors pointed out in several different ways that the non-relativistic quantum many-body problem 
is equivalent to the solution of a very complicated differential equation, the many-body Schr\"odinger equation.

As it was illustrated, in the few-body case ($A<6$) it possible to find compute exact solutions. At the very least, one can expand the eigenfunctions on a basis set including $\cal M$ elements,
diagonalize the Hamiltonian matrix, and try to reach convergence as a function of $\cal M$. Unfortunately, this procedure becomes more and more expensive
when the number of bodies $A$ increases. There are many ingenuous ways to improve the speed of convergence and the quality of the results. The price to pay often is the introduction of more or less controlled approximations.

All these approaches have one common feature: they end up with some closed expression for the eigenfunctions. However, we should remember that the wavefunction {\it per se} is not an observable. In order to make predictions to be compared with experiments, we only need a way to compute {\it expectations} of operators $\hat{O}$ describing the observables we are interested in. 

Given a many-body Hamiltonian $\hat{H}$, we might want, for instance, to look for the ground state eigenfunction and eigenvalue. This means that we want to solve the following equation:
\begin{equation}
\label{eqchap9.s}
\hat{H}\vert \Psi_0\rangle=E_0\vert \Psi_0\rangle.
\end{equation}  
At this point we to provide a representation of the Hilbert space in term of some basis set.
This set will be denoted as $\{|X\rangle\}$. Its elements could be eigenstates of the position
or of the momentum operators, or eigenstates of a simpler Hamiltonian of which we know the exact spectrum. 
In order to make the notation less cumbersome, we will assume that the quantum numbers $X$ characterizing the basis states are in the continuum. In the case of a discrete spectrum, integrals in the following have to be replaced by sums over all their possible values, without any loss of generality. As an example, $X$ could include the positions or the momenta of $A$ nucleons, and their spin and isospin values.  

All the physical information we need about the time-independent problem is then included in integrals of the form:
\begin{equation}
\langle O\rangle \equiv\langle \Psi_0\vert\hat{O}\Psi_0\rangle =\frac{\displaystyle\int\; dX dX'\langle \Psi_0\vert X\rangle\langle X\vert \hat{O}\vert X'\rangle\langle X'\vert\Psi_0\rangle}
{\displaystyle\int\; dX \vert\langle X\vert \Psi_0\rangle\vert^2}.
\end{equation}


These integrals are apparently as hard to solve as the Schroedinger equation itself, even if we had access to the explicit form of the wavefunction. Is there any real gain in reformulating the problem this way?

We can first notice that expectations can in general be written in a slightly different form, independent of the nature of the operator $\hat{O}$:
\begin{equation}
\langle O\rangle =\displaystyle\frac{\displaystyle\int\; dX \vert \langle X\vert \Psi_0\rangle\vert^2\displaystyle\displaystyle\frac{\langle X \vert \hat{O}\Psi_0\rangle}{\langle X\vert \Psi_0\rangle}}{\displaystyle\int\; dX \vert\langle X \vert \Psi_0\rangle \vert^2}.
\end{equation}
For the moment we will just assume that the quotient appearing at numerator of the expectation is always well defined, and we will later discuss this aspect in more detail.
The standard quantum mechanical interpretation of the wavefunction tells us that the quantity:
\begin{equation}
P[X]=\frac{\vert \langle X\vert \Psi_0\rangle\vert^2}{\displaystyle\int\; dX \vert\langle X \vert \Psi_0\rangle \vert^2},
\end{equation}  
is the probability density of finding the system in the state $\vert X \rangle$ labeled by the set of quantum numbers $X$. Thereby, the expectation integral has the general form:
\begin{equation}
\langle O\rangle =\int\; dX P[X]\displaystyle\frac{\langle X \vert \hat{O}\Psi_0\rangle}{\langle X\vert \Psi_0\rangle},
\label{eq5}
\end{equation}
i.e. the average of what we will call the {\it local} operator $O_{loc}\equiv \frac{\langle X \vert \hat{O}\Psi_0\rangle}{\langle X\vert \Psi_0\rangle}$ weighted with the probability of finding the system in a given state $\vert X\rangle$. 
Integrals like that in Eq. (\ref{eq5}) have a direct physical interpretation. In a measurement process what we would observe is essentially the result of a {\it sampling process} of $P[X]$. The expectation of our operator is approximated by:
\begin{equation}
\langle O \rangle \simeq \frac{1}{M}\sum_{k=1}^M{O(X_k)},
\end{equation}
where $M$ is the number of measurements performed, and $O(X_k)$ is a shorthand notation to 
indicate the value assumed by the observable $\hat{O}$ in the state labeled by the quantum numbers $X_k$. The laws of statistics also give us a way of estimating a {\it statistical} error on $\langle O \rangle$, and we know that the error decreases by increasing 
the number of measurements. 

There is here an important point to notice: in a physical measurement process we have {\it no direct knowledge  of the wavefunction}, we just {\it sample} its squared modulus! 

This argument suggests that if we had a numerical way of sampling the squared modulus of a wavefunction, we could in principle compute expectations and make comparisons with experiments
without needing an explicit expression of the wavefunction itself. Quantum Monte Carlo methods
aim exactly at solving the many-body Schr\"odinger equation by sampling its solutions, eventually
without any need of an explicit analytical form.

The remainder of this chapter will be organized as follows. First we will discuss
how to perform calculations based on an accurate, explicit ansatz for the wavefunction of an $A$-body system interacting via a purely central potential, exploiting the variational principle of quantum mechanics (Variational Monte Carlo methods). Then we will discuss how to sample the exact ground state of the system by
projecting it out of an initial ansatz (Projection Monte Carlo methods). Finally, we will
see how these methods need to be extended when we are interested in studying Hamiltonians
that have an explicit dependence on the spin and isospin states of the particles, as it 
happens for the modern interactions employed in nuclear physics.

\section{Variational wavefunctions and VMC for central potentials}



\subsection{Coordinate space formulation}
As previously discussed, we are in principle free to choose any representation of the Hilbert
space of the system we like, in order to compute expectations. The most convenient choice, for a system of particles interacting via a purely central potential, with no explicit dependence on the spin or isospin state, is to use the eigenstates of the
position operator. If $R={{\bf r}_1,\dots{\bf r}_A}$ are the coordinates of the $A$ 
(identical)\footnote{We will always refer to systems of identical particle throughout the text. The generalization to mixtures is normally straightforward, and it will not be discussed here.} particles of mass $m$ constituting the system, we have that:
\begin{equation}
\vert X\rangle \equiv \vert R \rangle
\end{equation}
with the normalization:
\begin{equation}
\langle R'\vert R\rangle = \delta(R-R') \,.
\end{equation}
Notice that we are here considering a $3A$-dimensional Cartesian space, without decomposing it
in the product of $A$ $3$-dimensional spaces. In this representation the wavefunction 
is simply given by:
\begin{equation}
\langle R\vert\Psi_0\rangle \equiv \Psi_0(R)=\Psi_0({\bf r}_1,\dots{\bf r}_A).
\end{equation}
The Hamiltonian instead reads:
\begin{equation}
\hat{H}=\sum_{i=1}^A \frac{p_i^2}{2m}
+V({\bf r}_1,\dots{\bf r}_A),
\end{equation}
or
\begin{equation}
\hat{H} = \int dR \vert R\rangle
\left [
-\frac{\hbar^2}{2m}\sum_{i=1}^A \nabla^2_i +V({\bf r}_1,\dots{\bf r}_A)
\right ] \langle R\vert\,,
\end{equation}
where $V$ is the interparticle potential. 
Substituting this form into Eq. \ref{eqchap9.s}, operating from the
left with $\langle R\vert$ gives the Schr\"odinger differential equation
\begin{equation}
\left [
-\frac{\hbar^2}{2m}\sum_{i=1}^A \nabla^2_i +V({\bf r}_1,\dots{\bf r}_A)
\right ] \Psi_0(R) = E_0\Psi_0(R) \,.
\end{equation}
We will often use the same symbol for the Hilbert space operator
and its differential form and write this simply as
$\hat H\Psi_0(R)=E_0\Psi_0(R)$; whether the
operator or differential form is used can be discerned readily
from context.
In this representation the states of the Hilbert space are sampled by sampling the particle
positions from the squared modulus of the wavefunction $\vert\Psi_0(R)\vert^2$. 

\subsection{Variational principle and variational wavefunctions}
As already seen in the previous chapters, one of the possible ways to approximate
a solution of the many-body Schroedinger equation is to exploit the variational principle.
Given a {\it trial state} $|\Psi_T\rangle$, the following inequality holds:
\begin{equation}
E_T=\frac{\langle \Psi_T\vert \hat{H}\Psi_T\rangle}{\langle\Psi_T\vert\Psi_T\rangle}\geq E_0,
\end{equation}
where $E_0$ is the ground state eigenvalue of the Hamiltonian $\hat{H}$. The equality holds
if and only if $\vert \Psi_T\rangle = \vert \Psi_0\rangle$. The variational principle holds for the ground state, but also for excited states, provided that $\vert \Psi_T\rangle$ is 
orthogonal to all the eigenstates having eigenvalue lower than that of the state one wants
to approximate.

In coordinate space the formulation of the variational principle can be directly
transformed in a form equivalent to that of Eq. (\ref{eq5}):
\begin{equation}
E_T=  \displaystyle\frac{\displaystyle\int\; dR \vert \Psi_T(R)\vert^2\displaystyle\displaystyle\frac{\hat{H}\Psi_T(R)}{ \Psi_T(R)}}{\displaystyle\int\; dR \vert \Psi_T(R) \vert^2}\geq E_0, \label{eq911}
\end{equation}
where $\frac{\hat{H}\Psi_T(R)}{ \Psi_T(R)}$ is called the {\it local energy}. 
Contrary to what happens in functional 
minimization approaches (such as the Hartree-Fock method), the variational 
principle is used to determine the best trial wavefunction within a class defined
by some proper ansatz. The wavefunction will depend on a set of 
{\it variational parameters} $\{\alpha\}$. The solution of the variational problem
will therefore be given by the solution of the Euler problem:
\begin{equation}
\frac{\delta E_T(\{\alpha \})}{\delta \{\alpha\}}=0.
\end{equation} 
This means that in order to find the variational solution to the Schroedinger problem we need to evaluate many times the integral of Eq.(\ref{eq911}) using
different values of the variational parameters, and find the minimum trial eigenvalue. 
\subsection{Monte Carlo evaluation of integrals}
The integral in Eq. (\ref{eq911}) is in general defined in a $3A$-dimensional space. Since particles interact, we expect that the solution cannot be expressed as a product of single particle functions, and therefore the integral cannot be factorized in a product of simpler integrals. In this sense, the problem is strictly analogous to that of a classical gas at finite temperature $\beta=1/{K_B T}$. In that case, given a classical Hamiltonian $H(p,q)=\sum_{i=1}^{A}\frac{p_i^2}{2m}+V(q_1\dots q_A)$, the average energy of the system is given by:
\begin{equation}
E=\frac{3A}{2}K_B T+\frac{1}{Z}\int\;dq_1\cdots dq_A V(q_1\cdots q_A)e^{-\beta V(q_1\cdots q_A)},
\end{equation}
where 
\begin{equation}
Z\equiv\int\;dq_1\cdots dq_A e^{-\beta V(q_1\cdots q_A)}
\end{equation}
is the {\it configurational partition function} of the system. Also in this case the integral to be evaluated is of the same form as Eq. (\ref{eq911}). We can distinguish in the integrand the product of a {\it probability density}:
\begin{equation}
P(q_1\dots q_A)=\frac{e^{-\beta V(q_1\cdots q_A)}}{Z},
\end{equation}
and a function to be integrated which is the potential energy $V$. For classical systems we have a quite intuitive way of proceeding, which is at the basis of statistical mechanics. If we are able to compute (or measure) the potential for some given set of particle coordinates, and we average over many different configurations (sets of particle positions), we will obtain the estimate of the potential energy we need. 

This fact can be easily formalized by making use of the Central Limit Theorem. Given a probability density $P[X]$ defined in a suitable event space $X$, let us consider an arbitrary function $F(X)$. One can define a stochastic variable:
\begin{equation}
S_N(F)=\frac{1}{N}\sum_{i=1}^{N}F(X_i),
\end{equation}
where the events $X_i$ are assumed to be {\it statistically independent}, and are distributed according to $P[X]$. The stochastic variable $S_N(F)$ will in turn have its own probability density $P[S_N]$, which in general depends on the index $N$. The Central Limit Theorem states that for large $N$ the probability density $P[S_N]$ will be a Gaussian, namely:
\begin{equation}
\lim_{N\rightarrow \infty} P[S_N]=\frac{1}{\sqrt{2\pi\sigma^2_N(F)}}\exp\left\{\displaystyle-\frac{(S_N-\langle F\rangle)^2}{2\sigma^2_N(F)}\right\},
\end{equation}
where we define the expectation of $F$ as:
\begin{equation}
\begin{split}
&\langle F\rangle=\int P[X]F(X)dX,
\\
&\langle F^2\rangle=\int P[X]F^2(X)dX,
\end{split}
\end{equation}
and
\begin{equation}
\sigma^2_N(F)=\frac{1}{N}\left[\langle F^2\rangle-\langle F\rangle^2\right]
\end{equation}
is the variance of the Gaussian.
The reported average is estimated as $S_N(F)$, while 
$\langle F^2\rangle-\langle F\rangle^2$ is estimated by
$\frac{N}{N-1}\left [S_N(F^2)-S_N^2(F)\right ]$.
This well known result is at the basis of all measurement theory. Averages over a set of measurements of a system provide the correct expectation of the measured quantity with an error that can be in turn estimated, and that decreases with the square root of the number of measurements $N$. 

This result is very important from the point of view of numerical evaluation of integrals. If we had a way to numerically sample an arbitrary probability density $P[X]$, we could easily estimate integrals like that in Eq. (\ref{eq911}). The statistical error associated with the estimate would decrease as the square root of the sampled points {\it regardless of the dimensionality of the system}. 

For a classical system, configurations might be generated by solving Newton's equations, possibly adding a thermostat in order to be consistent with the canonical averaging. However, this is not certainly possible for a quantum system. The solution is to use an artificial dynamics, provided that it generates (at least in some limit) configurations that are distributed according to the probability density we want to use. Once again, in order to simplify the following description we will work in the space of the coordinates of the $A$ particles, but the argument can be generalized to arbitrary spaces.

A very detailed description of what follows in this section can be found in the book of Kalos and Whitlock\cite{Kalos08} and references therein. 

We start defining a {\it transition matrix} $T_k(R_{k+1}\leftarrow R_k)$ expressing the probability that in the $k$-th step of the dynamics the system moves from the configuration $R$ to a configuration $R'$. If at the first step the system is in a configuration $R_0$, sampled from an arbitrary distribution $P_0[R_0]$, the probability density of finding the system in a configuration $R_1$ at the next step will be given by:
\begin{equation}
P_1[R_1] = \int\; dR_0 P_0[R_0]T_0(R_1\leftarrow R_0).
\end{equation}
We the introduce an integral operator $\hat{T}_0$ such that:
\begin{equation}
P_1[R_1]=\hat{T}_0 P[R_0].
\end{equation}
With this notation, the probability density of the configuration at an
arbitrary step $k$ will become:
\begin{equation}
P_k[R_k]=\hat{T}_{k-1} P[R_{k-1}]=\hat{T}_{k-1}\cdots\hat{T}_{1}
\hat{T}_{0} P_0[R_0].
\end{equation}
The sequence of stochastic variables $R_k$ generated at each step of this procedure is called a {\it Markov Chain}. Let us assume that $\hat{T}_k$ does not depend on the index $k$. What we will generate is then a {\it stationary} Markov Chain, for which the probability density generated at each step will only depend on the transition matrix and the probability density of the first element. In fact:
\begin{equation}
P_k[R_k]=\hat{T} P[R_{k-1}]=\hat{T}\cdots\hat{T}
\hat{T} P_0[R_0]=\hat{T}^kP_0[R_0].
\end{equation}
Under these assumptions one might wonder if the sequence is convergent (in functional sense), i.e. if a limiting probability density $P_\infty[R]$ exists. It is interesting to notice that if such function exists, it has to be an eigenvector of the integral operator $\hat{T}$. In fact, since we assume $\hat{T}$ to be independent of $k$ we have:
\begin{eqnarray}
\begin{array}{rcl}
\lim_{k\rightarrow\infty}\hat{T}P_k[R_k]&=&\lim_{k\rightarrow\infty}P_{k+1}[R_{k+1}]\nonumber\\
\\
\hat{T}P_\infty[R]&=&P_\infty[R]\nonumber.
\end{array}
\end{eqnarray}
It is also easy to realize that the eigenvalue is indeed 1. In fact, let us consider the general relation:
\begin{equation}
\hat{T}P_\infty[R]=\gamma P_\infty[R].
\end{equation}
The recursive application of $\hat{T}$ would give:
\begin{equation}
\hat{T}^k P_\infty[R]=\gamma^kP_\infty[R].
\end{equation}
If $\gamma\neq 1$ we would lose the normalization property of $P\infty[R]$.

These properties of stationary Markov chains can be exploited to sample a generic probability density $P[R]$. In fact, if we can determine the transition operator that has as eigenvector a {\em given} $P_\infty[R]$, a repeated application of such operator to an {\em arbitrary} initial distribution of points will eventually generate a chain in which each element is distributed according to $P_\infty[R]$. There is a simple recipe to construct such transition operator. We will assume that we have at hand a transition operator $\hat{\bar{T}}$ that we can sample (it could be as simple as a uniform probability within a given volume). We will split the searched transition operator
in the product of $\hat{\bar{T}}$ and an unknown factor $\hat{A}$ that we will call "acceptance probability", defined in such a way that:
\begin{equation}
\hat{\bar{T}}\hat{A}=\hat{T}.
\end{equation}

In order for the system to preserve its equilibrium state once the probability
distribution is reached, we expect that the dynamics described by the random walk will not change the density of sampled points anywhere in the events space.  Transitions carrying away from a state $R$ to anywhere must be balanced by transitions leading from anywhere to the same state $R$:
\begin{equation}
\label{equilibrium}
\int dR' P(R)T(R'\leftarrow R)= \int dR' P(R')T(R\leftarrow R') \,.
\end{equation}
One way to enforce this condition is to impose the more
stringent {\it detailed balance} condition, which requires the {\it integrands} in  Eq.(\ref{equilibrium}) be equal:
\begin{equation}
P(R)T(R'\leftarrow R)=P(R')T(R\leftarrow R') \,.
\end{equation}
The detailed balance condition can be in turn recast into a requirement on the acceptance probability. In fact:
\begin{equation}
\frac{A(R'\leftarrow R)}{A(R\leftarrow R')}=\frac{P(R')}{P(R)}\frac{\bar{T}(R\leftarrow R')}{\bar{T}(R'\leftarrow R)}.
\label{detb2}
\end{equation} 
The quantities on the r.h.s. of Eq. (\ref{detb2}) are all known. The configuration $R'$ has to be sampled originating in $R$ from the given transition probability $\bar{T}(R\leftarrow R')$. The probability density $P(R)$ is the one we actually want to asymptotically sample. 
If we interpret the $A$ values to be probabilities to actually keep the transition, then maximizing the possible $A$ values leads to the slightly modified version of Eq. (\ref{detb2}):
\begin{equation}
A(R'\leftarrow R)=\min\left(\frac{P(R')}{P(R)}\frac{\bar{T}(R\leftarrow R')}{\bar{T}(R'\leftarrow R)},1\right).
\label{detb2}
\end{equation} 
This expression is often called the {\it acceptance ratio}. In practice, it represents the probability according
to which we have to {\it accept} the new configuration as the new member of the Markov chain, rather than
keeping the original point as the next point in the chain.\footnote{The standard jargon refers to this as a "rejection" event. However one has not to be confused: this is the result of a {\it reversed} move, and generates a new element in the chain coincident with the starting point.}.
Further analysis shows that existence and uniqueness of the correct eigenvalue
1 solution and therfore convergence to the correct distribution
will be guaranteed if (1) every allowed state can be reached
from any other by a finite sequence of transitions and (2)
there are no cycle of states. The latter is guaranteed if there are
any transitions that leave the system in the same state, that is any rejections.

There is a case in which the Eq. (\ref{detb2}) further simplifies. If the transition matrix is taken to 
be symmetric in the arguments $R$ and $R'$, the ratio becomes unity, and one is left with:
\begin{equation}
A(R'\leftarrow R)=\min\left(\frac{P(R')}{P(R)},1\right).
\label{detb2}
\end{equation} 
At this point we have all the ingredients to describe an algorithm that performs a Monte Carlo evaluation
of an integral such that of Eq. (\ref{eq911}). In the following we will describe the simplest version,
i.e. the so called "Metropolis-Hastings algorithm"~\cite{Metropolis53,Hastings70}. 
\begin{enumerate}
	\item
	Start from an arbitrary configuration of the $A$ particles. If the potential has a strongly repulsive core one has to pay attention to avoid overlapping pairs.
	\item
	Sweep over the coordinates and generate new positions according to some transition probability. A simple choice is a uniform displacement within a cube of side $\Delta$, i.e.:
	\begin{equation}
	  \bar{T}(R'\leftarrow R)=\left\{ 
	  \begin{array}{cl}
	  \frac{1}{\Delta}&{\rm if \ \ \  } |R'^\alpha_i-R^\alpha_i|<\frac{\Delta}{2}\\
	  \\
	  0&{\rm otherwise}
	  \end{array}
	     \right.
	\end{equation}
	with $\alpha =x,y,z$, and $i=1\dots A$. This choice has the advantage of being symmetric. If we imagine
	to store our configuration in an array $R[0...2][0...A-1]$ the implementation of this step would read:
\begin{svgraybox}
	\begin{algorithmic} 
		\State{MC\_Move()}
		\For{$i \in \{0,A-1\}$}  
		\For{$j \in \{0,2\}$ }
	        \State{$R_{new}[i][j]\gets R[i][j]+(\text{rand}()-0.5)*\Delta$}
		\EndFor
		\EndFor
	\end{algorithmic}
\end{svgraybox}
    We will assume that the function ${\rm rand}()$ generates a random number uniformly distributed
    in $[0,1)$.
    \item 
    At this point we need to evaluate the acceptance ratio. This is easily done with our choice of the transition matrix, since we only need to evaluate the probability densities in $R$ and $R'$:
    \begin{equation}
    A(R'\leftarrow R)=\min\left(\frac{|\Psi_T(R')|^2}{|\Psi_T(R)|^2},1\right).
    \end{equation}
    \item
    Next we need to decide whether we keep the proposed configuration as the next element in the chain or if we want to resort to the original one. If we define ${\rm acc} = A(R'\leftarrow R)$, then:
\begin{svgraybox}
    	\begin{algorithmic} 
    		\State{Accept\_reject()}
    		\State{$\xi = \text{rand}()$}
    		\If{$\text{acc} > \xi$}
	    		\State{$R[i][j]\gets R_{new}[i][j]$}
    		\EndIf
    	\end{algorithmic}
\end{svgraybox}
    	\item
    	According to the Central Limit Theorem, we now need to cumulate the values of the rest of the integrand. In the case of our variational calculation we need to sum up the local energies.
    	Notice that this step has to be taken whatever the result of the procedure described at the previous point. If we want to estimate the statistical error, we also need to cumulate the {\it square}
    	of the local energy. 
\begin{svgraybox}
    	\begin{algorithmic} 
    	   \State{Acuest()}
    	   \State{$\text{eloc}\gets \frac{\hat{H}\Psi_T(R)}{\Psi_T(R)}$}
    	   \State{$\text{ecum}\gets \text{ecum} + \text{eloc}$}
    	   \State{$\text{ecum2}\gets \text{ecum2} + \text{eloc}*\text{eloc}$}
    	   \end{algorithmic}
\end{svgraybox}
    	\item
    	Steps 2 to 5 need to be repeated $N_{steps}$ times, where $N_{steps}$ must be sufficiently large to provide a small enough statistical error. The final estimate of the energy is given by
    	$\langle E\rangle \pm \Delta E$, where:
    	\begin{eqnarray}
    	\begin{array}{c}
    	\langle E\rangle = \frac{1}{N_{steps}}\cdot\text{ecum}\\
    	\\ 
    	\Delta E = \sqrt{\frac{1}{N_{steps}-1}\left(\frac{1}{N_{steps}}\cdot\text{ecum2}-\langle E\rangle^2\right)}
    	\end{array}
    	\end{eqnarray}
  \end{enumerate}
  Notice that this algorithm could in principle be used to evaluate arbitrary integrals. In fact, it is
  always possible to multiply and divide the integrand by a probability density $P(X)$ that can be used
  to sample the values of $X$:
  \begin{equation}
  I=\int F(X) dX = \int P[X]\frac{F(X)}{P[X]}
  \end{equation}
  \subsubsection{Autocorrelations}
  The main hypothesis underlying the Central Limit Theorem is that data used to construct the averages are sampled independently. While in a measurement process this is a quite reasonable assumption, in the case of the computation of an integral by means of any method based on the Markov chain theory (including the Metropolis-Hastings method) this requirement is not satisfied by construction. In fact, data are sampled based on a transition matrix, and the resulting random walk has a certain degree of memory of the past events. What are the consequences of such memory? Let us consider a sequence of points $X_1,X_2,\cdots,X_N$ sampled via the Metropolis algorithm from some probability density $P[X]$. If we assume these data not to be independent, we have to consider the joint probability for the specific realization of the chain in order to estimate the integral of a given function $F$:
  \begin{equation}
  I = \frac{1}{N}\sum_{i=1}^{N}\int dX_1,dX_2,\cdots dX_N P[X_1,X_2,\cdots X_N]F(X_i).
  \end{equation}  
  If the samples are independent then $P[X_1,X_2,\cdots X_N]=P[X_1]P[X_2]\cdots P[X_N]$, and we are in the case previously discussed. However, since we can arbitrarily exchange the indexes of the integration variables, we can easily see that the value of $I$ is unchanged despite the presence of correlations.
  By construction,  in a Markov process two consecutive samples will always be correlated to each other. This seems to be inconsistent to the use we want to make of these samples, i.e. to apply the Central Limit Theorem to integration. However, we can hope that after a certain number of steps memory is lost, and data will become effectively independent.
  Is it possible to estimate this typical {\it autocorrelation length}?
  Based on the previous argument one can define a measure of the autocorrelation by looking at the variance of the expectation of $F$ with respect to $P$: 
  \begin{equation}
   (\Delta I)^2=\left \langle \displaystyle\frac{1}{N^2}\sum_{i=1}^N F(X_i)\sum_{i=1}^N F(X_j)\right \rangle-\langle F\rangle^2.
  \end{equation}
  The corresponding standard deviation  is the estimate of the statistical error on the integral of $F$. The first term can be recast in the following way:
  \begin{eqnarray}
  \begin{array}{c}
\left  \langle \displaystyle\frac{1}{N^2}\sum_{i=1}^N F(X_i)\sum_{i=1}^N F(X_j)\right \rangle=\\ \\
  \displaystyle \frac{1}{N^2}\sum_{i,j=1}^N\int P[X_1,X_2,\cdots X_N]F(X_i)F(X_j)dX_1\cdots dX_N=\\ \\
  \displaystyle \frac{1}{N^2}\sum_{i,j=1}^N \langle F(X_i)F(X_j)\rangle.
  \end{array}
  \end{eqnarray}
  Since the Markov chain is stationary, this quantity is expected to depend only on the difference of the indexes $\tau=i-j$. We will then define an {\it autocorrelation coefficient}:
  \begin{equation}
  c(F)_\tau =\frac{\langle F(X_i)F(X_{i+\tau})\rangle - \langle F\rangle^2}{\langle F^2\rangle - \langle F\rangle^2}.
  \end{equation}
  The coefficient is normalized to the variance  $\sigma^2(F)$, in such a way that $C(F)_0=1$. Correlation coefficients are related to the average of the product of the $F$ in the following way:
  \begin{equation}
  \langle F(X_i)F(X_{i+\tau})\rangle=c(F)_\tau\sigma^2(F)+\langle F\rangle^2.
  \end{equation} 
  We can use the previous expression to estimate the error on $I$:
  \begin{eqnarray}
  \begin{array}{c}
  (\Delta I)^2 = \displaystyle\frac{1}{N^2}\sum_{i,j=1}^N \langle F(X_i)F(X_j)\rangle-\langle F\rangle^2=\\ \\ 
  \displaystyle\frac{1}{N}\sigma^2(F)\sum_{\tau=1}^N c(F)_\tau +\langle F\rangle^2-\langle F\rangle^2= \frac{\sigma^2(F)}{N}\sum_{\tau=1}^N c(F)_\tau.
  \displaystyle 
  \end{array}
  \end{eqnarray}
As it can be seen the error not only depends on the variance of $F$, but also on the sum over all the autocorrelation coefficients of $F$. This is the main consequence of having autocorrelated samples: the statistical error is underestimated by the variance of $F$, and needs to be corrected by a factor that depends on the autocorrelation length. 

Usually the coefficients $c(F)_\tau$ have an exponential decay. If we approximate them as 
$c(F)_\tau \sim \exp(-\tau/{\bar{\tau}})$, the sum of the coefficients can be approximated as:
\begin{equation}
\sum_{\tau=1}^{N}c(F)_\tau\sim\int_{0}^{\infty} d\tau e^{-\frac{\tau}{\bar{\tau}}}=\bar{\tau}.
\end{equation}
This means that it is sufficient to fit the exponential decay of the autocorrelation coefficients in order to find an estimate of the characteristic autocorrelation length that corrects the estimate of the error on the integral. In particular the correct expression for the error is:
\begin{equation}
\Delta I\simeq\sqrt{\frac{1}{N-1}\sigma^2(F)\bar{\tau}},
\end{equation}
which has a simple interpretation: We are not generating $N$ independent samples of the variable $X$ during our Markov process, but rather $N/\bar{\tau}$ of them, and this number must be used as the correct count of events for the error estimation.

It is important to be extremely careful about the estimation of autocorrelations. in many cases an underestimation of the statistical errors leads to a wrong interpretation of the results and to wrong physical conclusions.

Autocorrelations also play a crucial role in choosing the step width $\Delta$ in the Metropolis-Hastings algorithm. A common criterion is to choose it in such a way that the fraction of accepted moves is about 50\%. However, the ideal value is clearly the one minimizing the autocorrelations among samples, and quite often this value corresponds to acceptances of the order 30 or 40\%. 

Once the value of $\bar{\tau}$ has been estimated, it is possible to organize the calculation in such a way that the statistical error computed by the code is more realistic by using a {\it reblocking}
technique. In practice the values of the quantity to be averaged are summed up in blocks of $N_b$ elements each:
\begin{equation}
F^b_l=\sum_{i=1}^{N_b} F(X_i).
\end{equation} 
Then, the $F^b_l$ are used as the data on which performing the computation of the variance and of the standard deviation. If $N_b\gtrsim\bar{\tau}$, the standard deviation will be corrected by the effects of the autocorrelation of the original data.
Typically calculations store block values so that the values can be
``reblocked'' for example by combining pairs of blocks. The estimated
error should be unchanged if the blocks are uncorrelated. In addition,
the ratio of
the block variance to the variance of the original
function can be used to estimate the number of independent saqmples,
and therefore the autocorrelation time.

\subsection{Construction of the wavefunction and computational procedures}

When performing a variational calculation, the first step consists of deciding which model wavefunction we intend to use. 

First of all we have to take care of the symmetry of the particles. Nucleons are Fermions, and therefore it is necessary to build an antisymmetric wavefunction. If the Hamiltonian does not contain terms acting on the spin or isospin state of a nucleon or of a pair of nucleons, each particle will preserve its own initial state. In this case it is easy to write an antisymmetric wavefunction simply using a product of Slater determinants, one for each species. 

To build the determinants one needs some single particle orbitals. There are several possible choices. For nuclei linear combinations of Gaussians or the eigenstates of the harmonic oscillator are definitely an option. Another choice might be that of using orbitals coming from a Hartree-Fock calculation. In this case the orbitals contain some information about the fact that nucleons interact, but there usually is no consistency between the Hamiltonian used to compute the orbitals and the Hamiltonian we are interested in.

The basic starting point is then a wavefunction of the form:
\begin{equation}
\varphi(R)=\text{det}[\phi_j({\bf r}_{p^\uparrow_i})]\text{det}[\phi_j({\bf  r}_{p^\downarrow_i})]\text{det}[\phi_j({\bf r}_{n^\uparrow_i})]\text{det}[\phi_j({\bf r}_{n^\downarrow_i})],
\end{equation}
If we just limited ourselves to this kind of wavefunction we would miss most of the interesting physics that happens when particles are close together. A seen in the previous chapters, a very important role is played by the {\it short range correlations}, that should introduce the many-body effects due to repulsion/attraction of particles at short distance. Contrarily to what one does in other methods, such as coupled clusters, in Quantum Monte Carlo calculations it is easier to work with wavefunctions containing {\it explicit} two-, three- or many-body correlations. 

Here we will use the so-called {\it Jastrow} factor, i.e. a product of two-body functions  that helps to reproduce the correlations from the pair-wise potential. The simplest version of a trial wavefunction therefore reads:
\begin{equation}
\Psi_T(R)=\varphi(R)\prod_{i<j}^Af(r_{ij}),
\end{equation}
where $R=({\bf r_1,\cdots,r_A})$, and $f$ is the so called {\it Jastrow function} (JF). How do we determine the JF? We have some information that we can exploit. In particular we might seek for analytic forms of $f$ that satisfy what is commonly called the {\it cusp condition},(see e.g. Ref.\cite{Cep77}) i.e. we must have:
\begin{equation}
\frac{\hat{H}f(r_{ij})}{f(r_{ij})}<\infty
\end{equation}
everywhere in space. It is easy to realize that satisfying the cusp condition helps to prevent the local energy from fluctuating too much even in presence of a divergence of the potential, thereby reducing the variance and the statistical error. Usually in nuclear physics problems it is customary to take a further step.
Recognizing that at small separations, the many-body Schr\"odinger equation
is dominated by the short-range pair potential,
the two-body problem is solved to determine the $f$. In particular one can solve the following Schr\"odinger equation in relative coordinates:
\begin{equation}
\label{EqJas}
-\frac{\hbar^2}{2m}\nabla^2+qV(r)f(r)=\epsilon f(r),
\end{equation}
and impose the boundary condition that the function becomes a constant at a distance $h$ from the origin, where other parts of the Hamiltonian become important.  The quantities $q$ and $h$ are two variational parameters. One could in principle consider a third variational parameter in the Jastrow factor by using a modified Jastrow function $\tilde{f}$ such that:\begin{equation}
\tilde{f}(r)=e^{-b\log f(r_{ij})}
\end{equation}
The function $f$ is usually determined by numerically solving Eq.(\ref{EqJas}) with the Numerov or Runge-Kutta methods. One has to be careful that the resulting table has to be interpolated to compute the function at an arbitrary distance. Therefore it is important to choose an appropriate number of points (usually of the order of a few thousands).  
\begin{figure}
	\
\end{figure}
Single particle orbitals can also be either tabulated or computed analytically. Tabulation guarantees in general a faster computation at the price of a loss in numerical accuracy.

In the code it is necessary to compute derivatives of the wavefunction in order to estimate the local energy. This can be done either numerically or analytically. A very good test for checking that there are no major mistakes either in the Monte Carlo evaluation of integrals or in the computation of the local energy is to use the so-called {\it Jackson-Feenberg} identity for the kinetic energy.
\begin{figure}
	\begin{center}
		\includegraphics[scale=0.5]{Chapter9-figures/qvar.eps}
	\end{center}
	\caption{An example of variational minimization of the energy. The estimate of the binding energy of a $^4$He nucleus described by the Minnesota potential is here plotted as a function of the quencher parameter $q$, for a fixed value of the healing distance $h$ (see text). The dotted line serves as a guide for the eye.}
	\label{fig.var}
\end{figure}
The expectation of the kinetic energy is an integral of the form:
\begin{equation}
\label{T_PB}
\langle T\rangle = \frac{\displaystyle-\frac{\hbar^2}{2m}\int_\Omega dR \Psi^*(R)\nabla^2\Psi(R)}{\displaystyle\int_\Omega dR |\Psi(R)|^2},
\end{equation}
where $\Omega$ is the integration volume. Integrating the numerator by parts one gets:
\begin{equation}
\langle T\rangle = \frac{\displaystyle\frac{\hbar^2}{2m}\int_\Omega dR \nabla\Psi^*(R)\cdot\nabla\Psi(R)}{\displaystyle\int_\Omega dR |\Psi(R)|^2} -\frac{\displaystyle\frac{\hbar^2}{2m}\int_{S(\Omega)} dS\Psi^*(R)\nabla\Psi(R) }{\displaystyle\int_\Omega dR |\Psi(R)|^2}.
\end{equation}
The surface term is zero if the wavefunction is well behaved. We are therefore left with the integral:
\begin{equation}
\label{T_JF1}
\langle T\rangle = \frac{\displaystyle\frac{\hbar^2}{2m}\int_\Omega dR \nabla\Psi^*(R)\nabla\Psi(R)}{\displaystyle\int_\Omega dR |\Psi(R)|^2}= \frac{
	\displaystyle\frac{\hbar^2}{2m}\int_\Omega dR|\Psi(R)|^2 \frac{\nabla\Psi(R)}{\Psi(R)}\cdot
\frac{\nabla\Psi^*(R)}{\Psi^*(R)}
}{\displaystyle\int_\Omega dR |\Psi(R)|^2}.
\end{equation}
We can sum Eq.(\ref{T_PB}) and Eq. (\ref{T_JF1}), and divide by 2 in order to obtain a new kinetic energy estimator:
\begin{equation}
\langle T\rangle_{JF} = \frac{
	\displaystyle\frac{\hbar^2}{4m}\int_\Omega dR|\Psi(R)|^2 \left[\frac{\nabla\Psi^*(R)}{\Psi^*(R)}\cdot
	\frac{\nabla\Psi(R)}{\Psi(R)}-\frac{\nabla^2\Psi(R)}{\Psi(R)}\right]
}{\displaystyle\int_\Omega dR |\Psi(R)|^2}.
\end{equation}
\begin{figure}
	\begin{center}
		\includegraphics[scale=0.5]{Chapter9-figures/jas.eps}
	\end{center}
	\caption{The central channel of the Minnesota potential (dashed-dotted line), and the corresponding numerical Jastrow function (solid line) evaluated for $h=3.1$, $q=1.4$, and $b=1$.}
	\label{fig.var}
\end{figure}
This is the {\it Jackson-Feenberg} kinetic energy estimator. It is easy to see that configuration by configuration the value of the integrand of $T$ and $T_{JF}$ are different. However, they have to be the same on average (i.e. always within the current statistical error). The equivalence of the two estimators checks the integration procedure, the correctness of the implementation of the boundary conditions, and the computation of derivatives. If any of these quantities are wrong, the two estimates of the kinetic energy will differ. This is an extremely useful consistency check, and should always be used in a variational calculation.

At this point it is necessary to perform several calculations varying the parameters in the wavefunction, and looking for a minimum of the energy. In the next subsection we will describe algorithms that allow for performing this search in an automatic way. However, when the number of parameters is small, it is also possible in principle to perform a scan on a grid. 

In Fig.\ref{fig.var}, as an example, we report the behavior of the variational energy computed in a $^4$He nucleus, modeled with a two body Minnesota potential, and a wavefunction containing only a central Jastrow product. The spatial part of the orbital is an s-wave Gaussian with half width equal to 1.1fm. The energies have been computed for a fixed value of the healing distance $h=3.1$fm as a function of the quencher parameter $q$, keeping fixed the amplitude parameter $b=1$. Each run consists of an average over 6.4$\times 10^5$ samples, preceded by $6.4\times 10^4$ equilibration steps. 

As it can be seen, there is a clear minimum of the energy. 
The minimum can be determined with sufficient accuracy by fitting the resulting curve. A fit with a quadratic function predicts a minimum at $q\sim1.4$. The corresponding eigenvalue is $E_T=-15.31(4)$MeV\footnote{the number in parenthesis indicates the statistical error on the last figure}. The procedure should be repeated for different values of all other variational parameters until an absolute minimum is found. 

The variational wavefunctions can be made arbitrarily richer in structure in order to improve the results, including what our physical intuition suggests as important terms to describe correlations. We will later discuss how to construct trial wavefunctions for realistic nuclear Hamiltonians.
A full variational calculation for the $^4$He nucleus with the Minnesota potential, including Jastrow factors with a spin/isospin dependence would give a binding energy $E_T=-25.52(4)$MeV.

 

\subsection{Wave function optimization}
\subsubsection{Reweighting methods}
The brute-force optimization of the trial wave function becomes quite cumbersome with more than a few parameters. In general the problem is equivalent to searching an absolute minimum in a multi-dimensional space, and does not admit a simple solution.
If one is interested in a quick search for local minima, it is possible to compute the gradient of the energy in the parameter space, and use for instance some variant of the steepest descent method. Computation of gradients is based on the so-called ``reweighting method''. If we have a trial function depending on a set of parameters $\{\alpha\}$, and another depending on a set $\{\alpha+\delta\alpha\}$, it s not necessary to perform two independent calculations to compute the difference (which would also be affected by rather large statistical errors). In fact, the following identity holds):
\begin{equation}
\frac{\int dR |\Psi_T(R,\{\alpha+\delta\alpha\})|^2 O(R)}{\int dR |\Psi_T(R,\{\alpha+\delta\alpha\})|^2}=
\frac{\int dR |\Psi_T(R,\{\alpha\})|^2 \frac{|\Psi_T(R,\{\alpha+\delta\alpha\}|^2}{|\Psi_T(R,\{\alpha\})|^2}O(R)}{\int dR |\Psi_T(R,\{\alpha\})|^2\frac{|\Psi_T(R,\{\alpha+\delta\alpha\})|^2}{|\Psi_T(R,\{\alpha+\delta\alpha\})|^2}}
\end{equation}
It is therefore possible to use the configurations sampled from a trial wavefunction with a given parametrization $\{\alpha\}$ to compute expectations over a wavefunction with a different parametrization $\{\alpha+\delta\alpha\}$ by simply reweighting the values of the operator with the ration between the square moduli of the two wavefunctions:
\begin{equation}
\langle O_{\{\alpha+\delta\alpha\}}\rangle\equiv
\frac{\langle \Psi_T(R_k,\{\alpha+\delta\alpha\})|O(R_k)|\Psi_T(R_k,\{\alpha+\delta\alpha\})\rangle}{\langle \Psi_T(R_k,\{\alpha+\delta\alpha\})|\Psi_T(R_k,\{\alpha+\delta\alpha\})\rangle}=
\frac{\sum_k \frac{|\Psi_T(R_k,\{\alpha+\delta\alpha\}|^2}{|\Psi_T(R_k,\{\alpha\}|^2}O(R_k)}{\sum_k \frac{|\Psi_T(R_k,\{\alpha+\delta\alpha\}|^2}{|\Psi_T(R_k,\{\alpha\}|^2}}
\end{equation}
where the $R_k$ are sampled from $|\Psi_T(R,\{\alpha\})|^2$. Besides the obvious advantage of avoiding multiple calculations to compute the derivatives, the use of this reweighting technique allows direct computation of expectations of the gradients in the parameter space with very high accuracy. The access to gradients opens the way to the use of automated minimization algorithms such as the already mentioned steepest descent method, the Levemberg-Marquardt algorithm \cite{Cyrus96} or the Linear Method \cite{Toulouse07} briefly sketched below 

\subsubsection{Power method}
\label{sec:pm}
There is another class of algorithms that have been recently introduced, and based on the power method. We will here discuss in particular the algorithm due to Sandro Sorella~\cite{Sorella01}.
This algorithm was originally discussed in terms of
the Lanczos method, but for
a single multiplication by his propagator it
becomes equivalent to the simpler power method that we discuss here.

For $\Lambda$ larger than the largest eigenvalue of the
eigenvectors contained in $|\psi_n\rangle$, operating
with $\Lambda -H$ will multiply the ground state by a larger number
than any other state. Therefore iterating the equation
\begin{equation}
|\psi_{n+1}\rangle = (\Lambda-H)|\psi_n\rangle
\end{equation}
will converge to the ground state.
One way to implement this is to use a set of test functions (which, in
principle, should be complete), $|\phi_m\rangle$. This gives the
set of equations
\begin{equation}
\label{eq.power}
\langle \phi_m |\psi_{n+1}\rangle = \langle \phi_m |(\Lambda -H)|\psi_n\rangle
\,.
\end{equation}


In his original paper Sorella assumes $|\psi_n \rangle = |\Psi_T\rangle$, and next approximates
$|\psi_{n+1}\rangle$ as a linear combination of the original state
and the derivatives with respect to the parameters
\begin{eqnarray}
\label{eq:psiprime}
|\psi_{n+1}\rangle &\simeq& \Delta \alpha_0 |\Psi_T\rangle
+\sum_{n=1} \Delta \alpha_n \partial_{\alpha_n} |\psi_T\rangle
\equiv \sum_{n=0} O^n |\psi_T\rangle \Delta \alpha_n
\end{eqnarray}
and he uses the same functions for $|\phi_m\rangle$, so that
\begin{equation}
|\phi_m\rangle = O^m |\Psi_T\rangle \,.
\end{equation}
When evaluated in the position representation, the $O^m$ for $m>0$ correspond
to multiplying by the derivative of the logarithm of the trial function.
Substituting these expressions, and dividing by $\langle \Psi_T|\Psi_T\rangle$,
Eq. \ref{eq.power} becomes
\begin{eqnarray}
\label{eq.first}
\frac{\langle \Psi_T |O^m (\Lambda-H) |\Psi_T\rangle}
{\langle \Psi_T |\Psi_T\rangle}
= \sum_{n=0} \frac{\langle \Psi_T |O^m O^n |\Psi_T\rangle}
{\langle \Psi_T |\Psi_T\rangle} \Delta \alpha_n\,.
\end{eqnarray}
The expectation values can be calculated and the linear equations solved
to get $\Delta \alpha_n$.

Alternatively, the $m=0$ and $n=0$ terms can be separated. Writing
the trial function expectation of an operator $O$ as $\langle O\rangle$,
Eq. \ref{eq.first} becomes
\begin{eqnarray}
\label{eq.m0}
\langle \Lambda - H \rangle = \Delta \alpha_0 + \sum_{n=1} \langle O^n\rangle
\Delta \alpha_n & m=0\\
\label{eq.mneq0}
\langle O^m (\Lambda-H) \rangle = \langle O^m\rangle \Delta \alpha_0
+ \sum_{n=1} \langle O^m O^n\rangle \Delta \alpha_n & m > 0 \,.
\end{eqnarray}
Substituting Eq. \ref{eq.m0} into Eq. \ref{eq.mneq0} gives
\begin{equation}
\label{eq:srpar}
\langle O^m(\Lambda - H) \rangle -\langle \Lambda - H\rangle \langle O^m\rangle
= \sum_{n=1} \left [ \langle O^mO^n\rangle - \langle O^m\rangle
\langle O^n\rangle \right ] \Delta \alpha_n \,.
\end{equation}
Solving gives $\Delta \alpha_{n>0}$ and
Eq. \ref{eq.m0} then gives the value for $\Delta \alpha_0$.

In either case, the result gives an approximation to the next trial
function as a linear combination of the original function and its
parameter derivatives. The new parameters are chosen to give this
same linear combination as the first two terms in the Taylor series.
Since dividing the approximate expression for $|\psi_{n+1}\rangle$
by $\Delta\alpha_0$ gives
an expression that is the first two terms in the Taylor series,
the new parameters are
\begin{equation}
\alpha^{(\rm new)}_{n>0} =
\alpha^{(\rm old)}_n + \frac{\Delta \alpha_{n>0}}{\Delta \alpha_0}
\end{equation}

%The method can be applied to DMC by substituting the DMC propagator for
%$\Lambda-H$. Using a symmetric function for the importance function,
%should allow better node optimization.

More recently Toulouse \& Umrigar~\cite{Toulouse07} proposed a much more efficient method
where the Hamiltonian is diagonalized in the reduced space spanned by the $|\phi_m\rangle$.
The parameter variation is then given by the solution of the generalized eigenvalue equation
\begin{equation}
\label{eq:lmpar}
\sum_{n=0}\frac{\langle \Psi_T |O^m H O^n|\Psi_T\rangle}
{\langle \Psi_T |\Psi_T\rangle}\Delta \alpha_n
= \Delta E \sum_{n=0} \frac{\langle \Psi_T |O^m O^n |\Psi_T\rangle}
{\langle \Psi_T |\Psi_T\rangle} \Delta \alpha_n\,.
\end{equation}
with the lowest eigenvalue $\Delta E^min$:
\begin{equation}
\alpha^{(\rm new)}_{n>0} =
\alpha^{(\rm old)}_n + \frac{\Delta \alpha^{min}_{n>0}}{\Delta \alpha^{min}_0} .
\end{equation}
The gradient of the local energy is required for the expectation values 
appearing in~Eq.\eqref{eq:lmpar}, and can be efficiently estimated using the
reweighting technique presented in the previous section.

When the parameters are far away from the minimum this approach can be less stable 
than the previous one giving rise to large parameter variations that
invalidate the linear approximation Eq.~\ref{eq:psiprime}. A quick strategy is then to use 
the solution of Eq.~\eqref{eq:srpar} early on in the optimization process and then switch to
Eq.~\eqref{eq:lmpar} when the resulting norm of the variation is below some threshold.

\section{Projection Monte Carlo Methods in coordinate space}
%\subsection{Propagation in imaginary time}
\subsection{General formulation}
\label{sec:generaldmc}
Variational calculations provide only an upper bound for the ground-state eigenvalue of a 
given Hamiltonian. However, it is possible to use Monte Carlo algorithms to actually solve
the Schr\"odinger equation for an arbitrary number of interacting particles. This class of algorithms is bsed on the idea of imaginary time propagation.

Let us consider a Hamiltonian $\hat{H}$. The imaginary time evolution of an arbitrary state is defined starting by the standard time-dependent Schroedinger equation:
\begin{equation}
-i\hbar\frac{\partial}{\partial t}|\Psi(t)\rangle = \hat{H}|\Psi(t)\rangle.
\end{equation}
It is possible to Wick rotate, and introduce an {\it imaginary time} $\tau=\frac{it}{\hbar}$. The 
time-dependent Schr\"odinger equation is transformed into an imaginary-time-dependent equation:
\begin{equation}
-\frac{\partial}{\partial \tau}|\Psi(\tau)\rangle = \hat{H}|\Psi(\tau)\rangle,
\end{equation}
where
$\tau$ is defined as an {\it inverse energy} that parametrizes the propagation of the quantum state.
The formal solution can be written using the imaginary time propagator
\begin{equation}
\vert\Psi(\tau)\rangle=e^{-\tau\hat{H}}\vert\Psi(0)\rangle
\end{equation}
It is possible to expand the initial state $\vert \Psi(0)\rangle$ in eigenstates $\vert\phi_n\rangle$ of the Hamiltonian itself, such that $\hat{H}\vert\phi_n\rangle=E_n\vert\phi_n\rangle$. The imaginary time propagation of $\vert\Psi(0)\rangle=\sum_nc_n \vert\phi_n\rangle$ becomes:
\begin{equation}
\vert\Psi(\tau)\rangle=e^{-\tau\hat{H}}\sum_n c_n\vert\phi_n\rangle=
\sum_n c_n e^{-\tau E_n}\vert\phi_n\rangle
\end{equation}   
Let us now consider the limit of the propagation for $\tau\rightarrow\infty$. The coefficients of the expansion $c_n e^{-\tau E_n}$ will either decrease (if $E_n>0$) or increase (if $E_n<0$) with the
imaginary time, but in the limit the coefficient corresponding to the ground state of $\hat{H}$, i.e.
$c_0 e^{-\tau E_0}$ will be dominant. This means that the imaginary time propagator has the interesting property of filtering out of an arbitrary state in the Hilbert space the ground state
of a given Hamiltonian, provided that the state is not orthogonal to the ground state to begin with.
We want to stress a very important point. The ground state we are referring to is the {\it mathematical} ground state of the Hamiltonian $\hat{H}$. The {\it physical} ground state needs to
take into account the symmetry of the particles, either bosons or fermions.
It is very easy to convince oneself that such mathematical ground state is always a nodeless function (i.e. it is zero nowhere but possibly on the boundaries of the domain of existence of the wavefunction expressed in some representation). This is because the propagator is a positive definite function, at least for a Hamiltonian of the standard form $\hat{H}=\hat{T}+\hat{V}$, where $\hat{T}$ is the kinetic energy of a system of free particles and $\hat{V}$ is a local potential. In this case
the eigenvector corresponding to the largest eigenvalue of the propagator is positive definite within the domain that defines the system. The largest eigenvalue of the propagator corresponds to the lowest eigenvalue of $\hat{H}$.

Notice that the imaginary time propagator is hermitian not unitary, and the normalization of the projected ground state is not guaranteed in general. By means of a small change in the propagator definition it is possible to guarantee the normalization of the projected ground state. In fact, let us define the propagator as:
\begin{equation}
\label{eq:gentauprop}
\vert\Psi(\tau)\rangle=e^{-\tau(H-E_0)}\vert\Psi(0)\rangle.
\end{equation}
It is easy to realize that in this case the amplitude of the component of the initial state along the ground state is preserved (while all other amplitudes decrease exponentially), and therefore the projected state is normalizable.

We will later discuss in detail the implications of these properties as concerns the application of imaginary-time propagation to many-fermion systems. 
\subsection{Imaginary time propagator in coordinate representation}
\label{sec:dmccoord}
We will focus on a practical implementation of imaginary time propagation, and we will limit ourselves to a system of bosons (or Boltzmannions) which do admit a ground-state wavefunction that is positive definite. We will also consider Hamiltonians of the form mentioned in the previous subsection, in which the interaction is local.
In this case the propagator is easily represented in coordinates. Formally we would have:
\begin{equation}
\langle R \vert\Psi(\tau)\rangle=\int dR' \langle R\vert e^{-\tau(H-E_0)}\vert R'\rangle\langle R'\vert\Psi(0)\rangle,\label{general_diff}
\end{equation}
where we have inserted a complete set of position eigenstates.The propagator
\begin{equation}
\langle R \vert e^{-\tau(\hat{H}-E_0)}\vert R'\rangle
\end{equation}
seems to be still quite difficult to evaluate. However, let us break up the imaginary time interval
$\tau$ in two equal intervals $\tau/2$. We can write
\begin{equation}
\langle R \vert e^{-\tau(\hat{H}-E_0)}\vert R'\rangle=\langle R \vert e^{-\frac{\tau}{2}(\hat{H}-E_0)}e^{-\frac{\tau}{2}(\hat{H}-E_0)}\vert R'\rangle,
\end{equation}
since $\hat{H}$ obviously commutes with itself. Inserting a complete set we obtain:
 \begin{equation}
\langle R \vert e^{-\tau(\hat{H}-E_0)}\vert R'\rangle=\int dR''\langle R \vert e^{-\frac{\tau}{2}(\hat{H}-E_0)}\vert R''\rangle\langle R''\vert e^{-\frac{\tau}{2}(\hat{H}-E_0)}\vert R'\rangle,
\end{equation}
This process can be iterated for an arbitrary large number of times $M$:
\begin{equation}
\langle R \vert e^{-\tau(\hat{H}-E_0)}\vert R'\rangle=\int\cdots\int dR''\cdots dR^M\langle R \vert e^{-\frac{\tau}{2}(\hat{H}-E_0)}\vert R''\rangle\cdots\langle R^M\vert e^{-\frac{\tau}{2}(\hat{H}-E_0)}\vert R'\rangle.\label{path}
\end{equation}
Each of the factors in the integrand corresponds to a propagation for a {\it short} imaginary time $\Delta\tau=\tau/M$.
In this case we can split the propagator using the Trotter-Suzuki formula:
\begin{equation}
e^{-\frac{\Delta\tau}{2}(\hat{H}-E_0)}\sim e^{-\frac{\Delta\tau}{2}(\hat{V}-E_0)}e^{-\Delta\tau\hat{T}}e^{-\frac{\Delta\tau}{2}(\hat{V}-E_0)}+o(\Delta\tau^3)
\end{equation}
The representation in coordinates of each factor is known. The factors containing the potential, under the hypotheses made, are diagonal in the coordinates themselves, and simply become:
\begin{equation}
e^{-\frac{\Delta\tau}{2}(\hat{V}-E_0)}\vert R\rangle = |R\rangle e^{-\frac{\Delta\tau}{2}(V(R)-E_0)},
\end{equation}
while the kinetic term is the propagator of a set of $A$ free particles obeying the equation:
\begin{equation}
-\frac{\partial}{\partial\tau}\Psi(R,t)=-\frac{\hbar^2}{2m}\nabla^2\Psi(R,t)\label{diffusion_eq}
\end{equation}
This is a classical {\it free diffusion} equation. If we interpret $\Psi(R,t)$ as a the density of the $A$ particles, its evolution in time will be given by the well known diffusion law:
\begin{equation}
\Psi(R,t)=\frac{1}{(2\pi\frac{\hbar^2}{m}\Delta\tau)^\frac{3A}{2}}\int dR'e^{-\frac{(R-R')^2}{2\frac{\hbar^2}{m}\Delta\tau}}\Psi(R',0).\label{free_propagator}
\end{equation}
The short-time approximation for the propagator, correct at order $\Delta\tau$ , will then become:
\begin{equation}
\langle R\vert e^{-\frac{\Delta\tau}{2}(\hat{H}-E_0)}\vert R'\rangle\sim\frac{1}{(2\pi\frac{\hbar^2}{m}\Delta\tau)^\frac{3A}{2}}e^{-\frac{\Delta\tau}{2}(V(R)-E_0)} e^{-\frac{(R-R')^2}{2\frac{\hbar^2}{m}\Delta\tau}}e^{-\frac{\Delta\tau}{2}(V(R)-E_0)}.\label{propagator}
\end{equation}

At this point it is possible to proceed in different ways. By substituting Eq.(\ref{propagator}) in Eq.(\ref{path}), one obtains an integral in which the integrand is a function of $M$ replicas of the coordinates of the particles in the system. The ground-state expectation value of an operator that is a function of the coordinates can then  be computed using on the left and on the right the imaginary time propagation started from an arbitrary state $\Psi(R,0)$. The resulting expression is:
\begin{eqnarray}
\begin{array}{c}
\langle \phi_0|O(R)|\phi_0\rangle = 
\displaystyle\lim_{\tau\rightarrow\infty}\langle \Psi(R,\tau)|O(R)|\Psi(R,\tau)\rangle\sim\left(\frac{1}{(2\pi\frac{\hbar^2}{m}\Delta\tau)^\frac{3A}{2}}\right)^M\times\\ \\
\times\displaystyle \int\int\cdots\int dR\;dR'\cdots dR^M\Psi(R,0)e^{-\frac{\Delta\tau}{2}(V(R)-E_0)} e^{-\frac{(R-R')^2}{2\frac{\hbar^2}{m}\Delta\tau}}e^{-\Delta\tau(V(R')-E_0)}
\cdots O(R^{M/2}) \\ \\
\displaystyle
\cdots e^{-{\Delta\tau}(V(R^{M-1})-E_0)} e^{-\frac{(R^{M-1}-R^M)^2}{2\frac{\hbar^2}{m}\Delta\tau}}e^{-\frac{\Delta\tau}{2}(V(R^M)-E_0)}\Psi(R^M,0)
\end{array}
\end{eqnarray}	
This expression is reminiscent of a path-integral formulation of the problem. The integral can in principle be computed by means of a  Metropolis-like algorithm, and gives the ground-state
expectation of an arbitrary observable, provided that the number of slices $M$ used is  large enough to guarantee a correct filtering of the ground state. This method is known as Path Integral Ground State Monte Carlo (PIGS-MC)~\cite{Sarsa00}.

However, there is a simpler way to implement the imaginary time propagation. Let us expand the initial state from which we want to project the ground state in eigenstates of the position:
\begin{equation}
|\Psi\rangle \simeq \sum_i\Psi(R_i)|R_i\rangle
\end{equation} 
We will call each of these points in coordinate space a {\it walker}, and we will refer to the whole ensemble of points as to the {\it population} of walkers. If we apply the short-time propagator to each walker, it is easy to understand its effect. We will call the application of the short-time propagator to the walker population an {\it imaginary time step} (or simply a {\it time step}). Each time step originates a new {\it generation} of walkers.

The Gaussian factor in the propagator tells us the probability that a walker positioned in $R'$ is displaced to a new position $R$. Since the probability density is a Gaussian of variance
$\sigma^2=\frac{\hbar^2}{m}\Delta\tau$, the RMS displacement will be proportional to $\sqrt{\Delta\tau}$ times a constant, which plays the role of a {\it diffusion constant} $D$, equal to $\frac{\hbar^2}{m}$. For each coordinate of each particle we need to extract a random number
$\eta$ distributed as:
\begin{equation}
\label{eq:gaussprop}
P[\eta]=\frac{1}{\sqrt{2\pi D \Delta\tau}}e^{-\frac{\eta^2}{2D\Delta\tau}}
\end{equation}
and add it to to the original coordinate. 
\begin{svgraybox}
	\begin{algorithmic} 
		\State{DMC\_Move()}
		\For{$i \in \{0,A-1\}$}  
		\For{$j \in \{0,2\}$ }
		\State{$R_{new}[i][j]\gets R[i][j]+D\Delta\tau[\text{rgaus}()]$}
		\EndFor
		\EndFor
	\end{algorithmic}
\end{svgraybox}
The function $\text{rgaus}()$ generating normally-distributed random numbers is now universally available as a library routine, but it can easily be implemented starting from a uniform distribution by using the Box-Muller formula. 
The part of the propagator depending on the potential has a slightly different interpretation. In the classical analogy we could say that the factor $W=e^{-\frac{\Delta\tau}{2}(V(R^M)-E_0)}$ represents 
the probability of a process to occur by which new points might be created in the time interval $\Delta\tau$ (if $W>1$) or destroyed (if $W<1$), or in other words, a process related to the presence of a source or a sink of walkers. $W$ is interpreted as the average number of walkers that this
process would generate over time at the position $R$. As we will later see, this creation/absorption (or {\it branching}) process is related to the fact that the normalization of the propagated state is not preserved.
Since we cannot work with a non-integer number of walkers, we can use the following strategy
\begin{enumerate}
	\item
	use the quantity $W$ as a weight for the contribution to the estimates from the walker at a given position. Since in the short-time propagator we have two such factors, one from the initial position and on from te final position of the walker, we can use the product of the two as the total weight:
	\begin{equation}
	\label{eq:propw}
	W=\exp\left\{ -\Delta\tau\left[\frac{V(R)+V(R')}{2}-E_0\right]\right\}
	\end{equation}
	Estimates will be integrals of the form $\langle O\rangle=\int \phi_0(R) O(R) dR$,
	and they can be computed as:
	\begin{equation}
	   \langle O\rangle =\frac{  \sum_l^{N_{wk}}W_{kl}O(R_{kl})}
	   	{\sum_l^{N_{wk}}W_{kl}},\label{dmc_averages}
	\end{equation} 
where $N_{wk}$ is the number of walkers in a given generation. We will discuss later the specific form of the function $O$ for interesting cases.
\item
In order to generate a number of points that is correct on average, we can sample $N_{mult}$, the number of points to be generated for the next generation, in the following way:
\begin{equation}
N_{mult}=\text{int}(W+\xi),
\end{equation}
where $\text{int}()$ is the function truncating the argument to an integer, and $\xi$ is a random number in $[0,1)$. $N_{mult}$ could be $\geq 1$, in which case the next generation will contain
$N{mult}$ copies of the walker, or $0$, in which case the walker is suppressed. 
\end{enumerate}	
The projection of the ground state will be achieved when propagating for a sufficiently long imaginary time. This means that we need to evolve the population of walkers for a large number of time steps, and eventually we will sample a density of points with a distribution {\it proportional} to the ground-state wavefunction. In the initial stage of the run, the energy and other estimators will have a value that is still strongly biased by the initial state. This means that the initial part of the propagation should be excluded from the averages. There is no automatic recipe to choose how much of the walk should be discarded. Usually it is convenient to monitor some observable (typically energy) and try to see where its value stops having a systematic trend as a function of the imaginary time. 

How is the constant $E_0$ fixed? In principle it should be equal to the ground-state energy. This would mean that we need to know the solution of the problem... before solving it! In practice it is not strictly necessary to use the exact value of $E_0$, but it is sufficient to use a realistic variational estimate. The value of $E_0$ can also be used to reduce the fluctuations in the population of walkers due to the branching process, at the cost of introducing additional bias.
For example, it is possible to modify the weight of a given configuration in the following way:
\begin{equation}
\tilde{W}=\frac{N_{t}}{N_g}\exp\left\{ -\Delta\tau\left[\frac{V(R)+V(R')}{2}-E_0\right]\right\}
\end{equation}
where $N_t$ is a "target" number of walkers in the population and $N_g$ is the number of walkers in the current generation. This modified weight reacts to the variations of the population, increasing or decreasing the weight depending on whether $N_g$ is smaller or larger than $N_t$, respectively.
This modification obviously introduces a bias in the results, since it modifies the propagator. However, this bias will be linearly decreasing with the time-step $\Delta\tau$. The weight can  also be rewritten as:
\begin{equation}
\tilde{W}=\exp\left\{ -\Delta\tau\left[\frac{V(R)+V(R')}{2}-\tilde{E}\right]\right\},\label{weight2}
\end{equation}
where
\begin{equation}
\tilde{E}=E_0+\frac{1}{\Delta\tau}\log\left(\frac{N_t}{N_g}\right)
\end{equation}
Therefore, at each generation the constant can be modified to keep the size of the population under control. 

The weight can also be used to estimate the energy. In fact if we take the logarithm of both members of Eq. (\ref{weight2}) we obtain:
\begin{equation}
\log{\tilde{W}}=-\Delta\tau\left[\frac{V(R)+V(R')}{2}-\tilde{E}\right]
\end{equation}
from which we obtain:
\begin{equation}
E_0=\frac{1}{\Delta\tau}\log\left(\frac{N_g\tilde{W}}{N_t}\right)+\frac{V(R)+V(R')}{2}.
\end{equation}
This is the so-called {\it growth energy} estimator, and it can be used in principle to evaluate the
ground-state eigenvalue.

A simpler way of evaluating the energy is to use a test function $\Psi_T(R)$. In this case the idea is
to evaluate the following matrix element:
\begin{equation}
\langle E\rangle =\frac{\langle \phi_0|\hat{H}|\Psi_T\rangle}{\langle \phi_0|\Psi_T\rangle}=\frac{\int dR \phi_0(R)\hat{H}\Psi_T(R)}{\int dR \phi_0(R)\Psi_T(R)}
\end{equation}
Both numerator and denominator integrals are suitable for Monte Carlo evaluation.
The probability density that we sample is $\phi_0(R)$, and the functions to be cumulated following the recipe in Eq. (\ref{dmc_averages}) are $\hat{H}\Psi_T(R)$ and $\Psi_T(R)$. The latter is necessary whenever $\Psi_T(R)$ is not normalized. We will then have
\begin{equation}
\label{Eq._dmc_ave}
\langle E\rangle=\frac{\langle \hat{H}\Psi_T\rangle}{\langle \Psi_T\rangle}.
\end{equation}
However, due to the hermiticity of the hamiltonian, one has:
\begin{equation}
\langle E\rangle =\frac{\langle \phi_0|\hat{H}|\Psi_T\rangle}{\langle \phi_0|\Psi_T\rangle}=\frac{\langle \Psi_T|\hat{H}|\phi_0\rangle}{\langle \phi_0|\Psi_T\rangle}=E_0,
\end{equation}
independent of the choice of $\Psi_T(R)$. This is the most practical way to evaluate the energy eigenvalue and its standard deviation. Other observables can be evaluated in a similar way. However the results will always depend on the choice of the test function. We will discuss this aspect later.

A last important remark remains to be made. In devising the algorithm we are making some approximations. First of all the imaginary time propagator is not exact, but is correct only at order $\Delta\tau^2$. This means that for any finite imaginary time step value, the answer will be biased of an amount proportional to $\Delta\tau^2$. The same holds for the population size whenever one wants to apply population control as described above. For any finite target population $N_t$ there will be a bias on the answer of order $1/N_t$. These biases can be corrected by performing several simulations with different values of $\Delta\tau$ and $N_t$, and then extrapolating to $\Delta\tau\rightarrow 0$ and $1/N_t\rightarrow 0$. As we will show in the last part of this chapter, methods exist to completely eliminate the time step bias. However, it is possible to reduce the bias with some minor modifications in the propagator and by introducing an acceptance/rejection mechanism (cite CYRUS TIME STEP).
\subsection{Application to the harmonic oscillator}

A very simple illustration of the sense of the algorithm can be made by implementing to the one-dimensional harmonic oscillator. We consider the Hamiltonian:
\begin{equation}
\hat{H}=-\frac{1}{2}\frac{\partial^2}{\partial x^2}+\frac{1}{2}x^2
\end{equation}
\begin{figure}
	\begin{center}
		\includegraphics[scale=0.5]{Chapter9-figures/hists.eps}
	\end{center}
	\caption{Histogram of the walker population after $N$ DMC imaginary time steps for the harmonic oscillator Hamiltonian described in text. Here we used $\Delta\tau=10^{-3}$, with a target population of 4000 walkers.}
	\label{fig.hist}
\end{figure}
The ground-state eigenvalue is $E_0=\tfrac{1}{2}$ and the ground-state eigenfunction is the Gaussian $\Psi_0(x)=\frac{1}{\pi^{1/4}}e^{-\frac{x^2}{2}}$. As we have illustrated in the previous section, the propagation can start from any distribution of points with a density not orthogonal to the ground state. A very simple choice in this case is a constant. In Fig.\ref{fig.hist} we can see how the histogram of the walkers evolves as a function of the imaginary time applying the algorithm described in the previous section, including population control. The initial uniform distribution of walkers in the interval $[-6,6]$ is transformed into the correct Gaussian density. The mechanism that leads to this result is easy to understand. Any walker finding itself after diffusion in a region where the potential is larger than the eigenvalue will tend to be suppressed, while walkers near the origin will tend to multiply themselves. This will result in a histogram peaked at the origin ad decaying fast to zero when moving away from it.
\begin{figure}
	\begin{center}
		\includegraphics[scale=0.5]{Chapter9-figures/dmc_decay.eps}
	\end{center}
	\caption{Logarithm of the estimated energy averaged over a single generation as a function of the imaginary time in a run with a population target of 4000 walkers, and with an imaginary time step $\Delta\tau=10^{-4}$}
	\label{fig.decay}
\end{figure}


In order to estimate the energy we need a test function. An approximation to the ground state might be given by the function:
\begin{equation}
\Psi_T(x)=\frac{1}{1+x^2}.
\end{equation}
We can therefore estimate the energy by means of the following quotient (see Eq. (\ref{Eq._dmc_ave})):
\begin{equation}
\langle E\rangle = \frac{\sum_{i} w(x_i)\frac{1-3x_i^2}{(1+x_i^2)^3}+\frac{1}{2}\frac{x_i^2}{1+x_i^2}}{\sum_iw(x_i)\frac{1}{1+x_i^2}},
\end{equation}
where the sums runs first over all the generations (i.e. the imaginary time steps performed) and then over all the walkers belonging to a given generation.

In Fig. \ref{fig.decay} we show the logarithm of the energy estimator averaged over each single generation as a function of the imaginary time. As we would expect from the general behavior of the coefficients of the excited states as function of the imaginary time, we see a clear exponential decay of the energy towards the exact eigenvalue. The figure clearly shows how the transient is not made up of a single exponential. The initial state needs includes a large number of excited states, that all need to be projected out before reaching the ground state.
\begin{figure}
	\begin{center}
		\includegraphics[scale=0.5]{Chapter9-figures/walkers_dmc.eps}
	\end{center}
	\caption{Typical fluctuations of the walker population in a DMC run for the one dimensional harmonic oscillator. The target population in this case is 2000. The imaginary time step is
		set to $\Delta\tau=0.075$. }
	\label{fig.walkers}
\end{figure}
In Fig. \ref{fig.walkers}the typical behavior of the fluctuation in the walker number is reported. In the specific case the time step was set to $\Delta\tau=0.3$. Nevertheless, the walker number never departs from the target by more than 3\%. This is the effect of the population control procedure described in the previous subsection. Unfortunately population control alone is not sufficient to guarantee a stable calculation. In presence of particle-particle interactions that diverge at the origin fluctuations in the number of walkers become extremely wide. This is the reason why it is necessary to introduce the so-called importance sampling, that we will discuss in a later section.
\begin{figure}
	\begin{center}
		\includegraphics[scale=0.5]{Chapter9-figures/dmc_extr.eps}
	\end{center}
	\caption{TIllustration of the imaginary time-step extrapolation procedure. The energy is computed for different values of $\Delta\tau$, and the results are fitted with a linear function. The intercept will give the correct prediction for the eigenvalue. Notice that the results should still be extrapolated for an infinite population. Here we use a target number of walkers equal to 2000, and the runs consist of $10^5$ generations each. Errorbars refer to one standard deviation. The plotted value for $\Delta\tau=0$ and the corresponding errorbar are obtained from the linear fit of the data. }
	\label{fig.extr}
\end{figure}

Finally, in Fig. \ref{fig.extr} we show one of the points discussed in the previous section, that is the bias of the result due to the finite imaginary time step. The difference between the energy estimate and the exact eigenvalue is plotted as a function of $\Delta\tau$ for a target population of 2000 walkers and a total of $10^5$ generations for each value of $\Delta\tau$. The observed bias is quite small, but well outside of the statistical error. The dependence on $\Delta\tau$ is quadratic, as expected from the analysis of the propagator. Interpolating the data with a function of the form $E=E_0+\alpha(\Delta\tau)^2$ we predict $E_0$ to be  $(-3\pm1)\times 10^{-5}$. As it can be seen there is still a small residual bias due to the finitness of the population. Further extrapolation would be needed to recover the exact answer. 
 
\subsection{Importance sampling}
The simple diffusion algorithm we have illustrated above suffers of a substantial deficiency when particles interact with a potential having a repulsive or attractive core. Since the free particle diffusion propagator does not have any information about the potential, particles have no restrictions to come close to each other. This means that the weights will suffer of large fluctuations whenever a pair of particles find themselves at short distance. The consequent fluctuations in the population make the computation unmanageable.  

The use of an importance function to guide the diffusion process  \cite{Anderson76}  was the key to make  Diffusion Monte Carlo algorithms usable. The idea is to give up on the request of sampling the ground-state wavefunction, and rather try to sample a distribution that, asymptotically in imaginary time, is the product of the ground-state wavefunction and of a known function that is the best possible approximation to the ground state obtained, for instance, by means of a variational calculation. We will call this function $\Psi_G$. Starting from Eq.(\ref{general_diff}) we can multiply both sides by $\Psi_G$ and obtain:
	\begin{equation}
	\Psi_G(R)\Psi(R,\Delta\tau)=\int dR' G'(R',R,\Delta\tau) \Psi_G(R)\Psi(R',0),\label{is1}
	\end{equation}
where we have defined:
\begin{equation}
G'(R,R',\Delta\tau)=\frac{1}{(2\pi\frac{\hbar^2}{m}\Delta\tau)^\frac{3A}{2}}e^{-\Delta\tau(V(R')-E_0)} e^{-\frac{(R'-R)^2}{2\frac{\hbar^2}{m}\Delta\tau}}.
\end{equation}
Since all the expressions we have written are correct at order $\Delta\tau$, for our purposes we can assume the equivalence of $G'$ and $G$. 
We can multiply and divide the  integrand in Eq.(\ref{is1}) by $\Psi_G(R')$ to obtain:
\begin{equation}
\Psi_G(R)\Psi(R,\Delta\tau)=\int dR' G'(R',R,\Delta\tau) \frac{\Psi_G(R)}{\Psi_G(R')}\Psi_G(R')\Psi(R',0),\label{is2}.
\end{equation}
In Eq. (\ref{is2}) we can identify a new walker density to be propagated, namely:
\begin{equation}
f(R,\tau)=\Psi_G(R)\Psi(R,\tau),
\end{equation}
and the corresponding propagator:
\begin{equation}
\tilde{G}(R,R',\Delta\tau)=G'(R,R',\tau)\frac{\Psi_G(R)}{\Psi_G(R')}.
\end{equation}
The quotient of the wavefunctions can be included in the weight, and provides a correction that prevents the walkers to excessively multiply or die near the divergent points of the potential
This point is better illustrated considering the short time limit it is possible to expand the ratio of the guiding functions. At first order in $\Delta\tau$ the result is:
\begin{equation}
\tilde{G}(R,R',\Delta\tau)\simeq G_0(R,R',\tau)\left[1+\frac{\nabla\Psi_G(R')}{\Psi_G(R')}(R-R')+\cdots\right]
\end{equation}
At the same order we can regard the terms in bracket as the expansion of an exponential and write:
 \begin{equation}
 \tilde{G}(R,R',\Delta\tau)\simeq G_0(R,R',\tau)e^{\frac{\nabla\Psi_G(R')}
 	{\Psi_G(R')}(R-R')}
 \end{equation}
This can be combined with the Gaussian factor in $G_0$, and by completing the square (which introduces a term at order $\Delta\tau2$), the propagator is modified as follows:
\begin{equation}
\tilde{G}(R,R',\Delta\tau)\simeq\frac{1}{(2\pi\frac{\hbar^2}{m}\Delta\tau)^\frac{3A}{2}}e^{-\frac{\Delta\tau}{2}(V(R')-E_0)} e^{-\frac{(R-R'-\frac{\hbar^2}{m}\Delta\tau\frac{\nabla\Psi_G(R')}{\Psi_G(R')})^2}{2\frac{\hbar^2}{m}\Delta\tau}}e^{-\frac{\Delta\tau}{2}(V(R)-E_0)}.
\end{equation}
The same expansion can be performed to compute the {\it change in normalization} of the propagated density after a time step. The change in normalization is given by:
\begin{equation}
{\cal N}=\int dR \tilde{G}(R,R',\tau),\label{norm}
\end{equation}
i.e. the total weight of the final points $R$ that can be reached starting from $R'$.
Once more we can expand the ratio of the guiding functions in the propagator, but this time up to second order:
\begin{equation}
\tilde{G}(R,R',\Delta\tau)\simeq G_0(R,R',\tau)\left[1+\frac{\nabla\Psi_G(R')}{\Psi_G(R')}(R-R')+\frac{1}{2}\frac{\partial_{i\alpha}
\partial_{j\beta}\Psi_G(R')(R-R')_{i\alpha}(R-R')_{j\beta}}{\Psi_G(R')}+\cdots\right]
\end{equation}
Inserting the previous equation in Eq.(\ref{norm}) we can see that after integrating over $R$ the terms containing odd powers of $(R-R')$ disappear by parity. We are therefore left with:
\begin{equation}
{\cal N}=e^{-\Delta\tau[V(R')-E_0]}\left[1+\frac{1}{2}\frac{\nabla^2\Psi_G(R')}{\Psi_G(R')}\frac{\hbar^2}{m}\Delta\tau+\cdots\right]
\end{equation}
We can now use the same trick used above to write the expression in square parenthesis as an exponential. The result is:
\begin{equation}
{\cal N}= \exp\left[-\Delta\tau\left(V(R')-\frac{\hbar^2}{2m}\frac{\nabla^2\Psi_G(R')}{\Psi_G(R')}-E_0z\right)\right]
\end{equation}
In the previous expression it is possible to immediately recognize the local energy. In fact, when using importance sampling, the normalization assumes the expression:
\begin{equation}
{\cal N}= \exp\left[-\Delta\tau\left(\frac{\hat{H}\Psi_G(R')}{\Psi_G(R')}-E_0\right)\right]
\end{equation}
This is the new form of the weight factor that one needs to compute in order to determine the multiplicity of the walker at a given position. It is immediately clear that the fact that in the exponential we have the difference between the local energy, instead of the potential energy, and the reference eigenvalue $E_0$ essentially resolves the issue related to the fluctuations of the population related to a divergent behavior of the interaction. In fact, if we knew the exact solution the exponent would be identically zero, and the population would be absolutely stable. However, by means of an accurate variational calculation it is possible to obtain a very good approximation of the ground-state wavefunction, thereby reducing the fluctuations in the population to a minimum. 

The algorithm including impotance sampling is modified in the following way.
\begin{enumerate}
\item
   For each walker, and for each coordinate perform a "drift" move along the
   gradient of the guiding function. This displacement is deterministic.
\begin{svgraybox}
 	\begin{algorithmic} 
 		\State{DMC\_Drift()}
 		\For{$i \in \{0,A-1\}$}  
 		\For{$j \in \{0,2\}$ }
 		\State{$R_{drift}[i][j]\gets R[i][j]+\frac{\nabla\Psi_G(R)}{\Psi_G(R)}\lvert_{[i][j]} D\Delta\tau$}
 		\EndFor
 		\EndFor
 	\end{algorithmic}
\end{svgraybox}
\item
Cycle again over coordinates and diffuse the position from $R_{drift}$ as in the non-importance sampled case.
\item
Compute the new multiplicity of the walker and the weight to assign to estimators using
\begin{equation}
W=\exp\left[-\Delta\tau\left(\frac{\hat{H}\Psi_G(R')}{\Psi_G(R')}-E_0\right)\right]
\end{equation}
\end{enumerate}
In this way the walkers will asymptotically sample the distribution:
\begin{equation}
f(R)=\Psi_G(R)\phi_0(R).
\end{equation}
This means that it is possible to evaluate integrals of the form:
\begin{equation}
\langle O\rangle=\frac{\int dR f(R) O(R)}{\int dR f(R)}.
\end{equation}
As in the previous case the evaluation of the exact energy eigenvalue can be easily obtained by using the local energy. In fact, the matrix element of the Hamiltonian between the guiding function\footnote{It is always possible to project the energy from a function $\Psi_T$ other than $\Psi_G$, by introducing a further weighing factor $\frac{\Psi_T}{\Psi_G}$. However this is very rarely used in standard applications.}  and the ground-state wavefunction is:  
\begin{equation}
\langle E\rangle =\frac{\langle \phi_0|\hat{H}|\Psi_G\rangle}{\langle \phi_0|\Psi_G\rangle}=\frac{\int dR f(R)\frac{\hat{H}\Psi_G(R)}{\Psi_G(R)}}{\int dR f(R)}
\end{equation}
Once more, because of the hermiticity of the Hamiltonian we will have that $\langle E\rangle =E_0$. All other estimators will be matrix elements of the operator between $\Psi_G$ and $\phi_0$. 

\subsection{The fermion sign problem}
\label{sec:signprob}
As we have mentioned, imaginary time propagation projects out of an arbitrary initial state the absolute (mathematical) ground state of a given Hamiltonian $\hat{H}$, which is always a nodeless function. One might correctly object that if the initial state is chosen in such a way not to have any overlap with this ground state, the projection will correctly give back some excited state of $\hat{H}$. More rigorously, if our initial state has components only within a certain subspace of the total Hilbert space, which could be selected, for instance, by the wavefunction symmetry, then imaginary time propagation will end up projecting out the eigenstate with lowest eigenvalue within that given subspace.

This seems to be particularly useful when thinking of applying DMC-like algorithm to the study of many-fermion systems, as the nuclear systems we are interested in. The antisymmetry property of the fermionic ground state suggests that it should be sufficient to start from an arbitrary antisymmetric state $|\Psi_A\rangle$(provided it is not orthogonal to the fermion ground state) to obtain the sought solution. In fact, one might speculate that antisymmetry itself would guarantee that there is no overlap with the symmetric ground state since the beginning:
\begin{eqnarray}
\begin{array}{c}
\displaystyle\lim_{\tau\rightarrow\infty} e^{-\tau(\hat{H}-E_0^A)}|\Psi_A(0)\rangle=
\sum_n  e^{-\tau(E_n-E_0^A)}\langle \phi_n|\Psi_A\rangle|\phi_n\rangle=\\ \\
=\langle \phi_0^A|\Psi_A\rangle|\phi_0^A\rangle+\displaystyle\lim_{\tau\rightarrow\infty}\langle \phi_0|\Psi_A\rangle|\phi_0\rangle e^{-\tau(E_0-E_0^A)}
\end{array}
\end{eqnarray}
However, this abstract formulation forgets that eventually we need to {\it sample a probability density} in order to operate with a Monte Carlo integration, and any excited state will have a wavefunction changing sign somewhere, thereby breaking this requirement. If we had an exact knowledge of the {\it nodal surface} of the ground state (i.e. of the set of points such that $\phi_0^A(0)=0$), we could use an antisymmetric function $\Psi_G^A(R)$ having the same nodal surface, and obtain by importance function the required positive definite density to sample:
\begin{equation}
\langle E\rangle =\frac{\displaystyle\langle \phi_0^A|\hat{H}|\Psi_G^A\rangle}{\langle \phi_0^A|\Psi_G^A\rangle}=\frac{\displaystyle\int dR \phi_0^A(R)\Psi_G^A(R)\frac{\hat{H}\Psi_G^A(R)}{\displaystyle\Psi_G^A(R)}}{\displaystyle\int dR \phi_0^A(R)\Psi_G^A(R)}.
\end{equation}
If $\Psi_G^a(R)$ does not have the same nodal surface as $\phi_0^A(R)$, we are once again in trouble. 

We might have then the idea of separately sampling the positive and the negative part of the wave function. It is always possible to split an antisymmetric function as:
\begin{equation}
\psi^A(R)=\Psi^+(R)-\Psi^-(R),
\end{equation}
where both $\Psi^+$ and $\Psi^-$ are positive definite functions. It is easy to see that each one, by linearity, is a solution of the Schroedinger equation with the same eigenvalue as the fermionic ground state. We can call $|R^+\rangle$ the walkers sampling the positive part and $|R^-\rangle$ the walkers sampling the negative part of $\Psi^A$. The energy expectation could be computed as:
\begin{equation}
E_0^A=\frac{\displaystyle\int dR^+ f^{+}(R^+)\frac{\hat{H}\Psi_G^A(R^+)}{\Psi_G^A(R^+)}-\int dR^+ f^{-}(R^-)\frac{\hat{H}\Psi_G^A(R^-)}{\Psi_G^A(R^-)}}{\displaystyle
 \int dR^+ f^+(R^+)\Psi_G^A(R^+)-\int dR^+ f^-(R^-)\Psi_G^A(R^-)},\label{signedaverage}
\end{equation}
where $f^\pm$, as above, has the meaning of the importance sampled density of walkers. However, once more we have to notice that since both $f^+$ and $f^-$ will obey the same imaginary time Schroedinger equation, the two densities will both converge to the ground-state density for $\hat{H}$. This means that both the numerator and the denominator of Eq.(\ref{signedaverage}) will tend to 0 in the limit $\tau\rightarrow\infty$, and the ratio becomes undetermined.
The major effect that one can observe during the calculation is that the variance of the energy will become exponentially large, and the integral will be dominated by statistical noise. This is the so called {\it fermion sign problem}. For some authors there is a prove that the computation of estimates such as Eq.(\ref{signedaverage}) is an NP complex problem~\cite{Troyer05}, and a solution will always require computer time that is exponentially increasing with the dimension of the system. However there are hints that by using methods that break this {\it plus/minus symmetry}, based on correlated dynamics and cancellation methods it is possible to reduce the cost to a polynomial dependence \cite{Kalos00,Assaraf07}.

\subsubsection{Fixed-node approximation}
\label{sec:fn}
A possible way of circumventing the sign problem in the case in which the antisymmetric ground-state wavefunction has to be real is to use some artificial boundary conditions~\cite{Ceperley80}.

We can define a nodal pocket $\Omega(R)$ as the set of points that can be reached from $R$ without crossing the nodal surface at any point. For a standard Hamiltonian we can expect that for any pair of points $R',R$ not on the nodal surface of the wavefunction, there exist a permutation $P$ of the coordinates such that $PR'\in \Omega(R)$. This in turn means that all the space (but for the nodal surface, which has zero measure) can be covered by summing over all the permutations of the points lying in a single nodal pocket $\Omega(R)$. This the so-called {\it tiling theorem}. The tiling theorem implies that the fermion ground-state eigenvalue of the Schroedinger equation solved inside any $\Omega(R)$ is the same as the eigenvalue of the problem solved on the whole space. 

The prove of the tiling theorem is quite simple. If the tiling property does not hold for the antisymmetric ground state $\phi^A_0(R)$, then $\sum_P\Omega(PR)$ will not completely cover the space, leaving out some regions. This means that somewhere there are two regions $\Omega(R_a)$ and $\Omega(R_b)$ that share part of the nodal surface and are not equivalent. It is then possible to construct a function with a lower eigenvalue in the region $\Omega(R_a)\bigcup\Omega(R_b)$ by simply removing the common node and solving for the ground state of $\hat{H}$ within that region. Let us call $\phi^0_{ab}$ this function. Constructing an antisymmetric function $\Psi_A(R)=\sum_P(-1)^P\phi^0_{ab}$ we will have an antisymmetric function with an eigenvalue lower than that of $\phi^A_0(R)$, thereby violating the assumption that $\phi_A^0$ is the antisymmetric ground state of $\hat{H}$.

By the same kind of construction it is also possible to prove that the solution of the Schroedinger equation within a given nodal pocket $\Omega(R)$ is always an upper bound of the true antisymmetric eigenvalue, and that the exact result is recovered if and only if the nodal surface of the wavefunction generated by replicating the pocket coincides with that of the exact eigenfunction.

The previous considerations suggest that solving  for the ground state of a given Hamiltonian within a nodal pocket $\Omega(R)$ will provide an upper bound of the energy of a many-fermion system, which can in principle can by improved by improving the nodal structure of the test function used to determine the boundary conditions. This is called the {\it fixed-node approximation}. In order to have zero density at the nodal surface we have to assume that at the border of the nodal pocket an infinite absorbing potential exists, such that walkers never cross that surface. From the point of view of the algorithm this introduces a very tiny modification in the code. We have to remember that we can solve for the ground state in {\it any} pocket. This means that we do not need to care either of the initial position of the walkers or of the associated sign of the wavefunction. We said that the fixed-node approximation corresponds to modify the Hamiltonian as follows
\begin{equation}
\hat{\tilde{H}}=\hat{H}+V_\Omega(R),
\end{equation}  
where
\begin{equation}
V_\Omega(R)=\left\{ 
\begin{array}{ll}
\infty \text{\ \ \ if\ \ \ }R\in S(\Omega)\\
0\text{\ \ \ \ otherwise}
\end{array}
 \right.
\end{equation}
 This means that every time the walker crosses the border of the nodal pocket $S(\Omega)$ its weight becomes zero, and the walker is simply canceled from the population. Fixed node calculations are presently very widely employed especially in quantum chemistry and solid state physics applications (for a review of applications to many electron systems see Ref. \cite{Foulkes01}). When the wavefunction needs to be complex it is no longer possible to define a nodal surface, and a different kind of approach has to be used. This will be discussed in the next section concerning the applications to the nuclear physics case.
 \section{Quantum Monte Carlo for Nuclear Hamiltonians in coordinate space}
 %$Id: afdmc.tex,v 1.7 2011/02/03 14:49:52 schmidt Exp$  
 %\subsection{Green's Function Monte Carlo}
 %
 %
 %The GFMC method works well for calculating the low lying states of nuclei
 %up to and including $^{12}$C. Its major problem is that the computational
 %costs scale exponentially with the number of particles, because of the
 %full spin isospin summations. Full spatial integrations
 %would also scale exponentially, and this problem is solved by Monte Carlo
 %sampling the positions. The obvious solution to the exponential scaling
 %of the spin-isospin sums is to perform them using Monte Carlo sampling
 %as well.
 %
 %Our goal is then to find a low variance method to sample the spin-isospin
 %degrees of freedom. The usual historical route would be to begin with a
 %variational Monte Carlo calculation with spin-isospin summations.
 %For example, using Eq. \ref{xxx}, and instead of summing over the spin
 %degrees of freedom, we could sample them using a Metropolis et al.\cite{Metropolis53}
 %method. For example, the energy expectation would be
 %\begin{eqnarray}
 %E_V &=&
 %\frac{\sum_{S} \int dR \langle \Psi_T|H|RS\rangle \langle RS|\Psi_T\rangle}
 %{\sum_{S} \int dR \langle \Psi_T|RS\rangle \langle RS|\Psi_T\rangle}
 %\nonumber\\
 %&=& \sum_S \int dR P(R,S) E_L(R,S)
 %\end{eqnarray}
 %where
 %\begin{eqnarray}
 %P(R,S) &= &
 %\frac{\langle \Psi_T|RS\rangle \langle RS|\Psi_T\rangle}
 %{\sum_{S} \int dR \langle \Psi_T|RS\rangle \langle RS|\Psi_T\rangle}
 %\nonumber\\
 %E_L(R,S) &=& \frac{\langle \Psi_T|H|RS\rangle}{\langle \Psi_T|RS\rangle} \,.
 %\end{eqnarray}
 %We could then sample the spin-isospin states. We could begin with any
 %given up/down and proton/neutron spin state for each particle that gives
 %a nonzero $\langle RS|\Psi_T\rangle$ value, and make Metropolis et al.
 %moves by flipping spins and exchanging isospins. With a high quality
 %trial function, the local energy $E_L(R,S)$ will have low variance, and
 %the method would give an accurate variational energy upperbound.
 %
 %The problem here is not in devising a Monte Carlo method or finding a way
 %of sampling the spin-isospin variables. The problem, instead, is in the
 %calculation of the trial function. The Calculation of
 %the good trial functions described in section \ref{xxx} scales exponentially
 %with particle number even for a single spin-isospin state. In fact the difference
 %in computational cost
 %of calculating the value of these trial functions for one spin-isospin state
 %and all off them is completely negligible.  The VMC and GFMC methods described
 %in section \ref{xxx} can therefore lower the variance without increasing the
 %computational complexity by summing over all of the spin-isospin states.
 %
 %This tells us that in order to have an algorithm that has polynomial
 %scaling with particle number, we require a basis that is complete or overcomplete,
 %a trial state, and the evaluation of the wave function given by the
 %overlap of our trial and basis states must be calculable in polynomial time.
 %
 %To date, we have used the basis given by the outer product of nucleon position
 %states, and the outer product of single nucleon spin-isospin spinor states.
 %That is an element of this overcomplete basis is given by specifying the
 %$3A$ cartesian coordinates for the $A$ nucleons, and specifying four complex
 %amplitudes for each nucleon to be in a $|p\uparrow$, $p\downarrow$, $n\uparrow$,
 %$n\downarrow$ spin-isospin state. A basis state is then
 %\begin{equation}
 %|R_n S_n\rangle =
 %|r_1 s_1\rangle \otimes |r_2 s_2 \rangle \dots \otimes |r_n s_n\rangle
 %\end{equation}
 %
 %Our trial functions must be antisymmetric under interchange. The only such
 %functions with polynomial scaling that we know how to write down are Slater
 %determinants or Pfaffians (BCS pairing functions), for example,
 %\begin{eqnarray}
 %\langle R S|\Phi_{\rm Slater}\rangle &= &
 %{\cal A} \left [
 %\langle \vec r_1 s_1|\phi_1\rangle 
 %\langle \vec r_2 s_2|\phi_2\rangle 
 %\dots
 %\langle \vec r_A s_A|\phi_n\rangle  \right ]
 %\nonumber\\
 %\langle R S|\Phi_{\rm BCS}\rangle &=&
 %{\cal A} \left [ \langle \vec r_1 s_1 \vec r_2 s_2|\psi_{\rm cooper}\rangle
 %\langle \vec r_3 s_3 \vec r_4 s_4|\psi_{\rm cooper}\rangle
 %\dots
 %\langle \vec r_{A-1} s_{A-1} \vec r_A s_A|\psi_{\rm cooper}\rangle
 %\right ]
 %\end{eqnarray}
 %or linear combinations of them. Operating on these with the product of
 %correlation operators, Eq. \ref{xxx}, again gives a state with
 %exponential scaling with nucleon number. For our AFDMC calculations, we
 %multiply these wave functions by a state-independent, or
 %central, Jastrow correlation. Calculations of the Slater determinants
 %and Pfaffians scale like $A^3$ when using standard dense matrix methods,
 %while the central Jastrow requires $A^2$ operations if its range is
 %the same order as the system size.
 %
 %These trial functions capture only the physics of the
 %gross shell structure of the nuclear
 %problem and the state independent part of the two-body interaction.
 %Devising trial functions that are both computationally efficient to
 %calculate and that capture the state-dependent two- and three-body
 %correlations that we know are important would greatly improve both
 %the statistical and systematic errors of quantum Monte Carlo methods
 %for nuclear problems.
 %
 %The trial wave functions above can be used for variational calculations.
 %However, the results are poor since the functions miss the
 %physics of the important tensor interactions. These functions can be
 %used as importance functions for AFDMC calculations where they have been
 %found adequate for this purpose in a variety of problems.
 
 \subsection{General Auxiliary Field Formalism}
 
 
 We begin by looking at the auxiliary field formalism without importance
 sampling.
 All such diffusion Monte Carlo methods can be formulated as
 \begin{eqnarray}
 \label{eq.afdmc}
 |\Psi(t+\Delta t)\rangle = \int dX P(X) T(X) |\Psi(t)\rangle
 \end{eqnarray}
 where $X$ is a set of variables which will become our auxiliary fields,
 $P(X)$ is a probability density,
 \begin{eqnarray}
 P(X) &\geq& 0
 \nonumber\\
 \int dX P(X) &=& 1\,,
 \end{eqnarray}
 and $T(X)$ is an operator that operates
 in the Hilbert space of $|\Psi(t)\rangle$.
 We are free to choose the variables $X$, the probability density $P(X)$,
 and the operator $T(X)$ subject only to the
 constraint that the integral gives the desired propagator
 \begin{equation}
 e^{-(H-E_T)\Delta t} = \int dX P(X) T(X) \,,
 \end{equation}
 at least in the limit that $\Delta t \rightarrow 0$.
 
 In diffusion Monte Carlo methods, we represent the state $|\Psi(t)\rangle$
 as a linear combination of basis states which obviously must span the
 Hilbert space. These can be a complete set. An example is the position
 eigenstates used for diffusion Monte Carlo for central potentials.
 They can also form an overcomplete set such as
 or the position and spin/isospin bases used in the nuclear GFMC
 method and the position and
 overcomplete outer product of single particle
 spinor basis used in AFDMC, or the
 overcomplete single particle bases used in auxiliarly field methods
 such as those developed by Zhang and coworkers. For either case, we
 can denote these basis states as possible ``walkers.'' We will denote
 one of these walker states
 as $|RS\rangle$ since we will be applying the method to
 systems where the basis is given by the positions of the particles, $R$,
 and a spinor for each spin-isospin of the particles, $S$.
 
 The state,
 $|\Psi(t)\rangle$, at
 time $t$ is represented in diffusion Monte Carlo methods as a
 linear combination of walker states
 \begin{equation}
 \label{eq.walkers}
 |\Psi(t)\rangle = \sum_{i=1}^{N_W} w_i |R_i S_i\rangle
 \end{equation}
 where $w_i$ is a coefficient, often called the weight, and $N_W$ is the
 number of walkers.
 
 The key ingredient to implementing a diffusion Monte Carlo method is
 to choose the walker basis and the operator $T(X)$ such that when
 $T(X)$ operates on a walker basis state, it gives one and only one
 new walker basis state. That is we want
 \begin{equation}
 \label{eq.afop}
 T(X) |R S\rangle = W(X,R,S) |R' S'\rangle
 \end{equation}
 where $|R' S'\rangle$ is normalized in the same way as $|RS\rangle$,
 and $W(X,R,S)$ is the change in the normalization from the propagation.
 
 Once we have arranged for Eq. \ref{eq.afop} to be true, we can implement
 the diffusion Monte Carlo by starting with $|\Psi(0)\rangle$ written, as
 in Eq. \ref{eq.walkers}, as any, not
 unreasonable, linear combination of walkers. For each walker, we sample
 $X$ values from $P(X)$, and use Eq. \ref{eq.afop} to propagate to a new
 walker $|R_i' S_i'\rangle$, with a new weight $w_i'$
 given by the proportionality
 constant of Eq. \ref{eq.afop} multiplied by
 the original weight $w_i$. We branch on the magnitude of the weight,
 so usually, after branching, $w_i'=1$, where we are ignoring the
 fermion sign or phase problem for now and assuming that all of the
 weights are greater than or equal to zero. We will deal with the
 fermion case below.
 
 \subsection{Operator expectations and importance sampling}
 \subsubsection{Mixed averages}
 \label{sec:mixav}
 Diffusion Monte Carlo methods efficiently calculate ground-state
 mixed averages
 \begin{equation}
 \label{eq:mixedobs}
 \bar O_{\rm mixed} = \frac{\langle \Psi_T | O |\Psi(t)\rangle}{\langle
 	\Psi_T|\Psi(t)\rangle}
 \end{equation}
 where $|\Psi_T\rangle $ is  trial state.
 If $O$ is the Hamiltonian, operating on $|\Psi(t)\rangle$ shows
 that the result
 is the ground-state energy for large $t$. For other operators, for which
 the ground state is not an eigenstate, either
 approximate extrapolation methods or forward walking or its equivalent
 must be used to extract the correct ground-state expection value.
 
 Given a set of walkers as in Eq. \ref{eq.walkers}, the mixed estimate can
 be calculated by
 \begin{equation}
 \bar O_{\rm mixed} \simeq
 \frac{ \sum_{i=1}^{N_w} w_i \langle \Psi_T|O|R_i S_i\rangle}
 { \sum_{i=1}^{N_w} w_i \langle \Psi_T|R_i S_i\rangle}
 \end{equation}
 where the right hand side differs from the correct result because
 of statistical errors from the sampling which decreases as $N_W^{-1/2}$,
 and possible population size bias which decreases as $N_W^{-1}$. Statistical
 errors can be minimized by reducing the variance through importance
 sampling.  Population
 bias also can be controlled with importance sampling, and, since it decays
 faster with population size, can be readily detected
 and removed by either taking larger numbers of walkers or extrapolation.
 
 Efficient Monte Carlo methods need to have low variance so that the
 statistical error bars can be made small. For our walker propagation,
 this means that we should sample new walkers not only corresponding to
 the weight they will receive from our algorithm, but with this weight
 multiplied by their expected survival probability. The imaginary time
 Schr\"odinger equation is self adjoint, so the optimum importance
 function is the desired function\cite{xxx2010}. Typically, a trial
 function that can be efficiently evaluated is determined variationally
 and used as an approximation to the optimum trial function. Usually
 this trial wave function is used as the importance
 function. Sometimes a different importance function is used, so we will
 write this more general case. 
 
 \subsubsection{Importance sampling}
 To add importance sampling, we arrange to sample our walkers from
 a new state which we call $|\Psi_I \Psi(t)\rangle$ such that
 \begin{equation}
 \label{eq.iswf}
 \langle R S |\Psi_I\Psi(t)\rangle
 = \langle \Psi_I|RS\rangle \langle RS|\Psi(t)\rangle
 \end{equation}
 so that
 \begin{equation}
 \label{eq.iswalker}
 |\Psi_I \Psi(t)\rangle = \sum_{i=1}^{N_w} w_i |R_iS_i\rangle
 \end{equation}
 An alternative way of looking at this is that
 the sampling probability for the walkers at $R_iS_i$ has been
 modified so that
 \begin{equation}
 \label{eq.imp}
 |\Psi(t)\rangle = \sum_{i=1}^{N_w} w_i \langle \Psi_I|R_iS_i\rangle^{-1}
 |R_i S_i\rangle \,.
 \end{equation}
 Calculating a mixed average now becomes
 \begin{equation}
 \bar O_{\rm mixed} = \frac{\sum_{i=1}^{N_w} w_i
 	\frac{\langle \Psi_T|R_iS_i\rangle}{\langle \Psi_I|R_iS_i\rangle}
 	\frac{\langle \Psi_T|O|R_iS_i\rangle}{\langle \Psi_T|R_iS_i\rangle}}
 {\sum_{i=1}^{N_w} w_i
 	\frac{\langle \Psi_T|R_iS_i\rangle}{\langle \Psi_I|R_iS_i\rangle}
 } \,.
 \end{equation}
 For the usual case where $|\Psi_I\rangle = |\Psi_T\rangle$, and
 $w_i = 1$, we have
 \begin{equation}
 \bar O_{\rm mixed} = \frac{1}{N_w} \sum_{i=1}^{N_w}
 \frac{\langle \Psi_T|O|R_iS_i\rangle}{\langle \Psi_T|R_iS_i\rangle} \,.
 \end{equation}
 
 
 We substitute Eqs. \ref{eq.iswf} and \ref{eq.iswalker} into Eq. \ref{eq.afdmc}
 \begin{eqnarray}
 |\Psi_I \Psi(t+\Delta t)\rangle &=& \sum_{i=1}^{N_w} w_i
 \int dX P(X)
 \frac{\langle \Psi_I|R_i'S_i'\rangle}{\langle \Psi_I|R_iS_i\rangle}
 T(X)|R_i S_i\rangle
 \nonumber\\
 &=&
 \sum_{i=1}^{N_w} w_i
 \int dX P(X)
 \frac{\langle \Psi_I|T(X)|R_iS_i\rangle}{\langle \Psi_I|R_iS_i\rangle}
 \frac{T(X)}{W(X,R_i,S_i)}|R_i S_i\rangle
 \end{eqnarray}
 where $|R_i'S_i'\rangle$ is defined as in Eq. \ref{eq.afop}.
 Notice that the operator $T(X)/W(X,R_i,S_i)$ operating on $|R_iS_i\rangle$
 gives a normalized walker. The additional weight of this walker is given by
 $P(X)\frac{\langle \Psi_I|T(X)|R_iS_i\rangle}{\langle \Psi_I|R_iS_i\rangle}$.
 We want to minimize fluctuations in this weight factor, and to do this
 we normalize it and sample from the normalized distribution. The
 normalization will be the weight.
 
 We write
 \begin{eqnarray}
 {\cal N} &=& \int dX P(X)\frac{\langle \Psi_I|T(X)|R_iS_i\rangle}
 {\langle \Psi_I|R_iS_i\rangle}
 \nonumber\\
 &=& 
 \frac{\langle \Psi_I|e^{-(H-E_T)\Delta t} |R_iS_i\rangle}
 {\langle \Psi_I|R_iS_i\rangle}
 \nonumber\\
 &=& e^{-(E_L(R_i,S_i) -E_T)\Delta t} + O(\Delta t^2)
 \end{eqnarray}
 where the local energy $E_L(R_i,S_i)$ is defined by
 \begin{eqnarray}
 E_L(R_i,S_i) &=& 
 \frac{\langle \Psi_I|H|R_iS_i\rangle}
 {\langle \Psi_I|R_iS_i\rangle}
 \end{eqnarray}
 and we now sample $X$ variables from the normalized distribution
 \begin{eqnarray}
 \tilde P(X) &=& {\cal N}^{-1}
 P(X)\frac{\langle \Psi_I|T(X)|R_iS_i\rangle}{\langle \Psi_I|R_iS_i\rangle}\,.
 \end{eqnarray}
 
 
 The importance sampled diffusion Monte Carlo in the auxiliary field
 formalism becomes
 \begin{equation}
 \label{eq.afdmcimp}
 |\Psi_I\Psi(t+\Delta t)\rangle = \sum_{i=1}^{N_w}
 w_i \int dX \tilde P(X) e^{-(E_L(R_i,S_i) -E_T)\Delta t}
 \frac{T(X)}{W(X,R_i,S_i)} |R_iS_i\rangle \,.
 \end{equation}
 We  propagate a walker by sampling an $X$ value from $\tilde P(X)$,
 we include the local energy expression in the weight, and
 construct the new normalized
 walker position and spin state as $W^{-1}(X,R_i,S_i)T(X)|R_iS_i\rangle$.
 In each of the equations above,
 the ratio of the wave function terms gives the walker weight.
 In Eq. \ref{eq.afdmcimp} these terms have been combined to give a
 weight that depends on the local energy expectation value.
 All of the expectation values and weights
 contain ratios of trial wave functions
 so that any normalization factor multiplying the $|R S\rangle$ cancels
 and any convenient normalization can be used. We can therefore drop
 the $W$ factors and normalize our walker kets at the end of a step.
 Typically just the 
 walker positions are stored and the walker spinors are normalized to
 have magnitude 1.
 
 \subsubsection{Importance sampling with a Hubbard-Stratonovich transformation}
 We often have Hamiltonians where
 the Hubbard-Stratonovich transformation
 \begin{equation}
 e^{\frac{O^2}{2}} = \frac{1}{\sqrt{2\pi}}
 \int_{-\infty}^\infty dx e^{-\frac{x^2}{2}} e^{x O}
 \end{equation}
 can be used
 to write a propagatator in the form of Eq. \ref{eq.afop}. Examples are
 writing the kinetic energy as an integral over translations, or writing
 terms like $\vec \sigma_i \cdot \vec \sigma_j =
 (\vec \sigma_i +\vec \sigma_j)^2 -6$ as an integral over
 spin rotations.
 
 Since we primarily use the Hubbard-Stratonovich transformation to define
 our auxiliary fields, it is useful to work out how importance sampling
 can be included within the short-time approximation for this particular
 case. We begin with a Hamiltonian that is quadratic in a set of $N_O$ operators
 (which for our nuclear problems will be momentum and spin-isospin operators)
 $O_n$,
 \begin{equation}
 \label{eq.sumofsquares}
 H = \frac{1}{2}\sum_{n=1}^{N_O} \lambda_n O_n^2
 \end{equation}
 so that the imaginary time propagator is
 \begin{eqnarray}
 e^{-H \Delta t} &=&  \int dx \frac{1}{(2\pi)^{N_O/2}}
 e^{-\frac{1}{2}\sum_{n=1}^{N_O} x_n^2}
 e^{-i\sum_{n=1}^{N_O} x_n \sqrt{\lambda_n \Delta t} O_n}
 \nonumber\\
 && + ~ O(\Delta t^2)
 \end{eqnarray}
 where the $\Delta t^2$ terms comes from the possible noncommutivity of the
 $O_n$.
 
 As before, we choose our walker basis and the operators $O_n$ such
 that operating on a walker,
 $|R S\rangle$,
 with a term sampled from the integrand, gives a result
 proportional to another walker
 \begin{equation}
 \label{eq.walkerprop}
 e^{-i \sum_{n=1}^{N_O} x_n\sqrt{\lambda \Delta t} O_n}|RS\rangle =
 W(\{x_n\},R,S) |R'S'\rangle
 \end{equation}
 where $\{x_n\}$ represents the set of sampled $x_n$ values.
 
 We now sample $\tilde P(X)$ which is
 \begin{eqnarray}
 \tilde P(X) &=&
 {\cal N}^{-1}
 e^{-\frac{1}{2} \sum_{n=1}^{N_O} x_n^2}
 \frac{\langle \Psi_T |e^{-i\sum_{n=1}^{N_O}
 		x_n \sqrt{\lambda_n \Delta t} O_n}|R S\rangle}
 {\langle \Psi_T|RS\rangle}
 \nonumber\\
 &=& {\cal N}^{-1}
 e^{-\frac{1}{2} \sum_{n=1}^{N_O} x_n^2}
 \left (1 -i\sum_{n=1}^{N_O} x_n \sqrt{\lambda_n \Delta t}
 \frac{\langle \Psi_T|O_n|R S\rangle}
 {\langle \Psi_T|R S\rangle}
 - \frac{1}{2}\sum_{n=1,m=1}^{N_O}
 x_n x_m\sqrt{\lambda_m \lambda_n} \Delta t
 \frac{\langle \Psi_T|O_nO_m|R S\rangle}
 {\langle \Psi_T|R S\rangle}
 + ... \right )
 \nonumber\\
 \end{eqnarray}
 
 Notice that if we were to expand $T(X)/W(X,R_i,S_i)$ it would
 have the form $1 + O(x_n\Delta t^{1/2}) + O(x_nx_m \Delta t)+ ...$.
 Therefore if we drop terms of order $\Delta t^2$, the $O(\Delta t)$ term of
 $P(X)$ contributes only when it multiplies the 1 term from $T(X)/W(X,R_i,S_i)$.
 We can therefore integrate it
 over $X$ without changing the result to this
 order in $\Delta t$. This term cancels the normalization, so that
 \begin{eqnarray}
 \tilde P(X) &=&
 e^{-\frac{1}{2} \sum_{n=1}^{N_O} x_n^2}
 \left (1 -i\sum_{n=1}^{N_O} x_n \sqrt{\lambda_n \Delta t}
 \frac{\langle \Psi_T|O_n|R S\rangle}
 {\langle \Psi_T|R S\rangle} + O(\Delta t^{3/2}) \right )
 \nonumber\\
 &=& e^{-\frac{1}{2} \sum_{n=1}^{N_O} x_n^2}
 e^{ -i\sum_{n=1}^{N_O} x_n \sqrt{\lambda_n \Delta t}
 	\frac{\langle \Psi_T|O_n|R S\rangle}{\langle \Psi_T|R S\rangle}
 	+\sum_{n=1}^{N_O} \lambda_n 
 	[\frac{\langle \Psi_T|O_n|R S\rangle}{\langle \Psi_T|R S\rangle}]^2}
 +O(\Delta t^{3/2})
 \nonumber\\
 &=&
 \exp\left \{-\frac{1}{2} \sum_{n=1}^{N_O} 
 \left [x_n+i\sqrt{\lambda_n \Delta t}
 \frac{\langle \Psi_T|O_n|R S\rangle}{\langle \Psi_T|R S\rangle}
 \right ]^2 \right \}
 \nonumber\\
 \end{eqnarray}
 where in the last line, we have written the linear term in $x$ in the
 exponent, and included a canceling term so that only
 the linear term survives integration to order $\Delta t$.
 
 We sample our expression by sampling $x_n$ from the shifted gaussian
 (Again, we assume here that
 $i\sqrt{\lambda_n \Delta t}\langle O_n \rangle$ is real.
 We will discuss what to do for the complex case below.)
 \begin{equation}
 x_n = \chi_n -i \sqrt{\lambda_n\Delta t} \langle O_n\rangle
 \end{equation}
 where $\chi_n$ is sampled from a gaussian with unit variance.
 The new unnormalized ket is
 \begin{equation}
 |R'S'\rangle = e^{-i\sum_{n=1}^{N_O} x_n\sqrt{\lambda \Delta t} O_n}
 |RS\rangle \langle \Psi_T |RS\rangle
 \end{equation}
 and its weight is given by the local energy expression
 \begin{equation}
 W(R',S') = e^{-[\langle H \rangle-E_T] \Delta t}
 \end{equation}
 
 \subsection{Application to standard diffusion Monte Carlo}
 \subsubsection{Diffusion Monte Carlo without importance sampling}
 It is helpful to apply the formalism above to derive
 the well known central potential
 diffusion Monte Carlo algorithm\cite{anderson1976}.
 The Hamiltonian is
 \begin{equation}
 H = \sum_{j=1}^A \sum_{\alpha=1}^3 \frac{p_{j\alpha}^2}{2m} + V(R)
 \end{equation}
 where $p_{j\alpha}$ and $R$ operate on Hilbert space, and $p_{j\alpha}$
 is the $\alpha$ component of the momentum operator for the $j$th
 particle.
 Making the short-time approximation, the propagator can be written
 as
 \begin{equation}
 e^{-(H-E_T)\Delta t} = e^{\sum_{j=1}^A\sum_{\alpha=1}^3
 	\frac{p_{j\alpha}^2}{2m} \Delta t}
 e^{-[V(R)-E_T]\Delta t} + O(\Delta t^2) \,.
 \end{equation}
 Since the Hamiltonian does not operate on the spin, we can drop the
 spin variable
 from the our walker expressions and take just a position basis $|R\rangle$.
 Operating with the potential term
 \begin{equation}
 e^{-[V(R)-E_T]\Delta t}|R_j\rangle = 
 e^{-[V(R_j)-E_T]\Delta t}|R_j\rangle
 \end{equation}
 clearly satisfies Eq. \ref{eq.afop}. The kinetic energy part of the
 propagator does not satisfy Eq. \ref{eq.afop}. However, by
 using the Hubbard-Stratonovich transformation,
 we can write the kinetic energy in terms of the translation operators
 $e^{-\frac{i}{\hbar} p_{j\beta} a}$.
 We introduce the auxiliary field or Hubbard-Stratonovich variables,
 $x_{j\alpha}$, and write
 \begin{eqnarray}
 &&
 e^{-\sum_{j=1}^A\sum_{\alpha=1}^3 \frac{p_{j\alpha}^2}{2m} \Delta t} = 
 \nonumber\\
 &&
 \prod_{j\alpha} \frac{1}{(2\pi)^{3/2}} \int d x_{j\alpha}
 e^{-\frac{x_{j\alpha}^2}{2}}
 e^{-\frac{i}{\hbar} \vec p_{j\alpha}
 	x_{j\alpha} \sqrt{\frac{\hbar^2 \Delta t}{m}}}
 \end{eqnarray}
 
 With this definition, $X$ is the set $\{x_{j\alpha}\}$,
 for the $A$ particles,
 \begin{equation}
 P(X) =
 \prod_{j\alpha} \frac{1}{\sqrt{2\pi}} e^{-\frac{x_{j\alpha}^2}{2}} \,,
 \end{equation}
 and
 \begin{eqnarray}
 T(X)|R\rangle =
 e^{-[V(R)-E_T]\Delta t} |R+\Delta R\rangle
 \end{eqnarray}
 where $R' = R+\Delta R$ is given by translating each particle's position in $R$
 \begin{equation}
 r_{j\alpha}' = r_{j\alpha} + x_{j\alpha} \frac{\hbar^2 \Delta t}{m} \,.
 \end{equation}
 This is identical to the standard diffusion Monte Carlo algorithm without
 importance sampling. We move
 each particle with a gaussian distribution of variance
 $\frac{\hbar^2 \Delta t}{m}$, and include a weight of
 $e^{-[V(R)-E_T]\Delta t}$. We would then include branching on the weight
 to complete the algorithm.
 
 While the Hubbard-Stratonovich transformation is the most common, there are
 many other possibilities. For example, the propagator for the
 relativistic kinetic energy
 $\sqrt{p^2 c^2 + m^2 c^4}-mc^2$ can be sampled by using
 \begin{equation}
 e^{-\left [\sqrt{p^2 c^2 + m^2 c^4}-mc^2 \right ] \Delta t} =
 \int d^3x f(x) e^{-\frac{i}{\hbar} \vec p \cdot \vec x}
 \end{equation}
 with
 \begin{eqnarray}
 f(x) &=& \int \frac{d^3p}{(2\pi)^3} e^{\frac{i}{\hbar} \vec p \cdot \vec x}
 e^{-\left [\sqrt{p^2 c^2 + m^2 c^4}-mc^2 \right ] \Delta t}
 \nonumber\\
 &=& e^{mc^2 \Delta t}
 K_2 \left ( \frac{m c}{\hbar} \sqrt{x^2+c^2 \Delta t^2} \right )
 \end{eqnarray}
 where $K_2$ is the modified Bessel function of order 2\cite{carlson1993}.
 
 
 \subsubsection{Importance sampled Diffusion Monte Carlo in the auxiliary field
 	formulism}
 
 We break up the Hamiltonian as a kinetic and potential part. The potential
 part gives the usual $e^{-V(R) \Delta t}$ weight, and we need to work only
 with the importance sampled kinetic energy part.
 The kinetic energy operator is already written as a sum of squares,
 \begin{equation}
 KE = \sum_{j\alpha} \frac{p_{j\alpha}^2}{2m}
 \end{equation}
 where $j$ is the particle label and $\alpha$ is the $x$, $y$, or $z$ coordinate.
 We can identify $\lambda_{j\alpha} = m^{-1}$, and $O_{j\alpha} = p_{j\alpha}$.
 Substituting this into our previous formalism, we have
 \begin{eqnarray}
 i \sqrt{\lambda_{j\alpha} \Delta t}\langle O_{j\alpha} \rangle
 &=& i \sqrt{\frac{\Delta t}{m}} \frac{\langle \Psi_T |p_{j\alpha}|RS\rangle}
 {\langle \Psi_T |RS\rangle} 
 \nonumber\\
 &=& -\sqrt{\frac{\hbar^2 \Delta t}{m}}
 \frac{\partial_{j\alpha} \langle \Psi_T |RS\rangle}
 {\langle \Psi_T |RS\rangle}  \,.
 \end{eqnarray}
 The sampled value of $x_{j\alpha}$ will be
 \begin{equation}
 x_{j\alpha} = \chi_{j\alpha} + 
 \sqrt{\frac{\hbar^2 \Delta t}{m}}
 \frac{\partial_{j\alpha} \langle \Psi_T |RS\rangle}
 {\langle \Psi_T |RS\rangle}
 \end{equation}
 where the $\chi_{j\alpha}$ are sampled from a gaussian with unit variance.
 The new walker will be
 \begin{equation}
 |R'S'\rangle = e^{-\frac{i}{\hbar} \sum_{j\alpha} x_{j\alpha}
 	\sqrt{\frac{\hbar^2 \Delta t}{m}} p_{j\alpha} }|R S\rangle \,.
 \end{equation}
 Since
 $e^{-\frac{i}{\hbar} p_{j\alpha} a}$ is the translation operator that
 translates the ket's $j\alpha$ position coordinate by $a$.
 We have
 \begin{eqnarray}
 S' &=& S
 \nonumber\\
 R'_{j\alpha} &=& R_{j\alpha} + x_{j\alpha}\sqrt{\frac{\hbar^2 \Delta t}{m}}
 \nonumber\\
 &=& R_{j\alpha}+ \chi_{j\alpha} \sqrt{\frac{\hbar^2 \Delta t}{m}}
 +
 \frac{\hbar^2 \Delta t}{m}
 \frac{\partial_{j\alpha} \langle \Psi_T |RS\rangle}
 {\langle \Psi_T |RS\rangle}
 \end{eqnarray}
 which is the standard diffusion Monte Carlo propagation. The weight factor
 is the local energy.
 
 \subsection{Fixed-phase importance-sampled Diffusion Monte Carlo}
 \label{sec:fixedph}
 The fixed-phase approximation\cite{ortiz1993} was developed to extend the
 fixed-node approximation to electrons in a magnetic field where the
 ground-state wave function is complex. The approximation enforces the
 trial function's phase as the phase for the calculated ground state.
 Diffusion Monte Carlo is used to sample the magnitude of the ground
 state.
 
 If the walker phase has been chosen so that $\langle \Psi_T|R\rangle$
 is real, the fixed-phase approximation requires that after propagation
 $\langle \Psi_T|R'\rangle$ would also be real since an imaginary
 part would correspond to the calculated ground-state having a different
 phase than the trial function. Therefore in the implementation of the
 fixed-phase approximation
 we discard the imaginary part of the weight of a propagated walker.
 For an arbitrary initial phase, we discard the imaginary part of the
 ratio $\frac{\langle \Psi_T|R'\rangle}{\langle \Psi_T|R\rangle}$ which
 means that the we replace the importance sampled factor in
 Eq. \ref{eq.hsimp} with its real part
 \begin{eqnarray}
 \frac{\langle \Psi_T |
 	e^{-i\sum_{n=1}^{N_O} x_n \sqrt{\lambda_n \Delta t} O_n}|R S\rangle}
 {\langle \Psi_T|RS\rangle}
 &\rightarrow&
 \nonumber\\
 {\rm Re} \left [ \frac{\langle \Psi_T |
 	e^{-i\sum_{n=1}^{N_O} x_n \sqrt{\lambda_n \Delta t} O_n}|R S\rangle}
 {\langle \Psi_T|RS\rangle} \right ] \,.
 \end{eqnarray}
 
 The fixed-phase algorithm for propagating a walker is then
 \begin{enumerate}
 	\item
 	Propagate to the new position (the spin does not change with a central
 	potential)
 	\begin{eqnarray}
 	S' &=& S
 	\nonumber\\
 	R'_{j\alpha} &=&
 	R_{j\alpha}+ \chi_{j\alpha} \sqrt{\frac{\hbar^2 \Delta t}{m}}
 	+
 	\frac{\hbar^2 \Delta t}{m}
 	{\rm Re} \left [ \frac{\partial_{j\alpha} \langle \Psi_T |RS\rangle}
 	{\langle \Psi_T |RS\rangle} \right ]
 	\nonumber\\
 	\end{eqnarray}
 	\item
 	Include a weight factor for the walker of
 	\begin{equation}
 	W = e^{ -({\rm Re} \langle H\rangle -E_T)\Delta t}
 	\end{equation}
 \end{enumerate}
 This is identical to the fixed-phase algorithm of Ortiz et al.
 
 We will see that similar approximations can be used for our spin-isospin
 dependent problems.
 
 \subsection{Application to quadratic forms}
 \label{sec.quad}
 Quadratic forms in operators that change from one walker to another
 can be diagonalized to produce the sum of squares needed for
 Eq. \ref{eq.sumofsquares}. That is for
 \begin{equation}
 H = \frac{1}{2} \sum_{ij} O_i A_{ij} O_j
 \end{equation}
 with $A_{nm}$ real and symmetric, we can calculate the normalized
 real eigenvectors and eigenvalues of the matrix $A$,
 \begin{eqnarray}
 \sum_j A_{ij} \psi_j^{(n)} &=& \lambda_n |\psi_i\rangle
 \nonumber\\
 \sum_j \psi_j^{(n)} \psi_j^{(m)} &=& \delta_{nm} \,.
 \end{eqnarray}
 The matrix is then
 \begin{equation}
 A_{ij} = \sum_n \psi_i^{(n)} \lambda_n\psi_j^{(n)}
 \end{equation}
 and substituting back we have
 \begin{eqnarray}
 H = \frac{1}{2} \sum_n \lambda_n {\cal O}_n^2
 \nonumber\\
 {\cal O}_n = \sum_j \psi_j^{(n)} O_j \,,
 \end{eqnarray}
 which is now in the form of Eq. \ref{eq.sumofsquares}.
 
 \subsection{Auxiliary Field Breakups}
 There are many possible ways to break up the nuclear Hamiltonian using
 the auxiliary field formalism. As a concrete example let's look at
 the spinor propagator when we have a spin-exchange potential between $A$
 neutrons
 \begin{equation}
 V = \sum_{i<j}  v^{\sigma}(r_{ij}) \vec \sigma_i \cdot \vec \sigma_j
 \end{equation}
 Taking the operators to be the $x$, $y$, and $z$ components of the
 Pauli operators for each particle, we have a quadratic form in these
 $3A$ operators. Since walker gives the positions of the neutrons, we know
 the value of $v^{\sigma}(r_{ij})$ for all pairs. We can then write
 \begin{equation}
 V = 
 \frac{1}{2} \sum_{ij}^A B_{ij}\sigma_{ix}\sigma_{jx}
 +\frac{1}{2} \sum_{ij}^A B_{ij}\sigma_{iy}\sigma_{jy}
 +\frac{1}{2} \sum_{ij}^A B_{ij}\sigma_{iz}\sigma_{jz}
 \end{equation}
 where $B_{ii} = 0$, and $B_{ij} = v^{\sigma}(r_{ij})$ for  $i \neq j$.
 Finding the eigenenvectors $\psi^{(n)}_i$ and eigenvalues $\lambda_n$
 of the $B$ matrix, we
 can write
 \begin{eqnarray}
 V &=&
 \frac{1}{2} \sum_n \lambda_n ({\cal O}_{nx})^2
 +\frac{1}{2} \sum_n \lambda_n ({\cal O}_{ny})^2
 +\frac{1}{2} \sum_n \lambda_n ({\cal O}_{nz})^2
 \nonumber\\
 {\cal O}_{nx} &=& \sum_{i=1}^A \psi^{(n)}_i \sigma_{ix}
 \nonumber\\
 {\cal O}_{ny} &=& \sum_{i=1}^A \psi^{(n)}_i \sigma_{iy}
 \nonumber\\
 {\cal O}_{nz} &=& \sum_{i=1}^A \psi^{(n)}_i \sigma_{iz} \,.
 \end{eqnarray}
 Using the Hubbard-Stratonovich transformation would give us $3A$ auxiliary
 fields.
 
 We can modify this transformation. For example, the diagonal
 elements of the $B$ matrix are zero. Adding a nonzero diagonal term $B_{jj}$,
 would give us additional terms proportonal to
 $\sigma_{jx}^2 =\sigma_{jy}^2 =\sigma_{jz}^2 = 1$, that is,
 these would be additional purely central terms. Subtracting a corresponding
 central contribution would then give an identical interaction, but
 different eigenvectors and therefore different spin rotation operators.
 
 Another alternative would be to look at each term in the sum separately as
 a quadratic form of two operators. The resulting $2\times 2$ matrices
 have two eigenvalues and eigenvectors so that
 \begin{equation}
 v^\sigma(r_{ij}) \sigma_{ix} \sigma_{jx}  =
 \frac{1}{4} v^\sigma(r_{ij})[\sigma_{ix}+\sigma_{jx}]^2
 -\frac{1}{4} v^\sigma(r_{ij})[\sigma_{ix}-\sigma_{jx}]^2
 \end{equation}
 and each of the $3A(A-1)/2$ terms would require 2 auxiliary fields or
 $3A(A-1)$ total. We can reduce the number of auxiliary fields by including
 diagonal terms to our $2\times 2$ matrix equal to the off diagonal terms.
 These make the eigenvector $(1,-1)/\sqrt{2}$ have a zero eigenvalue,
 which then does not contribute
 \begin{eqnarray}
 v^\sigma(r_{ij}) \sigma_{ix} \sigma_{jx}  =
 \frac{1}{2} v^\sigma(r_{ij})[\sigma_{ix}+\sigma_{jx}]^2
 -v^\sigma(r_{ij})
 \end{eqnarray}
 where the second term on the right hand side
 is a central potential counter term that would be added
 to the physical central potential, and $3A(A-1)/2$ auxiliary fields
 would be required. This form could also be derived by expanding
 the square $[\sigma_{ix}+\sigma_{jx}]^2$.
 
 Each of these breakups gives the same net propagator after integration
 of the auxiliary fields. If a good importance function is used, and the
 sampling can be carried out, we would expect the local energy for a 
 complete step to have low variance, and therefore the propagation to
 have low variance. The trade off then would be the complexity of constructing
 the operator combination versus the number of auxiliary fields needed
 in the propagation. In our work to date, we have used the full
 diagonalization to minimize the number of auxiliary field integrations.
 The cost of the diagonalization is order $A^3$ which is the same order
 as the cost for calculating a Slater determinant for the trial functions
 we need. However, it is easy to imagine having more complicated Hamiltonians
 where the cost of full diagonalization would be prohibitive (for example
 adding $\Delta$ degrees of freedom to the nuclei) and a simpler breakup
 using more auxiliary fields would be more efficient.
 
 The best break up will be the one which optimizes the accuracy and variance
 of the results for a given amount of computational resources.
 
 \subsection{AFDMC with the $v_6'$ potential for nuclear matter}
 The Argonne $v_6'$ potential includes central, spin and isospin exchange,
 and tensor interactions. Writing out the components, the Hamiltonian is
 \begin{eqnarray}
 \label{eq.hv6}
 H &=& \sum_{i\alpha} \frac{p_{i\alpha}^2}{2m}
 + \sum_{i<j} v^c(r_{ij})
 + \sum_{i<j,\alpha\beta} \left \{
 v^\sigma(r_{ij})\delta_{\alpha\beta}+v^t(r_{ij})
 \left [  3 \hat \alpha \cdot \hat r_{ij}
 \hat \beta \cdot \hat r_{ij} -\delta_{\alpha\beta} \right] \right \}
 \sigma_{i\alpha}\sigma_{j\beta}
 \nonumber\\
 &&
 + \sum_{i<j,\alpha\beta\gamma} \left \{
 v^{\sigma\tau}(r_{ij})\delta_{\alpha\beta}+v^{t\tau}(r_{ij})
 \left [  3 \hat \alpha \cdot \hat r_{ij}
 \hat \beta \cdot \hat r_{ij} -\delta_{\alpha\beta} \right] \right \}
 [\sigma_{i\alpha}\tau_{i\gamma}][\sigma_{j\beta}\tau_{j\gamma}]
 + \sum_{i<j,\gamma}
 v^\tau(r_{ij})\tau_{i\gamma}\tau_{j\gamma}
 \end{eqnarray}
 where $\alpha$ and $\beta$ refer to the $x$, $y$, and $z$ components
 and $\hat \alpha$ $\hat \beta$ are the corresponding unit vectors.
 We work in a position basis. The potential is quadratic in
 the 15A spin-isospin operators
 $\sigma_{i\alpha}$, $\tau_{i\gamma}$, $\sigma_{i\alpha}\tau_{i\gamma}$.
 Since each spin-isospin operator can rotate the corresponding spin-isospinor
 the natural basis is the overcomplete basis of the outer product of these
 spin-isospinors -- one for each particle. A walker consists of an overall
 weight factor, and
 $x$, $y$, and $z$ coordinates
 and four
 complex numbers for the components of
 $|p\uparrow\rangle$, $|p\downarrow\rangle$,
 $|n\uparrow\rangle$, $|n\downarrow\rangle$ for
 each of the $A$ particles.
 
 \subsubsection{The $v_6'$ Hamiltonian as a sum of operator squares}
 We now follow section \ref{sec.quad} and define matrices
 \begin{eqnarray}
 C^\sigma_{i\alpha,j\beta} &=& 
 v^\sigma(r_{ij})\delta_{\alpha\beta}+v^t(r_{ij})
 \left [  3 \hat \alpha \cdot \hat r_{ij}
 \hat \beta \cdot \hat r_{ij} -\delta_{\alpha\beta} \right]
 \nonumber\\
 C^{\sigma\tau}_{i\alpha,j\beta} &= &
 v^{\sigma\tau}(r_{ij})\delta_{\alpha\beta}+v^{t\tau}(r_{ij})
 \left [  3 \hat \alpha \cdot \hat r_{ij}
 \hat \beta \cdot \hat r_{ij} -\delta_{\alpha\beta} \right]
 \nonumber\\
 C^\tau_{i,j} &=& v^\tau(r_{ij})
 \end{eqnarray}
 which have zero matrix elements when $i=j$. Their eigenvalues and
 normalized eigenvectors are defined as
 \begin{eqnarray}
 \sum_{j\beta}
 C^\sigma_{i\alpha,j\beta} \psi^{\sigma\ (n)}_{j\beta} &=& \lambda^\sigma_n
 \psi^{\sigma\ (n)}_{i\alpha}
 \nonumber\\
 \sum_{j\beta}
 C^{\sigma\tau}_{i\alpha,j\beta} \psi^{\sigma\ (n)}_{j\beta} &=&
 \lambda^{\sigma\tau}_n \psi^{\sigma\tau\ (n)}_{i\alpha}
 \nonumber\\
 \sum_{j}
 C^{\tau}_{i,j} \psi^{\tau\ (n)}_{j} &=&
 \lambda^{\tau}_n \psi^{\tau\ (n)}_{i}
 \end{eqnarray}
 with operator combinations
 \begin{eqnarray}
 O^\sigma_n &=& \sum_{i\alpha} \psi^{\sigma\ (n)}_{i\alpha} \sigma_{i\alpha}
 \nonumber\\
 O^{\sigma\tau}_{n\beta} &=& 
 \sum_{i\alpha} \psi^{\sigma\tau\ (n)}_{i\alpha} \sigma_{i\alpha}\tau_{i\beta}
 \nonumber\\
 O^\tau_{n\alpha} &=&
 \sum_{i} \psi^{\tau\ (n)}_{i} \tau_{i\alpha}
 \end{eqnarray}
 The Hamiltonian becomes
 \begin{eqnarray}
 H&=&\sum_{i=1}^A\sum_{\alpha=1}^3 \frac{p_{i\alpha}^2}{2m}
 +\sum_{i<j} v^{c}(r_{ij})
 +\frac{1}{2} \sum_{n=1}^{3A} \lambda^\sigma_n (O^\sigma_n)^2
 \nonumber\\
 &&
 +\frac{1}{2} \sum_{n=1}^{A}\sum_{\alpha=1}^3
 \lambda^\tau_n(O^\tau_{n\alpha})^2
 +\frac{1}{2} \sum_{n=1}^{3A} \sum_{\alpha=1}^3
 \lambda^{\sigma\tau}_n(O^{\sigma\tau}_{n\alpha})^2
 \nonumber\\
 \end{eqnarray}
 This is, of course, identical to the original Hamiltonian given in
 Eq. \ref{eq.hv6}, but now it is in a form that makes the propagator
 easy to sample using auxiliary fields.
 
 \subsubsection{Complex auxiliary fields}
 In realistic nuclear physics problems,
 the fermion sign problem necessarily becomes a phase problem
 since conservation of angular momentum requires that flipping a spin changes
 the orbital angular momentum, which induces an angular phase to the wave
 function. Various fixed-phase approximations can be used. The
 Hubbard-Stratonovich transformation integrates the auxiliary field over
 all real values with a gaussian weight. With importance sampling, the
 gaussian for $x_n$ is shifted by $i\sqrt{\lambda_n \Delta t}\langle O_n\rangle$
 as shown in the last line of Eq. \ref{eq.hsimpexpanded}.
 As Zhang and Krakauer\cite{zhang2003} showed for electronic structure problems,
 it is equally valid to integrate the auxiliary field over any shifted
 contour, and by shifting the contour so that $x_n$ becomes complex
 and takes on the values
 $x_n  = z + i\sqrt{\lambda_n \Delta t}\langle O_n\rangle$,
 $-\infty < z < \infty$. Integrating over these values does not change the
 result. However, now this factor is real. We implement the fixed phase
 approximation by taking the real part of $\langle H \rangle$.
 
 
 Note that this method cannot be used for the momentum operator. This is
 because the operator
 $e^{-\frac{i}{\hbar} p_{j\alpha} a}$ is not bounded if $a$ has an imaginary
 part. We therefore implement the kinetic energy terms exactly as in the
 central potential fixed-phase approximation.
 
 There are of course other possible approximations that can be used. The
 auxiliary fields can be kept real. We find that the approximation is
 more accurate with Zhang-Krakauer prescription for auxiliary fields for
 the spin operators.
 
 \subsubsection{The $v_6'$ algorithm}
 We can now give the complete algorithm used for the $v_6'$ potential.
 \begin{enumerate}
 	\item
 	We begin with a set of walkers $|R_iS_i\rangle$ which we sample from
 	our trial function magnitude
 	squared, $|\langle R S|\Psi_T\rangle|^2$, with Metropolis
 	Monte Carlo. The walkers consist of the $3A$ coordinates of the
 	$A$ particles, and $A$ 4-component normalized spinors.
 	\item
 	For each walker in turn we calculate the $C^\sigma$, $C^\tau$ and
 	$C^{\sigma\tau}$ matrices, their eigenvalues, and their eigenvectors.
 	\item
 	From the trial function and spinor values
 	we evaluate $\langle \sigma_{j\alpha}\rangle$,
 	$\langle \sigma_{j\alpha}\tau_{j\beta}\rangle$, $\langle \tau_{j\alpha}\rangle$,
 	$\langle p_{j\alpha}\rangle$, and $\langle H\rangle$.
 	\item
 	We sample the complex values for the spin-isospin auxiliary fields
 	\begin{equation}
 	x_n  = \chi_n + i\sqrt{\lambda_n \Delta t}\langle O_n\rangle
 	\end{equation}
 	and transform our walker spinors using
 	\begin{equation}
 	|RS'\rangle =
 	e^{-i\sum_{n=1}^{N_O} x_n\sqrt{\lambda \Delta t} O_n} |RS\rangle
 	\end{equation}
 	and normalize the spinors
 	\item
 	We sample the new positions from
 	\begin{equation}
 	r_{j\alpha}' = 
 	r_{j\alpha}+ \chi_{j\alpha} \sqrt{\frac{\hbar^2 \Delta t}{m}}
 	+
 	\frac{\hbar^2 \Delta t}{m}
 	{\rm Re} \frac{\partial_{j\alpha} \langle \Psi_T |RS\rangle}
 	{\langle \Psi_T |RS\rangle}
 	\end{equation}
 	\item
 	The weight of the new walker is given by
 	W = $e^{- [{\rm Re} \langle H \rangle - E_T]\Delta t}$
 	\item
 	We branch on the walker weight, taking the number of new walkers
 	to be the integer part of $W$ plus a uniform random value on $(0,1)$.
 	If the weight $W$ is negative, we discard the walker.
 \end{enumerate}
 
 \subsection{Isospin-independent spin-orbit interaction}
 
 Without isospin exchange, the spin orbit term for particles $j$ and $k$
 is
 \begin{equation}
 \frac{1}{4\hbar}
 v_{LS}(r_{jk}) [(\vec r_j-\vec r_k) \times (\vec p_j-\vec p_k)
 ]\cdot ( \vec\sigma_j+\vec \sigma_k )
 \end{equation}
 We can write the kinetic energy plus spin-orbit interaction Hamiltonian
 as
 \begin{eqnarray}
 && \sum_{j\alpha} \frac{p^2_{j\alpha}}{2m} +
 \frac{1}{4\hbar}\sum_{j<k}
 v_{LS}(r_{jk}) [(\vec r_j-\vec r_k) \times (\vec p_j-\vec p_k)
 ]\cdot ( \vec\sigma_j+\vec \sigma_k )
 \nonumber\\
 &=& \sum_{j\alpha} \frac{(p_{j\alpha}+\frac{m}{4\hbar}\sum_{k\neq j}
 	v_{LS}(r_{jk})
 	[(\vec \sigma_j+\vec \sigma_k) \times (\vec r_j-\vec r_k)]_\alpha)^2}{2m}
 +V_{\rm Counter}
 \nonumber\\
 V_{\rm Counter} &=& -\frac{1}{2m} \sum_{j\alpha} \left [
 \frac{m}{4\hbar}\sum_{k\neq j}
 v_{LS}(r_{jk})
 [(\vec \sigma_j+\vec \sigma_k) \times (\vec r_j-\vec r_k)]_\alpha \right ]^2
 \end{eqnarray}
 where the counter terms subtract off the unwanted interaction from
 completing the square.
 The counter terms do not depend on $\vec p_j$, so they can be included
 with the rest of the local potential, and will contribute to the
 drift and the local energy for that part. However, we will see that the local
 energy part is canceled below (that is the final weight will be just the
 correct total local energy which does not include the counter terms).
 
 Using the Hubbard-Stratonovich break up with importance sampling, we have
 $\lambda_{j\alpha} = m^{-1}$, and
 \begin{eqnarray}
 i\sqrt{\lambda_{j\alpha}\Delta t}\langle O_{j\alpha} \rangle
 &=& -\sqrt{\frac{\hbar^2 \Delta t}{m}}
 \frac{\partial_{j\alpha} \langle \Psi_T |RS\rangle}
 {\langle \Psi_T |RS\rangle} +i \sqrt{\frac{m \Delta t}{16 \hbar^2}}
 \sum_{k\neq j}[(\langle \vec \sigma_j\rangle+\langle \vec \sigma_k\rangle)\times
 \vec r_{jk} ]_{\alpha} v_{LS}(r_{jk})\,.
 \nonumber\\
 \end{eqnarray}
 The sampled value of $x_{j\alpha}$ will be
 \begin{equation}
 x_{j\alpha} = \chi_{j\alpha} + 
 \sqrt{\frac{\hbar^2 \Delta t}{m}}
 \frac{\partial_{j\alpha} \langle \Psi_T |RS\rangle}
 {\langle \Psi_T |RS\rangle}
 -i \sqrt{\frac{m \Delta t}{16 \hbar^2}}
 \sum_{k\neq j}[(\langle \vec \sigma_j\rangle+\langle \vec \sigma_k\rangle)\times
 \vec r_{jk} ]_{\alpha} v_{LS}(r_{jk})\,.
 \end{equation}
 where our fixed-phase like approximation
 will modify this to keep the translation real, so that
 \begin{equation}
 x_{j\alpha} = \chi_{j\alpha} +  {\rm Re}\left \{
 \sqrt{\frac{\hbar^2 \Delta t}{m}}
 \frac{\partial_{j\alpha} \langle \Psi_T |RS\rangle}
 {\langle \Psi_T |RS\rangle}
 -i \sqrt{\frac{m \Delta t}{16 \hbar^2}}
 \sum_{k\neq j}[(\langle \vec \sigma_j\rangle+\langle \vec \sigma_k\rangle)\times
 \vec r_{jk} ]_{\alpha} v_{LS}(r_{jk}) \right \}\,.
 \end{equation}
 The walker propagator is
 \begin{eqnarray}
 |R'S'\rangle = e^{-\frac{i}{\hbar} \sum_{j\alpha} x_{j\alpha}
 	\sqrt{\frac{\hbar^2\Delta t}{m}} p_{j\alpha} }
 e^{i \sum_{j\alpha} x_{j\alpha} \sqrt{\frac{m \Delta t }{16\hbar^2}}
 	\sum_{k\neq j}[(\vec \sigma_j+ \vec \sigma_k)\times
 	\vec r_{jk} ]_{\alpha} v_{LS}(r_{jk}) }  |R S\rangle
 \end{eqnarray}
 
 The local energy term for the spin orbit will contain the kinetic energy,
 the spin orbit, and the negative of the counter terms. Therefore, the
 counter term contribution cancels in the weight, and the final weight is
 the local energy. 
 
 \section{GFMC with full spin-isospin summation}
 As mentioned above, current 
 high quality trial wave functions for the coordinate space
 nuclear Hamiltonians require the same computational complexity to
 calculate either one or all of the spin-isospin amplitudes at a specified
 position for the particles. Very roughly for $A$ nucleons, each of which
 can be a proton or neutron with spin up or down, the number of
 spin-isospin amplitudes is $4^A$. Symmetries can lower
 this factor but not change its overall exponential character.
 
 Typically these calculations are done in either a good charge or good
 isospin basis. In a good charge basis, with $A$ nucleons, with $Z$ protons,
 the number of combinations of protons and neutrons is $\frac{A!}{Z!(A-Z)!}$,
 while the tensor force can flip any of the spins so there are $2^A$ spin
 states. The total number of allowed spin-isospin states is
 the product of these factors. Sometimes the initial calculations are done
 with a Hamiltonian that conserves isospin and the charge symmetry breaking
 components are added perturbatively. In this case the number of states
 can be further reduced. Since $T_z = \frac{2Z-A}{2}$, the number of
 isospin states $T$ states for a given $T_z \le T$
 is given by the difference in the number
 of charge states with $T_z=T$ and $T_z=T+1$, which is
 $\frac{A!}{(\frac{A}{2}-T)!(\frac{A}{2}+T)!}\frac{2T+1}{\frac{A}{2}+T+1}$.
 
 Time-reversal invariant states have a further factor of 2 reduction, since
 in that case, the time reversal operator
 \begin{equation}
 {\cal T} = \left [ \prod_{i=1}^A \sigma_{ix}\sigma_{iz}\right ] K
 \end{equation}
 relates the amplitudes of the states given by flipping all the spins.
 Here $K$ is the complex conjugating operator.
 
 Table \ref{chapter9.t1}
 \begin{table}
 	\begin{center}
 	\begin{tabular}{|c|c|c|c|c|}
 		\hline
 		Nucleus & Spin & Charge states & Total & Isospin/T Reversal\\
 		\hline
 		$^4$He & 16 & 6 & 96 & 16\\
 		$^8$Be & 256 & 70 & 17920 & 1792 \\
 		$^{12}$C & 4096 & 924 & 3784704 & 270336 \\
 		$^{16}$O & 65536 & 12870 & $8.4 \times 10^8$ & $4.7 \times 10^7$ \\
 		\hline
 	\end{tabular}
 	\end{center}
 	\caption{The number of spin-isospin amplitudes for the
 		ground states of some representative
 		nuclei.}
 	\label{chapter9.t1}
 \end{table}
 
 To see how this works, we can look at a straightforward generalization
 of a Jastrow-Slater trial state,
 \begin{equation}
 |\Psi_T\rangle = \left [ {\cal S} \prod_{i<j} \sum_p
 f^{(p)}_{ij} O^{(p)}_{ij} \right ] |\Phi\rangle
 \end{equation}
 where $|\Phi\rangle$ is a model state, typically one or
 a small linear combination of antisymmetric
 products of single particle orbitals. The $p$ sum is over the
 same sort of operators as those in the potential (usually operators
 with gradients are either omitted or kept only at lowest order), with
 the Jastrow correlations $f^{(p)}_{ij}$ depending only on the spatial
 operator $|\vec r_i - \vec r_j|$, while the $O^{(p)}_{ij}$ contain
 spin-isospin operators and the unit vector operators
 $\frac{\vec r_i - \vec r_j}{ |\vec r_i - \vec r_j|}$. The ${\cal S}$
 is a symmetrizing operator applied to the Jastrow product, since
 the operators in general do not commute, so that the trial function
 is properly antisymmetric under interchange.
 
 To form a trial wave
 function we take the inner product with $\langle R S|$ to obtain
 $\Psi_T(R,S) = \langle R S|\Psi_T\rangle$. The spatial operators
 operating to the left on their eigenstate $\langle R S|$ are replaced
 by their eigenvalues. This leaves just the spin-isospin matrix elements.
 The model state is evaluated for all possible spin-isospin states as
 enumerated above, $\langle R S'|\Phi\rangle$. In our spin-isospin
 basis, each of
 the operators $\langle S''| O^{(p)}_{ij}|S'\rangle$ is a sparse matrix which
 can either be tabulated or easily calculated as needed. For example,
 in the charge basis, acting on a single basis state, the interaction
 can change the spins of a pair to any of the 4 values. If the particles
 of the pair are a neutron and a proton, they can be interchanged. This
 shows that there are at most either 4 or 8 nonzero entries per row or
 column of the matrix representation. The construction of the Jastrow
 product is obtained by these repeated sparse-matrix multiplications.
 
 The symmetrizing operator has the factorial of the number of pairs
 terms. It would be prohibitive to calculate explicitly. However, the
 commutator terms are small, so the sum over orders of the operators
 is done by Monte Carlo sampling.
 
 Since much of the compuational time is spent in evaluating the trial
 wave functions, wave functions that include more complicated correlations
 as well as alpha particle clustering are often included. The simplest
 wave function above is adequate for the alpha particle.
 
 A GFMC calculation uses walkers given by positions for all the
 particles, and amplitudes for each of the possible spin-isospin
 states in the basis.
 
 In the simplest GFMC implementation, the so-called primitive approximation
 can be used. Here the propagator is 
 \begin{equation}
 \left [
 \prod_{i<j} e^{-\tfrac{1}{2}\tau \sum_p v^{(p)}_{ij}} 
 \right ]
 e^{-\tau \sum_i \frac{p_i^2}{2m}}
 \left [
 \prod_{i<j} e^{-\tfrac{1}{2}\tau \sum_p v^{(p)}_{ij}} 
 \right ]
 \end{equation}
 where the opposite order of the pairs is taken in the two products
 to minimize the time-step errors. The exponentials of the pair operators
 can be written as a linear combination of pair operators, and these
 are then operated on the walker states giving new amplitudes. The
 kinetic energy term is implemented by sampling a gaussian to give new
 positions.
 

\section{General projection algorithms in Fock space and non-local interactions}
In recent years, a number of projection algorithms working in a discrete Fock space (configuration 
space) rather than in coordinate space have been proposed~\cite{Booth09,Cleland10,Petruzielo12,Booth13,Mukherjee13,Roggero13}. While more similar to
more standard many-body techniques like Coupled Cluster (CC) and Many Body Perturbation Theory already covered in previous chapters
the adoption of statistical techniques in a configuration space has some advantage. First of all
Monte Carlo techniques can be implemented with a much milder scaling with the system size enabling the
possibility with a much larger number of basis states that build up the total Hilbert space. Contrary
to eg. CC theory we can ensure that the final QMC estimate for the ground-state energy would
be an upper bound of the true eigenvalue, thus providing useful benchmark results. Also, working 
on a finite many-body space allows practical calculations with non-local interactions, like those developed 
within the Chiral Effective Field Theory approach to nuclear forces, in a far more controllable way than not 
with the continuous coordinate-space formulation exposed so far (as was done in~\cite{Roggero14}).
Finally, another great advantage of performing the Monte Carlo on a discrete Hilbert space is the possibility to
devise an efficient strategy to reduce the impact of the sign-problem by using cancellation techniques~\cite{Booth09,Cleland10,Petruzielo12}
in an analogous fashion to what was sketched at the end of Sec.~\ref{sec:signprob}. Unfortunately we won't have space here to cover these aspects.

\subsection{Fock space formulation of Diffusion Monte Carlo}
\label{sec:cimc}
To set the stage let us take a finite set $\mathcal{S}$ of single-particle (sp) states of size ${\cal N}_s$ and 
consider a general second--quantized fermionic Hamiltonian including two and possibly many--body interactions
\begin{equation}
\label{eqham}
H=\sum_{\alpha \in \mathcal{S}} \epsilon_\alpha a^{\dagger}_\alpha a_\alpha + \sum_{\alpha\beta\gamma\delta\in \mathcal{S}} V_{\alpha\beta\gamma\delta} a^{\dagger}_\alpha a^{\dagger}_\beta a_\delta a_\gamma + \dots \;.
%\label{eqham}
\end{equation}
In this expression Greek letter indices indicates sp states (ie. $\alpha$ is a collective label for all sp quantum 
numbers), the operator $a^{\dagger}_\alpha$ ($a_\alpha$) creates (destroys) a particle in the sp state $\alpha$ 
and the $V_{\alpha\beta\gamma\delta}$ are general (anti--symmetrized) two-body interaction matrix elements:
\begin{equation}
\label{eq_ham}
V_{\alpha\beta\gamma\delta} = \langle \alpha \beta \lvert \hat{V} \rvert \gamma \delta \rangle - \langle \alpha \beta \lvert \hat{V} \rvert \gamma \delta \rangle .% \equiv \langle i j \vert \vert a b \rangle 
\end{equation}.
For an $N$-fermion system the resulting Fock space would be spanned by the full set of $N$-particle 
Slater determinants that can be generated using the sp orbitals $\alpha \in \mathcal{S}$. We will denote these 
Slater--determinants in the occupation number basis by $\rvert \mathbf{n} \rangle$, where $\mathbf{n} \equiv \{ n_\alpha \}$ and $n_\alpha = 0,1$ 
are occupation number of the single--particle orbital $\alpha$ satisfying $\sum_\alpha n_\alpha=N$. For
example in a system composed by 2 identical fermions and with ${\cal N}_s=4$ available sp states we will write
\begin{equation}
\rvert 0110 \rangle \equiv a^{\dagger}_3 a^{\dagger}_2 \rvert 0\rangle  
\end{equation}
where $\rvert 0 \rangle$ is our vacuum state (that can be conveniently set to the Hartree-Fock ground state $\Phi_{HF}$), while
$a^{\dagger}_2$ and $a^{\dagger}_3$ creates a particle in sp state $2$ and $3$ respectively.

We can now use these states as a complete basis in our many--body Hilbert space and express a generic state in it as 
\begin{equation}
\lvert \Psi \rangle = \sum_{\bf n} \langle \mathbf{n} \vert \Psi \rangle \lvert \mathbf{n} \rangle \equiv \sum_{\bf n} \Psi(\bf{n}) \lvert {\bf n} \rangle
\end{equation}
where the sum is over all possible basis vectors that one can obtain from the ${\cal N_S}$ single-particle orbitals.

It is important to notice at this point that no assumption is made on the locality of the 
interaction, which translates into restrictions on the structure of the tensor $V_{\alpha\beta\gamma\delta}$. This 
shows already that possible non-local interactions can be cleanly incorporated in the formalism.

As was already introduced in Section~\ref{sec:generaldmc}, the core idea behind a Diffusion Monte Carlo algorithm is
to extract ground-state informations on the system by evolving in imaginary-time an initial guess for the lowest
eigenstate of the hamiltonian $H$: 
\begin{equation}
\label{ci_evol}
\Psi_{\tau + \Delta\tau} (\mathbf{m}) = \sum_{\mathbf{n}} \langle \mathbf{m} \lvert P \rvert \mathbf{n} \rangle \Psi_{\tau} (\mathbf{n}) .
\end{equation}
with a suitable projecton operator $P$ (cf. Eq.~\eqref{eq.afdmc} and discussion above). 
In order to illustrate how the evolution in \eqref{ci_evol} can be implemented in a stochastic way, it will be usefull
first to express the matrix elements of $P$ as follows
\begin{equation}
\langle \mathbf{m} \lvert P \rvert \mathbf{n} \rangle = p(\mathbf{m},\mathbf{n}) g(\mathbf{n})
\end{equation}
with
\begin{equation}
\label{eq:CIMC_branching_factor}
g(\mathbf{n}) = \sum_{\mathbf{m}} \langle \mathbf{m} \lvert P \rvert \mathbf{n} \rangle 
\end{equation}
and
\begin{equation}
\label{eq:CIMC_prob}
p(\mathbf{m},\mathbf{n}) = \frac{\langle \mathbf{m} \lvert P \rvert \mathbf{n} \rangle}{\sum_{\mathbf{m}} \langle \mathbf{m} \lvert P \rvert \mathbf{n} \rangle }
\end{equation}.
At this point, provided the matrix elements $\langle \mathbf{m} \lvert P \rvert \mathbf{n} \rangle \geq 0$ we can interpret 
$p(\mathbf{m},\mathbf{n})$ for fixed $\mathbf{n}$ as (normalized) probability distribution for the states $\mathbf{m}$ and 
$g(\mathbf{n})$ as a weight factor. This is analogous to what was done in Section~\ref{sec:dmccoord} for the conventional coordinate--space formulations  
where now $p$ takes the place of the gaussian Eq.~\eqref{eq:gaussprop} while $g$ replaces the weight Eq.~\eqref{eq:propw}.

Imagine now that at a given imaginary--time $\tau$ the wave--function $\Psi_{\tau}$ is non--negative in configuration 
space
\begin{equation}
\Psi_{\tau}(\mathbf{n}) \geq 0 \forall\mathbf{n} ,
\end{equation}
then we can represent it as an ensemble of configurations. Due to the non--negativity of the matrix elements of $P$, we also have
that the evolution described in \eqref{ci_evol} preserves the signs
\begin{equation}
\Psi_{\tau+\Delta\tau}(\mathbf{m}) \geq 0 \; \forall \mathbf{m} .
\end{equation}
This suggests the following procedure for the stochastic imaginary--time evolution: 
\begin{enumerate}
 \item walker starts at configuration $\mathbf{n}$ with weight $w(\mathbf{n})$
 \item a new configuration $\mathbf{m}$ is chosen from the probability distribution $p(\mathbf{m},\mathbf{n})$
 \item the walker's weight gets rescaled as $w(\mathbf{n}) \to w(\mathbf{m})=w(\mathbf{n})g(\mathbf{n})$
 \item reapeat from $1.$
\end{enumerate}
In order to improve efficiency one can include a {\it branching} step where the new configuration in $\mathbf{m}$
is replicated according to its weight as explained in Sec.~\ref{sec:dmccoord}.

Expectation values of observables can then be estimated as usual (cf. Eq.~\eqref{eq:mixedobs}) with the mixed estimator
\begin{equation}
\begin{split}
\langle O\rangle_{mixed} &= \frac{\langle \Psi_T\lvert O \rvert \Psi(\tau)\rangle}{\langle \Psi_T\vert \Psi(\tau)\rangle}= \frac{  \sum_l^{N_{w}} w(\mathbf{m}_{l}) \langle \Psi_T\lvert O \rvert \mathbf{m}_{l}\rangle}{\sum_l^{N_{w}}w(\mathbf{m}_{l})\Psi_T(\mathbf{m}_{l})}
\end{split}
\end{equation} 
where $\Psi_T$ is a trial state and the sums run over the walker population of size $N_w$.

In practice we have to choose some form for the evolution operator that appears in \eqref{ci_evol}, a common choice in 
discrete spaces is on operator very similar to the one already encountered in the discussion of the Power Method Sec.~\ref{sec:pm}:
\begin{equation}
\begin{split}
\label{eq:prop}
\langle \mathbf{m} \lvert P \rvert \mathbf{n} \rangle & = \langle \mathbf{m} \lvert 1 - \Delta\tau \left( H - E_T \right) \rvert \mathbf{n} \rangle \\
& = \delta_{\mathbf{m},\mathbf{n}} - \Delta\tau \langle \mathbf{m} \lvert H - E_T \rvert \mathbf{n} \rangle
\end{split}
\end{equation}
where $E_T$ is an energy shift used in the simulation to preserve the norm of the solution (the constant $E_0$ introduced in Sec.~\ref{sec:dmccoord}). 
Convergence to the ground--state by repeated application of the projector $P$ to the initial state $\rvert \Psi_0 \rangle$
\begin{equation}
\vert\Psi_{gs}\rangle = \lim_{M \to \infty} P^M \vert \Psi_0 \rangle
\end{equation}
is guaranteed provided that the eigenvalues of $P$ lie between $-1$ and $1$ in order to ensure the diagonal part remains positive definite.
This requirement translates into a condition on the imaginary-time step $\Delta\tau$ which has to satisfy the bound
\begin{equation}
\label{eq_bound_on_tau}
\Delta\tau < 2/(E_{max}-E_{min}) 
\end{equation}
where $E_{max}$ and $E_{min}$ are respectively the maximum and minimum eigenvalue of $H$ in our finite basis. This upper bound becomes 
tighter and tighter as we increase the number of particle $N$ and/or the number of sp--states ${\cal N}_s$. As a consequence the number $M$ of 
iterations needed for convergence to the ground state increases dramatically. A way to deal with this problem is to
employ a different algorithm proposed in \cite{Trivedi90} (see also \cite{TenHaaf95,Sorella00} ) that allows us to sample directly 
from the exponential propagator
\begin{equation}
\langle \mathbf{m} \lvert P \rvert \mathbf{n} \rangle = \langle \mathbf{m} \lvert e^{-\Delta\tau(H-E_T)} \rvert \mathbf{n} \rangle
\end{equation}
in analogy to Eq.~\eqref{eq:gentauprop}, but now without any limitation on the choice of the imaginary time step $\Delta \tau$ that can be chosen
arbitrarily large. We leave the discussion of its details in Sec.~\ref{sec:expprop}.

In our discussion so far we have assumed that the matrix elements on the projector that defines $p(\mathbf{m},\mathbf{n})$ in Eq.~\eqref{eq:CIMC_prob} are
actually positive definite. Under general circumstances however this is not the case. This clearly prevents the interpretation 
of $p({\bf m},{\bf n})$ as a probability distribution invalidating the naive approach employed above. In order to circumvent the problem we can use the same idea behind the
fixed node (phase) approximation introduced in Sec.~\label{sec:fn}

Before continuing it is worth to mention that in principle one can still produce a stochastic evolution by absorbing the signs into the weight factor $g(\mathbf{n})$
while sampling off-diagonal moves using $\left \vert \langle \mathbf{m} \lvert P \rvert \mathbf{n} \rangle \right\vert$. However as briefly explained in Sec.~\ref{sec:fn} 
this is accompanied by an exponential decay of the signal to noise ratio as a function of the total projection time $\tau = M \Delta\tau$. Recently it was shown 
that by employing an annihilation step in the evolution this problem can be substantially alleviated \cite{Booth09,Petruzielo12,Booth13}. At the end however these 
algorithms have still an exponential scaling with system-size, though with a reduced exponent.

\subsection{Importance sampling and fixed-phase approximation}
\label{subsect:CCDMC-IS}
As we just mentioned, we can deal with the sign--problem in a way which is similar to standard coordinate--space QMC: we will use an initial ansatz $\Phi_T$ 
for the ground--state wave--function and use that to constrain the random walk in a region of the many--body Hilbert space where 
\begin{equation}
\langle \mathbf{m} \lvert P \rvert \mathbf{n} \rangle \geq 0
\end{equation}
is satisfied. In order for this scheme to be practical one needs a systematic way for reducing the bias coming from this approximation, e.g. we want the bias 
to go to zero as the ansatz $\Phi_T$ goes towards the ground--state $\Psi_{gs}$. That's exactly what is done in coordinate-space fixed-node(fixed-phase) QMC simulations 
presented in the previous sections.

In this derivation we will follow the work in \cite{TenHaaf95,Sorella00} and generalize it to the case of complex--hermitian hamiltonians usually found in nuclear theory.
Similarly to what was done in Sec.~\ref{sec:fixedph} the imaginary part of the solution is constrained to be the same of that of the trial wave--function
\begin{equation}
\Re [\Psi^*(\mathbf{n})\Phi_T(\mathbf{n})] = 0 
\end{equation}
for every distribution $\Psi(\mathbf{n})$ sampled in the random walk. In this expression $\Re$ stands for the real part and $^*$ is complex--conjugation

We start by defining for any configurations $\mathbf{n}$ and $\mathbf{m}$ for which $|\Phi_T(\mathbf{n})| \neq 0$ the following quantity:
\begin{equation}
\begin{split}
\label{CCDMC:ham_sign}
\mathfrak{s}_{\mathbf{m}\mathbf{n}} &= \mbox{sign}\; \Re \left [ \Phi_T^*(\mathbf{m}) H_{\mathbf{m}\mathbf{n}} \Phi_T^*(\mathbf{n})^{-1}\right ]  \\
 &= \mbox{sign} \; \frac{\Re \left [ \Phi_T^*(\mathbf{m}) H_{\mathbf{m}\mathbf{n}} \Phi_T(\mathbf{n})\right ]}{ \lvert \Phi_T (\mathbf{n}) \rvert ^2} = \mathfrak{s}_{\mathbf{n}\mathbf{m}} .
 \end{split}
\end{equation}
Now define a one--parameter family of Hamiltonians $\mathcal{H}_{\gamma}$ defined over configurations $\mathbf{n}$ (again such that
$|\Phi_T(\mathbf{n})| \neq 0$) with off--diagonal matrix elements given by
\begin{equation}
  \label{mh1}
  \langle \mathbf{m} | \mathcal{H}_{\gamma} | \mathbf{n} \rangle  =\left \{ \begin{array}{rl} -\gamma \langle \mathbf{m} | H | \mathbf{n} \rangle&  \quad    \mathfrak{s}(\mathbf{m},\mathbf{n})  > 0 \\
  \langle \mathbf{m} | H | \mathbf{n} \rangle&   \quad  \text{otherwise} \end{array} \right . \;,
\end{equation}
while the diagonal terms are
\begin{equation}
\begin{split}
\label{mh2}
\langle \mathbf{n} | \mathcal{H}_{\gamma} | \mathbf{n} \rangle &= \langle \mathbf{n} | H | \mathbf{n} \rangle+ (1+\gamma) \displaystyle \sum_{\stackrel{ \mathbf{m} \neq
\mathbf{n}}{\mathfrak{s}(\mathbf{m},\mathbf{n}) > 0}} \mathfrak{s} (\mathbf{m},\mathbf{n})\\
&= \langle \mathbf{n} | H | \mathbf{n} \rangle + \sum_{\mathbf{m}} h_{\mathbf{m}\mathbf{n}}\\\;.
\end{split}
\end{equation}
In the limit where $\gamma\to-1$ we clearly recover the original Hamiltonian: 
\begin{equation}
\mathcal{H}_{\gamma = -1} \equiv H .
\end{equation}

We proceed to define a corresponding family of propagators $\mathcal{P}_{\gamma}$ for configurations $\mathbf{n}$ with $|\Phi_T(\mathbf{n})| \neq 0$ by
\begin{equation}
\label{eq:CIMC_IS_prop}
\langle \mathbf{m} \lvert \mathcal{P}_{\gamma} \rvert \mathbf{n} \rangle = \delta_{\mathbf{m},\mathbf{n}} - \Delta\tau \frac{\Re \left[ \Phi^*_T(\mathbf{m}) \langle \mathbf{m} \lvert \mathcal{H}_{\gamma} - E_T \rvert \mathbf{n} \rangle \Phi_T(\mathbf{n}) \right]}{\lvert \Phi_T(\mathbf{n}) \rvert^2}\; .
\end{equation}
It is clear now that for any $\gamma \geq 0$ we have 
\begin{equation}
\langle \mathbf{m} \lvert \mathcal{P}_{\gamma} \rvert \mathbf{n} \rangle \geq 0 
\end{equation}
and so the propagator $\mathcal{P}$ is, by construction, free from the sign--problem. Performing the corresponding random--walk allows us to filter the state
\begin{equation}
\Phi_T(\mathbf{n})\phi_{\gamma}^0(\mathbf{n}) ,
\end{equation}
where now $\phi_{\gamma}^0(\mathbf{n})$ is the ground--state of the hamiltonian $\mathcal{H}_{\gamma}$. The ground--state energy $E_{\gamma}$ 
obtained following this procedure can be proved (the proof is left to the Appendix) to be a strict upper bound for the true ground--state 
energy $E_{0}$ of the true hamiltonian $H$. Moreover, this upper bound is tighter than the variational upper--bound provided by 
\begin{equation}
E_T = \frac{\langle \Phi_T \lvert H \rvert \Phi_T \rangle}{\langle \Phi_T \vert \Phi_T \rangle} \ge E_0 .
\end{equation}

As you can show in Problem~\ref{prob:egamma} any linear extrapolation of $E_{\gamma}$ from any two values $\gamma \geq 0$ to $\gamma = -1$ (which would correspond to the original hamiltonian) also
provides an upper--bound on $E_{gs}$ that is tighter than the individual $E_{\gamma}$'s. A good compromise between the tightness of the upper--bound and the statistical noise in
the extrapolation is to choose two values of $\gamma$: $0$ and $1$, thus giving the following energy estimator:
\begin{equation}
\label{cimc_extrap}
E_{extr} = 2 E_{\gamma=0} - E_{\gamma=1}
\end{equation}

To ensure the success of the proposed method a good choice for the importance function $\rvert \Phi_T \rangle$ is critical.

% , we need a wave--function flexible enough to account for the relevant correlations
% in the system and that at the same time can be evaluated sufficiently quickly on a computer.% In many strongly--interacting systems coupled--cluster theory provide an ansatz that fulfills the first criterion. 

\subsection{Trial wave-functions from Coupled Cluster ansatz}
As have been pointed out before, a crucial role is played by the
importance function $\Phi_T$ used to impose the constraint. This is especially true if we want
to estimate expectation values of operators other than the energy (cf. discussion in Sec.~\ref{sec:mixav}).

Fundamental prerequisites for a viable importance function are
\begin{enumerate}
 \item enough flexibility to be able to account for the relevant correlations in the system
 \item availability of an efficient way to evaluate its overlap with states explored during the random walk
\end{enumerate}

Within a Fock space formulation, an excellent choice for $\Phi_T$ that satisfy the first requirement is given by the wave 
function generated in a Coupled Cluster calculation. Starting from a reference state,
which usually is the Hartree-Fock solution of the problem, CC theory allows to
include dynamical correlations into a new state as
\begin{equation}
\label{eq:ccwf}
\vert \Psi_{CC}\rangle = e^{\hat{T}}\vert \Phi_{HF} \rangle.
\end{equation}
In the above equation, correlations are introduced trough the excitation operator $\hat{T}$ which
in CC theory is hierarchically divided as 
\begin{equation}
\hat{T}=\hat{T}_1+\hat{T}_2 + \hat{T}_3+\cdots 
\end{equation}
counting the number of creation/annihilation operators that compose them. The firt two terms are:
\begin{equation}
\hat{T}_1=\sum_{\alpha,\beta \in \mathcal{S}} t_\alpha^\beta a^{\dagger}_\beta a_\alpha
 \;\quad\; \hat{T}_2=\frac{1}{4}\sum_{\alpha,\beta,\gamma,\delta \in \mathcal{S}} t_{\alpha\beta}^{\gamma\delta} a^{\dagger}_\gamma a^{\dagger}_\delta a_\alpha a_\beta \;\;\; \cdots
\end{equation}
The final state $\vert \Psi_{CC}\rangle$ will then be uniquely identified by the coefficients $t_\alpha^\beta$ and $t_{\alpha\beta}^{\gamma\delta}$ corresponding
to single and double particle-hole excitations respectively. The exponentiated form of the CC wave-functions enables to effectively include
some correlations up to the maximum N-particle N-hole in a relatively compact way.

But is the wave--function in Eq.~\eqref{eq:ccwf} also quick to evaluate?
In order to simplify the discussion we will focus here on the case of a homogeneous system 
that can be described dropping the one-particle--one-hole excitation operator $\hat{T}_1$ in 
the expansion (which do not contribute due to translational invariance)\footnote{Extension to singlets (p-h states) and triplets (3p-3h states) is simple}. 
In this situation the lowest order of CC theory is the Coupled Cluster Doubles (CCD) approximation.

To set the notation, we will express a generic Slater-Determinant state describing an M-particle--M-hole state as
\begin{equation}
\rvert {\bf m}\rangle= a^\dagger_{p_1}\dots,a^\dagger_{p_M}a_{h_1}\dots,a_{h_M} \vert \Phi_{HF} \rangle \equiv \; \rvert \Phi^{p_1,\dots,p_M}_{h_1,\dots,h_M} \rangle.
\end{equation}
%---------------------
The required amplitude can then be expressed as a superposition of $M-2$ particle/hole states
that can be generated from ${\bf m}$. Eventually (the proof is tedious but straightforward) one obtains:
\begin{equation}
\label{eq:ccdeval}
\langle {\bf m}\vert \Psi_{CC}\rangle = \sum_{\gamma=2}^M\sum_{\mu<\nu}^M (-1)^{\gamma+\mu+\nu}t^{p_\mu p_\nu}_{h_1 h_\gamma}\Psi_{CC}^{M-2}\left(\substack{p_1,p_2,\dots,p_{\mu-1},p_{\mu+1},\dots,p_{\nu-1},p_{\nu+1},\dots,p_M\\ h_2\dots,h_{\gamma-1},h_{\gamma+1},\dots,h_M}\right)
\end{equation}
assuming $p_1<p_2<\dots<p_M$ and $h_1<h_2<\dots<h_M$. The normalization is fixed in such a way that $\langle \Phi_{HF}\vert \Psi_{CC}\rangle = 1$.

One way to implement Eq.~\eqref{eq:ccdeval} is for instance trough a recursive function that takes as input some K-particle--K-hole state and 
returns $1.0$ for $K=0$, the correct amplitude $t_{ij}^{ab}$ for $K=2$ and for $K>2$ calls itself again removing two particle and two hole states.
Clearly this approach becomes slow when states with large values of $K$ are sampled often during the random walk. Just to give an idea, for
calculations of pure neutron matter with soft Chiral EFT interactions we have $K\leq6$ at densities $\rho\approx0.08 fm^{-3}$ (cf. discussion in~\cite{Rrapaj16})
and the calculation can be made very efficient.

Within CC theory the coefficients $t_{\alpha\beta}^{\gamma\delta}$ appearing in the equations above are to be obtained as the self--consistent solutions
of the following non--linear equation:
\begin{equation}
\label{eq_CCD}
\langle \Phi^{\gamma\delta}_{\alpha\beta} \lvert \hat{H} \left( 1+\hat{T}_2+\frac{1}{2}\hat{T}^2_2\right) \rvert\Phi_{HF} \rangle= \left( \frac{1}{4}\sum_{\alpha,\beta,\gamma,\delta \in \mathcal{S}} \langle \alpha\beta\lvert\rvert \gamma\delta\rangle t_{\alpha\beta}^{\gamma\delta}\right) t_{\alpha\beta}^{\gamma\delta}
\end{equation}
where $\langle \alpha\beta\lvert\rvert \gamma\delta\rangle$ are the anti-symmetrized two-body matrix elements of the interaction defined in Eq.~\eqref{eq_ham}.

Solving Eq.~\eqref{eq_CCD} is in general a very expensive computational problem and within 
the fixed--node approach all that matters are the signs in Eq.~\eqref{sign_function}. It could 
then be possible to find cheaper approximate ways to determine the doubles coefficients $t_{\alpha\beta}^{\gamma\delta}$ 
while still preserving a good quality in the fixed--node approximation. A quite precise and very
cheap approximation that have been used successfully is to obtain the coefficients within second 
order Moeller--Plesset perturbation theory:
\begin{equation}
t_{\alpha\beta}^{\gamma\delta} = \frac{\langle \alpha\beta\lvert\rvert \gamma\delta\rangle }{\eta_\alpha+\eta_\beta-\eta_\gamma-\eta_\delta} \quad  \text{with} \quad \eta_i=\epsilon_i+\sum_{k\in\mathcal{S}} \langle ik\lvert\rvert ik\rangle
\end{equation}
and $\epsilon_i$ are the single particle energies appearing in the one body part of the Hamiltonian Eq.~\eqref{eqham}.
This is equivalent to truncating the self--consistent solution of \eqref{eq_CCD} after the first iteration.

\subsection{Propagator sampling with no time-step error}
\label{sec:expprop}
As we pointed out before, in simulations employing the linear propagator \eqref{eq:prop} raising the dimension of the basis set has a detrimental effect on the efficiency 
of the algorithm since in order to satisfy the bound Eq.~\eqref{eq_bound_on_tau} we are forced to employ
an exceedingly small time step. Moreover, in practice values of $\tau$ much smaller than the
maximum value are usually employed due to the difficulty in obtaining reliable estimates of $E_{max}$ in realistic situations.

To further complicate the scenario, when lattice fixed-node(fixed-phase) methods are employed this maximum value is reduced even further because the diagonal
matrix elements of $P$ gets pushed towards the negative region by the addition of the sign--violating contributions $\sum_{\mathbf{m}} h_{\mathbf{m}\mathbf{n}}$ in Eq.~\eqref{mh2}. 
If this method is used to control the sign--problem additional care has to be devoted in the choice of the time--step, greatly deteriorating the efficiency of the overall scheme.

In a discrete space however we can cope with the problem by using an algorithm firstly introduced by Trivedi and Ceperley~\cite{Trivedi90}, which shares 
similarities with the Domains Green's Function Monte Carlo by Kalos, Levesque, and Verlet~\cite{Kalos74}. The idea is to use directly (meaning sample from) the exponential propagator
\begin{equation}
\label{exp_prop}
P^{exp}(\tau,\mathbf{m},\mathbf{n}) = \langle \mathbf{m} \lvert e^{-\tau(H - E_T)} \rvert \mathbf{n}  \rangle,
\end{equation}
that clearly has no problem with negative diagonal elements. These schemes usually come with the name of {\it continuous--time} evolution.

For simplicity let us forget the sign--problem for the time being and imagine we are working with the positive-definite importance-sampled greens function \eqref{eq:CIMC_IS_prop} 
with $\gamma=0$ and the corresponding Hamiltonian $\widetilde{H}$ which then satisfies
\begin{equation}
\label{eq:hoffdpd}
\widetilde{H}_{\mathbf{m},\mathbf{n}} \leq 0 \quad\quad \forall \; \mathbf{m}\neq \mathbf{n} .
\end{equation}
Furthermore, we will neglect the energy shift $E_T$ since its addition is straightforward.

Recall that the propagator can be written as a product of a stochastic matrix $\widetilde{p}_{\mathbf{m},\mathbf{n}}$ and a weight factor $\widetilde{g}_{\mathbf{n}}$ (cf. Sec.~\ref{sec:cimc}):
\begin{equation}
\widetilde{P}_{\mathbf{m},\mathbf{n}}(\Delta\tau) = \delta_{\mathbf{m},\mathbf{n}}- \Delta  \tau \widetilde{H}_{\mathbf{m},\mathbf{n}} = \widetilde{p}_{\mathbf{m},\mathbf{n}} \widetilde{g}_{\mathbf{n}}
\end{equation}
where the two factors are given by:
\begin{equation}
\begin{split}
\widetilde{p}_{\mathbf{m},\mathbf{n}}&=\frac{\widetilde{P}_{\mathbf{m},\mathbf{n}}(\Delta\tau)}{\widetilde{g}_{\mathbf{n}}},\\
\widetilde{g}_{\mathbf{n}} &= \sum_{\mathbf{m}} \widetilde{P}_{\mathbf{m},\mathbf{n}}(\Delta\tau)=1-\Delta\tau E_L({\mathbf{n}})
\end{split}
\end{equation}
and in the last equation we have used the expression for the local energy
\begin{equation}
\label{eq:elocal}
E_L({\mathbf{n}}) = \frac{\langle \Phi_T \lvert H \rvert \mathbf{n}\rangle}{\langle\Phi_T\vert\mathbf{n}\rangle} = \sum_\mathbf{m} \frac{\Phi_T (\mathbf{m}) \langle\mathbf{m}\lvert H \rvert \mathbf{n}\rangle}{\Phi_T\vert\mathbf{n}\rangle} \equiv \sum_\mathbf{m} \widetilde{H}_{\mathbf{m},\mathbf{n}} .
\end{equation}

The continuous--time limit is recovered by applying $M$ times $\widetilde{P}(\Delta\tau)$ and letting $\Delta \tau \to 0$ while preserving constant the product $\tau=M\Delta\tau$:
\begin{equation}
% \begin{split}
\lim_{M \to \infty} \widetilde{P}_{\mathbf{m},\mathbf{n}}(\tau)^M = \lim_{M \to \infty} \left( 1 - \frac{\tau}{M}\widetilde{H}_{\mathbf{m},\mathbf{n}} \right)^{M}
= \lim_{\Delta\tau \to 0} \left( 1 - \Delta\tau \widetilde{H}_{\mathbf{m},\mathbf{n}} \right)^{\frac{\tau}{\Delta\tau}}
= \langle \mathbf{m} \lvert e^{-\tau \widetilde{H}} \rvert \mathbf{n} \rangle .% = P^{exp}(\tau,\mathbf{m},\mathbf{n}).
% \end{split}
\end{equation}

Now note that if we let $\Delta\tau \to 0$ the probability to make a diagonal move in a single step among the $M$ will accordingly go to $\approx 1$, in fact:
\begin{equation}
% \begin{split}
P_{diag} = \frac{\widetilde{P}_{\mathbf{n},\mathbf{n}}(\Delta\tau)}{\widetilde{g}_{\mathbf{n}}}
= \frac{1-\Delta\tau \widetilde{H}_{\mathbf{n},\mathbf{n}}}{1-\Delta\tau E_L(\mathbf{n})}
\xrightarrow{\Delta\tau \to 0} 1
% \end{split}
\end{equation}
since the local--energy $E_L$ does not depend on the time step but just on the current configuration $\mathbf{n}$. Accordingly, the probability of making $K$ consecutive diagonal moves will be:
\begin{equation}
\begin{split}
P_{diag}^K &=\left( \frac{\widetilde{P}_{\mathbf{n},\mathbf{n}}(\Delta\tau)}{\widetilde{g}_{\mathbf{n}}} \right)^K = \left( \frac{1-\Delta\tau \widetilde{H}_{\mathbf{n},\mathbf{n}}}{1-\Delta\tau E_L(\mathbf{n})} \right)^K\\
&\xrightarrow{K \to \infty} \exp{\left(\tau (E_L(\mathbf{n}) - \widetilde{H}_{\mathbf{n},\mathbf{n}})\right)} = \exp{\left(\tau \widetilde{H}^{off}_{\mathbf{n}}\right)} = f_{\mathbf{n}}(\tau)
\end{split}
\end{equation}
where we have implicitly defined the off--diagonal sum 
\begin{equation}
\label{eq:hoffd}
\widetilde{H}^{off}_{\mathbf{n}} = \sum_{\mathbf{m}\neq \mathbf{n}} \widetilde{H}_{\mathbf{m},\mathbf{n}} <0
\end{equation}
and the inequality holds thanks to Eq.~\eqref{eq:hoffdpd}.

The elapsed time between consecutive off--diagonal moves is therefore distributed as an exponential 
distribution $f_{\mathbf{n}}(\tau)$ with average time given by
\begin{equation}
\int_{0}^{\infty} \tau f_{\mathbf{n}}(\tau) = -\frac{1}{\widetilde{H}^{off}_{\mathbf{n}}} = \left\vert \frac{1}{\widetilde{H}^{off}_{\mathbf{n}}}\right\vert .
\end{equation}
We can then sample the time when the off-diagonal move happens by using a transformation technique: suppose we have a way to sample values $\xi$ from a uniform distribution 
$g(\xi) = \text{const}$, due to conservation of probability the samples $\tau$ drawn from the wanted $f_{\mathbf{n}}(\tau)$ will satisfy:
\begin{equation}
\label{eq_cdfsampling}
\vert f(\tau) d\tau \vert = \vert g(\xi) d\xi \vert \quad \longrightarrow \left\vert \frac{d \xi(\tau)}{d \tau} \right\vert = f_{\mathbf{n}}(\tau)
\end{equation}
where $\tau$ are the samples drawn from the wanted PDF $f_{\mathbf{n}}$. By solving now equation \eqref{eq_cdfsampling} for $\xi(\tau)$ and performing the inversion to
$\tau=\tau(\xi)$ we obtain the following relation
\begin{equation}
\label{sampled_tau}
\tau_{\xi}=\frac{log(\xi)}{\widetilde{H}^{off}_{\mathbf{n}}}.
\end{equation}
that allows to sample exactly from $f_{\mathbf{n}}$ using only samples from a uniform distribution $\xi \in (0,1)$.

Walkers undergoing such random walk accumulate weight during the $K$ diagonal--moves as well as from performing the off--diagonal step. The weight coming from the diagonal
moves is given by
\begin{equation}
w_{\mathbf{n}}=\widetilde{g}_{\mathbf{n}}^K = \left( 1-\Delta\tau E_L(\mathbf{n})\right)^K \xrightarrow{\Delta\tau \to 0} e^{-\tau E_L(\mathbf{n})}.
\end{equation}
For the off--diagonal moves instead we have at least two options for sampling the new state $\rvert \mathbf{m}\rangle$:
\begin{itemize}
 \item heat-bath sampling: 
 \begin{equation}
 \label{eq:heba}
P_1(\mathbf{m},\mathbf{n})=\widetilde{H}_{\mathbf{m},\mathbf{n}}/ \widetilde{H}^{off}_{\mathbf{n}}  
 \end{equation} 
\begin{enumerate}
 \item new configuration $\rvert \mathbf{m}\rangle$ is chosen using the normalized probability $P_1$
 \item the off-diagonal weight would be $w_{\mathbf{m},\mathbf{n}}=1$
\end{enumerate}

 \item uniform sampling: 
 \begin{equation}
P_2(\mathbf{m},\mathbf{n})=1/N_{conn}
\end{equation}
\begin{enumerate}
 \item new configuration $\rvert \mathbf{m}\rangle$ is chosen among the $N_{conn}$ states connected to $\rvert\mathbf{n}\rangle$
 \item reweight the new walker using $w_{\mathbf{m},\mathbf{n}} = P_1(\mathbf{m},\mathbf{n})/P_2(\mathbf{m},\mathbf{n})$
\end{enumerate}
\end{itemize}

The first option is clearly more expensive per iteration than the second since an explicit calculation of the off-diagonal sum $\widetilde{H}^{off}_{\mathbf{n}}$ is 
needed in order to normalize $P_1$. In the uniform sampling case however the weights $w_{\mathbf{m},\mathbf{n}}$ can have large fluctuations
forcing the use of smaller time-steps to keep them under control. In our case since we already need to compute the off-diagonal
sum in order to generate the fixed-phase hamiltonian Eq.~\eqref{mh1} and Eq.~\eqref{mh2} the heat-bath sampling comes with no additional
cost. It is worth noting that other choice can be made that are more efficient when fixed-node(phase) is not employed at all \cite{Holmes16}
or when the transformation that produces $\mathcal{H}_\gamma$ is carried out only approximately \cite{Kolodrubetz12}.

Finally, in order for the measurements along the path to be unbiased we want to define equidistant "time-slices" along the random walk. In 
order to this we simply choose a target time-step $\tau_t$ at the beginning then for each move we first sample a value of $\tau_{\xi}$ 
from Eq.~\eqref{sampled_tau}, if $\tau_{\xi} > \tau_t$ we set $\tau=\tau_t$ and use correspondingly a diagonal move if instead $\tau_{\xi} < \tau_t$ 
we have to sample an off--diagonal move. The process is preformed until the sum of all the sampled $\tau_{\xi}$ reaches the target time $\tau_t$. 
The final algorithm for a single walker at $\rvert \mathbf{n}\rangle$ is then as follows: 
\begin{svgraybox}
\begin{algorithmic} 
\State{EXP\_Move()}
\State{$\tau=\tau_t$}
\Loop
  \State{$E_L(\mathbf{n}) = \sum_\mathbf{m} \widetilde{H}_{\mathbf{m},\mathbf{n}}$}
  \Comment{Eq.~\eqref{eq:elocal}}
  \State{$\widetilde{H}^{off}_\mathbf{n} = E_L(\mathbf{n}) - \widetilde{H}_{\mathbf{n},\mathbf{n}}$}
  \Comment{Eq.~\eqref{eq:hoffd}}
  \State{$\xi = \text{rand}()$}
  \State{$\tau_\xi=log(\xi)/\widetilde{H}^{off}_\mathbf{n}$}
  \Comment{Eq.~\eqref{sampled_tau}}
  \If{$\tau_\xi\geq\tau$}
    \State{$w(\mathbf{n})\to w(\mathbf{n}) \exp{\left(-\tau \;E_L(\mathbf{n})\right)}$}
    \State{\bf exit}
  \EndIf
  \State{$w(\mathbf{n})\to w(\mathbf{n}) \exp{\left(-\tau_\xi \;E_L(\mathbf{n})\right)}$}
  \State{$\tau\to\tau-\tau_\xi$}
  \State{$\mathbf{m} \gets \text{HeatBath}[P_1,\mathbf{n}]$}%(\mathbf{m},\mathbf{n})=\widetilde{H}_{\mathbf{m},\mathbf{n}}/\widetilde{H}^{off}_{\mathbf{n}}]$}
%   \State{Choose new state $\mathbf{m}\neq \mathbf{n}$ according to $P_1(\mathbf{m},\mathbf{n})=\widetilde{H}_{\mathbf{m},\mathbf{n}}/\widetilde{H}^{off}_{\mathbf{n}}$}
  \Comment{Eq.~\eqref{eq:heba}}
  \State{$\mathbf{n}\to\mathbf{m}$}
\EndLoop
\end{algorithmic}
\end{svgraybox}
where the function $HeatBath[P,\mathbf{n}]$ generates a new configuration according to the probability $P$ (eg. Eq.~\eqref{eq:heba}) starting 
from the current state $\mathbf{n}$. In Problem~\ref{prob:heatbath} you will try to devise an implementation of this function.

As a final remark, it is evident that the most expensive part of the algorithm is the computation of the local energy $E_L(\mathbf{n}$ since it will require
a sum over all states connected to $\mathbf{n}$ from the Hamiltonian and for each one $\mathbf{m}$ of these we have to compute both the matrix element of 
the Hamiltonian and the overlap with the trial function $\Psi_T(\mathbf{m})$. The use of symmetries to reduce the size of the sum is thus
of fundamental importance to reach medium-sized systems. We can show this for the simple case of a homogeneous system with only two-body interactions
so that the connected states will be all the possible 2-particle--2-hole excitations that can be obtained from the initial state $\rvert\mathbf{n}\rangle$.
Neglecting the construction of the transformed matrix $\widetilde{H}$, we can then implement the calculation of the local energy as
\begin{svgraybox}
\begin{algorithmic} 
\State{EL\_calc1()}
\State{$E_L=0$}
\For{$i\in occ(\mathbf{n})$}
  \For{$j\in occ(\mathbf{n})$}
    \For{$a\in \mathcal{S}\setminus occ(\mathbf{n})$}
      \For{$b\in \mathcal{S}\setminus occ(\mathbf{n})$}
        \State{$\rvert \mathbf{m} \rangle = a^{\dagger}_a a^{\dagger}_b a_i a_j \rvert \mathbf{n}\rangle$}
        \State{$E_L = E_L + \widetilde{H}_{\mathbf{m},\mathbf{n}}$}
      \EndFor
    \EndFor
  \EndFor
\EndFor
\State{$E_L=E_L/4$}
\end{algorithmic}
\end{svgraybox}
where $occ(\mathbf{n})$ is the set of single-particle states that are occupied in the initial state $\mathbf{n}$. The above algorithm requires $O(N_{occ}^2{\cal N}_s^2)$
evaluations of the Hamiltonian. Many of these are however equivalent to other ones or just zero. For instance all the terms with $i=j$ or $a=b$ give zero
due to the Pauli principle. If we fix an ordering of the single particle orbitals in the many--body states and use anti-symmetrized matrix elements the configurations 
obtained interchanging eg. $i\leftrightarrow j$ are equivalent. Finally if both momentum and spin are conserved, given the triple $(i,j,a)$ there exist only one single
particle state $b$ allowed. An implementation like
\begin{svgraybox}
\begin{algorithmic} 
\State{EL\_calc2()}
\State{$E_L=0$}
\For{$i\in occ(\mathbf{n})$}
  \For{$j<i\in occ(\mathbf{n})$}
    \For{$a\in \mathcal{S}\setminus occ(\mathbf{n})$}
      \State{$b\gets FourthState[i,j,a]$}
      \If{$b\in\mathcal{S}\setminus occ(\mathbf{n})\;$ \bf{and} $\;b<a$}
        \State{$\rvert \mathbf{m} \rangle = a^{\dagger}_a a^{\dagger}_b a_i a_j \rvert \mathbf{n}\rangle$}
        \State{$E_L = E_L + \widetilde{H}_{\mathbf{m},\mathbf{n}}$}
      \EndIf
    \EndFor
  \EndFor
\EndFor
\end{algorithmic}
\end{svgraybox}
will take now only $O(N_{occ}^2{\cal N}_s)$ evaluations of the Hamiltonian at most, and with a reduced prefactor with respect to the previous version. The function $FourthState$ returns 
the only single particle state allowed by simmetry. 

\subsection{Results}
The combination of imaginary time projection, use of importance function derived from CC calculations and no time-step error propagator make up the algorithm that goes under the name of Configuration Interaction Monte Carlo (CIMC).
Actual calculations with CIMC require a substantial amount of CPU time. Here we report some results obtained by making use of a simplified Hamiltonian in which the nucleon-nucleon interaction is described by the Minnesota interaction.
\begin{figure}
	\begin{center}
		\includegraphics[scale=0.5]{Chapter9-figures/cimc_convergence.eps}
	\end{center}
	\caption{Convergence of the CIMC energies as a function of the number of shells used for a periodic cell of 66 neutrons at different densities.}
	\label{fig.cimc_conv}
\end{figure}
The system under investigation is homogeneous pure neutron matter (PNM). In QMC calculations PNM is typically modeled as a periodic system containing A neutrons. The cell size is adjusted in such a way that the average density of the system is $\rho$.

In Fig. \ref{fig.cimc_conv} we show how the computed energy depends on the number of plane wave shells included in the model space. As it can be seen, it is necessary to pay attention to the convergence of th results, which can strongly depend on the specific details of the system. In this case, for instance, one can easily see how convergence is faster when the density is increased. 

\begin{figure}
	\begin{center}
		\includegraphics[scale=0.5]{Chapter9-figures/cimcccd.pdf}
	\end{center}
	\caption{Equation of state of neutron matter modeled as a periodic cell containing A=66 neutrons using the CIMC method and coupled cluster theory with doubes correlations. Single-particle states up to $N_{\mathrm{max}}=36$ have been included.}
	\label{fig.cimc_eos}
\end{figure}

In Fig. \ref{fig.cimc_eos} the energy computed by CIMC shown for the same neutron matter model as a function of the density (the so called "Equation of State" of neutron matter) is compared with the coupled theory results with doubles (CCD) only discussed in the previous chapter. 
In this calculation single-particle states up to $N_{\mathrm{max}}=36$ have been used. The CIMCC and CCD results are converged to the fifth digit as function of 
$N_{\mathrm{max}}$. The agreement between the two methods is at the level of the third digit after the decimal point for neutron matter with the 
Minnesota interaction. This is a striking agreement between such different many-body methods, in particular for larger densities where correlations and contributions from states above and below the Fermi level play a larger role, as seen from the difference between the reference energy and the CIMC and CCD energies.
Most likely, there will be larger differences between different many-body methods when proton correlations are brought in, as well as when more realistic interaction models will be used. Such results will be presented elsewhere. In the next two chapters we will add results using two  additional many-body methods, the in-medium SRG approach described in chapter 10 and the Green's function approach of chapter 11.  

\section{Conclusions and perspectives}

Quantum Monte Carlo methods are still one of the most powerful tools to attack general many body problems, and in particular the many-nucleon problem. Despite the fact that the Fermion sign problem prevents us so far from having strictly exact results for the solution of the Schroedinger equation, the accuracy that can be reached is very high, and in any cases it constitutes the current benchmark.

Another important general feature of QMC calculations is that they provide a very flexible framework in which it is possible to explore from low temperature condensed helium, to trapped fermions, from atoms and molecules ad solid state devices to nuclei and eventually lattice QCD. It is not rare that technical improvements spread across different disciplines, and the development of te method itself is a common ground that is often the subject of interdisciplinary workshops and conferences.  

In the field of nuclear physics it is possible that Fock-space based methods will eventually become the standard. Their main feature is the possibility of dealing with non-local interactions, which makes it possible to extend the use of QMC to the original formulations of $\xi$-EFT potentials, and a whole class of soft-core interactions that so far have never been used in this context. On the other hand, the availability of more and more accurate versions of the AFDMC codes will open the access of accurate studies of the equation of state f neutron and nuclear matter, and of general baryonic matter of extreme importance for astrophysical applications, concerning in particular the physics of neutron stars. The possibility of extending accurate calculations to large $A$ systems is also crucial for understanding the phenomenology of exotic beams.

In this chapter we did not deal with the problem of evaluating excited states and dynamical quantities within a QMC framework. Several methods are nowadays available, mostly based on the evaluation of the Laplace transform of a given response function by means of the calculation of imaginary time correlation functions. Many technical advances have been recently made in this field (see e.g. Refs. \cite{Galli, RoggeroHe, Lovato1,Lovato2}), and the subject is still under very active investigation. 

Finally, the hardest wall to climb remains the solution of the Fermion sign problem. Although there are claims that the problem is NP complete (which is true in general), thereby preventing any solution within standard classical computation, there are hints that many Hamiltonians of interest might admit a viable solution with polynomial scaling in $A$. This problem would definitely deserve more efforts than those that are presently devoted to its solution.  
\section{Problems}
 \begin{prob}
  Evaluate the following integral by means of the Metropolis algorithm
  \[
  I=\int_0^1 e^{x}-1\, dx
  \]
  sampling points from
  \begin{enumerate}
  	\item P(x) = 1 for $x\in[0,1]$
  	\item P(x) = x for $x\in[0,1]$
  \end{enumerate}
  \begin{itemize}
  	\item
  Compare the average and the statistical error for the cases 1) and 2). Which is the best estimate?
  \item
  Try to figure out a way to sample a probability density proportional to $x^n$, and reevaluate the integral
  $I$. How is the convergence and the statistical error behaving by increasing $n$? Try to give an explanation of the result.  
  \end{itemize}
  \end{prob}
  \begin{prob}
  Try to sketch the general proof that for a generic integral $I$ defined as in problem 9.1 the best statistical error is obtained when sampling from a probability density proportional to F(x).
  \end{prob}
  \begin{prob}
  Consider the one-dimensional Hamiltonian:
  \[
  \hat{H}=\frac{1}{2}\frac{d^2}{dx^2}+\frac{1}{2}x^2
  \]
  and consider the parametrized family of trial solutions $\psi(x,\alpha,\beta)=e^{-\alpha^2 x^2}-\beta$. Compute by means of the  Metropolis algorithm the energy and the standard deviation of the energy as a function of $\alpha$ keeping $\beta=0.01$. Is the minimum found at the same value than for $\beta =0$? Why?
  \end{prob}
  \begin{prob}
  	Prove that the propagator defined in the integrand of Eq.(\ref{free_propagator}) is the Green's function of the differential equation (\ref{diffusion_eq}).
  \end{prob}
  \begin{prob}
  \label{prob:egamma}
	Show that the given two fixed-phase energies $E_{a}$ and $E_{b}$ obtained using the hamiltonians $\mathcal{H}_{\gamma}$ defined in Eq.~\eqref{mh1} and Eq.~\eqref{mh2} 
	with $\gamma=a$ and $\gamma=b$ ($a,b\geq 0$) the linear extrapolation to $\gamma=-1$ (remember that $\mathcal{H}_{-1}=H$) is still an upper bound. 
	(Hint: show that $E_\gamma$ is a convex function of the parameter $\gamma$).
  \end{prob}
  \begin{prob}
  \label{prob:heatbath}
        Implement the function $HeatBath[P,\mathbf{n}]$ that appears in the algorithm EXP\_Move() in Sec.~\ref{sec:expprop}.
  \end{prob}
\begin{prob}
In the case of nuclear Hamiltonians spin is not a conserved quantity. Referring to the calculation of the local energy in Sec. 9.6.4, how would you modify the subroutine EL\_calc2 to take into account the non-conservation of spin?
\end{prob}
  

\section*{Appendix}
\addcontentsline{toc}{section}{Appendix}
In this appendix we give the proof of the upper--bound property for the auxiliary hamiltonians $\mathcal{H}_{\gamma}$, defined in Sec.~\ref{subsect:CCDMC-IS}, for
the general complex--hermitian case (see \cite{TenHaaf95} for the original proof in the real symmetric case).
We will concentrate in the simpler case $\gamma=0$ in equations \eqref{mh1} and \eqref{mh2}, extension to the generic $\gamma \geq 0$ is then straightforward. In what follows
 we will use the shorthand $\mathcal{H}_{\gamma=0} \equiv \widetilde{H}$. Let $\Psi(\mathbf{n})$ be any arbitrary wave function, our goal is to show that 
\begin{equation}
\Re [\langle \Psi \lvert \widetilde{H}\rvert\Psi \rangle]\geq\Re\left[ \langle \Psi |H|\Psi \rangle\right]\;. 
\end{equation}

Let us proceed by considering the following difference:
\begin{equation}
\begin{split}
 \Re [\langle \Psi& \lvert \widetilde{H}\rvert\Psi \rangle] -\Re\left[ \langle \Psi |H|\Psi \rangle\right]=  \sum_{\mathbf{m}\mathbf{n}} \Re \left[\Psi^*(\mathbf{m}) (\widetilde{H}_{\mathbf{m}\mathbf{n}}-H_{\mathbf{m}\mathbf{n}})\Psi(\mathbf{n})\right]\\ 
&= \sum_{\mathbf{m}\mathbf{n}} h_{\mathbf{m}\mathbf{n}}  \lvert\Psi(\mathbf{n})\rvert^2 +\sum_{\mathbf{m}\neq \mathbf{n}}  \Re \left[\Psi^*(\mathbf{m}) (\widetilde{H}_{\mathbf{m}\mathbf{n}}-H_{\mathbf{m}\mathbf{n}})\Psi(\mathbf{n})\right]\\
&= \sum_{\mathbf{n}} \sum_{\mathfrak{s}_{\mathbf{m}\mathbf{n}} \neq -} \lvert\Psi(\mathbf{n})\rvert^2  \frac{\Re \left [ \Phi_T^*(\mathbf{m}) H_{\mathbf{m}\mathbf{n}} \Phi_T(\mathbf{n})\right ]}{ \lvert \Phi_T (\mathbf{n}) \rvert ^2} - \Re \left [ \Psi^*(\mathbf{m}) H_{\mathbf{m}\mathbf{n}} \Psi(\mathbf{n})\right ]\\ 
%&= \sum_n \sum_{s_{mn} \neq -} \lvert\Psi(n)\rvert^2  (-p_{mn}) - \Re \left [ \Psi^*(m) \Phi(m) \Phi(m)^{-1} H_{mn} \Phi^*(n)^{-1}\Phi^*(n)\Psi(n)\right ]
\end{split}
\end{equation}
where the second sum is over all ${\mathbf{m}\mathbf{n}}$ pairs such that $\mathfrak{s}_{\mathbf{m}\mathbf{n}}$ of \eqref{CCDMC:ham_sign} is positive--definite. The last term
 can now be rewritten as:
\begin{equation}
\begin{split}
\Re \left [ \Psi^*(\mathbf{m}) H_{\mathbf{m}\mathbf{\mathbf{n}}} \Psi(\mathbf{\mathbf{n}})\right ] &= \Re \left [ \Psi^*(\mathbf{m}) \Phi_T(\mathbf{m}) \Phi_T(\mathbf{m})^{-1} H_{\mathbf{m}\mathbf{n}} \Phi_T^*(\mathbf{n})^{-1}\Phi_T^*(\mathbf{n})\Psi(\mathbf{n})\right ]\\
&= (\Psi^*(\mathbf{m}) \Phi(\mathbf{m}))\Re \left [\Phi_T(\mathbf{m})^{-1} H_{\mathbf{m}\mathbf{n}} \Phi_T^*(\mathbf{n})^{-1}\right ] (\Phi_T^*(\mathbf{n})\Psi(\mathbf{n}))\\
&= (\Psi^*(\mathbf{m}) \Phi(\mathbf{m}))\Re \left [\frac{\Phi_T^*(\mathbf{m})}{\lvert\Phi_T(\mathbf{m})\rvert^2} H_{\mathbf{m}\mathbf{n}} \frac{\Phi_T(\mathbf{n})}{\lvert\Phi_T(\mathbf{n})\rvert^2}\right ] (\Phi_T^*(\mathbf{n})\Psi(\mathbf{n}))\\
\end{split}
\end{equation}
where in the second step we used the fact that by employing a real propagator we are imposing a fixed--phase constraint,  ie $\Im (\Phi_T^*(\mathbf{n})\Psi(\mathbf{n})) = 0$ for every $\mathbf{n}$ explored in the random walk. The equation for the difference becomes:
\begin{equation}
\begin{split}
 \Re [\langle \Psi& \lvert \widetilde{H}\rvert\Psi \rangle] -\Re\left[ \langle \Psi |H|\Psi \rangle\right]=  \sum_{\mathbf{m}\mathbf{n}} \Re \left[\Psi^*(\mathbf{m}) (\widetilde{H}_{\mathbf{m}\mathbf{n}}-H_{\mathbf{m}\mathbf{n}})\Psi(\mathbf{n})\right]\\ 
&= \sum_{\mathbf{n}} \sum_{\mathfrak{s}_{\mathbf{m}\mathbf{n}} \neq -} \frac{\Re \left [ \Phi_T^*(\mathbf{m}) H_{\mathbf{m}\mathbf{n}} \Phi_T(\mathbf{n})\right ]}{ \lvert \Phi_T (\mathbf{n}) \rvert ^2}\left( \lvert\Psi(\mathbf{n})\rvert^2  - \frac{(\Psi^*(\mathbf{m}) \Phi_T(\mathbf{m}))  (\Phi_T^*(\mathbf{n})\Psi(\mathbf{n}))}{\lvert\Phi(\mathbf{m})\rvert^2 }\right) \;.\\
\end{split}
\end{equation}

Using again the fixed--phase constraint (ie. $(\Phi_T^*(\mathbf{n})\Psi(\mathbf{n})) \equiv (\Phi_T(\mathbf{n})\Psi^*(\mathbf{n}))$) we can rewrite the numerator of the second term as:
\begin{equation}
\begin{split}
(\Psi^*(\mathbf{m}) \Phi_T(\mathbf{m}))  (\Phi_T^*(\mathbf{n})\Psi(\mathbf{n})) &= -\frac{1}{2} \big( \lvert \Psi^*(\mathbf{m}) \Phi_T(\mathbf{n}) - \Phi_T^*(\mathbf{m})\Psi(\mathbf{n})\rvert^2 \\
&\quad- \lvert \Phi_T(\mathbf{n})\rvert^2\lvert \Psi(\mathbf{m})\rvert^2 -\lvert \Phi_T(\mathbf{m})\rvert^2\lvert \Psi(\mathbf{n})\rvert^2 \big)\\
\end{split}
\end{equation}
and then we have:
\begin{equation}
\begin{split}
 \Re \left[\langle \Psi \lvert \widetilde{H}\rvert\Psi \rangle\right] & -\Re\left[ \langle \Psi |H|\Psi \rangle\right]=  \sum_{\mathbf{m}\mathbf{n}} \Re \left[\Psi^*(\mathbf{m}) (\widetilde{H}_{\mathbf{m}\mathbf{n}}-H_{\mathbf{m}\mathbf{n}})\Psi(\mathbf{n})\right]\\ 
&= \sum_{\mathbf{n}} \sum_{\mathfrak{s}_{\mathbf{m}\mathbf{n}} \neq -} \frac{\Re \left [ \Phi_T^*(\mathbf{m}) H_{\mathbf{m}\mathbf{n}} \Phi_T(\mathbf{n})\right ]}{ \lvert \Phi_T(\mathbf{n}) \rvert ^2}\bigg( \lvert\Psi(\mathbf{n})\rvert^2  + \frac{ \lvert \Psi^*(\mathbf{m}) \Phi_T(\mathbf{n}) - \Phi_T^*(\mathbf{m})\Psi(\mathbf{n})\rvert^2}{2 \lvert\Phi_T(\mathbf{m})\rvert^2} \\  
&- \frac{ \lvert \Phi_T(\mathbf{n})\rvert^2\lvert \Psi(\mathbf{m})\rvert^2}{2\lvert\Phi_T(\mathbf{m})\rvert^2}- \frac{ \lvert \Psi(\mathbf{n})\rvert^2}{2}\bigg)\\
&= (\mbox{positive terms}) + \sum_{\mathbf{n}} \sum_{\mathfrak{s}_{\mathbf{m}\mathbf{n}} \neq -} \frac{\Re \left [ \Phi_T^*(\mathbf{m}) H_{\mathbf{m}\mathbf{n}} \Phi_T(\mathbf{n})\right ]}{ 2\lvert \Phi_T(\mathbf{n}) \rvert ^2}\left( \lvert\Psi(\mathbf{n})\rvert^2  - \frac{ \lvert \Phi_T(\mathbf{n})\rvert^2\lvert \Psi(\mathbf{m})\rvert^2}{\lvert\Phi_T(\mathbf{m})\rvert^2} \right) \\  
\end{split}
\end{equation}
Now we note that 
\begin{equation*}
\Re \left [ \Phi_T^*(\mathbf{m}) H_{\mathbf{m}\mathbf{n}} \Phi_T(\mathbf{n})\right ] = \Re \left [ \Phi_T^*(\mathbf{n}) H_{\mathbf{n}\mathbf{m}} \Phi_T(\mathbf{m})\right ]
\end{equation*}
for a complex--hermitian hamiltonian, we can then express the sums by allowing only unique $\mathbf{m}\mathbf{n}$ combinations:
\begin{equation}
\begin{split}
 \Re \left[\langle \Psi \lvert \widetilde{H}\rvert\Psi \rangle\right] & -\Re\left[ \langle \Psi |H|\Psi \rangle\right]=  \sum_{\mathbf{m}\mathbf{n}} \Re \left[\Psi^*(\mathbf{m}) (\widetilde{H}_{\mathbf{m}\mathbf{n}}-H_{\mathbf{m}\mathbf{n}})\Psi(\mathbf{n})\right]\\ 
&= (\mbox{positive terms}) + \\
&\sum_{\mathbf{n}} \sum'_{\mathfrak{s}_{\mathbf{m}\mathbf{n}} \neq -} \Re \left [ \Phi_T^*(\mathbf{m}) H_{\mathbf{m}\mathbf{n}} \Phi_T(\mathbf{n})\right ] \bigg( \frac{\lvert\Psi(\mathbf{n})\rvert^2}{2\lvert\Phi_T(\mathbf{n})\rvert^2} + \frac{\lvert\Psi(\mathbf{m})\rvert^2}{2\lvert\Phi_T(\mathbf{m})\rvert^2}-\\
& \frac{ \lvert \Phi_T(\mathbf{n})\rvert^2\lvert \Psi(\mathbf{m})\rvert^2}{2\lvert\Phi_T(\mathbf{m})\rvert^2 \lvert \Phi_T(\mathbf{n})\rvert^2} - \frac{ \lvert \Phi_T(\mathbf{m})\rvert^2\lvert \Psi(\mathbf{n})\rvert^2}{2 \lvert \Phi_T(\mathbf{n})\rvert^2\lvert\Phi_T(\mathbf{m})\rvert^2} \bigg)\\
&=  (\mbox{positive terms})
\end{split}
\end{equation}
which by definition is positive. The extension to the case with $\gamma>0$ is straightforward since we are basically adding a positive constant to the difference.

\begin{thebibliography}{99.}%
\bibitem{Kalos08}
M.H. Kalos and P.A Whitlock, \emph{Monte Carlo Methods-Second Edition}, Wiley-VCH (2008)
\bibitem{Metropolis53}
N. Metropolis1, A. W. Rosenbluth1, M. N. Rosenbluth1, A. H. Teller1 and E. Teller, \emph{Equation of State Calculations by Fast Computing Machines}, J. Chem. Phys. {\bf 21}, 1087 (1953)
\bibitem{Hastings70}
W. K. Hastings, \emph{Monte Carlo sampling methods using Markov chains and their applications}, Biometrika {\bf 57}, 97 (1970)
\bibitem{Cep77}
D. Ceperley, G. V. Chester, and M. H. Kalos, \emph{Monte Carlo simulation of a many-fermion study}, Phys. Rev. B {\bf 16}, 3081 (1977)
\bibitem{Cyrus96}
M.P. Nightingale and C.J. Umrigar,\emph{Monte Carlo Optimization of Trial Wave Functions in Quantum Mechanics and Statistical Mechanics} in \emph{"Recent Advances in Quantum Monte Carlo Methods"}, edited by W.A. Lester, Jr., (World Scientific, Singapore, 1996). 
\bibitem{Toulouse07}
J. Toulouse and C. J. Umrigar, \emph{Optimization of quantum Monte Carlo wave functions by energy minimization}, J. Chem. Phys. {\bf 126}, 084102 (2007)
\bibitem{Sorella01}
S. Sorella, {\em Generalized Lanczos algorithm for variational quantum Monte Carlo}, Phys. Rev. B {\bf 64}, 024512 (2001).
\bibitem{Sarsa00}
A. Sarsa, K. E. Schmidt and W. R. Magro, \emph{Path Integral Ground State Monte Carlo}, J. Chem. Phys. {\bf 113}, 1366 (2000)
\bibitem{Anderson76}
J. B. Anderson, \emph{Quantum chemistry by random walk. H 2 P, H+ 3 D 3h 1 A′1, H2 3Σ+ u , H4 1Σ+ g , Be 1 S}, J. Chem. Phys {\bf 65}, 4121 (1976)
\bibitem{Troyer05}
M. Troyer and U.-J. Wiese, \emph{Computational Complexity and Fundamental Limitations to Fermionic Quantum Monte Carlo Simulations}, Phys. Rev. Lett. {\bf 94}, 170201 (2005)
\bibitem{Kalos00}
M. H. Kalos and F. Pederiva, \emph{Exact Monte Carlo Method for Continuum Fermion Systems}, Phys. Rev. Lett. {\bf 85}, 3547 (2000)
\bibitem{Assaraf07}
R. Assaraf, M. Caffarel and A. Khelif, \emph{The fermion Monte Carlo revisited}, J. Phys. A Math. Theor. {\bf 40}, 1181 (2007)
\bibitem{Ceperley80}
D. M. Ceperley and B. J. Alder, \emph{ Ground State of the Electron Gas by a Stochastic Method}, Phys. Rev. Lett. {\bf 45}, 566 (1980)
\bibitem{Foulkes01}
W. M. C. Foulkes, L. Mitas, R. J. Needs, and G. Rajagopal, \emph{Quantum Monte Carlo Simulations of Solids} Rev. Mod. Phys. 73, 33 (2001)
\bibitem{Ortiz93}
G. Ortiz, D. M. Ceperley, and R. M. Martin, \emph{New stochastic method for systems with broken time-reversal symmetry: 2D fermions in a magnetic field}, Phys. Rev. Lett. {\bf 71}, 2777 (1993)
\bibitem{Zhang03}
S. Zhang and H. Krakauer, \emph{Quantum Monte Carlo Method using Phase-Free Random Walks with Slater Determinants}, Phys. Rev. Lett. {\bf 90}, 136401 (2003)
\bibitem{Booth09}
G. H. Booth, A. J. W. Thom, and A. Alavi, \emph{Fermion Monte Carlo without fixed nodes: a Game of Life, death and annihilation in Slater Determinant space}, J. Chem. Phys. {\bf 131}, 054106 (2009).
\bibitem{Cleland10}
D. Cleland, G. H. Booth, and A. Alavi, \emph{ Survival of the Fittest: Accelerating Convergence in Full Configuration-Interaction Quantum Monte Carlo },  J. Chem. Phys. {\bf 132}, 041103 (2010).
\bibitem{Petruzielo12}
F. R. Petruzielo, A. A. Holmes, H. J. Changlani, M. P.
Nightingale, and C. J. Umrigar, \emph{Semistochastic Projector Monte Carlo Method}, Phys. Rev. Lett. {\bf 109},
230201 (2012).
\bibitem{Booth13}
G. H. Booth, A. Gr\"uneis, G. Kresse, and A. Alavi,\emph{ Towards an exact description of electronic wavefunctions in real solids}, Nature
{\bf 493}, 365 (2013).
\bibitem{Mukherjee13}
A. Mukherjee and Y. Alhassid, \emph{Configuration-interaction Monte Carlo method and its application to the trapped unitary Fermi gas},
 Phys. Rev. A {\bf 88}, 053622 (2013).
\bibitem{Roggero13}
A. Roggero, A. Mukherjee, and F. Pederiva , \emph{Quantum Monte Carlo with Coupled-Cluster wave functions},
 Phys. Rev. B {\bf 88}, 115138 (2013).
\bibitem{Roggero14}
A. Roggero, A. Mukherjee, and F. Pederiva, \emph{Quantum Monte Carlo calculations of neutron matter with non-local chiral interactions },
 Phys. Rev. Lett. {\bf 112}, 221103 (2014).
\bibitem{Trivedi90}
N. Trivedi and D. M. Ceperley, \emph{Ground-state correlations of quantum antiferromagnets: A Green-function Monte Carlo study} ,Phys. Rev. B {\bf 41}, 4552 (1990)
\bibitem{Sorella00}
S. Sorella and L. Capriotti, \emph{Green function Monte Carlo with stochastic reconfiguration: An effective remedy for the sign problem}, Phys. Rev. B {\bf 61}, 2599 (2000)
\bibitem{TenHaaf95}
D. F. B. ten Haaf, H. J. M. van Bemmel, J. M. J. van Leeuwen, W. van Saarloos and D. M. Ceperley, \emph{Proof for an upper bound in fixed-node Monte Carlo for lattice fermions}, Phys. Rev. B {\bf 51}, 13039 (1995)
\bibitem{Rrapaj16}
E. Rrapaj, A. Roggero, and J. W. Holt, \emph{Microscopically constrained mean-field models from chiral nuclear thermodynamics}, Phys. Rev. C {\bf 93}, 065801 (2016)
\bibitem{Kalos74}
M. H. Kalos, D. Levesque, and L. Verlet, \emph{Helium at zero temperature with hard-sphere and other forces}, Phys. Rev. A {\bf 9}, 2178 (1974)
\bibitem{Holmes16}
A. Holmes, H. J. Changlani and C.J. Umrigar, \emph{Efficient heat-bath sampling in Fock space}, J. Chem. Theory Comput. {\bf 12}, 1561 (2016) 
\bibitem{Kolodrubetz12}
M. Kolodrubetz and B. K. Clark, \emph{FCI-QMC approach to the Fermi polaron}, Phys. Rev. B {\bf 86}, 075109 (2012).
\bibitem{Galli}
E. Vitali, M. Rossi, L. Reatto, and D. E. Galli, \emph{Ab initio low-energy dynamics of superfluid and solid $^4$He}, Phys. Rev. B {\bf 82}, 174510 (2010).
\bibitem{RoggeroHe}
A. Roggero, F. Pederiva, and G. Orlandini, \emph{Dynamical structure functions from quantum Monte Carlo calculations of a proper integral transform}, Phys. Rev. B {\bf 88}, 094302 (2013)
\bibitem{Lovato1}
A. Lovato, S. Gandolfi, R. Butler, J. Carlson, E. Lusk, S. C. Pieper, and R. Schiavilla, \emph{Charge Form Factor and Sum Rules of Electromagnetic Response Functions in $^{12}$C}, Phys. Rev. Lett. {\bf 111}, 092501, 2013
\bibitem{Lovato2}
A. Lovato, S. Gandolfi, J. Carlson, S. C. Pieper, and R. Schiavilla, \emph{Electromagnetic and neutral-weak response functions of $^4$He and $^{12}$C}, Phys. Rev. C {\bf 91},062501(R) (2015)
\end{thebibliography}







\label{chap:chapter9}

\title{In-medium SRG approaches to infinite nuclear matter}
\author{Scott K.~Bogner, Heiko Hergert, Justin Leitz, Titus Morris, Sam Novario, Nathan Parzuchowski, and Fei Yuan}
\institute{Scott Bogner  \at Department of Physics and Astronomy and National Superconducting Cyclotron Laboratory, Michigan State University, East Lansing, Michigan USA, \email{bogner@nscl.msu.edu}, \and Heiko Hergert  \at Department of Physics and Astronomy and National Superconducting Cyclotron Laboratory, Michigan State University, East Lansing, Michigan USA, \email{hergert@nscl.msu.edu}, \and Justin G.~Lietz \at Department of Physics and Astronomy and National Superconducting Cyclotron Laboratory, Michigan State University, East Lansing, Michigan,  USA, \email{lietz@nscl.msu.edu}, \and Titus Morris  \at Department of Physics and Astronomy and National Superconducting Cyclotron Laboratory, Michigan State University, East Lansing, Michigan USA, \email{morrist@nscl.msu.edu}, \and Samuel Novario \at Department of Physics and Astronomy and National Superconducting Cyclotron Laboratory, Michigan State University, East Lansing, Michigan,  USA, \email{novarios@nscl.msu.edu},\and Nathan Parzuchowski  \at Department of Physics and Astronomy and National Superconducting Cyclotron Laboratory, Michigan State University, East Lansing, Michigan USA, \email{parzuchowski@frib.msu.edu}, \and Fei Yuan  \at Department of Physics and Astronomy and National Superconducting Cyclotron Laboratory, Michigan State University, East Lansing, Michigan USA, \email{yuan@nscl.msu.edu}}
\maketitle
\abstract{We present applications  of the In-Medium Similarity
Renormalization Group (IM-SRG) method to studies of infinite nuclear matter. 
The IM-SRG method employs a continuous unitary transformation of the
many-body Hamiltonian to decouple the ground state from all
excitations, thereby solving the many-body problem. Starting from a
pedagogical introduction of the underlying concepts, the IM-SRG flow 
equations are developed and we study different IM-SRG generators that
achieve the desired decoupling, and how they affect the details of the IM-SRG results. We compare with the coupled cluster theory results of
chapter 8, the Monte Carlo results of chapter 9 and the Green's function results of chapter 11.}

\section{Introduction}

The Similarity Renormalization Group (SRG) method was first formulated by
Wegner \cite{wegner1994} and Glazek and Wilson \cite{glazek1993}
to study condensed matter systems and light-front quantum field
theories, respectively.  From a mathematical point of view, the
philosophy behind the SRG is to render the Hamiltonian $\hat{H}(s)$
diagonal via a continuous unitary transformation
\begin{equation}\label{eq:cut}
  \hat{H}(s)=\hat{U}(s)\hat{H}(0)\hat{U}^{\dagger}(s)\,,
\end{equation}
where $H(s=0)$ is the starting Hamiltonian and $s$ denotes the so-called flow
parameter, for reasons that will become apparent shortly. In practice, the demand 
for strict diagonality is usually relaxed to requiring band- or block-diagonality 
of the Hamiltonian matrix in a chosen basis
These specific cases are realized in nuclear
physics applications, where the SRG is used to decouple momentum or
energy scales, and thereby render the nuclear Hamiltonian more
suitable for \emph{ab initio} many-body calculations \cite{bogner2007,bogner2010,morris2015,bogner2016}.

In the following sections we outline the basic ingredients behind the
SRG method, with an emphasis on applications to infinite nuclear
matter, coupling our final results with those of chapter 8, 9 and
11. The next section presents the essential philosophy of the method,
which in practice means to solve Schr\"odinger's equation for a
many-body system in terms of coupled ordinary differential equations
(ODEs). We present also two simple demonstrations of the SRG method with the true vacuum as reference state. This corresponds in principle to finding all eigenvalues of a matrix by the solution of coupled differential equations (ODEs). 
However, compared with standard eigenvalue solvers, this is a less efficient way of finding the full spectrum of a matrix. 
Moreover, with growing dimensionalities, it becomes quickly impossible to store the matrix elements and solve the coupled ODEs. 
Introducing a reference vacuum as done in chapters 8 and 9, allows us to find approximate solutions to properties  like the  ground state energy and the correlation energy, or selected excited states. This leads us to the
to the in-medium SRG (IM-SRG) approach discussed in this chapter. 
We apply thereafter the IM-SRG approach to the simplified pairing  model (presented in chapter 8) and 
to pure neutron matter. We compare our IM-SRG results to those obtained from coupled cluster theory and many-body perturbation theory of  chapter 8 and quantum Monte Carlo of the previous chapter.  In the final section, we present our conclusions and point  to further
perspectives and research directions.

\section{Similarity Renormalization Group method}

We give here a brief overview of the SRG method applied to two simple systems,  a $2\times 2$ matrix and its pertinent eigenvalue problem and the simple pairing model discussed in chapter 8, the latter as a numerical case study in this chapter. 

In the similarity renormalization group (SRG) method, one performs a
continuous sequence of unitary transformations on a Hamiltonian
operator $\hat H$ to evolve it into a more ``amenable'' form, which
generally implies decoupling a small model space of interest from its
complement.  The sequence is parameterized by a continuous variable
$s$ known as the \textit{flow parameter}, which by convention we
define to be $0$ at the beginning of the sequence.  Upon reaching $s$,
the transformed Hamiltonian is given by
\begin{align*}
  \tilde H(s) \equiv \hat U(s) \hat H \hat U(s)
\end{align*}
where $U(s)$ describes the product of all such transformations since $s = 0$.
At every step $s'$, the differential unitary transformation is described by
the exponential of an antihermitian operator $\eta(s)$, known as the
\textit{generator} of the transformation.  When ``integrated'' as a product,
the full transformation $\hat U(s)$ is recovered
\begin{align*}
  \hat U(s) = \lim_{\Delta s' \to 0} \prod_{s' = 0}^{s' = s}
  \mathrm e^{\hat \eta(s') \Delta s'}
\end{align*}
The evolution of the Hamiltonian itself is governed by a differential
equation, commonly referred to as the flow equation:
\begin{gather} \label{eq:imsrgode}
  \frac{d}{d s} \tilde H(s) = [\hat \eta(s), \tilde H(s)]
\end{gather}
which allows $\tilde H(s)$ to be evaluated without explicitly constructing the
full transformation $\hat U(s)$.

 Introducing a flow parameter $s$, there exits a unitary
 transformation $U_s$, such that
\begin{equation}
 \hat{H}_s = U_s^\dagger \hat{H} U_s \equiv \hat{H}^{\rm d}_s + \hat{H}^{\rm od}_s,
\end{equation}
with the relations $U_{s=0} = \mathbf{1}$, and $\hat{H}_{s= 0} =
\hat{H}$.  The transformation $U_s$ is parametrized as
\[
U_s = T_s \exp \left(\int_0^s \! ds'\hat{\eta}_{s'} \right),
\]
where the anti-hermitian operator $\hat{\eta}_s$ serves as generator
of the transformation. With $T_s$ we denote $s$-ordering, which is
defined equivalently to usual time-ordering.  Taking the derivative of
$\hat{H}_s$ with respect to $s$ gives
\begin{equation}
 \frac{d \hat{H}_s}{ds} = \frac{d U_s}{ds}\hat{H} U_s^\dagger + U_s
 \hat{H} \frac{d U_s^\dagger}{ds}.
\label{eq:flowlong}
\end{equation}
Utilizing that for our particular form of $U_s$, we have that 
\begin{equation}
\hat{\eta}_s = \frac{d U_s}{ds} U_s^\dagger = - U_s \frac{d
  U_s^\dagger}{ds} = -\hat{\eta}_s,
\label{eq:eta}
\end{equation} 
we obtain that 
\begin{equation} 
\frac{d \hat{H}_s}{ds} = \hat{\eta}_s \hat{H}_s
- \hat{H}_s \hat{\eta}_s = \left[\hat{\eta}_s, \hat{H}_s \right].
\label{eq:flowEquations}
\end{equation}

This is the key expression of the SRG method, describing the flow
of the Hamiltonian.  The specific unitary transformation is determined
by the choice of $\hat{\eta}_s$.  Through different choices of
$\hat{\eta}_s$, the SRG evolution can be adapted to the features of a
particular problem.
Naively, one could try to solve the flow equation \eqref{eq:flowEquations} by
choosing a suitable basis of the many-body Hilbert space and turning
Eq.~\eqref{eq:flowEquations} into a matrix differential equation, but such an
approach would ultimately amount to a diagonalization of the many-body
Hamiltonian. To make matters worse, implementing the flow means we
would deal with the Hamiltonian's full spectrum rather than just some
extremal eigenvalues that can be extracted efficiently in
state-of-the-art, large-scale large scale diagonalization approaches based on iterative techniques
\cite{golubvaloan1996}.

\subsection{Simple demonstration of the SRG method}

In order to get a better understanding of the SRG method, let us first consider  
a simple $2\times 2$ matrix and compare \eqref{eq:flowEquations} with standard diagonalization algorithms like 
Jacobi's rotation method, see for example Ref.~\cite{golubvanloan1996}.

We define a  symmetric matrix  $H\in {\mathbb{R}}^{2\times 2}$
\[ 
H = \begin{bmatrix} H_{11} & H_{12} \\ H_{21} & H_{22}\end{bmatrix}. 
\]
The standard Jacobi rotation method allows us to find the eigenvalues via the orthogonal matrix
$\mathbf{U}$ 
\[ 
\mathbf{U} = \begin{bmatrix} c & s \\ -s & c
\end{bmatrix}, 
\]
with $c = \cos \gamma$ and $s = \sin \gamma$. We have then that  $H' = UHU^T$ is diagonal. 

To transform the matrix into one with zero  nondiagonal matrix $H'$ we need to solve
\[ 
(H_{22} - H_{11})cs + H_{12}(c^2 - s^2) = 0, 
\]
and using $c^2-s^2 = \cos(2\gamma)$ and $cs = \sin(2\gamma)/2$
this is equivalent with 
\[ \tan(2\gamma) = \frac{2 H_{12}}{H_{11}-H_{22}}. \]
Solving the equation we have
\begin{equation} 
\gamma = \frac{1}{2} \tan^{-1} \left( \frac{2H_{12}}{H_{11}-H_{22}}
\right) + \frac{k\pi}{2}, \quad k=\ldots,-1,0,1,\ldots, \label{eq:0} 
\end{equation}
where $k\pi/2$ is added due to the periodicity of the $\tan$ function.

Note that  $k=0$ gives a diagonal matrix on the form
\begin{equation} 
H'_{k=0} = \begin{bmatrix} \lambda_1 & 0 \\ 0 & \lambda_2 \end{bmatrix},
\label{eq:1} 
\end{equation}
while  $k=1$ changes the diagonal elements  
\begin{equation} 
H'_{k=1} = \begin{bmatrix} \lambda_2 & 0 \\ 0 & \lambda_1 \end{bmatrix}.
\label{eq:2}
\end{equation}

We switch now to the SRG method and 
let $H(s) = T + V(s)$, where $T= \mathrm{diag}(E_1,E_2)$ is diagonal. We want to solve the ordinary differential equations (ODEs) 
for $U(s)$ using the flow parameter $s$. 
We are searching for a transformation 
\[ 
H(s) = U(s)H(0)U(s)^T, 
\]
and we have
\[ 
\frac{d}{ds} H(s) = [\eta(s),H(s)],  \quad \eta(s) = [T,H(s)], 
\]
which gives
\[ 
\frac{d}{ds} U(s) = \eta(s) U(s). 
\]
Note that $\eta(s)^T = -\eta(s)$, that is
\[ 
\eta(s) = \begin{bmatrix} 0 & a(s) \\ -a(s) & 0 \end{bmatrix}. 
\]

To make the link with the Jacobi transformation method
we can parametrize $U(s)$ as
\[ 
U = \begin{bmatrix} \cos(\gamma(s)) & \sin(\gamma(s)) \\ -\sin(\gamma(s)) & \cos(\gamma(s)) \end{bmatrix}. 
\]
Setting up $\eta(s)U(s)$ og $\frac{d}{ds} U(s)$ we arrive at 
\[ 
\frac{d}{ds} \gamma(s) = a(s). 
\]
We apply this to a simple Hamiltonian $H(s)$ (and $T$ and $V(s)$)  defined in term of Pauli matrices  as
\[ 
T = \mathcal{E} I + \Omega \sigma_z, \quad \mathcal{E} = \frac{E_1+ E_2}{2}, \; \Omega = \frac{E_1-E_2}{2}, 
\]
and
\[ 
V(s) = c I \omega_z(s)\sigma_z + \omega_x(s)\sigma_x, 
\]
with $c = (V_{11}+V_{22})/2$, $\omega_z(s) = (V_{11}-V_{22})/2$ and $\omega_x(s) = V_12$. The quantities depend on
$s$. 

We obtain then
\[ \eta(s) = [T, H(s)] = 2\Omega\omega_x(s)\sigma_y, \]
where we have used $\sigma_i\sigma_j = i\epsilon_{ijk}\sigma_k$.
It results in
\[ a(s) = 2\Omega \omega_x(s), \]
and $\gamma(s)$,
\begin{equation} \frac{d}{ds} \gamma(s) = 2\Omega\omega_x(s). \label{eq:3}\end{equation}
We introduce next the  variables $\omega(s)$ and $\theta(s)$ instead of
$\omega_x(s)$ and $\omega_y(s)$. This allows us to write the ODE 
for $\theta(s)$ as
\begin{equation} 
\frac{d}{ds} \theta(s) = -4\Omega\omega \sin(\theta(s)), \label{eq:4}
\end{equation}
and noting that
\[ 
\frac{d}{dx} \ln \tan \frac{x}{2} = \frac{1}{\sin x}, 
\]
we get
\[ 
\tan\left(\frac{\theta(s)}{2}\right) = \exp{-4\Omega\omega s} \tan\left(
  \frac{\theta(0)}{2}\right). 
\]
If $\Omega<0$ ($E_1-E_2<0$) we obtain  $\theta(s)\rightarrow \pi$, and if 
$\Omega>0$ we get  $\theta(s)\rightarrow 0$ and exponential convergence. In a more 
compact form we have
\begin{equation} \theta(s) \rightarrow \pi \vartheta(E_2 - E_1),
  \label{eq:5} \end{equation}
where $\vartheta(x)$ is the  step function.

If we now recall that 
\[ \tan \theta(0) = \frac{2 V_{12}(0) }{[E_1 + V_{11}(0)]- [E_2 +
  V_{22}(0)]},
\]
we can make the link with the Jacobi rotation method. Comparing the  ODE of (\ref{eq:3}) with 
(\ref{eq:4}), we have
\[ \frac{d}{ds} \gamma(s) = -\frac{1}{2} \frac{d}{ds} \theta(s). \]
The initial condition is 
$\gamma(0) = 0$. Integrating we obtain 
\[ \gamma(s) = \frac{1}{2}\theta(0) - \frac{1}{2}\theta(s), \]
and using (\ref{eq:5}) results in 
\[ \gamma(s) \rightarrow \frac{1}{2}\theta(0) -
\frac{\pi}{2}\vartheta(E_2-E_1). \]

From this we see that the flow equation selects the solution of (\ref{eq:0}) with
$k=0$ and $k=-1$, depending on whether  $E_1<E_2$ or $E_2<E_1$.

The flow equation yields $H(s)$ as a continuous function of $s$ and the solution
$H(\infty)$ results in the eigenvalues sorted the same way as
$T = \mathrm{diag}(E_1,E_2)$. The shift $\pi/2$ is connected with this; if $k=0$ in 
(\ref{eq:0}) gives the wrong sequence, we choose $k=-1$ instead!

Let us now apply the above flow equations to the simple pairing model discussed in chapter 8. 

\subsection{The pairing model}
In chapter 8 we presented a simple pairing model, which for the case of four doubly degenerate single-particle states and 
four fermions resulted in the following Hamiltonian matrix
 \[
  H = \begin{bmatrix}
  2\delta -g & -g/2 & -g/2 & -g/2 & -g/2 & 0 \\ -g/2 & 4\delta -g &
  -g/2 & -g/2 & -0 & -g/2 \\ -g/2 & -g/2 & 6\delta -g & 0 & -g/2 &
  -g/2 \\ -g/2 & -g/2 & 0 & 6\delta-g & -g/2 & -g/2 \\ -g/2 & 0 & -g/2
  & -g/2 & 8\delta-g & -g/2 \\ 0 & -g/2 & -g/2 & -g/2 & -g/2 &
  10\delta -g
  \end{bmatrix}
  \]
where $g$ represents the strength of the two-body interaction and $\delta$ is a constant that defines the single-particle energies, see Eqs.~(\ref{eq:sppairing}) and 
(\ref{eq:intpairing}) and problem \ref{problem:pairingmodel}.  The above Hamiltonian matrix represents the case where no pairs of fermions are broken, resulting in six possible Slater determinats that span the actual model space.

We restate the equations for the various parts of the Hamiltonian here, namely
 \[
  \hat{H}_0 = \xi \sum_{p \sigma} (p-1) a^{\dagger}_{p \sigma} a_{p\sigma},
  \]
for the one-body part
and 
  \[
  \hat{V} = -\frac{1}{2}g \sum_{pq} a^{\dagger}_{p+}a^{\dagger}_{p-}a_{q-}a_{q+},
  \]
for the two-body interaction. The parameters $\xi$ and $g$ are constants.
The single-particle operator $\hat{H}_0$ defines the operator $\hat{T}$ above. We will let $\hat{V}$ have an explicit $s$ dependence. 
In order to solve the coupled ODEs, we use the ODE solver developed by Gordon and Shampine, see for example Ref.~\cite{shampine1976}. 
As we let $s$ evolve, the Hamiltonian matrix becomes more and more diagonal. The code to solve the coupled ODEs is at
\url{https://github.com/ManyBodyPhysics/LectureNotesPhysics/blob/master/doc/src/Chapter10-programs/f95}. For $s=$ and $g=-0.5$ and $\delta=1$
we obtain 

1.42704949,   3.48610223,  11.55722946,   9.53080786,
         7.47925918,   5.51955178]


0.142705E+01    -.413102-179    0.000000E+00    0.000000E+00    0.516773-184    -.652307E-07
    -.413102-179    0.348610E+01    -.556534-177    0.000000E+00    0.000000E+00    -.101304-186
    0.000000E+00    -.556534-177    0.551955E+01    -.119801-173    0.000000E+00    0.000000E+00
    0.000000E+00    0.000000E+00    -.119801-173    0.747926E+01    -.168573-178    0.000000E+00
    0.516773-184    0.000000E+00    0.000000E+00    -.168573-178    0.953081E+01    -.209242-176
    -.652307E-07    -.101304-186    0.000000E+00    0.000000E+00    -.209242-176    0.115572E+02

with s=0.1

with s =1.0

0.142706E+01    -.473950E-02    -.213293E-07    0.160910E-10    -.139603E-10    0.127066E-07
    -.473950E-02    0.348610E+01    -.401318E-02    -.274470E-07    -.103994E-11    0.174969E-10
    -.213293E-07    -.401318E-02    0.551954E+01    -.393739E-05    -.278160E-07    -.129686E-10
    0.160910E-10    -.274470E-07    -.393739E-05    0.747927E+01    -.437695E-02    -.204824E-07
    -.139603E-10    -.103994E-11    -.278160E-07    -.437695E-02    0.953081E+01    -.375229E-02
    0.127066E-07    0.174969E-10    -.129686E-10    -.204824E-07    -.375229E-02    0.115572E+02



As we can from this study, we solve the flow equations with
respect to the physical vacuum state by setting up the full  Hamiltonian matrix in
and solve Eq.~(\ref{eq:flowEquations}) as a set of coupled
first-order differential equations. However, since the size of the
problem grows enormously with the number of particles and the size of
the model space, the applicability of this free-space SRG method is
restricted to comparatively small systems. This leads us to approximations to the solution of the full set of coupled ODEs. We discuss this in the next section. 


\section{The In-Medium SRG approach}



Instead of performing SRG in free space, the evolution can be done at
finite density, that is directly in the $A$-body system
\cite{kehrein2006}. This approach has recently been applied very
successfully in nuclear physics, see the recent review of Bogner, Hergert, Morris and collaborators 
\cite{hergert2016},
and is called in-medium SRG (IM-SRG). The method allows the evolution
of $3,...,A$-body operators using only two-body machinery, with the
simplifications arising from the use of normal-ordering with respect
to a reference state.

In our case, we assume that the problem can be modelled by a
Hamiltonian containing maximally two-body interaction, as done in chapters 8 and 9. In
second-quantized form, we defined in chapter 8 the normal-ordered Hamiltonian as
\[
\hat{H}_N = \sum_{pq} \langle p|\hat{h}_0|q\rangle a^\dagger_p
a_q+\frac{1}{4} \sum_{pqrs} \langle pq|\hat{v}|rs\rangle a^\dagger_p a^\dagger_q a_s a_r+\sum_{pq,i\le F}
\langle pi|\hat{v}|qi\rangle \left\{a^\dagger_p a_q.
\]
In chapter 8 we rewrote the normal-ordered Hamiltonian 
in terms of a new one-body operator and a two-body operator
\[
\hat{H}_N=\hat{F}_N+\hat{V}_N,
\]
with
\[
\hat{F}_N=\sum_{pq} \langle p|\hat{f}|q\rangle a^\dagger_pa_q,
\]
where
\[
\langle p|\hat{f}|q\rangle= \langle p|\hat{h}_0|q\rangle +\sum_{i\le F}
\langle pi|\hat{v}|qi\rangle.
\]
The last term on the right hand side represents a medium modification
to the single-particle Hamiltonian due to the two-body interaction.
Finally, the two-body interaction is given by
\[
\hat{V}_N = \frac{1}{4} \sum_{pqrs} \langle pq|\hat{v}|rs\rangle a^\dagger_p a^\dagger_q a_s a_r.
\]
In the equations below we will use the following shorthand notation for the matrix elements
\[
f_{pq}=\langle p|\hat{f}|q\rangle,
\]
and 
\[
v_{pqrs}= \langle pq|\hat{v}|rs\rangle.
\]
As discussed in chapter 8, the indices $i,j,k,...$
denote hole states below the Fermi level, indices $a,b,c,...$
particles states above the Fermi level, and indices $ p,q,r,...$ can
be used for both particle and hole states. As reference state
$|\Phi_0\rangle$ we choose the ground state of the non-interacting
system, where all single-particle orbitals below the Fermi level are
occupied.

Integrating the flow equations~(\ref{eq:flowEquations}), we face one
of the major challenges of the SRG method, namely the generation of
higher and higher order interaction terms during the flow. With each
evaluation of the commutator, the Hamiltonian gains terms of higher
order, and these induced contributions will in subsequent integration
steps contribute to terms of lower order. In principle, this continues
to infinity. 
To make the method computationally possible, we have to
close the IM-SRG flow equations, suggesting that we are forced to
truncate the equations to a certain order. We choose to truncate both
$\hat{H}_s$ and $\hat{\eta}_s$ at the two-body level, an approach
which is referred to as IM-SRG(2).  This normal-ordered two-body
approximation seems to be sufficient in many cases and has yielded
excellent results for several nuclei
\cite{tsukiyama2011,tsukiyama2012,hergert2016}.

The
commutator in the flow equations (\ref{eq:flowEquations}) guarantees
that the IM-SRG wave function $U_s^\dagger|\Phi\rangle$ can be
expanded in terms of linked diagrams only
\cite{shavittbartlett2009,hergert2016}, which suggests that IM-SRG
is size-extensive. Regarding the quality of the SRG results, it means
that the error introduced by truncating the many-body expansions
scales linearly with the number of particles~$N$.

With this
truncation, the generator $\hat{\eta}$ can be written as
\[
\hat{\eta} = \sum_{pq} \eta_{pq}^{(1)} \lbrace  a_p^{\dagger}a_q\rbrace  +
\frac{1}{4}\sum\limits_{pqrs}\eta_{pqrs}^{(2)} \lbrace  a_p^{\dagger}a_q^{\dagger}a_s
a_r \rbrace ,
\]
where $\eta_{pq}^{(1)}$ and $ \eta_{pqrs}^{(2)}$ are the one- and
two-body elements, respectively. Making use of the permutation
operator $ \hat{P}_{pq}f(p,q) = f(q,p)$ defined in chapter 8, the IM-SRG(2) flow equations
are given by
\begin{widetext}
\begin{align}
\frac{d E_0}{ds} &= \sum_{ia}\left( \eta_{ia}^{(1)}f_{ai} -
\eta_{ai}^{(1)}f_{ia}\right) +
\frac{1}{2}\sum_{ijab}\eta_{ijab}^{(2)}v_{abij},
\label{eq:flow1}\\
\frac{d f_{pq}}{ds}&= \sum_r \left(\eta_{pr}^{(1)}f_{rq} +
\eta_{qr}^{(1)}f_{rp}\right) + \sum_{ia}\left(1-\hat{P}_{ia}\right)\lb
\eta_{ia}^{(1)}v_{apiq} - f_{ia}\eta_{apiq}^{(2)} \right)\notag \\ &
+\frac{1}{2} \sum_{aij} \left(1+\hat{P}_{pq}\rb
\eta_{apij}^{(2)}v_{ijaq} + \frac{1}{2}\sum_{abi}\left(1+\hat{P}_{pq}\rb
\eta_{ipab}^{(2)}v_{abiq},
\label{eq:flow2}
\end{align}
\begin{align}
\frac{d v_{pqrs}}{ds} &= \sum_t \left(1-\hat{P}_{pq} \right)\lb
\eta_{pt}^{(1)}v_{tqrs}-f_{pt}\eta_{tqrs}^{(2)}\right)-\sum_t \lb
1-\hat{P}_{rs} \right)\left(\eta_{tr}^{(1)} v_{pqts} - f_{tr}
\eta_{pqts}^{(2)}\right)\notag \\ & +\frac{1}{2}\sum_{ab}
\lb\eta_{pqab}^{(2)} v_{abrs} - v_{pqab}\eta_{abrs}^{(2)}\right)-
\frac{1}{2}\sum_{ij} \lb\eta_{pqij}^{(2)} v_{ijrs} -
v_{pqij}\eta_{ijrs}^{(2)}\right)\notag \\ & -\sum_{ia} \left(1-
\hat{P}_{ia} \right)\left(1-\hat{P}_{pq}\right)\left(1-\hat{P}_{rs} \rb
\eta_{aqis}^{(2)}v_{ipar}.
\label{eq:flow3}
\end{align}
\end{widetext}
Note that for brevity, we have skipped the explicit $s$-dependence in the equations.

\subsection*{Choice of generator}
To determine the specific unitary transformation, one needs to specify
the generator~$\hat{\eta}$. Through different choices, the SRG flow
can be adapted to the features of a particular problem.\\


\paragraph{Wegner's canonical generator}
The original choice, suggested by Wegner \cite{PhysRepWegner0}, reads
\begin{equation} 
\hat{\eta} = \left[ \hat{H}^{\rm d}, \hat{H}^{\rm od} \right] = \left[ \hat{H}^{\rm d}, \hat{H}
  \right].
\label{eq:etaWegner}
\end{equation} 
As commutator between two Hermitian operators, $\hat{\eta}$
fulfils the criterion of antihermiticity and can be shown to suppress
the off-diagonal matrix elements \cite{kehrein2006}. In general,
matrix elements far off the diagonal, where the Hamiltonian connects
states with large energy differences, are suppressed much faster than
elements closer to the diagonal. 

Evaluating the commutator, we get
for the one- and two-body elements
\begin{align*}
\eta_{pq}^{(1)} = & \sum_{r}\left(f_{pr}^d f_{rq} - f_{pr} f_{rq}^d \rb
+ f_{pq} v_{qppq}^d \left(n_q - n_p \right)\\ \eta_{pqrs}^{(2)} = & -
\sum_t \left\lbrace \left(1-\hat{P}_{pq} \right)f_{pt} v_{tqrs}^d - \left(1 -
\hat{P}_{rs} \right)f_{tr} v_{pqts}^d \right\rbrace \notag \\ & + \sum_t
\left\lbrace  \left(1 - \hat{P}_{pq} \right)f_{pt}^d v_{tqrs} - \left(1 -
\hat{P}_{rs} \right)f_{tr}^d v_{pqts} \right\rbrace  \notag \\ & +
\frac{1}{2} \sum_{tu} (1 - n_t - n_u) \left(v_{pqtu}^d v_{turs} -
v_{pqtu} v_{turs}^d \right)\notag \\ & + \sum_{tu} \left(n_t - n_u \right)\lb
1 - \hat{P}_{pq} \right)\left(1 - \hat{P}_{rs} \right)v_{tpur}^d v_{uqts},
\end{align*}
where we use the standard notation
\[
n_p = \begin{cases} 1, & \mbox{if } p< \epsilon_F \quad(p\;\mbox{is
    hole state})\\ 0, &\mbox{if } p> \epsilon_F \quad(p\;\mbox{is
    particle state})\\
\end{cases}.
\]

\paragraph{White's generator}
Apart from this canonical generator, there exist several other ones in
literature. One of them is White's choice
\cite{white2002}, which makes numerical approaches much
more efficient.  The problem with Wegner's generator are the widely
varying decaying speeds of the elements, removing first terms with
large energy differences and then subsequently those ones with smaller
energy separations.  That way the flow equations become a stiff set of
coupled differential equations, which often gets numerically
unstable.\\ White takes an alternative approach, which is especially
suited for problems where one is interested in the ground state of a
system. Instead of driving all off-diagonal elements of the
Hamiltonian to zero, he focuses solely on those ones that are
connected to the reference state $|\Phi_0\rangle$, aiming to decouple
the reference state from the remaining Hamiltonian. With a suitable
transformation, the elements get similar decaying speeds, which solves
the problem of stiffness of the flow equations.  The generator is
explicitly constructed the following way \cite{white2002}
\begin{align}
\hat{\eta} &= \sum_{ai} \frac{f_{ai}}{f_a-f_i-v_{aiai}}\lbrace a_a^{\dagger}a_i\rbrace -\text{hc} \notag \\ & + \sum_{abij}
\frac{v_{abij}}{f_a+f_b-f_i-f_j+A_{abij}}\lbrace a_a^{\dagger}a_b^{\dagger}a_j
a_i\rbrace - \text{hc},
\label{eq:WhiteFull}
\end{align}
with $f_p \equiv f_{pp}$, 'hc' denoting the Hermitian conjugate, and
\[
A_{abij} = v_{abab} + v_{ijij} - v_{aiai} - v_{ajaj} - v_{bibi} -
v_{bjbj}.
\label{eq:White7}
\]
Compared to Wegner's canonical generator, where the final flow
equations involve third order powers of the $f$- and $v$-elements,
these elements contribute only linearly with White's generator, which
results in much better numerical properties.


{\bf add more? Magnus? derivation of the equations?}




\section{In-medium SRG studies of the pairing model and infinite neutron matter}

\section{Conclusions and perspectives}


\begin{acknowledgement}
This work was supported by NSF Grant No.~PHY-1404159 (Michigan State University).
\end{acknowledgement}

\begin{thebibliography}{99}
\bibitem{wegner1994}
\bibitem{glazek1993}
\bibitem{bogner2007}
\bibitem{bogner2010}
\bibitem{morris2015}
\bibitem{bogner2016}
\bibitem{navratil2000}
\bibitem{barrett2013}
\bibitem{golubvanloan1996}
\bibitem{weinberg1996}
\bibitem{tsukiyama2011}
\bibitem{tsukiyama2012}
\bibitem{hergert2013}
\bibitem{day1967}
\bibitem{brandow1967}
\bibitem{fetter2003}
\bibitem{shavittbartlett2009}
\bibitem{hagen2014}
\bibitem{dickhoff2004}
\bibitem{barbieri2007}
\bibitem{cipollone2013}
\bibitem{white2002}
\bibitem{yanai2006}
\bibitem{anderson2010}
\bibitem{bogner2010}
\bibitem{blanes2009}
\bibitem{morris2015}
\bibitem{jurgenson2010}
\bibitem{hebeler2012}


\end{thebibliography}

\label{chap:chapter10}
\include{chapter11}\label{chap:chapter11}


\end{document}

\printindex

%%%%%%%%%%%%%%%%%%%%%%%%%%%%%%%%%%%%%%%%%%%%%%%%%%%%%%%%%%%%%%%%%%%%%%

\end{document}





